% Note to Tom:  Must fix the parenthesis and brackets in the expression
% for the Lagrangian.  Here texsis is causing problems. Ask Francois.
% See marker below




%\input texsis
%\book
%\input robustin
%\input grafinp3
%\input grafinput8
%\input psfig
%\eqnotracetrue




%\showchaptIDtrue
%\def\@chaptID{4.}


%\eqnotracetrue


%\hbox{}
\def\sumst{{\sum_{t=0}^\infty \sum_{s^t}}}
\def\Ast{{A_t(s^t)}}
\def\cst{{c_t(s^t)}}
\def\kst{{k_t(s^{t-1})}}
\def\kstone{{k_{t+1}(s^t)}}
\def\nst{{n_t(s^t)}}
\def\xst{{x_t(s^t)}}
\def\qst{{q_t^0(s^t)}}
\def\qstone{{q_{t-1}^0(s^{t-1})}}
\def\wst{{w_t^0(s^t)}}
\def\rst{{r_t^0(s^t)}}
\def\rstilde{{r_t^0(\tilde s^t)}}
\def\pist{{\pi_t(s^t)}}
\def\must{{\mu_t(s^t)}}
\def\mustone{{\mu_{t+1}(s^{t+1})}}

\def\thetast{{\theta_{t}(s^{t})}}
\def\thetastilde{{\theta_{t}(\tilde s^{t})}}
\def\thetastone{{\theta_{t-1}(s^{t-1})}}
\def\lhake{{\Bigl[}}




\chapter{Recursive Competitive Equilibrium: II\label{growth1}}
\footnum=0
\section{Endogenous aggregate state variable}  %%Introduction}
For pure endowment stochastic economies, chapter \use{recurge}  described
two types of competitive equilibria, one  in the style of Arrow and Debreu
with markets that convene
at time $0$ and trade
a complete set of history-contingent securities, another with markets
that meet each period and trade a complete set
of one-period-ahead state-contingent securities called  Arrow
securities.  Though their price systems and trading protocols
differ, both types of equilibria support identical
equilibrium allocations.
Chapter \use{recurge} described how to transform the Arrow-Debreu price
system into one for pricing Arrow securities.
 The key
step in transforming an equilibrium with time $0$ trading into one
with sequential trading was to account for how individuals' wealth
evolve as time passes in a time $0$ trading economy.
In a time $0$ trading economy, individuals do not make any
 trades other than those executed in period $0$, but the present
value of those portfolios change as time passes and as uncertainty gets
resolved.
So in period $t$ after some history $s^t$, we used the Arrow-Debreu
prices to compute the value of an individual's purchased
claims to current and future goods net of his outstanding liabilities.
We could then show that these wealth levels (and the
associated consumption choices) could also be attained in a sequential-trading
economy where there are only markets in one-period Arrow securities that reopen in each period.


In chapter \use{recurge} we also demonstrated how to obtain a
recursive formulation of the equilibrium with sequential trading.
This required us to assume that individuals' endowments
were governed by a Markov process. Under that assumption
we could identify a state vector
in terms of which the Arrow securities could be cast.  This
(aggregate)
state vector  then became a component of the state
vector for each individual's problem.
This transformation of price systems is easy in
    the pure exchange economies of chapter \use{recurge}
because in equilibrium,  the relevant state variable, wealth, is a
function solely of the
current realization of the exogenous Markov state variable.
%%%history of an exogenous Markov state variable.
The transformation is more subtle in economies in which
part of the aggregate state is endogenous in the sense that it
emerges from the
{\it history}
of
equilibrium interactions of agents' decisions. In
this chapter, we use the basic stochastic growth model (sometimes
also called the real business cycle model) as a laboratory for
moving from an equilibrium with time $0$ trading to a sequential
equilibrium with trades of Arrow securities.\NFootnote{The stochastic
growth model was formulated and fully analyzed by Brock and Mirman (1972).  It
is a workhorse for studying macroeconomic fluctuations.
Kydland and Prescott (1982) used the framework to study quantitatively
the importance of persistent technology shocks for business cycle
fluctuations. Other researchers have used the stochastic
growth model as a point of departure when exploring the implications of
injecting various frictions into that otherwise frictionless environment.} %
We also formulate  a
recursive competitive equilibrium with trading in Arrow securities
by using a version of
  the ``Big $K$, little $k$'' device that is often used
in macroeconomics.

\index{growth model!stochastic}
\auth{Brock, William A.}
\auth{Mirman, Leonard J.}
\index{real business cycle model}
\auth{Kydland, Finn E.}
\auth{Prescott, Edward C.}







\section{The stochastic growth model}
Here we spell out the basic ingredients of the stochastic growth model:
preferences, endowment, technology, and information. The
environment is the same as in chapter \use{linappro} except
that we now allow for a stochastic technology level.
In each period $t\geq 0$, there is a realization of a stochastic
event $s_t \in S$. Let the history of events up to time $t$
be denoted $s^t = [s_t, s_{t-1}, \ldots, s_0]$. The unconditional
probability of observing a particular sequence of events $s^t$ is
given by a probability measure $\pi_t(s^t)$. We write conditional
probabilities as $\pi_\tau(s^\tau\vert s^t)$, which is the probability
of observing $s^\tau$ conditional on the realization of $s^t$.
In this chapter, we assume that the state $s_0$ in period 0 is
nonstochastic, and hence $\pi_0(s_0)=1$ for a particular
$s_0\in{\bf S}$. We use $s^t$ as a commodity space in which
goods are differentiated by histories.


A representative household has preferences over nonnegative
streams of consumption $\cst$ and leisure $\ell_t(s^t)$ that
are ordered by
$$ \sum_{t=0}^\infty \sum_{s^t} \beta^t u[\cst,\ell_t(s^t) ]\pist  \EQN gr1 $$
where $\beta \in (0,1)$ and $u$ is strictly increasing in its
two arguments, twice continuously differentiable, strictly concave,
and satisfies the Inada conditions
$$
\lim_{c\to 0} u_c(c,\ell)   \,=\,
\lim_{\ell\to 0} u_\ell(c,\ell)   \,=\, \infty.
$$

In each period, the representative household is endowed with one
unit of time that can be devoted to leisure $\ell_t(s^t)$
or labor $\nst$:
$$             1     = \ell_t(s^t) +\nst.      \EQN gr2_endow $$
The only other endowment is a capital stock $k_0$ at the
beginning of period $0$.


The technology is
$$\EQNalign{\cst + \xst& \leq A_t(s^t) F(\kst, \nst), \EQN gr2;a \cr
            \kstone & = (1-\delta) \kst + \xst, \EQN gr2;b \cr}$$
%             1      & = \ell_t(s^t) +\nst,      \EQN gr2;c \cr
%            \Ast & =  s_0 s_1 \cdots s_t  A_0 , \EQN gr2;d \cr
%            &\hskip-1cm \cst, \ell_t(s^t), \nst, k_{t+1}(s^t) \geq 0,\cr
%            &\hskip-1cm \hbox{\rm given} \;\, k_0.\cr}$$
where $F$ is a twice continuously differentiable,
constant-returns-to-scale production function  with inputs capital
$\kst$ and labor $\nst$, and $A_t(s^t)$ is a stochastic process of
Harrod-neutral technology shocks. Outputs are the consumption good
$\cst$ and the investment good $\xst$.  In \Ep{gr2;b}, the
investment good augments a capital stock that is depreciating at
the rate $\delta$. Negative values of $\xst$ are permissible, which
means that the capital stock can be reconverted into the
consumption good.




We assume that the production function satisfies standard assumptions
of positive but diminishing marginal products,
$$
F_i(k,n)>0, \quad F_{ii}(k,n)<0, \quad {\rm for}\; i=k,n;
$$
and the \idx{Inada condition}s,
$$\EQNalign{
\lim_{k\to 0}F_k(k,n) \,&=\, \lim_{n\to 0}F_n(k,n) =\infty, \cr
\lim_{k\to \infty}F_k(k,n) \,&=\, \lim_{n\to \infty}F_n(k,n) =0. \cr}
$$
Since the production function has constant returns to scale, we
can define
$$
F(k,n) \,\equiv\, n f(\hat k) \quad {\rm where} \quad
                               \hat k\equiv {k \over n}.   \EQN ff_fratio
$$
Another property of a linearly homogeneous
function $F(k,n)$ is that its first derivatives are homogeneous of degree 0,
and thus the first derivatives are functions only of the ratio $\hat k$. In
particular, we have
$$\EQNalign{
&F_k(k,n) = {\partial \, n f\left({k / n}\right) \over \partial \, k }
         = f'(\hat k),                                  \EQN ff_partial1;a   \cr
\noalign{\vskip.3cm}
&F_n(k,n) = {\partial \, n f\left({k / n}\right) \over \partial \, n }
         = f(\hat k) - f'(\hat k) \hat k.            \EQN ff_partial;b  \cr}
$$






\section{Lagrangian formulation of the planning problem}
The  planner chooses an allocation
$\{\cst, \ell_t(s^t), \xst,  \nst, k_{t+1}(s^t)\}_{t=0}^{\infty}$ to
maximize \Ep{gr1} subject to \Ep{gr2_endow} and \Ep{gr2},
the initial capital stock $k_0$, and the stochastic process for the
technology level $A_t(s^t)$.
To solve this planning problem, we
form the Lagrangian
%% Ask Francois how to fix this!
\offparens
$$\eqalign{ L & = \sumst \beta^t \pist  \{ u(\cst,1-\nst) \cr
   &+ \must [ \Ast F(\kst,\nst)
 + (1-\delta)\kst   -\cst -\kstone ]\} \cr} $$
\autoparens
where $\mu_t(s^t)$ is a process of Lagrange multipliers  on
the technology constraint.
First-order conditions with respect to $\cst$, $\nst$, and $\kstone,$ respectively,
are
$$\EQNalign{  u_c(s^t) & = \must, \EQN pp1;a \cr
           u_\ell(s^t) & = u_c(s^t) \Ast F_n(s^t), \EQN pp1;b \cr
          u_c(s^t) \pist & = \beta \sum_{s^{t+1} \vert s^t}
             u_c(s^{t+1}) \pi_{t+1}(s^{t+1}) \cr
    & \hskip1cm\left[A_{t+1}(s^{t+1})
     F_k(s^{t+1}) + (1-\delta)\right], \EQN pp1;c \cr}$$
where the summation over
$s^{t+1} \vert s^t$ means that we sum
over all possible histories $\tilde s^{t+1}$ such that
$\tilde s^t=s^t$.


%%\section{Decentralization after Arrow-Debreu}
\section{Time $0$ trading: Arrow-Debreu securities}
In the style of Arrow and Debreu, we can support the allocation
that solves the planning problem by a competitive equilibrium with
time $0$ trading of a complete set of date- and
history-contingent securities.  Trades occur among a
representative household and two types of representative
firms.\NFootnote{
One can also support the allocation that solves the planning
problem with a less decentralized setting with only the first of
our two types of firms, and in which the decision for making
physical investments is assigned to the household. We assign that
decision to a second type of firm because we want to price more
items, in particular, the capital stock.}


We let $[q^0,  w^0, r^0, p_{k0}]$ be a price system where
$p_{k0}$ is the price of a unit of the initial capital stock, and
each of $q^0$, $w^0$, and $r^0$ is a stochastic process of prices for
output and for renting labor and capital, respectively, and   the
time $t$ component of each is indexed by the history $s^t$. A
representative household purchases consumption goods from a type I
firm and sells labor services to the type I firm that operates
the production technology \Ep{gr2;a}.
The household owns the initial capital stock $k_0$ and at date
$0$ sells it to a type II firm.  The type II firm operates the
capital storage technology \Ep{gr2;b}, purchases new investment
goods $x_{t}$ from a type I firm, and rents stocks of capital back
to the type I firm.


 We now describe the problems of the
representative household and the two types of firms in the economy
with time $0$ trading.


\subsection{Household}
The household maximizes
$$ \sum_t\sum_{s^t} \beta^t u\left[\cst, 1-\nst\right]\,\pist  \EQN hh1 $$
subject to
$$ \sumst \qst \cst \leq \sumst \wst \nst + p_{k0} k_0. \EQN hh2  $$
First-order conditions with respect to $\cst$ and $\nst$, respectively,
are
$$\EQNalign{ \beta^t u_c(s^t) \pist & = \eta\qst, \EQN hhh1;a \cr
                   \beta^t u_\ell(s^t) \pist & = \eta \wst, \EQN hhh1;b \cr}$$
where $\eta>0$ is a multiplier on the budget constraint
\Ep{hh2}.


\subsection{Firm of type I}
The representative firm of type I operates the production technology
 \Ep{gr2;a}
with capital and labor that it rents at market prices. For each period
$t$ and each realization of history $s^t$, the firm enters into
state-contingent contracts at time $0$ to rent capital $k^{I}_t(s^t)$
and labor services $n_t(s^t)$. The type I firm
seeks to maximize
$$ \sumst \left\{\qst [\cst +\xst]   -\rst k^I_t(s^t)
   -\wst \nst \right\} \EQN fff1 $$
subject to
$$ \cst +\xst \leq \Ast F(k^I_t(s^t), \nst). \EQN fff2 $$
After substituting \Ep{fff2} into \Ep{fff1} and invoking
\Ep{ff_fratio}, the firm's objective function can be
expressed alternatively as
$$ \sumst \nst\left\{\qst \Ast f\!(\hat k^I_t(s^t))
    -\rst \hat k^I_t(s^t)  -\wst \right\} \hskip-.1cm\EQN fff1prime $$
and the maximization problem can then be decomposed into two parts.
First, conditional on operating the production technology
in period $t$ and history $s^t$, the firm solves for the
profit-maximizing capital-labor ratio, denoted $k^{I\star}_t(s^t)$.
Second, given that capital-labor ratio $k^{I\star}_t(s^t)$,
the firm determines the profit-maximizing level of its operation
by solving for the optimal employment level, denoted
$n^\star_t(s^t)$.


The firm finds the profit-maximizing capital-labor
ratio by maximizing the expression
in curly brackets in \Ep{fff1prime}. The first-order condition
with respect to $\hat k^{I}_t(s^t)$ is
$$ \qst \Ast f^\prime(\hat k^I_t(s^t))
    -\rst   = 0 \,.                          \EQN ff_focIk
$$
At the optimal capital-labor ratio $\hat k^{I\star}_t(s^t)$
that satisfies \Ep{ff_focIk}, the firm evaluates
the expression in curly brackets in \Ep{fff1prime} in order
to determine the optimal level of employment $n_t(s^t)$.
In particular, $n_t(s^t)$ is optimally set equal to zero or
infinity if the expression in curly brackets in \Ep{fff1prime}
is strictly negative or strictly positive, respectively. However,
if the expression in curly brackets is zero in some period $t$ and
history $s^t$, the firm would be indifferent to the level of
$n_t(s^t)$, since profits are then equal to zero for all levels of
operation in that period and state. Here, we summarize the optimal
employment decision by using equation \Ep{ff_focIk} to eliminate
$r^0_t(s^t)$ in the expression in curly brackets in \Ep{fff1prime};
$$\EQNalign{\hbox{\rm if}\hskip.2cm
&\left\{\qst \Ast \left[ f\!(\hat k^{I\star}_t(s^t))
       - f^\prime(\hat k^{I\star}_t(s^t))\,\hat k^{I\star}_t(s^t)\right]
                                                 -\wst \right\} \cr
 &\hskip2.5cm \cases{\,<0, &then $n^{\star}_t(s^t)=0$; \cr
                        \,=0, &then $n^{\star}_t(s^t)$ is indeterminate; \cr
                        \,>0, &then $n^{\star}_t(s^t)=\infty$. \cr}
                                                      \hskip1cm\EQN fff_nstar }
$$
In an equilibrium, both $k^{I}_t(s^t)$ and $n_t(s^t)$ are strictly positive
and finite, so expressions \Ep{ff_focIk} and \Ep{fff_nstar}
imply the following equilibrium prices:
$$\EQNalign{\qst \Ast F_k(s^t) & = \rst \EQN fff3;a \cr
            \qst \Ast F_n(s^t) & = \wst  \EQN fff3;b \cr} $$
where we have invoked \Ep{ff_partial1}.




\subsection{Firm of type II}
The representative firm of type II operates technology \Ep{gr2;b}
to transform output into capital. % then rents capital  to a type I firm.
The type II firm purchases
capital at time
$0$ from the household sector and thereafter invests in new capital,
earning revenues by renting capital to the type I firm.  It maximizes
$$ - p_{k0}k_0^{II} +
\sumst \left\{ \rst  k_t^{II}(s^{t-1}) - \qst \xst \right\}
                                         \EQN f2f1 $$
subject to
$$ k_{t+1}^{II}(s^t) = (1-\delta) k_t^{II}(s^{t-1}) + x_t(s^t). \EQN f2f2 $$
Note that the firm's capital stock in period $0$, $k_0^{II}$, is bought
without any uncertainty about the rental price in that period while
the investment in capital for a future period $t$, $k_t^{II}(s^{t-1})$,
is conditioned on the realized history $s^{t-1}$.
Thus, the type II firm manages the risk associated with technology
constraint \Ep{gr2;b} that states that capital must be assembled one
period prior to becoming an input for production.
In contrast, the type I firm of the previous
subsection can choose how much capital $k_t^{I}(s^{t})$ to rent
in period $t$ conditioned on history $s^t$.




After substituting \Ep{f2f2} into \Ep{f2f1} and rearranging,
the type II firm's objective function can be written as
$$\EQNalign{&   k_0^{II}\left\{-p_{k0}
         +r_0^0(s_0) + q_0^0(s_0)(1-\delta)\right\}
+ \sum_{t=0}^\infty \sum_{s^{t}} k_{t+1}^{II}(s^{t})    \cr
  &  \hskip.5cm \cdot\Biggl\{-q^0_{t}(s^{t})
+ \sum_{s^{t+1} \vert s^{t}} \left[r^0_{t+1}(s^{t+1})
              + q^0_{t+1}(s^{t+1}) (1-\delta) \right]\Biggr\} ,
                                         \hskip1cm               \EQN ff_IIk \cr}
$$
where the firm's profit is a linear function of investments in
capital. The profit-maximizing level of the capital stock
$k^{II}_{t+1}(s^{t})$ in expression \Ep{ff_IIk} is equal to
zero or infinity if the associated multiplicative term in
curly brackets is strictly negative or strictly positive,
respectively. However, for any expression in curly brackets
in \Ep{ff_IIk} that is zero, the firm would be
indifferent to the level of $k^{II}_{t+1}(s^{t})$, since profits
 then equal  zero for all levels of investment.
In an equilibrium, $k^{II}_0$ and $k^{II}_{t+1}(s^{t})$ are
strictly positive and finite, so each expression in curly brackets
in \Ep{ff_IIk} must equal zero, and hence equilibrium prices must satisfy
$$\EQNalign{  p_{k0} &=r_0^0(s_0) + q_0^0(s_0)(1-\delta), \EQN ffprice;a \cr
\noalign{\vskip.2cm}
    q^0_{t}(s^{t})& =\sum_{s^{t+1} \vert s^{t}}
      \left[r^0_{t+1}(s^{t+1}) + q^0_{t+1}(s^{t+1}) (1-\delta) \right]. \EQN ffprice;b \cr}
$$


\subsection{Equilibrium prices and quantities}
According to equilibrium conditions \Ep{fff3}, each input
in the production technology is paid its marginal product,
and hence profit maximization of the type I firm ensures an efficient
allocation of labor services and capital. But nothing is said about
the equilibrium quantities of labor and capital.
Profit maximization of the type II firm imposes no-arbitrage restrictions
\Ep{ffprice} across prices $p_{k0}$ and $\{r^0_t(s^t), q^0_t(s^t)\}$.
But nothing is said about the specific equilibrium value of an
individual price. To solve for equilibrium prices and quantities, we
turn to the representative household's first-order conditions \Ep{hhh1}.


After substituting \Ep{fff3;b} into the household's first-order condition
\Ep{hhh1;b}, we obtain
$$\EQNalign{
 \beta^t u_\ell(s^t) \pist  &= \eta q^0_{t}(s^{t})
    A_{t}(s^{t}) F_n(s^{t}) ;           \EQN ff_final;a \cr
%\noalign{\vskip.3cm}
\noalign{\noindent \rm  and then by substituting \Ep{ffprice;b} and
\Ep{fff3;a} into \Ep{hhh1;a},}
\noalign{\vskip.3cm}
\beta^t u_c(s^t) \pist  &= \eta \sum_{s^{t+1} \vert s^{t}}
    \left[r^0_{t+1}(s^{t+1}) + q^0_{t+1}(s^{t+1}) (1-\delta) \right] \cr
=\eta \sum_{s^{t+1} \vert s^{t}} &q^0_{t+1}(s^{t+1})
    \left[A_{t+1}(s^{t+1}) F_k(s^{t+1}) + (1-\delta) \right]. \hskip1cm
                                                       \EQN ff_final;b\cr}
$$
Next, we use $q^0_t(s^t) =\beta^t u_c(s^t) \pist/\eta$ as given
by the household's first-order condition \Ep{hhh1;a} and the
corresponding expression for $q^0_{t+1}(s^{t+1})$ to substitute
into \Ep{ff_final;a} and \Ep{ff_final;b}, respectively. This step
produces expressions identical to the planner's first-order
conditions \Ep{pp1;b} and \Ep{pp1;c}, respectively. In this way,
we have verified that the allocation in the competitive
equilibrium with time $0$ trading is the same as the allocation
that solves the planning problem.


Given the equivalence of allocations, it is standard to compute
the competitive equilibrium allocation by solving the
planning problem since the latter problem is a simpler one.
We can compute equilibrium  prices by substituting
the allocation from the planning problem into the household's and
firms' first-order conditions. All relative prices are then
determined, and in order to pin down absolute prices, we would also
have to pick a numeraire. Any such normalization of prices is
tantamount to setting the multiplier $\eta$ on the
household's present value budget constraint equal to an
arbitrary positive number. For example, if we set $\eta=1$,
we are measuring prices in units of marginal utility of the
time $0$ consumption good.  Alternatively, we can set $q_0^0(s_0) =1$
by setting
$\eta = u_c(s_0)$.  We can compute $q_t^0(s^t)$ from \Ep{hhh1;a},
$w_t^0(s^t)$ from \Ep{hhh1;b}, and $r_t^0(s^t)$ from
\Ep{fff3;a}.  Finally, we can compute $p_{k0}$
from
\Ep{ffprice;a}  to get
$p_{k0} = r_0^0(s_0) + q_0^0(s_0)(1-\delta)$.




\subsection{Implied wealth dynamics}
Even though trades are only executed at time $0$ in the Arrow-Debreu
market structure, we can study how the representative household's
wealth evolves over time. For that purpose, after a given
history $s^t$, we convert all prices, wages, and rental rates that
are associated with current and future deliveries so that they
are expressed in terms of time $t$, history $s^t$ consumption goods,
i.e., we change the numeraire:
$$ \EQNalign{  q^t_\tau(s^\tau) &\equiv
{q_\tau^0(s^\tau) \over q_t^0(s^t)}
         = \beta^{\tau - t} {u_c(s^\tau) \over u_c(s^t)}
          \,\pi_\tau\!(s^\tau|s^t),     \EQN price_renorm;a \cr
w^t_\tau(s^\tau) &\equiv
{w_\tau^0(s^\tau) \over q_t^0(s^t)}, \EQN price_renorm;b \cr
r^t_\tau(s^\tau) &\equiv
{r_\tau^0(s^\tau) \over q_t^0(s^t)}. \EQN price_renorm;c \cr}
$$
%$$ \EQNalign{  q^t_\tau(s^\tau) &\equiv
%{q_\tau^0(s^\tau) \over q_t^0(s^t)}
%         = \beta^{\tau - t} {u'[c_\tau(s^\tau)] \over u'[c_t(s^t)]}
%          \,\pi_\tau\!(s^\tau|s^t),     \EQN price_renorm;a \cr
%w^t_\tau(s^\tau) &\equiv
%{w_\tau^0(s^\tau) \over q_t^0(s^t)}, \EQN price_renorm;b \cr
%r^t_\tau(s^\tau) &\equiv
%{r_\tau^0(s^\tau) \over q_t^0(s^t)}. \EQN price_renorm;c \cr}
%$$


In chapter \use{recurge} we asked the question, what is
the implied wealth of a household at time $t$ after history
$s^t$ when excluding the endowment stream?
Here we ask the same question except  that now instead of
endowments, we exclude
the value of labor. For example, the household's net claim
to delivery of goods in a future period $\tau \geq t$,
contingent on history $s^\tau$, is given by
$[q_\tau^t(s^\tau)c_\tau(s^\tau) - w_\tau^t(s^\tau)n_\tau(s^\tau)]$,
as expressed in terms of time $t$, history $s^t$ consumption
goods. Thus, the household's wealth, or the value of all its
current and future net claims, expressed in terms of the date $t$,
history $s^t$ consumption good, is
\offparens
$$\EQNalign{
&\Upsilon_t(s^t) \equiv
\sum_{\tau=t}^\infty \; \sum_{s^\tau \vert s^t} \Bigl\{
    q_\tau^t(s^\tau) c_\tau(s^\tau) - w_\tau^t(s^\tau)n_\tau(s^\tau)\Bigr\}\cr
&=
\sum_{\tau=t}^\infty \; \sum_{s^\tau \vert s^t} \Bigl\{
q_\tau^t(s^\tau) \Bigl[ A_\tau(s^\tau) F(k_\tau(s^{\tau-1}), n_\tau(s^\tau)) \cr
&\hskip4cm     + (1-\delta) k_\tau(s^{\tau-1}) - k_{\tau+1}(s^\tau) \Bigr]
             - w_\tau^t(s^\tau)n_\tau(s^\tau) \Bigr\}\cr
&=
\sum_{\tau=t}^\infty \; \sum_{s^\tau \vert s^t} \Bigl\{
 q_\tau^t(s^\tau) \Bigl[ A_\tau(s^\tau)
 \Bigl( F_k(s^\tau) k_\tau(s^{\tau-1}) + F_n(s^\tau) n_\tau(s^\tau) \Bigr)     \cr
&\hskip4cm     + (1-\delta) k_\tau(s^{\tau-1}) - k_{\tau+1}(s^\tau) \Bigr]
             - w_\tau^t(s^\tau)n_\tau(s^\tau) \Bigr\}\cr
&=
\sum_{\tau=t}^\infty \; \sum_{s^\tau \vert s^t} \Bigl\{
    r_\tau^t(s^\tau) k_\tau(s^{\tau-1}) +
    q_\tau^t(s^\tau) \Bigl[ (1-\delta) k_\tau(s^{\tau-1})
                            - k_{\tau+1}(s^\tau) \Bigr] \Bigr\} \cr
\noalign{\vskip.2cm}
&=  r_t^t(s^t) k_t(s^{t-1}) + q_t^t(s^t) (1-\delta) k_t(s^{t-1})   \cr
\noalign{\vskip.2cm}
&\hskip.5cm   + \sum_{\tau=t+1}^\infty \; \sum_{s^{\tau-1} \vert s^t}
    \biggl\{ \sum_{s^{\tau} \vert s^{\tau-1}}
    \Bigl[r_\tau^t(s^\tau) +
    q_\tau^t(s^\tau) (1-\delta) \Bigr] - q_{\tau-1}^t(s^{\tau-1}) \biggr\}
                            k_{\tau}(s^{\tau-1})  \cr
\noalign{\vskip.2cm}
&=  \Bigl[ r_t^t(s^t) + (1-\delta) \Bigr] k_t(s^{t-1}),   \EQN seq_wealth \cr}
$$
where the first equality uses the equilibrium outcome that consumption
is equal to the difference between production and investment in each period,
the second equality follows from Euler's theorem on linearly homogeneous
functions,\NFootnote{According to Euler's theorem on linearly
homogeneous functions, our constant-returns-to-scale production function
satisfies
$$
 F(k,n) = F_k(k,n)\,k +  F_n(k,n)\,n.
$$}
the third equality invokes equilibrium input
prices in \Ep{fff3}, the fourth equality is merely a rearrangement of
terms, and the final, fifth equality acknowledges that $q_t^t(s^t) = 1$
and that each term in curly brackets is zero because of equilibrium price
condition \Ep{ffprice;b}.




\section{Sequential trading: Arrow securities}
As in chapter \use{recurge}, we now demonstrate that sequential
trading in one-period Arrow securities provides an alternative
market structure that preserves the allocation from
the time $0$ trading equilibrium.
In the production economy with sequential trading, we
will also have to include markets for labor and capital services
that reopen in each period.


We guess that at time $t$ after history $s^t$, there exist a
wage rate $\tilde w_t(s^t)$, a rental rate $\tilde r_t(s^t)$,
and Arrow security prices $\tilde Q_t(s_{t+1} | s^t)$.
The {\it pricing kernel} $\tilde Q_t(s_{t+1} | s^t)$ is to be
interpreted as follows:
$\tilde Q_t(s_{t+1} | s^t)$ gives the price of one unit of
time $t+1$ consumption,
contingent on the realization $s_{t+1}$ at $t+1$, when the
history at $t$ is $s^t$.


\subsection{Household}
At each date $t \geq 0$ after history $s^t$,
the representative household buys consumption
goods $\tilde c_t(s^t)$, sells labor services $\tilde n_t(s^t)$, and
trades claims to date $t+1$ consumption,
whose payment is  contingent
on the realization of $s_{t+1}$.   Let $\tilde a_t(s^t)$ denote the
claims to time $t$ consumption
that the household brings into time $t$ in history $s^t$. Thus,
the household faces a sequence of budget constraints
for $t \geq 0$, where the time $t$, history $s^t$ budget constraint is
$$ \tilde c_t(s^t) + \sum_{s_{t+1}} \tilde a_{t+1}(s_{t+1},s^t)
\tilde Q_t(s_{t+1} | s^t)
     \leq  \tilde w_t(s^t) \tilde n_t(s^t) + \tilde a_t(s^t) , \EQN seq_hh0   $$
 where
$\{\tilde a_{t+1}(s_{t+1},s^t)\}$  is a vector of claims on
time $t+1$ consumption,
one element of the vector for each  value of the time $t+1$ realization of
$s_{t+1}$.


To rule out Ponzi schemes, we must impose borrowing constraints on
the household's asset position. We could follow the approach of
chapter \use{recurge} and compute state-contingent
{\it natural debt limits,\/} where the counterpart to the earlier present
value of the household's endowment stream would be the present
value of the household's time endowment. Alternatively, we  just
impose that the household's indebtedness in any state
next period, $-\tilde a_{t+1}(s_{t+1},s^t)$, is bounded by some
arbitrarily large constant. Such an arbitrary debt limit works
well for the following reason. As long as the household is constrained
so that it cannot run a true Ponzi scheme with an unbounded budget
constraint, equilibrium forces will ensure that the representative
household willingly holds the market portfolio. In the present setting, we
can for example set that arbitrary debt limit equal to zero, as will
become clear as we go along.


Let $\eta_t(s^t)$ and $\nu_t(s^t;s_{t+1})$ be the nonnegative
Lagrange multipliers on the budget constraint \Ep{seq_hh0} and the
borrowing constraint with an arbitrary debt limit of zero,
respectively, for time $t$
and history $s^t$. The Lagrangian can then be formed as \offparens
$$ \eqalign{ L =
 &\sum_{t=0}^\infty \sum_{s^t}
   \Bigr\{  \beta^t u(\tilde c_t(s^t), 1 - \tilde n_t(s^t) ) \,\pi_t(s^t)  \cr
 & +\eta_t(s^t) \Bigl [\tilde w_t(s^t) \tilde n_t(s^t)
                               + \tilde a_t(s^t) - \tilde c_t(s^t)
 - \sum_{s_{t+1}} \tilde a_{t+1}(s_{t+1},s^t) \tilde Q_t(s_{t+1} | s^t) \Bigr] \cr
 & +\nu_t(s^t;s_{t+1})  \tilde a_{t+1}(s^{t+1})
                                                             \Bigl\}, \cr} $$
for a given initial wealth level $\tilde a_0$. In an equilibrium,
the representative
household will choose interior solutions for
$\{\tilde c_t(s^t), \, \tilde n_t(s^t)\}_{t=0}^\infty$ because of
the assumed Inada conditions. The Inada conditions on the utility function
ensure that the household will  set neither $\tilde c_t(s^t)$
nor $\ell_t(s^t)$ equal to zero, i.e., $\tilde n_t(s^t)<1$. The Inada
conditions on the production function guarantee that the
household will always find it desirable to supply some labor,
$\tilde n_t(s^t)>0$. Given these interior solutions,
the first-order conditions for maximizing $L$ with respect to
$\tilde c_t(s^t)$, $\tilde n_t(s^t)$ and
$\{\tilde a_{t+1}(s_{t+1},s^t)\}_{s_{t+1}}$ are
$$\EQNalign{
\beta^t u_c(\tilde c_t(s^t),1 - \tilde n_t(s^t)  ) \,\pi_t(s^t)
                              -\eta_t(s^t) &=0\,,\EQN seq_hhFOC;a \cr
\noalign{\vskip.2cm}
-\beta^t u_\ell(\tilde c_t(s^t),1 - \tilde n_t(s^t)  ) \,\pi_t(s^t)
             +\eta_t(s^t) \tilde w_t(s^t) &=0\,,\EQN seq_hhFOC;b \cr
\noalign{\vskip.2cm}
-\eta_t(s^t) \tilde Q_t(s_{t+1} | s^t) + \nu_t(s^t;s_{t+1})
+ \eta_{t+1}(s_{t+1},s^t)  &=0\,, \hskip1cm \EQN seq_hhFOC;c \cr}
$$
for all $s_{t+1}$, $t$, $s^t$. Next, we proceed under the
conjecture that the arbitrary debt limit of zero will not
be binding, and hence the Lagrange multipliers $\nu_t(s^t;s_{t+1})$
are all equal to zero.  After
setting those multipliers equal to zero in equation \Ep{seq_hhFOC;c}, the
first-order conditions imply the following conditions for the
optimal choices of consumption and labor:
$$\EQNalign{
\tilde w_t(s^t) &= { u_\ell(\tilde c_t(s^t),1 - \tilde n_t(s^t)  ) \over
                       u_c(\tilde c_t(s^t),1 - \tilde n_t(s^t)  ) },
         \EQN seq_hhFOCc;a \cr
\noalign{\vskip.2cm}
 \tilde Q_t(s_{t+1} | s^t) &= \beta
{ u_c(\tilde c_{t+1}(s^{t+1}),\, 1-\tilde n_{t+1}(s^{t+1}) ) \over
  u_c(\tilde c_t(s^t),\, 1-\tilde n_t(s^t) ) }\,
\pi_t(s^{t+1} | s^t),                                \EQN seq_hhFOCc;b \cr}
$$
for all $t$, $s^t$, and $s_{t+1}$.




\subsection{Firm of type I}\label{sec:FirmI}%
At each date $t \geq 0$ after history $s^t$,
a type I firm is a production firm that
chooses a quadruple $\{\tilde c_t(s^t),\, \tilde x_t(s^t), \,
\tilde k_t^I(s^t),\, \tilde n_t(s^t)\}$
to solve a static optimum problem:
$$ \max\Bigl\{ \tilde c_t(s^t) + \tilde x_t(s^t)
- \tilde r_t(s^t) \tilde k^I_t(s^t)
- \tilde w_t(s^t) \tilde n_t(s^t) \Bigr\}      \EQN seq_f1a  $$
subject to
$$ \tilde c_t(s^t) + \tilde x_t(s^t) \leq  A_t(s^t)
F( \tilde k^I_t(s^t),\, \tilde n_t(s^t) ). \EQN seq_f1b $$
The zero-profit conditions are
$$ \EQNalign{ \tilde r_t(s^t) & = A_t(s^t) F_k(s^t),  \EQN seq_f1c;a \cr
              \tilde w_t(s^t) & = A_t(s^t) F_n(s^t).  \EQN seq_f1c;b \cr} $$
If conditions \Ep{seq_f1c} are violated, the type I firm
either makes infinite profits by hiring infinite capital
and labor, or else it makes negative profits for any positive
output level, and therefore shuts down.  If conditions \Ep{seq_f1c}
are satisfied, the firm makes zero profits and its size is
indeterminate.  The firm of type I is willing to produce
any quantities of $\tilde c_t(s^t)$ and $\tilde x_t(s^t)$ that the
market demands, provided
that conditions \Ep{seq_f1c} are satisfied.




\subsection{Firm of type II}\label{sec:FirmII}%
A type  II firm transforms output into capital, stores capital,
 and earns its revenues by renting
capital to the type I firm. Because of the technological
assumption that capital can be converted back into the consumption
good, we can without loss of generality consider a
two-period optimization problem where a type II firm decides how
much capital $\tilde k_{t+1}^{II}(s^t)$ to store at the end of
period $t$ after history $s^t$ in order to earn a stochastic
rental revenue $\tilde r_{t+1}(s^{t+1}) \,\tilde k_{t+1}^{II}(s^t)$
and liquidation value $(1-\delta) \,\tilde k_{t+1}^{II}(s^t)$
in the following period. The firm finances itself by issuing state-contingent
 debt to the households, so future income streams can
be expressed in today's values by using prices
$\tilde Q_t(s_{t+1} | s^t)$.  Thus, at each date $t \geq 0$ after
history $s^t$, a type II firm chooses $\tilde k_{t+1}^{II}(s^t)$
to solve the optimum problem
$$ \max   \; \tilde k_{t+1}^{II}(s^t) \Bigl\{ -1 +
\sum_{s_{t+1}} \tilde Q_t(s_{t+1} | s^t)
\left[ \tilde r_{t+1}(s^{t+1}) + (1-\delta)
                           \right]\Bigr\} .    \EQN seq_f2a  $$
The zero-profit condition is
$$ 1 = \sum_{s_{t+1}} \tilde Q_t(s_{t+1} | s^t)
\Bigl[ \tilde r_{t+1}(s^{t+1}) + (1-\delta)
                           \Bigr]. \EQN seq_f2b $$


The size of the type II firm is indeterminate.
So long as condition \Ep{seq_f2b} is satisfied, the firm breaks even
at any level of $\tilde k_{t+1}^{II}(s^t)$.
If condition \Ep{seq_f2b} is not satisfied, either it
can earn infinite profits by setting $\tilde k_{t+1}^{II}(s^t)$
to be  arbitrarily large (when
the right side exceeds the left),
or it earns negative profits for any positive level of capital (when
the right side falls short of the left), and
so chooses to shut down.




\subsection{Equilibrium prices and quantities}
We leave it to the reader to follow the approach taken in
chapter \use{recurge} to show the equivalence of
allocations attained in the sequential equilibrium and the
time $0$ equilibrium,
$\{\tilde c_t(s^t), \tilde \ell_t(s^t), \tilde x_t(s^t),
\tilde n_t(s^t), \tilde k_{t+1}(s^t)\}_{t=0}^{\infty}=
\{\cst, \ell_t(s^t), \xst, $ $\nst, k_{t+1}(s^t)\}_{t=0}^{\infty}$.
The trick is to guess that the prices in the sequential
equilibrium satisfy
$$ \EQNalign{\tilde Q_t(s_{t+1} | s^t) &= q^t_{t+1}(s^{t+1}),\EQN price_seq;a \cr
             \tilde w_t(s^t) &= w_t^t(s^t),                \EQN price_seq;b \cr
             \tilde r_t(s^t) &= r_t^t(s^t).                \EQN price_seq;c \cr}
$$
The other set of guesses is that the representative household
chooses asset portfolios given by
$\tilde a_{t+1}(s_{t+1},s^t) = \Upsilon_{t+1}(s^{t+1})$ for all
$s_{t+1}$. When showing that the household can afford these asset
portfolios together with the prescribed quantities of consumption
and leisure, we will find that the required initial wealth is equal to
$$
\tilde a_0 = [r^0_0(s_0) + (1-\delta) ] k_0 = p_{k0} k_0,
$$
i.e., the household in the sequential equilibrium starts out
at the beginning of period $0$ owning
the initial capital stock, which is then sold to a type II firm at the
same competitive price as in the time $0$ trading equilibrium.






\subsection{Financing a type II firm}
A type II firm finances purchases of $\tilde k_{t+1}^{II}(s^t)$
units of capital in period $t$ after history $s^t$ by issuing
one-period state-contingent claims that promise
to pay
$$\left[ \tilde r_{t+1}(s^{t+1}) + (1-\delta) \right]
                                 \tilde k_{t+1}^{II}(s^t)$$
consumption goods tomorrow in state
$s_{t+1}$.  In units of today's time $t$ consumption good,
these payouts are worth \hfil\break
$$\sum_{s_{t+1}} \tilde Q_t(s_{t+1} | s^t)
\left[ \tilde r_{t+1}(s^{t+1}) + (1-\delta) \right]
                                \tilde k_{t+1}^{II}(s^t)$$
(by virtue of \Ep{seq_f2b}).
The firm breaks even by issuing these claims.
Thus, the firm of type II is entirely owned by its creditor, the
household, and it earns zero profits.


Note that the economy's end-of-period wealth as embodied
in $\tilde k_{t+1}^{II}(s^t)$ in period $t$ after history $s^t$
is willingly held by the representative household. This follows
immediately from fact that the household's desired
beginning-of-period wealth next period is given by
$\tilde a_{t+1}(s^{t+1})$ and is equal to $\Upsilon_{t+1}(s^{t+1})$,
as given by \Ep{seq_wealth}. Thus, the equilibrium prices entice the
representative household to enter each future period with
a strictly positive net asset level that is equal to the value
of the type II firm. We have then confirmed the correctness
of our earlier conjecture that the arbitrary
debt limit of zero is not binding in the household's optimization
problem.






\section{Recursive formulation}
Following the approach taken in chapter \use{recurge},
we have established that the
equilibrium allocations are the same in the Arrow-Debreu
economy with complete markets at time $0$ and in a
sequential-trading economy with complete one-period
Arrow securities. This finding holds for an arbitrary
technology process $A_t(s^t)$, defined as a measurable
function of the history of events $s^t$ which in turn are
governed by some arbitrary probability measure $\pi_t(s^t)$.
At this level of generality,
all prices $\{\tilde Q_t(s_{t+1} | s^t),\, \tilde w_t(s^t),\,
\tilde r_t(s^t)\}$ and the capital stock $k_{t+1}(s^t)$
in the
sequential-trading economy depend on the history $s^t$. That is,
these objects are time-varying functions of all past events
$\{s_\tau\}_{\tau=0}^t$.


In order to obtain a recursive formulation and solution to both the
social planning problem and the sequential-trading equilibrium,
we make the following specialization of the
exogenous forcing process for the technology level.




\subsection{Technology is governed by a Markov process}
Let the stochastic event $s_t$ be governed by a Markov process,
$[s\in {\bf S}, \pi(s'|s),$ $\pi_0(s_0)]$. We keep our earlier
assumption that the state $s_0$ in period 0 is
nonstochastic and hence $\pi_0(s_0)=1$ for a particular
$s_0\in{\bf S}$. The sequences of probability measures $\pi_t(s^t)$
on histories $s^t$ are induced by the Markov
process via the recursions
$$ \pi_t(s^t) = \pi(s_t | s_{t-1}) \pi(s_{t-1} | s_{t-2}) \ldots
       \pi(s_1 | s_0)   \pi_0(s_0).  $$


Next, we assume that the aggregate technology level $A_t(s^t)$
in period $t$ is a time-invariant measurable function of its
level in the last period and the current stochastic event $s_t$,
i.e., $A_t(s^t) = A\left( A_{t-1}(s^{t-1}),\, s_t\right)$. For
example, here we will proceed with the multiplicative version
$$
A_t(s^t) = s_t A_{t-1}(s^{t-1}) =  s_0 \,s_1 \cdots s_t \, A_{-1},
$$
given the initial value $A_{-1}$.






\subsection{Aggregate state of the economy}
The specialization of the technology process enables us to
adapt the recursive construction of chapter \use{recurge} to incorporate
additional components of the state of the economy. Besides information
about the current value of the stochastic event $s$, we need to know
last period's technology level, denoted $A$, in order to determine
current technology level, $s\,A$, and to forecast future technology
levels. This additional element $A$ in the aggregate state vector
does not constitute any conceptual change from what we did in
chapter \use{recurge}. We are merely including one more state
variable that is a direct mapping from exogenous stochastic events,
and it does not depend on any endogenous outcomes.


But we also need  to expand the aggregate state vector with
an endogenous component of the state of the economy, namely, the
beginning-of-period capital stock $K$. Given the new state
vector $X\equiv \left[ K \;  A \; s\right]$, we are ready to
explore recursive formulations of both the planning problem
and the sequential-trading equilibrium. This state vector is
a complete summary of the economy's current position. It is
all that is needed for a planner to compute an optimal
allocation and it is all that is needed for the ``invisible hand''
to call out prices and implement the first-best
allocation as a competitive equilibrium.


We proceed as follows. First, we display the Bellman equation
associated with a recursive formulation of the planning problem.  Second, we
use the
same state vector $X$
for the planner's problem as a state vector
in which to cast the Arrow securities in a competitive economy with sequential
trading.    Then we define a competitive equilibrium
and show how the prices for the sequential equilibrium are embedded in
the decision rules and the value function of the planning problem.




%We want to decentralize the solution of the planning problem via
%a competitive equilibrium with sequential trading in current period
%commodities and one-period Arrow securities.   Accomplishing this requires
%that we adapt the construction of chapter \use{recurge}  to incorporate
%an endogenous component of the state of the economy, namely, the
%capital stock.
%We proceed as follows. First, we display the Bellman equation
%associated with a recursive formulation of the planning problem.  We
%use the state vector for the planner's problem to define a state vector
%in which to cast the Arrow securities in a competitive economy with sequential
%trading.    Then we define a competitive equilibrium
%and show how the prices for the sequential equilibrium are embedded in
%the decision rules and the value function of the planning problem.




\section{Recursive formulation of the planning problem}
We use capital letters $C,N,K$ to denote objects in the planning problem
that correspond to $c,n,k$, respectively, in the household's and firms'
problems.  We shall eventually equate them, but not until we have
derived an appropriate formulation of the household's and firms' problems
in a recursive competitive equilibrium.
The Bellman equation for the planning problem is
$$ v(K,A,s) = \max_{C,N, K'} \left\{  u(C,1-N)
   +\beta \sum_{s'} \pi(s'|s) v(K', A', s') \right\} \EQN plan1 $$
subject to
$$\EQNalign{  K' + C &  \leq A s F(K, N) + (1-\delta )K , \EQN plan2;a \cr
            A' & = A s. \EQN plan2;b \cr} $$
Using the definition of the state vector
$X = \left[ K \;  A \; s\right]$, we denote  the optimal
policy functions as
$$ \EQNalign{ C  & = \Omega^C(X), \EQN plan3;a \cr
              N & = \Omega^N(X), \EQN plan3;b \cr
             K' & = \Omega^K(X) .  \EQN plan3;c \cr} $$
Equations  \Ep{plan2;b}, \Ep{plan3;c}, and the Markov transition
density $\pi(s'|s)$ induce a transition density $\Pi(X'|X)$ on the
state $X$.


  For convenience, define  the functions
$$ \EQNalign{ U_c(X) & \equiv u_c(\Omega^C(X), 1 - \Omega^N(X)), \EQN plan4;a \cr
   U_\ell(X) & \equiv u_\ell(\Omega^C(X), 1 - \Omega^N(X)), \EQN plan4;b \cr
   F_k(X) & \equiv F_k(K, \Omega^N(X)), \EQN plan4;c \cr
   F_n(X) & \equiv F_n(K, \Omega^N(X)). \EQN plan4;d \cr}$$
The  first-order conditions for the planner's problem can be represented
as\NFootnote{We are using the envelope condition
$$v_K(K,A,s) = U_c(X)[A s F_k(X) +(1-\delta)].$$}%
$$ \EQNalign{ U_\ell(X) &  = U_c(X) A s F_n(X), \EQN plan5;a \cr
         1&  =  \beta \sum_{X'} \Pi(X'| X) {U_c(X') \over U_c(X)}
    [A' s' F_K(X') + (1-\delta)]. \hskip1cm   \EQN
      plan5;b \cr} $$





\section{Recursive formulation of sequential trading}\label{sec:bigKlittlek}%
We seek a competitive equilibrium with sequential trading of
one-period-ahead state-contingent securities (i.e., Arrow securities).
To do this, we must use a ``Big $K$, little $k$'' trick of the type used in
chapter \use{recurpe}. \index{Big $K$, little $k$}%


\subsection{A ``Big $K$, little $k$'' device}
Relative to the setup described in chapter \use{recurge}, we have
augmented the time $t$ state of the economy by both last period's
technology level $A_{t-1}$ and the current aggregate value of
the endogenous state variable $K_t$.  We assume that decision
makers act as if their decisions do not affect current or future prices. In a
sequential market setting, prices depend on the state, of which
$K_t$ is part. Of course, {\it in the aggregate\/}, decision
makers choose the motion of $K_t$, so that we  require a device
that makes them ignore this fact when they solve their decision
problems (we want them to behave as perfectly competitive price
takers, not monopolists). This consideration induces us to carry
along both ``Big $K$'' and ``little $k$'' in our computations.
Big $K$ is  an endogenous
state variable\NFootnote{More generally, Big $K$ can be a vector  of
endogenous state variables that impinge on equilibrium prices.}
that is useful for forecasting  prices.
Big $K$ is a component of the state that agents
regard as beyond their control when solving their optimum
problems. Values of little $k$ are chosen by firms and consumers.
While we distinguish $k$ and $K$
 when posing the decision problems of the household and firms,
to impose equilibrium we set $K=k$ {\it after\/} firms
and consumers have optimized.


\subsection{Price system}
To decentralize the economy in terms of one-period Arrow securities,
we need a description of the aggregate state in terms of which one-period
state-contingent payoffs are defined.  We proceed by guessing that the
appropriate description of the state is the same vector $X$ that
constitutes the state for the planning problem.  We temporarily forget
about the optimal policy functions for the planning problem and focus
on a decentralized economy with sequential trading and one-period
prices that depend on $X$. We specify {\it price functions\/}
$r(X), w(X), Q(X'|X)$, that represent, respectively, the rental price
of capital, the wage rate for labor, and the price of a claim to one
unit of consumption next period when next period's state is $X'$ and
this period's state is $X$.
(Forgive us for recycling the notation
for $r$ and $w$ from the previous sections on the formulation of
history-dependent competitive equilibria with commodity space
$s^t$.)
%%%%%in the space of sequences.)
The prices are
all measured in units of this period's consumption good.
We also take as given an arbitrary candidate for the law of
motion for $K$:
$$ K' = G(X). \EQN Kmotion $$
Equation \Ep{Kmotion}  together with \Ep{plan2;b} and a given
subjective transition
density $\hat \pi(s'|s)$ induce a subjective transition
density $\hat \Pi(X' |X)$ for the state $X$.
For now, $G$ and $\hat \pi(s'|s)$ are arbitrary. We wait until
later to impose other equilibrium conditions, including rational
expectations in the form of some restrictions on $G$ and $\hat \pi$.


\subsection{Household problem}
The perceived law of motion \Ep{Kmotion}
for $K$ and the induced transition probabilities $\hat \Pi(X'|X)$
describe the beliefs of a representative household.  The Bellman
equation of the household is
$$ J(a,X) = \max_{c,n, \overline a(X')} \left\{ u(c,1-n) + \beta
  \sum_{X'} J(\overline a(X'), X') \hat \Pi(X'| X) \right\} \EQN hh1 $$
subject to
$$ c + \sum_{X'} Q(X'| X) \overline a(X') \leq w(X) n + a. \EQN hh2  $$
Here $a$ represents the wealth of the household denominated in
units of current consumption goods and $\overline a(X')$ represents
next period's wealth denominated in units of next period's consumption
good.
Denote the household's optimal policy functions as
$$ \EQNalign{ c &= \sigma^c(a,X), \EQN hh3;a \cr
             n& = \sigma^n(a,X), \EQN hh3;b \cr
             \overline a(X') & = \sigma^a(a, X;X').   \EQN hh3;c \cr }
$$
Let
$$ \EQNalign{ \overline u_c(a,X) & \equiv u_c(\sigma^c(a,X), 1-\sigma^n(a,X)),
   \EQN hh10;a \cr
 \overline u_\ell(a,X) & \equiv  u_\ell(\sigma^c(a,X), 1-\sigma^n(a,X)).
   \EQN hh10;b \cr  } $$
Then we can represent the first-order conditions for the household's
problem as
$$\EQNalign{ \overline u_\ell(a,X) & = \overline u_c(a,X)  w(X), \EQN hh11;a \cr
\noalign{\vskip.2cm}
 Q(X'|X)  & =  \beta  {\overline u_c(\sigma^a(a,X;X'), X') \over \overline
   u_c(a,X)} \hat \Pi(X'|X) .\EQN hh11;b\cr} $$




\subsection{Firm of type I}
Recall from subsection \use{sec:FirmI} the static optimum problem
of a type I firm in a sequential equilibrium. In the recursive
formulation of that equilibrium, the optimum problem of a type I
firm can be written as
$$ \max_{c,x,k,n}  \left\{ c+x -r(X) k - w(X)n \right\}  \EQN f11  $$
subject to
$$ c + x \leq A s F(k,n). \EQN f12 $$
The zero-profit conditions are
$$ \EQNalign{ r(X) & = A s F_k(k,n),  \EQN f13;a \cr
              w(X) & = A s F_n(k,n). \EQN f13;b \cr} $$




\subsection{Firm of type II}
Recall from subsection \use{sec:FirmII} the optimum problem
of a type II firm in a sequential equilibrium. In the recursive
formulation of that equilibrium, the optimum problem of a type II
firm can be written as
%A type  II firm transforms output into capital, stores capital,
% and earns its revenues by renting
%capital to the type I firm. Because of the technological
%assumption that capital can be converted back into the consumption
%good, we can without loss of generality consider a sequence of
%two-period optimization problems where a type II firm decides how
%much capital $k'$ to store at the end of each period, in order to earn
%a rental revenue $r(X')k$ and liquidation value $(1-\delta)k'$
%in the following period. The firm finances itself by issuing state
%contingent debt to the households, so future income streams can
%be expressed in today's values by using prices $Q(X'|X)$.  The
%associated optimum problem of a type II firm becomes
$$ \max_{k'}   \; k' \left\{ -1 + \sum_{X'} Q(X'|X) \left[r(X') + (1-\delta)
                           \right]\right\} .    \EQN f21  $$
The zero-profit condition is
$$ 1 = \sum_{X'} Q(X'|X) \left[r(X') +  (1-\delta) \right]. \EQN f26 $$








\section{Recursive competitive equilibrium}
So far, we have taken the price functions
$r(X)$, $w(X)$, $Q(X'|X)$ and the perceived law of motion \Ep{Kmotion} for $K'$
and the associated induced  state transition probability $\hat \Pi(X'|X)$
as given arbitrarily. We now
impose equilibrium conditions on these objects and make
them outcomes of the analysis.\NFootnote{An important
function of the rational expectations hypothesis is to
remove agents' expectations in the form of $\hat \pi$
and $\hat \Pi$ from the list of free parameters of the model.}


When solving their optimum problems, the household and firms take
the endogenous state variable $K$ as given.  However, we want $K$
to be determined by the equilibrium interactions of households
and firms.  Therefore, we impose $K=k$ {\it after\/} solving the
optimum problems of the household and the two types of firms.
Imposing equality afterward makes the household and the firms
be price takers.


\subsection{Equilibrium restrictions across decision rules}
We shall soon define an equilibrium as a set of pricing functions,
a perceived law of motion for the $K'$, and an associated
$\hat \Pi(X'| X)$ such that when the firms and the household take
these as given, the household's and firms' decision rules {\it imply\/}
the law of motion for $K$ \Ep{Kmotion} after substituting $k=K$
and other market clearing conditions.  We shall remove the
arbitrary nature of both $G$ and $\hat \pi$ and therefore
also $\hat \Pi$ and thereby impose rational expectations.


We now proceed to find the restrictions that this notion of
equilibrium imposes across agents' decision rules, the pricing
functions, and the perceived law of motion \Ep{Kmotion}. If the
state-contingent debt issued by the type II firm is to match that
demanded by the household, we must have
%$$\EQNalign{
%\overline a(X') &= [r(X') + (1-\delta)]K', \EQN ff_equil;a \cr
%\noalign{\noindent \rm and consequently beginning-of-period assets in
%a household's budget constraint \Ep{hh2} have to satisfy}
%a &= [r(X) + (1-\delta)]K.           \EQN ff_equil;b \cr}
%$$
$$\overline a(X') = [r(X') + (1-\delta)]K', \EQN ff_equil;a $$
and consequently beginning-of-period assets in
a household's budget constraint \Ep{hh2} have to satisfy
$$a = [r(X) + (1-\delta)]K.           \EQN ff_equil;b
$$


By substituting equations \Ep{ff_equil} into a household's
budget constraint \Ep{hh2}, we get
$$\EQNalign{\sum_{X'} Q(X'|X)  &[r(X') + (1-\delta)]K' \cr
\noalign{\vskip.2cm}
         &= [r(X) + (1-\delta)]K + w(X) n - c. \EQN ff_equilbc \cr}$$
Next, by recalling equilibrium condition \Ep{f26} and the fact that
$K'$ is a predetermined variable when entering next period, it follows
that the left side of \Ep{ff_equilbc} is equal to $K'$. After also
substituting equilibrium prices \Ep{f13} into the right side
of \Ep{ff_equilbc}, we obtain
$$\EQNalign{
K' &= \left[A s F_k(k,n) + (1-\delta)\right]K + A s F_n(k,n) n  - c \cr
%\noalign{\vskip.2cm}
   &= A s F(K,\sigma^n(a,X) ) + (1-\delta)K  - \sigma^c(a,X),\EQN mat0  \cr}
$$
where the second equality invokes Euler's theorem on linearly homogeneous
functions and equilibrium conditions $K=k$, $N=n=\sigma^n(a,X)$ and
$C=c=\sigma^c(a,X)$.
To express the right side of equation \Ep{mat0} solely as a function of
the current aggregate state
$X = \left[K \;  A \; s \right]$, we also impose equilibrium
condition \Ep{ff_equil;b}
$$\EQNalign{
K' &= A s F\left(K,\sigma^n([r(X) + (1-\delta)]K,X) \right) \cr
%\noalign{\vskip.2cm}
     &+ (1-\delta)K
- \sigma^c([r(X) + (1-\delta)]K,X) .   \EQN mat1  \cr}$$
Given the arbitrary
perceived law of motion \Ep{Kmotion} for $K'$ that underlies
the household's optimum problem,
the right side of \Ep{mat1} is the {\it actual\/} law of motion
for $K'$ that is implied by the household's and firms' optimal decisions.
In equilibrium, we want $G$ in \Ep{Kmotion} not to be arbitrary
but to be an {\it outcome\/}.  We want
to find an equilibrium perceived law of motion \Ep{Kmotion}.
By way of imposing rational expectations, we require that
the perceived and actual laws of motion be identical.
Equating the right sides of \Ep{mat1} and the perceived law
of motion \Ep{Kmotion} gives
$$\EQNalign{
G(X) = &A s F\left(K,\sigma^n([r(X) + (1-\delta)]K,X) \right) \cr
%\noalign{\vskip.2cm}
     &+ (1-\delta)K
- \sigma^c([r(X) + (1-\delta)]K,X) .   \EQN mat2  \cr}$$
Please remember that the right side of this equation is
itself implicitly a function of $G$, so that \Ep{mat2}
is to be regarded as instructing us to find a fixed point equation of
a mapping from a perceived $G$ and a price
system  to an actual $G$.
This functional equation requires that the perceived law of motion
for the capital stock $G(X)$ equals the actual law of motion
for the capital stock that is determined jointly by the
decisions of the household and the firms in a competitive equilibrium.




\medskip\noindent{\sc Definition:}
A {\it recursive competitive equilibrium with Arrow securities\/}
is a price system
$r(X)$, $w(X)$, $Q(X'|X)$, a  perceived law of motion $K'=G(X) $
and associated induced transition density $\hat \Pi(X'|X)$,
  a household value function  $J(a,X)$,
and decision rules $\sigma^c(a,X)$,
$\sigma^n(a,x)$,  $\sigma^a(a,X;X')$ such that:
\medskip
\noindent {\bf a.}  Given $r(X)$, $w(X)$, $Q(X'|X)$, $\hat \Pi(X'|X)$, the
functions
 $\sigma^c(a,X)$, $\sigma^n(a,X)$, $\sigma^a(a,X;X')$ and the value
function $J(a,X)$ solve
 the household's optimum problem;
\medskip
\noindent {\bf b.} For all $X$, \hskip.2cm
 $r(X) = A F_k \!\Bigl(K, \sigma^n([r(X) + (1-\delta) ]K, X) \Bigr)$, and
\vskip.2cm
\noindent{} \hskip2.2cm
 $w(X) = A F_n  \!\Bigl(K, \sigma^n([r(X) + (1-\delta) ]K, X) \Bigr);$
\medskip
\noindent{\bf c.} $Q(X'|X)$ and $r(X)$ satisfy \Ep{f26};
\medskip
\noindent{\bf d.}
 The functions $G(X)$, $r(X)$, $\sigma^c(a,X)$, $\sigma^n(a,X)$ satisfy
\Ep{mat2};
\medskip
\noindent{\bf e.}  $\hat \pi  = \pi$.


\medskip
Item a enforces optimization by the household, given the prices
it faces and its expectations.  Item b requires that the type I firm break even
at every capital stock and at the labor supply chosen by
the household.  Item c requires that the type II firm break even.
Market clearing is implicit when item d
requires that the perceived and actual laws of motion of
capital  are equal. Item e and the equality of the perceived and actual
$G$ imply that $\hat \Pi  = \Pi$. Thus, items d and e impose
rational expectations.


\medskip
\subsection{Using the planning problem}
Rather than directly attacking the fixed point problem
\Ep{mat2} that is the heart of the equilibrium definition, we'll
guess a candidate $G$  as well as a price system, then describe
how to verify that they form an equilibrium. As our candidate for
$G$, we choose the decision rule \Ep{plan3;c} for  $K'$ from the
planning problem. As sources of candidates for the pricing
functions, we again turn to the planning problem and
 choose:
$$\EQNalign{ r(X) & =  A F_k(X), \EQN plan6;a \cr
      w(X) & = A F_n(X), \EQN plan6;b \cr
     Q(X'|X) & = \beta \Pi(X'|X) {U_c(X') \over U_c(X)}
   [A' s' F_K(X')+(1-\delta)]. \hskip1cm\EQN plan6;c} $$
In an equilibrium it will turn out that the household's decision rules
for consumption and labor supply will match those chosen by the
planner:\NFootnote{The two functional equations \Ep{plan7} state
restrictions that a recursive competitive equilibrium imposes
across the household's decision rules $\sigma$ and the planner's
decision rules $\Omega$.}
$$\EQNalign{ \Omega^C(X) & = \sigma^c
([r(X)+(1-\delta)]K, X), \EQN plan7;a  \cr
           \Omega^N(X) & = \sigma^n([r(X)+(1-\delta)]K, X). \EQN plan7;b  \cr  }
$$


 The key to verifying these guesses is to show that the first-order
conditions for both types of firms and the household are
satisfied at these guesses.  We leave the details to an exercise.
Here we are exploiting some consequences of the welfare
theorems, transported this time to a recursive setting with an
endogenous aggregate state variable.


\section{Concluding remarks}
  The notion of a recursive competitive equilibrium was introduced
by Lucas and Prescott (1971) and Mehra and Prescott (1979).
The application in this chapter is in the spirit of those papers
but differs substantially in details.  In particular, neither of
those papers worked with Arrow securities, while the focus of this
chapter has been to manage an endogenous state vector in terms of
which it is appropriate to cast Arrow securities.

\auth{Lucas, Robert E., Jr.}%
\auth{Mehra, Rajnish}%
\auth{Prescott, Edward C.}%


\index{permanent income model!general equilibrium version}%

\appendix{A}{The permanent income model revisited}\label{sec:perm_income_ge}%
This appendix is a variation on the theme that `many single agent models can
be reinterpreted as general equilibrium models'.
\subsection{Reinterpreting the single-agent model}\label{sec:pricesystemLQ}%
In this appendix, we cast the single-agent linear quadratic  permanent income model
of section \use{sec:LQmodel} of chapter \use{timeseries} as a competitive equilibrium
with time $0$ trading of a complete set of history-contingent securities.
We begin by reformulating the model in that chapter as a planning problem.
The planner has  utility functional
$$  E_0 \sum_{t=0}^\infty \beta^t u(\bar c_t)  \EQN  sprob100 $$
where $E_t$ is the mathematical expectation conditioned
on the consumer's time $t$ information,  $\bar c_t$ is time $t$ consumption,
$u(c)= -.5 (\gamma - \bar c_t)^2$, and
$\beta \in (0,1)$ is a discount factor.  The planner maximizes
\Ep{sprob100} by choosing a consumption, borrowing plan
 $\{\bar c_t, b_{t+1}\}_{t=0}^\infty$ subject to the sequence of budget constraints
$$ \bar c_t + b_t = R^{-1} b_{t+1}  + y_t \EQN sprob200 $$
where $y_t$ is an exogenous
 stationary endowment process, $R$ is a constant gross
risk-free interest rate, $-R^{-1} b_t \equiv \bar k_t$ is the stock of an asset that bears a risk free one-period
gross return of $R$, and $b_0$ is a given initial condition.  We  assume
that $R^{-1} = \beta$ and  that the endowment
process has the state-space representation
$$ \EQNalign{ z_{t+1} & = A_{22} z_t + C_2 w_{t+1} \EQN sprob150;a \cr
               y_t & = U_y  z_t \EQN sprob150;b \cr}$$
where $w_{t+1}$ is an i.i.d.\ process with mean zero and
identity contemporaneous covariance matrix, $A_{22}$ is a stable matrix,
its eigenvalues being strictly below unity in modulus, and
$U_y$ is a selection vector that identifies $y$ with a particular
linear combination of  $z_t$.  As shown in chapter \use{timeseries},
the solution of what we now interpret as a planning problem
can be represented as the following versions of equations
\Ep{sprob8} and \Ep{debt_evolution}, respectively:
$$\EQNalign{% \bar c_t & = (1-\beta)
%\left[ \sum_{j=0}^\infty \beta^j E_t y_{t+j} - R \bar k_t\right]
% \EQN sprob800   \cr
 \bar c_t & = (1-\beta)
\left[ U_y (I-\beta A_{22})^{-1} z_t  - R \bar k_t\right]
 \EQN sprob800   \cr
 \bar k_{t+1} & = \bar k_t +  R U_y (I - \beta A_{22})^{-1} (A_{22} - I) z_t . \EQN debt_evolution100 } $$
We can represent the optimal consumption, capital accumulation path  compactly as
$$\EQNalign{ \bmatrix{ \bar k_{t+1} \cr z_{t+1} } &= A \bmatrix{ \bar k_t \cr z_t } + \bmatrix{ 0 \cr C_2 } w_{t+1} \EQN kc_solnk \cr
   \bar c_t & = S_c \bmatrix{ \bar k_t \cr z_t} \EQN kc_solnc} $$
   where the matrices $A, S_c$ can readily be constructed from the solutions and specifications just mentioned.
In addition, it is   useful to have at our disposal the marginal utility of consumption process
$p_t^0 \equiv (\gamma - \bar c_t)$, which can be represented as
$$ p_t^0 (z^t) = S_p \bmatrix{ \bar k_t \cr z_t } \EQN eq:pLQrepres  $$
and where $S_p$ can be constructed easily from $S_c$.  Solving equation \Ep{debt_evolution100} recursively shows that  $k_{t+1}$ is a function
$k_{t+1}(z^t; k_0)$ of  history $z^t$.   In equation \Ep{eq:pLQrepres}, $\bar k_t$ encodes the history dependence of $p_t^0(z^t)$.

Equations \Ep{kc_solnk}, \Ep{kc_solnc}, \Ep{eq:pLQrepres} together with the equation $r_t^0 = \alpha$ to be explained
below
turn out to be representations of the equilibrium price system in the competitive equilibrium to which we
turn next.

\subsection{Decentralization and scaled prices}

Let $q_t^0(z^t)$ the time $0$ price of a unit of time $t$ consumption at history $z^t$.  Let
$\pi_t(z^t)$ the probability density of the history $z^t$ induced by the state-space
representation \Ep{sprob150}.  Define the adjusted  Arrow-Debreu price scaled by discounting and probabilities
as
$$ p_t^0(z^t) = {\frac{q_t^0(z^t)}{\beta^t \pi_t(z^t) }}. \EQN scaled_price $$
We find it convenient to express a representative  consumer's problem and a representative
firm's problem  in terms of these scaled Arrow-Debreu prices.

Evidently, the present value of consumption, for example,  can be represented as
$$ \eqalign{ \sum_{t=0}^\infty \sum_{z^t} q_t^0(z^t) c_t(z^t) & = \sum_{t=0}^\infty \sum_{z^t} \beta^t p_t^0(z^t) c_t(z^t) \pi_t(z^t) \cr
     & = E_0 \sum_{t=0}^\infty \beta^t p_t(z^t) c_t(z^t) .}$$
Below, it will be  convenient for us to represent present values as conditional expectations of discounted sums as is done
in the second  line.

We let $r_t^0(z^t)$ be the rental rate on capital, again scaled analogously to \Ep{scaled_price}.
Both the consumer and the firm take these processes as given.

The consumer owns and operates the technology for accumulating capital.  The consumer  owns the endowment process $\{y_t \}_{t=0}^\infty$, which
it sells to a firm that operates a production technology.  The consumer  rents capital to the firm.  The firm uses the endowment and
capital to produce output that it sells to the consumer at a competitive price. The consumer divides his time $t$ purchases between
consumption $c_t$ and gross investment $x_t$.
\subsubsection{The consumer}
Let $\{p_t^0(z^t), r_t^0(z^t)\}_{t=0}^\infty$ be a price system, each component of which takes the form of a `scaled
Arrow-Debreu price' (attained by dividing a time-$0$  Arrow-Debreu price by a discount factor times a probability, as in the previous
subsection).    The representative consumer's
problem is to choose processes $\{c_t, k_{t+1}\}_{t=0}^\infty$ to maximize
$$ -.5 E_0 \sum_{t=0}^\infty \beta^t (\gamma- c_t)^2 \EQN obj_c $$
subject to
$$ \EQNalign{ E_0 \sum_{t=0}^\infty \beta^t  p_t^0(z^t) o_t(z^t) & = E_0 \sum_{t=0}^\infty \beta^t \bigl( p_t^0(z^t) y_t + r_t^0(z^t) k_t(z^t) \bigr) \EQN eqn:budgetLQ \cr
      k_{t+1} & = (1-\delta) k_t + x_t \EQN eqn:capmotion \cr
      o_t(z^t) & = c_t(z^t) + x_t(z^t) \EQN eqn:defocx } $$
where $k_0$ is a given initial condition. Here $x_t$ is gross investment and $k_t$ is physical capital owned by the household and rented
to firms.  The consumer purchases output $o_t = c_t + x_t$ from competitive firms.  The consumer sells its endowment $y_t$ and rents its
capital $k_t$ to firms at prices $p_t^0(z^t)$ and $r_t^0(z^t)$.  Equation \Ep{eqn:capmotion} is the law of motion for physical capital, where $\delta \in (0,1)$
is a depreciation rate.

\subsubsection{The firm}

A competitive representative firm chooses processes $\{k_t, c_t, x_t\}_{t=0}^\infty$ to maximize
$$ E_0 \sum_{t=0}^\infty \beta^t \bigl\{ p_t^0(z^t) o_t(z^t) - p_t^0(z^t) y_t - r_t^0(z^t) k_t  \bigr\} \EQN firm_objLQ $$
subject to the physical technology
$$  o_t(z^t) = \alpha k_t + y_t(z_t), \EQN eqn:technologyLQ $$
where $\alpha >0$.  Since the marginal product of capital is $\alpha$,
a good guess is that
$$ r_t^0(z^t) = \alpha .\EQN eqnMPK $$


\subsection{Matching equilibrium and planning allocations}

We impose the condition
 $$\alpha + (1-\delta) = R . \EQN eqn:Rmatch $$
This makes the gross rates of return in investment identical in the planning and decentralized economies.
In particular, if we substitute equation \Ep{eqn:capmotion} into equation \Ep{eqn:technologyLQ} and
remember that $b_t \equiv R k_t$, we obtain \Ep{sprob200}.

It is straightforward to verify that the allocation $\{\bar k_{t+1}, \bar c_t\}_{t=0}^\infty$ that solves the planning problem
is a competitive equilibrium allocation.

As in chapter \use{recurpe},
we have distinguished between the planning allocation $\{\bar k_{t+1}, \bar c_t\}_{t=0}^\infty$ that determines the equilibrium
price functions  defined in subsection \use{sec:pricesystemLQ} and the allocation chosen by the representative firm and the
representative consumer who face those prices as price takers. This is yet another example of the `big K, little k' device
from chapter   \use{recurpe}.



\subsection{Interpretation}

As we saw in section \use{sec:LQmodel} of chapter \use{timeseries} and also in
representation \Ep{sprob800}
\Ep{debt_evolution100} here, what is now {\it equilibrium\/} consumption is a random walk.
Why, despite his preference for a {\it smooth\/} consumption path,  does the representative consumer accept
fluctuations in his consumption?   In the complete markets economy of this appendix, the consumer believes that it is possible for him
 completely to smooth consumption over time and across histories by purchasing and selling
history contingent claims. But at the equilibrium prices facing him, the consumer prefers to tolerate
 fluctuations in consumption over time and across histories.


\medskip

