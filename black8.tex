
\input grafinp3
%\input grafinput8
\input psfig
%\eqnotracetrue

\input psfig

\def\toone{{t+1}}
\def\ttwo{{t+2}}
\def\tthree{{t+3}}
\def\Tone{{T+1}}
\def\TTT{{T-1}}
\def\rtr{{\rm tr}}
\def\tone{{t+1}}

%\showchaptIDtrue
%\def\@chaptID{8.}

%\hbox{}
\footnum=0

  \auth{Samuelson, Paul}%
\chapter{Overlapping Generations\label{ogmodels}}

This chapter describes the pure exchange overlapping generations
model of Paul Samuelson (1958).
 We begin with an
abstract presentation  that treats the overlapping
generations model as a special case of the chapter \use{recurge}
general equilibrium model with complete markets and all
trades occurring at time $0$.
A peculiar type of heterogeneity across
agents distinguishes the model.
Each individual
cares about consumption only at  two adjacent dates, and the set of
individuals who care about consumption at a particular date
includes some who care about consumption one period earlier and others
who care about consumption one period later.   We shall study
how  this
special preference and demographic  pattern affects some
of the outcomes of the chapter \use{recurge} model.

  While it  helps to reveal the fundamental structure, allowing
complete markets with time $0$ trading in
an overlapping generations model strains
credulity.
The formalism envisions that equilibrium price and quantity sequences
are set at time $0$,
before the participants who are to execute the trades have been born.
  For that reason, most applied work with the overlapping generations
model adopts a sequential-trading arrangement, like
the sequential trade in Arrow securities described   in chapter
\use{recurge}.
The sequential-trading arrangement has all trades
executed by agents living in the here and now.
Nevertheless, equilibrium quantities and intertemporal prices
are  equivalent between these  two trading arrangements.
Therefore, analytical results found in one
setting transfer to the other.


Later in the chapter,
  we use versions of the model with
sequential trading  to tell how the overlapping
generations model provides  a framework for thinking about equilibria
with government debt and/or valued fiat currency, intergenerational
transfers, and fiscal policy.

\section{Endowments and preferences}

Time is discrete, starts at $t=1$,
and  lasts forever, so $t=1, 2, \ldots$. There is an infinity of agents
named $i=0, 1, \ldots$.  We can also regard $i$ as agent $i$'s period of
birth.
  There is a single good at each date. The good is not storable.  There is no uncertainty.
  Each agent has a
strictly concave, twice  continuously
differentiable, one-period utility function $u(c)$, which
is strictly increasing in consumption  $c$ of the one good.
  Agent $i$ consumes
a vector $c^i = \{c_t^i\}_{t=1}^\infty$ and has the
special utility function
$$\EQNalign{U^i(c^i) & = u(c_i^i) + u(c_{i+1}^i), \quad i \geq 1,
 \EQN olg1;a \cr
    U^0(c^0) &= u(c_1^0). \EQN olg1;b \cr} $$
Notice that agent $i$ only wants goods dated $i$ and $i+1$.
The interpretation of equations \Ep{olg1} is that agent $i$ lives during
periods $i$ and $i+1$ and wants to consume only when he is alive.

Each household has an endowment sequence $y^i$ satisfying
$y_i^i \geq 0, y^i_{i+1} \geq 0, y_t^i = 0 \ \forall t \neq i \ {\rm or}
\ i+1$.
Thus, households are endowed with goods only when
they are alive.


\section{Time $0$ trading}

We use the definition of competitive equilibrium from chapter \use{recurge}.
Thus, we temporarily suspend disbelief and proceed in the style
of Debreu (1959) with time $0$ trading.
\auth{Debreu, Gerard}%
Specifically, we imagine that there is a ``clearinghouse'' at
time $0$ that posts prices and, at those prices, aggregates demands and supplies for goods in different periods.
An equilibrium price vector makes markets for  all
periods $t\geq 2$ clear,  but there  may be
excess supply in period $1$; that is, the clearinghouse might
end up with goods left over in period $1$. Any such excess supply of
goods in period $1$ can be given to the initial old
generation without any effects on the equilibrium price vector,
since those old  agents optimally consume all
their wealth in period $1$ and do not want to buy  goods
in future periods. The reason for our special treatment of
period $1$  will become clear as we proceed.


Thus,
at date $0$, there are complete markets  in time $t$ consumption goods
with date $0$ price $q_t^0$.
A household's budget constraint is
$$ \sum_{t=1}^\infty q_t^0 c_t^i \leq \sum_{t=1}^\infty q_t^0 y_t^i .
   \EQN olgbud $$
Letting  $\mu^i$ be a Lagrange multiplier attached to  consumer $i$'s budget constraint,
the consumer's first-order conditions
are
$$\EQNalign{ \mu^i q_i^0 &= u'(c^i_i), \EQN olg2;a  \cr
             \mu^i q_{i+1}^0 & = u'(c^i_{i+1}), \EQN olg2;b \cr
c_t^i & = 0  \ {\rm if} \ t \notin \{i, i+1\} .\EQN olg2;c \cr} $$

 Evidently an allocation is feasible if for all $t \geq 1$,
$$ c_t^t + c_t^{t-1} \leq y_t^t + y_t^{t-1}. \EQN olg3 $$

\medskip
\noindent{\sc Definition:} An allocation is {\it stationary} if
$c^i_{i+1} = c_o, c^i_i = c_y \ \forall i \geq 1$.
\medskip
\noindent Here the subscript $o$ denotes old and $y$ denotes young.
Note that we do not require that $c^0_1 = c_o$.
We call an equilibrium with a stationary allocation a
{\it stationary equilibrium}.

\subsection{Example equilibria}

Let $\epsilon \in (0, .5)$.
The endowments are
$$ \eqalign{ y_i^i  & = 1 - \epsilon,  \ \forall i \geq 1, \cr
              y_{i+1}^i & = \epsilon, \ \forall i \geq 0, \cr
              y_t^i &=  0 \ {\rm otherwise.} \cr} \EQN example1  $$

This economy has many equilibria.    We describe two stationary
equilibria now, and later we shall describe some
nonstationary equilibria.  We can use a \idx{guess-and-verify method} to
confirm the following two equilibria.
\index{equilibrium!multiple}
\medskip
\item{1.}  Equilibrium H: a high-interest-rate
equilibrium.  Set $q_t^0 =1 \ \forall t \geq 1$ and
$c^i_i = c^i_{i+1} = .5$ for all $i \geq 1$ and $c^0_1 = \epsilon$.
   To verify
that this is an equilibrium, notice that
each household's first-order conditions are satisfied
and that the allocation is feasible. Extensive
intergenerational trade  occurs
at time $0$ at the equilibrium price vector
$q_t^0$.   Constraint \Ep{olg3} holds with  equality
for all $t \geq 2$ but with strict inequality for $t=1$.
Some of the $t=1$ consumption good is left unconsumed.

\medskip
\item{2.}  Equilibrium L: a low-interest-rate
equilibrium.  Set $q_1^0 = 1$, ${q^0_{t+1}  \over
q^0_t} = {u'(\epsilon) \over u'(1 - \epsilon)} = \alpha > 1$.
Set $c^i_t = y_t^i$ for all $i, t$.   This equilibrium is autarkic,
with prices being set to eradicate all trade.

\medskip
\subsection{Relation to welfare theorems}

As we shall explain in more detail later, equilibrium H Pareto dominates
equilibrium L.  In equilibrium H every generation after the initial
old one is better off and no generation is worse off than in equilibrium
L.
%As we shall explain more below,
%the equilibrium  1 Pareto dominates equilibrium 2.  Every generation
%after the initial old one is better off in the second equilibrium,
%and no generation is worse off.
The equilibrium H allocation
is strange because some of the time $1$ good is not consumed, leaving
 room to set up a giveaway program to the initial
old that makes them better off and costs subsequent generations
nothing.  We shall see how the institution of  either perpetual government debt or of fiat money
can accomplish this purpose.\NFootnote{See Karl Shell (1971) for an
investigation that characterizes why some competitive equilibria
in overlapping generations models fail to be Pareto optimal. Shell
cites earlier studies that had sought reasons why the welfare
theorems seem to fail in the overlapping generations structure.}
\auth{Shell, Karl}

  Equilibrium L is a competitive equilibrium that evidently fails to satisfy
  one of the assumptions needed to deliver the
   first fundamental theorem of welfare economics, which identifies conditions under which
 a competitive equilibrium allocation is Pareto
optimal.\NFootnote{See Mas-Colell, Whinston, and Green (1995) and Debreu (1954).
\auth{Mas-Colell, Andreu} \auth{Whinston, Michael D.} \auth{Green, Jerry R.}%
\auth{Debreu, Gerard}
}
  The condition of the theorem that is violated by  equilibrium L
is the assumption that the value of the aggregate endowment
at the equilibrium prices is finite.\NFootnote{Note
that if the horizon of the economy were finite, then the counterpart of
equilibrium H would not exist and the  allocation of the
counterpart of equilibrium L would be Pareto optimal.}

\subsection{Nonstationary equilibria}

Our example economy has more  equilibria.
To construct more equilibria, we summarize preferences and consumption
decisions in terms
of an offer curve.   We describe a graphical apparatus proposed
by David Gale (1973) and used  to good advantage by
William Brock (1990).
\auth{Gale, David}%
\auth{Brock, William A.}%

\medskip
\noindent{\sc Definition:}  The household's {\it offer curve} is
the locus of $(c_i^i, c^i_{i+1})$ that solves
$$ \max_{\{c_i^i, c^i_{i+1}\}} U(c^i)  $$
subject to
$$ c_i^i + \alpha_i c_{i+1}^i \leq y_i^i + \alpha_i y^i_{i+1} .$$
Here $\alpha_i \equiv {q^0_{i+1}\over q^0_i } $,
 the reciprocal
of the one-period gross rate of return from period $i$ to $i+1$, is treated
as a parameter.
\medskip
Evidently,
 the offer curve solves the following pair of equations:
$$\EQNalign{c^i_i + \alpha_i c^i_{i+1} & = y_i^i + \alpha_i y^i_{i+1} \EQN offer1;a
            \cr
            {u'(c_{i+1}^i) \over u'(c^i_i)} & = \alpha_i  \EQN offer1;b  \cr}$$
for $\alpha_i > 0$.
We denote the offer curve   by
$$ \psi(c^i_i, c^i_{i+1}) = 0 .$$

The graphical construction of the offer curve is illustrated in Figure \Fg{graph12f}. %Figure 8.1.
We trace it out by varying $\alpha_i$ in the household's problem and
reading tangency points between the household's indifference
curve and the budget line.  The resulting locus depends
on the endowment vector and  lies above the indifference curve through
the endowment vector.  By construction, the following property
is also true:  at the intersection between the offer curve and a straight
line through the endowment point, the straight line is tangent to an
indifference curve.\NFootnote{Given our assumptions on preferences and endowments,
the conscientious reader will note that  Figure \Fg{graph12f} %Figure 8.1
appears  distorted because  the offer curve really ought to intersect the feasibility line
along the 45 degree line  with $c^t_t=c^t_{t+1}$, i.e., at the allocation affiliated with equilibrium H above.}

%It depends implicitly on the endowment. We trace it out by
%varying $\alpha$ in the household's problem and reading
%tangency points between the household's indifference curve
%and the budget line. The offer curve lies
%above the indifference curve through the initial endowment (because
%trade is voluntary).  And once again, by construction,
%an indifference curve through the
%intersection of the offer curve and
%a straight line between   a point on the offer curve and the endowment
%must be tangent to that line (this follows from \Ep{offer1}).
%  See figure 1.

%%%%%%
%$$
%\grafone{graph12.eps,height=2.5in}{{\bf Figure 8.1} The offer curve and
%feasibility line.}
%$$
%%%%%%%

\midfigure{graph12f}
\centerline{\epsfxsize=3truein\epsffile{graph12.eps}}
\caption{The offer curve and feasibility line.}
\infiglist{graf12f}
\endfigure

\medskip \auth{Gale, David}%
Following Gale (1973),
 we can use the offer curve and
a straight line depicting feasibility in the $(c^i_i, c^{i-1}_i)$ plane
to construct a machine for computing equilibrium allocations and prices.
In particular, we  can use the following pair of difference equations to
solve for an equilibrium allocation.  For $i \geq 1$, the equations
are\NFootnote{By imposing equation \Ep{equilibr1;b} with
equality, we are implicitly possibly including a giveaway program
to the initial old.}
$$\EQNalign{  \psi(c^i_i, c^i_{i+1}) & = 0, \EQN equilib1;a \cr
              c^i_i + c^{i-1}_i & = y^i_i + y^{i-1}_i. \EQN equilibr1;b \cr}$$
We take $c_1^1$ as an initial condition.
After the allocation has been computed, the equilibrium price
system can be computed
from
$$ q^0_i = u'(c^i_i) $$
for all $i \geq 1$.


\subsection{Computing equilibria}


\noindent{\it Example 1:} \quad Gale's equilibrium computation machine:
%  Figure 8.2 is David Gale's (1973) machine for
%computing an equilibrium.
%It works as follows.
A procedure for constructing an equilibrium is illustrated
in Figure \Fg{graph22f}, %Figure 8.2,
which reproduces a version of a graph
of David Gale (1973).
  Start with
a proposed $c_1^1$, a time $1$ allocation to the initial young.
Then use the feasibility line to  find the {\it maximal\/}
feasible value for $c_0^1$, the time $1$  allocation to    the
initial old.  In the Arrow-Debreu equilibrium, the allocation to
the initial old will be less than this maximal value, so that some
of the time $1$ good is thrown away.  The reason for this is that
 the budget constraint of the initial old, $q_1^0( c^0_1 - y^0_1) \leq 0$, implies
that $c^0_1 = y^0_1$.\NFootnote{Soon we shall discuss another market structure that
avoids throwing away any of the initial endowment by augmenting the endowment of the
initial old with a particular zero-dividend infinitely durable asset.}
The candidate time $1$  allocation is thus feasible, but  the
time $1$  young will choose $c_1^1$ only if the price $\alpha_1$ is such that
$(c_2^1, c_1^1)$ lies
on the offer curve.      Therefore, we choose $c_2^1$ from
the point on the offer curve that cuts a vertical
line through  $c_1^1$.   Then we proceed to find $c_2^2$ from
the intersection of a horizontal line through $c_2^1$
and the  feasibility line.   We continue recursively in this way,
choosing $c_i^i$ as the intersection of the feasibility line
with a horizontal line through $c^{i-1}_i$, then
choosing $c_{i+1}^i$ as the intersection of a vertical
line through $c_i^i$   and the offer curve.    We can
construct a sequence of $\alpha_i$'s from the slope
of a straight line through the endowment point
and the sequence of $(c_i^i, c^i_{i+1})$  pairs that
lie on the offer curve.

 If the offer curve has the shape drawn in Figure \Fg{graph22f}, any %Figure 8.2, any
$c_1^1$ between the upper and lower intersections
of the offer curve and the feasibility line is
an equilibrium setting of $c_1^1$. Each such $c_1^1$ is
associated with a distinct allocation and $\alpha_i$ sequence, all but
one of them converging to the {\it low\/}-interest-rate
 stationary equilibrium allocation and interest rate.


%%%%%%%%
%$$
%\grafone{graph22.eps,height=2.5in}{{\bf Figure 8.2} A nonstationary
%equilibrium allocation.}
%$$
%%%%%%

\midfigure{graph22f}
\centerline{\epsfxsize=3truein\epsffile{graph22.eps}}
\caption{A nonstationary equilibrium allocation.}
\infiglist{graph22f}
\endfigure

\medskip
\noindent{\it Example 2:}  \quad  Endowment at $+\infty$:
 Take the preference and endowment structure
of the previous example and modify only one feature. Change the
endowment of the initial old to be  $y^0_1 =\epsilon>0 $ {\it and}
``$\delta=1-\epsilon >0$ units of consumption at $t=+\infty$,''
by which we mean that we take
$$ \sum_t q_t^0 y_t^0 = q_1^0 \epsilon + \delta \lim_{t\rightarrow
   \infty} q_t^0.$$
It is easy to verify that the only competitive equilibrium
of the economy with this specification of endowments
has $q_t^0 = 1 \ \forall t\geq 1$, and thus $\alpha_t = 1
\ \forall t \geq 1$.   The reason is that
  all  the ``low-interest-rate'' equilibria  that we computed in example 1
would assign an infinite value to the endowment
of the initial old.  Confronted with such prices, the
initial old would demand unbounded consumption. That is
not feasible.  Therefore, such a price system cannot be an equilibrium.

\medskip
\index{Lucas tree!in overlapping generations model}
\noindent{\it Example 3:}\quad  A Lucas tree:
  Take the preference and endowment
structure to be the same as example 1 and modify only one feature.
Endow the initial old with a ``Lucas tree,'' namely, a claim
to a constant stream of $d >0$ units of consumption for each
$t\geq 1$.\NFootnote{This is a version of an example of Brock (1990). The `Lucas tree' refers to a colorful interpretation
of a dividend stream as `fruit' falling from a `tree' in a pure exchange economy studied by Lucas (1978). See chapter \use{assetpricing1}.}
\auth{Brock, William A.}%
\auth{Lucas, Robert E., Jr.}%
   Thus, the budget constraint of the initial old person now becomes
$$ q_1^0 c_1^0 = d \sum_{t=1}^\infty q_t^0 + q_1^0 y_1^0. $$
The offer curve of each young agent remains as before,
but now the feasibility line is
$$ c_i^i + c^{i-1}_i =  y_i^i + y^{i-1}_i + d $$
for all $i \geq 1$.  Note that young agents are  endowed
below the feasibility line.  From Figure \Fg{graph32f}, %Figure 8.3,
it seems that there
are  two candidates for stationary equilibria, one
with constant  $\alpha < 1$, another with constant $\alpha >1$.
The one with $\alpha  <1$ is associated with the
steeper budget line in Figure \Fg{graph32f}. %Figure 8.3.
 However, the candidate stationary
equilibrium with $ \alpha >1 $ cannot be an equilibrium for a reason
similar to that encountered in example 2.  At the price
system associated with an $\alpha >1$, the wealth of the initial
old would be unbounded, which would prompt them to consume an
unbounded amount, which is not feasible.
This argument rules out not only the stationary $\alpha >1$ equilibrium
but also all nonstationary candidate equilibria that converge to
that constant $\alpha$.  Therefore, there is a unique equilibrium;
it is stationary and has $\alpha < 1$.

%%%%%%%
%\topinsert{
%$$
%\grafone{graph32.eps,height=2.8in}{{\bf Figure 8.3} Unique equilibrium with
%a fixed-dividend asset.}
%$$
%}\endinsert

\midfigure{graph32f}
\centerline{\epsfxsize=3truein\epsffile{graph32.eps}}
\caption{Unique equilibrium with a fixed-dividend asset.}
\infiglist{graph32f}
\endfigure

    If we interpret the gross rate of return on the
tree as $\alpha^{-1} = {p + d  \over p}$, where $p = \sum_{t=1}^\infty
q_t^0 d$, we can compute that $p = {d \over R-1}$ where
$R = \alpha^{-1}$.  Here $p$ is the price of the
Lucas tree.

    In terms of the logarithmic preference example 5 below, the difference equation
\Ep{alphadiff1} becomes modified to
$$ \alpha_i = {1 + 2d \over \epsilon} - {\epsilon^{-1} -1 \over \alpha_{i-1}}.
    \EQN alphadiff2 $$

\medskip
\noindent{\it Example 4:} \ \ Government expenditures:
  Take the preferences and endowments
to be as in example 1 again, but now alter the feasibility
condition to be
$$ c^i_i  + c^{i-1}_i + g = y^i_i + y^{i-1}_i $$
for all $i \geq 1$ where $g > 0$ is a positive level of government
purchases. The ``clearinghouse'' is now looking for an equilibrium
price vector such that this feasibility constraint is satisfied.
We assume that government purchases do not
give utility.
 The offer curve and the feasibility line look
as   in Figure \Fg{graph42f}. %Figure 8.4.
 Notice that the endowment point $(y^i_i, y^i_{i+1})$
lies {\it outside} the relevant feasibility line.  Formally,
this graph looks like example 3, but with a ``negative dividend $d$.''
Now there are two stationary equilibria with $\alpha > 1$, and
a continuum of equilibria converging to the higher $\alpha$ equilibrium
(the one with the lower slope $\alpha^{-1}$ of the associated budget line).
Equilibria with $\alpha >1$ cannot be ruled out
by the argument in example 3 because no one's endowment sequence
receives infinite value when $\alpha >1$.

\index{deficit finance}

  Later, we shall interpret this example as one in which a government
finances a constant deficit either by money creation or by
borrowing at a negative real net interest rate.  We shall discuss this and
other examples in a setting with sequential trading.

%%%%%%
%\midinsert{
%$$
%\grafone{graph42.eps,height=2.5in}{{\bf Figure 8.4} Equilibria
%with debt- or money-financed government deficit finance.}
%$$
%}\endinsert
%%%%%%%%%%

\midfigure{graph42f}
\centerline{\epsfxsize=3truein\epsffile{graph42.eps}}
\caption{Equilibria with debt- or money-financed government deficit finance.}
\infiglist{graph42f}
\endfigure

\medskip
\noindent{\it Example 5:} \quad Log utility:
Suppose that $u(c) = \ln c$ and that the
endowment is described by equations \Ep{example1}.   Then the offer curve
is given by the recursive formulas $c^i_i = .5(1 - \epsilon + \alpha_i
\epsilon), c^i_{i+1} = \alpha_i^{-1} c_i^i$.
Let $\alpha_i$ be the gross rate of return facing   the young
at $i$.   Feasibility at $i$ and the offer curves then
imply
$$ {1 \over 2 \alpha_{i-1}} (1 - \epsilon + \alpha_{i-1} \epsilon)
    + .5 (1 -\epsilon + \alpha_i \epsilon) =1. \EQN alphadiff $$
This implies the difference equation
$$ \alpha_i = \epsilon^{-1} - {\epsilon^{-1} -1 \over \alpha_{i-1}}.
  \EQN alphadiff1 $$
See Figure  \Fg{graph22f}. %Figure 8.2.
 An equilibrium $\alpha_i$ sequence must
satisfy  equation \Ep{alphadiff} and have $\alpha_i > 0$ for
all $i$.   Evidently, $\alpha_i =1$ for all $i\geq 1$ is
an equilibrium $\alpha$ sequence.   So is any $\alpha_i$ sequence  satisfying
equation \Ep{alphadiff} and $\alpha_1 \geq 1$; $\alpha_1 < 1$ will not work
because equation
\Ep{alphadiff} implies that the tail of $\{\alpha_i\}$ is an unbounded
negative sequence.  The limiting value of $\alpha_i$ for any
$\alpha_1 >1  $ is ${1-\epsilon \over \epsilon} = u'(\epsilon) /
u'(1-\epsilon)$,
which is the interest factor associated with the stationary
autarkic equilibrium.   Notice that Figure \Fg{graph22f} suggests that   the stationary
$\alpha_i=1$ equilibrium is not stable, while the autarkic equilibrium is.


\section{Sequential trading}

  We now alter the trading arrangement to bring them
into line with standard presentations of the overlapping
generations model.   We abandon the time $0$, complete markets
 trading  arrangement and replace it with sequential
trading in which a durable asset, either government debt or unbacked fiat money
or claims on a Lucas tree, is passed from old to young.
Some cross-generation transfers occur with voluntary exchanges,
 while others are engineered by government
tax and transfer programs.

\section{Money}
\auth{Samuelson, Paul}%
 In  Samuelson's (1958)  version of the model, trading occurs
 sequentially through a medium of exchange, an
inconvertible (or ``fiat'') currency.
   In Samuelson's model, preferences and  endowments
are as described above, with one important additional component of the
endowment.
At date $t=1$, old agents are endowed in the aggregate
with $M >0$ units of intrinsically worthless currency.
No one has promised to redeem the currency for goods.
The currency is  not ``backed'' by any government promise to redeem it
for goods.  But as
Samuelson showed, there  exists a   system of expectations that  makes
unbacked currency be valued. Currency will be valued today if people expect it
to be valued tomorrow.  Samuelson thus envisioned a situation in which
currency is backed
by expectations without promises.
\index{expectations!without promises}

For each date $t \geq 1$, young agents purchase
$m^i_t$ units of currency at a price of $1/p_t$ units
of the time $t$ consumption good.  Here $p_t \geq 0$ is
the time $t$ price level. At each $t\geq 1$, each
old agent exchanges his holdings of currency for the time $t$
 consumption good.    The budget constraints of a young
agent born in period $i \geq 1$ are
$$ \EQNalign{ c_i^i + {m_i^i \over p_i} & \leq y_i^i, \EQN samuels1 \cr
            c_{i+1}^i & \leq {m_i^i   \over p_{i+1} } + y^i_{i+1},
                \EQN samuels2 \cr
           m_i^i & \geq 0. \EQN samuels3 \cr} $$
If $m_i^i \geq 0$, inequalities \Ep{samuels1} and \Ep{samuels2} imply
$$ c_i^i + c^i_{i+1} \left( {p_{i+1} \over p_i}\right) \leq y_i^i + y^i_{i+1}
          \left(  {p_{i+1} \over p_i}\right) . \EQN samuels4 $$
Provided that we set
$$ {p_{i+1} \over p_i} = \alpha_i = {q^0_{i+1} \over q^0_i },$$
 this budget set is identical with equation \Ep{olgbud}.



  We use  the following definitions:
\medskip
\noindent{\sc Definition:}  A nominal price  sequence is a positive
sequence $\{p_i\}_{i \geq 1}$.
\medskip
\noindent{\sc Definition:}  An equilibrium with valued fiat money
is a feasible allocation and a nominal price sequence with $p_t < +\infty$
for all $t$
 such
that given the price sequence, the allocation solves the household's
problem for each $i \geq 1$.
\medskip


\noindent The qualification that $p_t < + \infty$ for all $t$  means that
fiat money is valued. Sometimes we call an equilibrium with valued fiat money
a `monetary equilibrium'.  If ${\frac{1}{p_t}} = + \infty$, we sometimes  call it a `nonmonetary equilibrium'.



\subsection{Computing  more equilibria with valued fiat currency}

   Summarize the  household's optimal decisions with
a saving function
$$ y_i^i - c_i^i = s(\alpha_i; y_i^i, y_{i+1}^i). \EQN savfn $$
Then the equilibrium conditions for the model are
$$ \EQNalign{ {M  \over p_i} & = s(\alpha_i; y_i^i, y_{i+1}^i) \EQN sameq1;a
                          \cr
                \alpha_i & = {p_{i+1} \over p_i}, \EQN sameq1;b \cr }$$
where it is understood that
$ c_{i+1}^i = y_{i+1}^i + {M \over p_{i+1}} $.
Equation \Ep{sameq1;a} states that at time $i$  the net of saving of generation $i$ (the expression on the right side) equals
the net dissaving of generation $i-1$ (the expression on the left side).  To compute an equilibrium, we solve the difference equations
\Ep{sameq1} for $\{p_i\}_{i=1}^\infty$,  then get the allocation
from the household's budget constraints evaluated at equality
at the equilibrium level of real balances.
As an example, suppose that $u(c) = \ln(c)$,
and that $(y^i_i, y^i_{i+1}) = (w_1, w_2)$ with $w_1 > w_2$.
  The saving function is
$ s(\alpha_i) = .5(w_1 - \alpha_i w_2)$.   Then
equation \Ep{sameq1;a} becomes
$$ .5(w_1 - w_2 {p_{t+1} \over p_t}) = {M \over p_t} $$
or
$$ p_t = 2 M/ w_1 + \left({w_2 \over    w_1}\right) p_{t+1} .\EQN samdiff1 $$
This is a difference equation whose solutions
with a positive price level are
$$ p_t = {2 M \over w_1 (1 - {w_2 \over w_1})} + c
  \left({w_1\over w_2}\right)^t, \EQN samsol1 $$
for any scalar $c >0$.\NFootnote{See the appendix
to chapter \use{timeseries}.}
The solution for $c=0$ is the unique stationary solution.  The solutions
with $c >0$ have uniformly higher price levels than the $c=0$ solution,
and have the value of currency approaching  zero in the limit as $t \rightarrow +\infty$.

\subsection{Equivalence of equilibria}
   We briefly look back at the equilibria with
time $0$ trading and note that  the equilibrium allocations are
the same under time $0$ and sequential trading.
Thus, the following proposition
asserts that  with an adjustment to the endowment and the consumption allocated
to the initial old, a competitive equilibrium
allocation with time $0$ trading
is an equilibrium  allocation in the fiat money economy (with
sequential trading).
\medskip
\noindent{\sc Proposition:}  Let $\overline c^i$ denote a competitive equilibrium
 allocation (with time $0$ trading)
  and suppose that it satisfies
$\overline c^1_1 <  y^1_1$.
Then there exists
an equilibrium (with sequential trading) of the monetary economy with
allocation that satisfies $c^i_i = \overline c^i_i, c^i_{i+1} = \overline
c^i_{i+1}$ for $i \geq 1$.

\medskip
\noindent{\sc Proof:}  Take the competitive equilibrium allocation
and  price system and form $ \alpha_i = q^0_{i+1}/q^0_i$.  Set
$m_i^i/p_i = y_i^i - \overline c_i^i$. Set $m_i^i = M$ for all $i \geq 1$,
and   determine $p_1$ from ${M \over p_1} = y_1^1  - \overline c_1^1$.
This last equation determines a positive initial price level
$p_1$ provided that $y_1^1 - \overline c_1^1 >0$.  Determine
subsequent price levels from $p_{i+1} = \alpha_i p_i$.
Determine the allocation to the initial old from
$c^0_1 = y^0_1 + {M \over p_1} = y^0_1 +(y_1^1 - \overline c_1^1)$.  \qed

\medskip
In the monetary equilibrium, time $t$ real balances
equal the per capita {\it saving\/} of the young and the
per capita {\it dissaving\/} of the old. To be a monetary equilibrium, both quantities must
be positive for all $t \geq 1$.


A converse of the proposition is true.
\medskip
\noindent{\sc Proposition:}  Let    $\overline c^i$ be an
equilibrium allocation for the fiat money economy.    Then
there is a competitive equilibrium with time $0$ trading
with the same allocation, provided that the endowment
of the initial old is augmented with an appropriate  transfer
from the clearinghouse.
\medskip

To  verify this proposition, we have to construct
the required transfer from the clearinghouse to the initial
old.  Evidently, it is $y^1_1 - \overline c^1_1$.  We invite
the reader to complete  the proof.



%To  verify this proposition, we have to construct
%the required transfer from the government to the initial
%old.  Evidently, it is $B =  y^1_1 - \overline c^1_1$.   The
%government awards  claims of this amount to the initial
%old.  These claims are denominated in units of time $1$ consumption
%goods, and are never to be redeemed.   By the mere act of issuing
%these claims,
%the government  creates wealth.      The government
%never has to  tax   to give these claims value.
%The market simply assigns positive  value  to them.
%This seems mysterious, because the government bonds
%are backed by no `fundamentals'.
%  That government bonds are valued without backing is
%closely connected to the fact that there is an
%equilibrium of the economy without government debt
%in which the equilibrium interest rate is below the
%growth rate of population.
%For now, we simply invite the
%reader to verify that the proposition is true.
%Later we'll shed more light on how such unbacked claims
%acquire value.






\section{Deficit finance}
For the rest of this chapter, we shall assume sequential trading.
With sequential trading of fiat currency,
this section reinterprets one of our earlier examples with time $0$ trading, the example with government spending.

Consider the following overlapping generations model:
The population is constant.
At each date $t \geq 1$, $N$ identical young agents are endowed
with $(y_t^t, y^t_{t+1}) = (w_1, w_2) $, where $w_1 > w_2 >0$.
A government   levies lump-sum taxes of
$\tau_1$ on each young agent and $\tau_2$ on each old agent
  alive at each $t \geq 1$.    There are $N$   old people at
time $1$ each of whom is endowed with $w_2$ units of the
consumption good and $M_0>0$ units of inconvertible, perfectly durable
fiat currency. The initial old have utility function
$c^0_1$.   The young have utility function
$u(c^t_t) + u(c^t_{t+1})$.
 For each date $t \geq 1$ the government
augments the currency supply according to
$$ M_t - M_{t-1} = p_t ( g - \tau_1 -   \tau_2), \EQN gbudg1 $$
where $g$ is a constant stream of government expenditures per capita
and $0 < p_t \leq + \infty$ is the price level.
If $p_t = +\infty$, we intend  that    equation
\Ep{gbudg1} be interpreted as
$$ g = \tau_1 + \tau_2.  \EQN gbudg2 $$


   For each $t \geq 1$, each young person's behavior
is summarized by
$$  s_t = f(R_t;\tau_1, \tau_2) = \argmax_{s\geq 0} \left[ u(w_1 - \tau_1
     -s)+u(w_2-\tau_2 + R_t s)\right] .\EQN savingsprob $$


\medskip
\noindent{\sc Definition:} An equilibrium with valued fiat currency is a pair  of
positive sequences $\{M_t, p_t\}$  such that (a) given the price level
sequence, $M_t/p_t =  f(R_t)$ (the dependence on $\tau_1,\tau_2$ being
understood); (b) $R_t = p_t / p_{t+1}$; and (c) the government budget
constraint \Ep{gbudg1} is satisfied for all $t\geq 1$.
\medskip

The condition $f(R_t) = M_t/p_t$  can be written
as $f(R_t) = M_{t-1} /p_t + (M_t - M_{t-1})/p_t$.  The left side
is the saving of the young.  The first term on  the right side is the
dissaving of the old (the real value of currency  that they
exchange for time $t$ consumption). The second term on the right
is the dissaving of the government (its deficit), which is
the real value of the additional currency that the government
prints at $t$ and uses to purchase time $t$ goods from the young.

   To compute an equilibrium, define $d = g- \tau_1 - \tau_2$ and
write equation \Ep{gbudg1} as
$$ {M_t \over  p_t} = {M_{t-1} \over p_{t-1} }  {p_{t-1} \over p_t}
     + d  $$
for $t \geq 2$
and $$ {M_1 \over p_1} = {M_0 \over p_1} + d $$
for $t=1$.
Substitute the equilibrium condition $M_t / p_t = f(R_t)$ into these equations to
get
$$ f(R_t) = f(R_{t-1})R_{t-1} + d  \EQN money1;a $$
for $t\geq 2$ and
$$ f(R_1) =  {M_0 \over p_1} + d. \EQN money1;b $$


 Given $p_1$, which determines an initial
$R_1$ by means of equation
\Ep{money1;b}, equations \Ep{money1} form an autonomous
difference equation in $R_t$.    With appropriate transformations of variables, this system can be
solved using Figure \Fg{graph42f}. %Figure 8.4.


\subsection{Steady states and the Laffer curve}

Let's seek a stationary solution of equations  \Ep{money1}, a quest
 rendered reasonable by the fact that $f(R_t)$ is time
invariant (because the endowment and the
tax patterns as well as the government deficit
$d$  are time-invariant).     Guess that $R_t = R$ for
$t \geq 1$.  Then equations \Ep{money1} become
$$\EQNalign{ f(R) (1-R) & = d, \EQN money2;a \cr
             f(R) &  = {M_0   \over p_1}  + d. \EQN money2;b \cr} $$

For example, suppose that $u(c) = \ln(c)$.  Then
$ f(R) = {w_1 - \tau_1 \over 2} - {w_2 - \tau_2 \over 2 R}$.  We
have graphed $f(R)(1-R)$ against $d$ in Figure \Fg{graph52f}. %Figure 8.5.
Notice that if there is one solution for equation \Ep{money2;a},  then there
are at least two.

%%%%%%
%$$
%\grafone{graph52.eps,height=2.3in,angle=-90}{{\bf Figure 8.5} The Laffer curve
%in revenues from the inflation tax.}
%$$
%%%%%%%%%%%%%%%%%%%

\midfigure{graph52f}
\centerline{\epsfxsize=3truein\epsffile{graph52new.eps}}
\caption{The Laffer curve in revenues from the inflation tax.}
\infiglist{graph52f}
\endfigure

  Here $(1-R)$ can be interpreted as  a tax rate on  real
balances, and $f(R)(1-R)$ is a Laffer curve for the inflation
tax rate.  The high-return (low-tax) $R = \overline R$
 is associated with
the good Laffer curve stationary equilibrium, and
the low-return (high-tax) $R = \underline R$ comes with the bad
Laffer curve stationary
equilibrium.  Once $R$ is determined, we can determine
$p_1$ from equation \Ep{money2;b}.

Figure \Fg{graph52f} %Figure 8.5
is isomorphic with Figure \Fg{graph42f}. %Figure 8.4.
  The saving rate function
$f(R)$ can be deduced from the offer curve. Thus, a version of
Figure \Fg{graph42f} %Figure 8.4
 can be used to solve the difference equation \Ep{money1;a}
graphically. If we do so, we discover a continuum of nonstationary solutions
of  equation  \Ep{money1;a}, all but one of which  have
$R_t \rightarrow \underline R$ as $t \rightarrow \infty$.  Thus, the
bad \idx{Laffer curve} equilibrium is stable.

The stability of the bad Laffer curve equilibrium arises under
perfect foresight dynamics.
Bruno and Fischer (1990) and Marcet and Sargent (1989) analyze
how the system behaves under two different types of adaptive
dynamics.  They find that either under a crude form
of  adaptive expectations  or under a least-squares learning scheme,
$R_t$ converges to  $\overline R$.   This finding  is comforting because
the comparative dynamics are more plausible at $\overline  R$
(larger deficits bring higher inflation).  Furthermore,
Marimon and Sunder (1993) present experimental evidence
pointing toward the selection  made by the adaptive dynamics.
Marcet and Nicolini  (2003) build and calibrate an adaptive model of several
Latin American hyperinflations that rests on this selection. Sargent, Williams, and Zha (2009)
extend and estimate the model.
\auth{Williams, Noah}%
\auth{Zha, Tao}%
\auth{Marcet, Albert}
\auth{Nicolini, Juan Pablo}
\auth{Marimon, Ramon}
\auth{Sunder, Shyam}
\auth{Bruno, Michael}
\auth{Fischer, Stanley}
\auth{Sargent, Thomas J.}

\section{Equivalent setups}

  This section describes some alternative asset structures
and trading arrangements that support the
same equilibrium  allocation.   We take a model with
a government deficit and show how it can be supported
with sequential trading in government-indexed bonds,
sequential trading in fiat currency, or time $0$ trading
of Arrow-Debreu dated securities.


\subsection{The economy}

Consider an overlapping generations economy with one agent
born at each $t\geq 1$ and an initial old  person at $t=1$.
Young agents born at date $t$ have endowment
pattern $(y_t^t, y^t_{t+1})$ and the  utility function described
earlier.   The initial old person is endowed with $M_0>0$ units
of unbacked currency and $y^0_1$ units of the consumption good.
There is a stream of per-young-person government purchases
$\{g_t\}$.

\medskip
\noindent{\sc Definition:}  An equilibrium with money-financed
government deficits is a sequence $\{M_t, p_t\}_{t=1}^\infty$
with $0 < p_t < +\infty$ and $M_t >0$ that satisfies
(a) given $\{p_t\}$,
$$ M_t = \argmax_{\tilde M \geq 0} \left[ u(y_t^t - \tilde M/p_t)
       + u(y^t_{t+1} + \tilde M/p_{t+1})\right]; \EQN equilib1;a $$
and (b) $$ M_t - M_{t-1} = p_t g_t. \EQN equilib1;b$$



\medskip
Now consider a version  of the same economy in which there is no
currency but rather indexed government bonds.  The demographics
and endowments are identical with the preceding economy, but
now each initial old   person is endowed with
$B_1$ units of a maturing bond, denominated in units of
time $1$ consumption good.  In period $t$, the government sells
new one-period  bonds   to the young to finance its purchases
$g_t$  of time $t$ goods and to pay off the one-period debt
falling due at time $t$.  Let $R_t >0$ be the gross real one-period
rate of return on government debt between $t$ and $t+1$.
\medskip
\noindent{\sc Definition:}  An equilibrium with bond-financed government
deficits is a sequence $\{B_{t+1}, R_t\}_{t=1}^\infty$ that
satisfies (a) given $\{R_t\}$,
$$ B_{t+1} = \argmax_{\tilde B} [ u(y_t^t -\tilde B/R_t)
    + u(y^t_{t+1} + \tilde B)]; \EQN equilib2;a $$
and (b)
$$ B_{t+1}/R_t = B_t + g_t,  \EQN equilib2;b $$
with $B_1 \geq 0$ given.

    These two types of equilibria are isomorphic in the following
sense:  Take an equilibrium of the economy with money-financed
deficits and transform it into an equilibrium of the
economy with bond-financed deficits as follows:  set
$ B_t = M_{t-1} / p_t, R_t = p_t / p_{t+1}$. It can be verified
directly that these settings of bonds and interest rates, together
with the original consumption allocation,
form an equilibrium of the economy with bond-financed deficits.

  Each of these two types of equilibria is evidently also
isomorphic to the following equilibrium
formulated with time $0$ markets:
\medskip
\noindent{\sc Definition:} Let $B^g_1$ represent claims to time $1$ consumption
owed by the government to the old at time $1$.
An equilibrium with  time $0$ trading
is an initial level of government
debt $B^g_1$, a price system $\{q_t^0\}_{t=1}^\infty$, and
a sequence $\{s_t\}_{t=1}^\infty$ such that (a)
given the price system,
%$$ s_t = \argmax_{\tilde s} \Bigl\{ u(y_t^t - \tilde s)
%     + u \Bigl[y^t_{t+1}
% + \Bigl({q_t^0 \over q^0_{t+1}}\Bigr) \tilde s \Bigr] \Bigr\};
%                                                    $$
$$ s_t = \argmax_{\tilde s} \left\{ u(y_t^t - \tilde s)
     + u \left[y^t_{t+1}
 + \left({q_t^0 \over q^0_{t+1}}\right) \tilde s \right] \right\};
                                                    $$
and (b)
$$ q_1^0 B_1^g + \sum_{t=1}^\infty q_t^0 g_t = 0 . \EQN arrowd0 $$
\medskip
Condition b is the  Arrow-Debreu version of the
government budget constraint.  Condition a is the optimality
condition for the intertemporal consumption decision of the young
of generation $t$.

  The government budget constraint in condition b
can be represented recursively as
$$ q^0_{t+1} B_{t+1}^g = q_t^0 B_t^g + q_t^0 g_t.\EQN arrowd1  $$
If we solve equation \Ep{arrowd1} forward and impose
$\lim_{T \rightarrow \infty} q^0_{t+T} B^g_{t+T} = 0$, we obtain
the budget constraint \Ep{arrowd0}  for $t=1$.
Condition \Ep{arrowd0} makes it evident that
when $\sum_{t=1}^\infty q_t^0 g_t >0$, $B_1^g < 0$, so that the
government has negative net worth.  This negative net worth
corresponds to  the unbacked
claims that the market nevertheless values in the sequential-trading
version of the model.

\subsection{Growth}

  It is easy to extend these models to the case  in which  there is
growth in the population.  Let there be $N_t = n N_{t-1}$ identical young
people at time $t$, with $n > 0$.   For example, consider the economy
with money-financed deficits.  The total money supply
is $N_t M_t$, and the government budget constraint is
$$ N_t M_t - N_{t-1} M_{t-1} = N_t p_t g, $$
where $g$ is per-young-person government   purchases.
Dividing both sides of the budget constraint by
$ N_t$ and rearranging gives
$$ {M_t \over p_{t+1}} { p_{t+1} \over p_t} = n^{-1} {M_{t-1} \over p_t}
     + g .\EQN equilib1c  $$
This equation replaces equation \Ep{equilib1;b} in the definition of
an equilibrium with money-financed deficits.  (Note that
in a steady state, $R=n$ is the high-interest-rate equilibrium.)
Similarly, in the economy with bond-financed deficits,
the government budget constraint would become
$$ {B_{t+1} \over R_t} = n^{-1} B_t + g_t .$$

  It is also easy to modify things to permit the government to
tax young and old people at $t$.  In that case, with government
bond finance the government budget constraint
becomes
$$ {B_{t+1} \over R_t} = n^{-1} B_t + g_t - \tau^t_t - n^{-1} \tau^{t-1}_t ,$$
where $\tau^s_t$ is the time $t$ tax on a person born in period $s$.


\section{Optimality and the existence of monetary equilibria}

\auth{Wallace, Neil}%
Wallace (1980)
discusses the connection between nonoptimality of
the equilibrium without valued money and existence of monetary
equilibria. Abstracting from his assumption of a storage technology,
we study how the arguments apply to a pure endowment economy. The
environment is as follows. At any
date $t$, the population consists of $N_t$ young agents and
$N_{t-1}$ old agents where $N_t=n N_{t-1}$ with $n > 0$. Each young
person is endowed with $y_1> 0$ goods, and an old person receives the
endowment $y_2> 0$. Preferences of a young agent at time $t$ are given
by the utility function $u(c^t_t,c^t_{t+1})$, which is twice
differentiable with indifference curves that are convex to the
origin. The two goods in the utility function are
normal goods, and $$\theta(c_1,c_2) \equiv u_1(c_1,c_2)/u_2(c_1,c_2),$$
the marginal rate of substitution function, approaches infinity as
$c_2/c_1$ approaches infinity and approaches zero as $c_2/c_1$ approaches
zero. The welfare of the initial old agents at time $1$ is
strictly increasing in $c^0_1$, and each one of them is endowed with
$y_2$ goods and $m^0_0> 0$ units of fiat money. Thus, the
beginning-of-period aggregate nominal money balances in the initial
period $1$ are $M_0=N_0 m^0_0$.

For all $t \geq 1$, $M_t$, the post-transfer time $t$ stock of fiat
money obeys $M_t = z M_{t-1}$ with $z > 0$. The time $t$ transfer
(or tax), $(z-1) M_{t-1}$, is divided equally at time $t$ among the
$N_{t-1}$ members of the current old generation. The transfers
(or taxes) are fully anticipated and are viewed as lump-sum: they do
not depend on consumption and saving behavior. The budget constraints
of a young agent born in period $t$ are
$$ \EQNalign{ c_t^t + {m_t^t \over p_t} & \leq y_1, \EQN wall1 \cr
              c_{t+1}^t & \leq y_2 + {m_t^t \over p_{t+1} }
               + {(z-1) \over N_t} {M_t \over p_{t+1} }, \EQN wall2 \cr
              m_t^t & \geq 0,                            \EQN wall3 \cr} $$
where $p_t> 0$ is the time $t$ price level. In a nonmonetary equilibrium,
the price level is infinite, so the real values of both money holdings and
transfers are zero.
Since all members in a generation are identical, the nonmonetary
equilibrium is autarky with a marginal rate of substitution equal to
$$
\theta_{\rm aut} \equiv { u_1(y_1,y_2) \over u_2(y_1,y_2)}.
$$
We ask two questions about this economy. Under what
circumstances does a monetary equilibrium exist? And, when it exists,
under what circumstances does it improve matters?

Let $\hat m_t$ denote the equilibrium real money
balances of a young agent at time $t$,
$\hat m_t \equiv M_t / (N_t p_t)$. Substitution of equilibrium money
holdings into budget constraints \Ep{wall1} and \Ep{wall2} at equality
yield $c^t_t= y_1 - \hat m_t$ and $c^t_{t+1}=y_2 + n \hat m_{t+1}$.
In a monetary equilibrium, $\hat m_t > 0$ for all $t$ and the
marginal rate of substitution $\theta(c^t_t,c^t_{t+1})$ satisfies
$$
\theta(y_1- \hat m_t, \ y_2 + n \hat m_{t+1}) = { p_t \over p_{t+1} }
> \theta_{\rm aut},   \quad  \forall t \geq 1 .            \EQN wall5
$$
The equality part of \Ep{wall5} is the first-order condition for money
holdings of an agent born in period $t$ evaluated at the equilibrium
allocation. Since $c^t_t <y_1$ and $c^t_{t+1} > y_2$ in a monetary
equilibrium, the inequality in
\Ep{wall5} follows from the assumption that the two goods in the
utility function are normal goods.

Another useful characterization of the equilibrium rate of return on
money, $p_t/p_{t+1}$, can be obtained as follows.
By the rule generating $M_t$ and the equilibrium condition
$M_t/p_t=N_t \hat m_t$, we have for all $t$,
$$
{p_t \over p_{t+1}} = {M_{t+1} \over z M_t} {p_t \over p_{t+1}}
= {N_{t+1} \hat m_{t+1} \over z N_t \hat m_t}
= {n \over z} {\hat m_{t+1} \over \hat m_t}.                 \EQN wall6
$$
We are now ready to address our first question, under what circumstances
does a monetary equilibrium exist?

\medskip
\noindent{\sc Proposition:}  $\theta_{\rm aut} z < n$ is necessary
and sufficient for the existence of at least one monetary equilibrium.
\medskip

\noindent{\sc Proof:}
We first establish necessity. Suppose to the contrary that there is
a monetary equilibrium and $\theta_{\rm aut} z / n \geq 1$. Then,
by the inequality part of \Ep{wall5} and expression \Ep{wall6},
we have for all $t$,
$$
{\hat m_{t+1} \over \hat m_t}  > { z \theta_{\rm aut} \over n} \geq 1. \EQN wallmt
$$
If $z \theta_{\rm aut} / n > 1$, one plus the net growth rate of $\hat m_t$ is
bounded uniformly above one and, hence,
the sequence $\{\hat m_t\}$ is unbounded, which is
inconsistent with an equilibrium because real money balances per capita
cannot exceed the endowment $y_1$ of a young agent.
If $z \theta_{\rm aut} / n = 1$, the strictly increasing sequence
$\{\hat m_t\}$ in \Ep{wallmt} might not be unbounded but converge
to some constant $\hat m_\infty$.
According to \Ep{wall5} and \Ep{wall6}, the marginal rate
of substitution will then converge to $n/z$, which by assumption is now
equal to $\theta_{\rm aut}$, the marginal rate of substitution in
autarky. Thus, real balances must be zero in the limit, which contradicts
the existence of a strictly increasing sequence of positive real balances
in \Ep{wallmt}.

To show sufficiency, we prove the existence of a unique
equilibrium with constant per capita real money balances when $\theta_{\rm aut} z < n$.
Substitute our candidate equilibrium,
$\hat m_t = \hat m_{t+1} \equiv \hat m$,  into \Ep{wall5} and \Ep{wall6},
which yields two equilibrium conditions,
$$
\theta(y_1- \hat m, \ y_2 + n \hat m) = { n \over z } > \theta_{\rm aut}.
$$
The inequality part is satisfied under the parameter restriction of
the proposition, and we only have to show the existence of
$\hat m \in [0, y_1]$ that satisfies the equality part.
But the existence (and uniqueness) of such a $\hat m$ is trivial.
Note that the marginal rate of substitution on the left side of
the equality is equal to $\theta_{\rm aut}$ when $\hat m=0$.
Next, our assumptions on preferences imply that the marginal rate
of substitution is strictly increasing in $\hat m$, and approaches
infinity when $\hat m$ approaches $y_1$. \qed
\medskip

The stationary monetary equilibrium in the proof will be referred to
as the $\hat m$ equilibrium. In general, there are other nonstationary
monetary equilibria when the parameter condition of the proposition
is satisfied. For example, in the case of logarithmic preferences and
a constant population, recall the continuum of equilibria
indexed by the scalar $c > 0$ in expression \Ep{samsol1}. But here we
choose to focus solely on the stationary $\hat m$ equilibrium and
its welfare implications. The $\hat m$ equilibrium will be compared
to other feasible
allocations using the Pareto criterion. Evidently, an allocation
$C = \{c^0_1; (c^t_t,c^t_{t+1}), t\geq 1\}$ is feasible if
$$ \EQNalign{
N_t c_t^t + N_{t-1} c_t^{t-1} &\leq N_t y_1 + N_{t-1} y_2,
                                       \quad  \forall t \geq 1, \cr
\noalign{\hbox{\rm or, equivalently,}}
n c_t^t + c_t^{t-1} &\leq n y_1 + y_2, \quad  \forall t \geq 1.
                                                        \EQN wall7 \cr}
$$
The definition of Pareto optimality is:

\medskip
\noindent{\sc Definition:} A feasible allocation $C$ is Pareto optimal
if there is no other feasible allocation $\tilde C$ such that
$$\EQNalign{
&\tilde c^0_1 \geq c^0_1, \cr
&u(\tilde c^t_t, \tilde c^t_{t+1}) \geq u(c^t_t,c^t_{t+1}),
                \quad  \forall t \geq 1, \cr}
$$
and at least one of these weak inequalities holds with strict inequality.
\medskip

We first examine under what circumstances the nonmonetary
equilibrium (autarky) is Pareto optimal.

\medskip
\noindent{\sc Proposition:}  $\theta_{\rm aut} \geq n$ is necessary
and sufficient for the optimality of the nonmonetary equilibrium
(autarky).
\medskip

\noindent{\sc Proof:}
To establish sufficiency, suppose to the contrary that there exists
another feasible allocation $\tilde C$ that is Pareto superior to
autarky and $\theta_{\rm aut} \geq n$. Without loss of generality,
assume that the allocation $\tilde C$ satisfies \Ep{wall7} with
equality. (Given an allocation that is Pareto superior to autarky but
that does not satisfy \Ep{wall7}, one can easily construct another
allocation that is Pareto superior to the given allocation,
and hence to autarky.) Let period $t$ be the first
period when this alternative allocation $\tilde C$ differs from the
autarkic allocation. The requirement that the old generation in this
period is not made worse off, $\tilde c^{t-1}_t \geq y_2$, implies
that the first perturbation from the autarkic allocation must be
$\tilde c^t_t < y_1$, with the subsequent implication that
$\tilde c^t_{t+1} > y_2$. It follows that the consumption of
young agents at time $t+1$ must also fall below $y_1$, and we define
$$
\epsilon_{t+1} \equiv y_1 - \tilde c^{t+1}_{t+1} > 0.      \EQN wall8
$$
Now, given $\tilde c^{t+1}_{t+1}$, we compute the smallest number
$c^{t+1}_{t+2}$ that satisfies
$$
u(\tilde c^{t+1}_{t+1},c^{t+1}_{t+2}) \geq u(y_1,y_2).
$$
Let $\overline c^{t+1}_{t+2}$ be the solution to this problem.
Since the allocation $\tilde C$ is Pareto superior to autarky,
we have $\tilde c^{t+1}_{t+2} \geq \overline c^{t+1}_{t+2}$. Before using
this inequality, though, we want to derive a convenient expression
for $\overline c^{t+1}_{t+2}$.

Consider the indifference curve of $u(c_1,c_2)$ that yields a fixed
utility equal to $u(y_1,y_2)$.
In general, along an indifference curve, $c_2=h(c_1)$,
where $h'=-u_1/u_2=-\theta$ and $h'' > 0$. Therefore, applying the
intermediate value theorem to $h$, we have
$$
h(c_1) = h(y_1) + (y_1-c_1) [-h'(y_1) + f(y_1-c_1)],       \EQN wall9
$$
where the function $f$ is strictly increasing and $f(0)=0$.

Now, since $(\tilde c^{t+1}_{t+1},\overline c^{t+1}_{t+2})$ and
$(y_1,y_2)$ are on the same indifference curve, we can use
\Ep{wall8} and \Ep{wall9} to write
$$
\overline c^{t+1}_{t+2} = y_2 + \epsilon_{t+1} [\theta_{\rm aut}
                                 + f(\epsilon_{t+1})],
$$
and after invoking $\tilde c^{t+1}_{t+2} \geq \overline c^{t+1}_{t+2}$,
we have
$$
\tilde c^{t+1}_{t+2} - y_2 \geq \epsilon_{t+1} [\theta_{\rm aut}
                                 + f(\epsilon_{t+1})].       \EQN wall10
$$
Since $\tilde C$ satisfies \Ep{wall7} at equality, we also have
$$
\epsilon_{t+2} \equiv y_1 - \tilde c^{t+2}_{t+2}
= {\tilde c^{t+1}_{t+2} - y_2 \over n}.                     \EQN wall11
$$
Substitution of \Ep{wall10} into \Ep{wall11} yields
{\ninepoint
$$\eqalign{ \epsilon_{t+2} & \geq  \epsilon_{t+1}
                 {\theta_{\rm aut} + f(\epsilon_{t+1}) \over n} \cr
                & > \epsilon_{t+1},\cr}\EQN wall12
$$
}% endninepoint
where the strict inequality follows from
$\theta_{\rm aut} \geq n$ and $f(\epsilon_{t+1})>0$ (implied by
$\epsilon_{t+1}>0$).
Continuing these computations of successive values of $\epsilon_{t+k}$ yields
{\ninepoint
$$
\epsilon_{t+k} \geq \epsilon_{t+1}
         \prod_{j=1}^{k-1}
{\theta_{\rm aut} + f(\epsilon_{t+j}) \over n}
 > \epsilon_{t+1} \left[{\theta_{\rm aut} +
 f(\epsilon_{t+1}) \over n} \right]^{k-1},
\ \hbox{\rm for}\;k>2,
$$
}%endninept
where the strict inequality follows from the fact that $\{\epsilon_{t+j}\}$
is a strictly increasing sequence.
Thus, the $\epsilon$ sequence is bounded below by a strictly increasing
exponential and hence is unbounded. But such an unbounded sequence violates
feasibility because $\epsilon$ cannot exceed the endowment $y_1$ of a
young agent. It follows that we can rule out the existence of a
Pareto superior allocation $\tilde C$, and conclude that
$\theta_{\rm aut} \geq n$ is
a sufficient condition for the optimality of autarky.

To establish necessity, we prove the existence of an alternative
feasible allocation $\hat C$ that is Pareto superior to autarky when
$\theta_{\rm aut} < n$. First, pick an $\epsilon > 0$ sufficiently
small so that
$$
\theta_{\rm aut} + f(\epsilon) \leq n,                      \EQN wall13
$$
where $f$ is defined implicitly by equation \Ep{wall9}.
Second, set $\hat c^t_t= y_1 - \epsilon \equiv \hat c_1$, and
$$
\hat c^t_{t+1} = y_2 + \epsilon [\theta_{\rm aut} + f(\epsilon)]
               \equiv \hat c_2,    \quad  \forall t \geq 1. \EQN wall14
$$
That is, we have constructed a consumption bundle
$(\hat c_1, \hat c_2)$ that lies on the same indifference curve as
$(y_1, y_2)$, and from \Ep{wall13} and \Ep{wall14}, we have
$$
\hat c_2 \leq y_2 + n \epsilon,
$$
which ensures that the condition for feasibility \Ep{wall7} is satisfied
for $t\geq 2$. By setting $\hat c^0_1 = y_2 + n \epsilon$, feasibility
is also satisfied in period $1$ and the initial old generation is
then strictly better off under the alternative allocation $\hat C$.
\qed

\medskip
With a constant nominal money supply, $z=1$, the two
propositions show that a monetary equilibrium exists if and only
if the nonmonetary equilibrium is suboptimal. In that case, the
following proposition establishes that the stationary $\hat m$
equilibrium is optimal.

\medskip
\noindent{\sc Proposition:}  Given $\theta_{\rm aut} z < n$, then
$z \leq 1$ is necessary and sufficient for the optimality of the
stationary monetary equilibrium $\hat m$.
\medskip

\noindent{\sc Proof:}
The class of feasible stationary allocations with
$(c^t_t,c^t_{t+1}) = (c_1,c_2)$ for all $t\geq 1$, is given by
$$
c_1 + {c_2 \over n} \leq y_1 + {y_2 \over n},        \EQN wall15
$$
i.e., the condition for feasibility in \Ep{wall7}. It follows that
the $\hat m$ equilibrium satisfies \Ep{wall15} at equality, and
we denote the associated consumption allocation of an agent born
at time $t\geq 1$ by $(\hat c_1, \hat c_2)$. It is also the case
that $(\hat c_1, \hat c_2)$ maximizes an agent's utility subject
to budget constraints \Ep{wall1} and \Ep{wall2}.
The consolidation of these two constraints yields
$$
c_1 + {z \over n} c_2 \leq y_1 + {z \over n} y_2
    +  {z \over n} {(z-1) \over N_t} {M_t \over p_{t+1} },
                                                     \EQN wall16
$$
where we have used the stationary rate or return in \Ep{wall6},
$p_t/p_{t+1}=n/z$. After also invoking $z M_t=M_{t+1}$,
$n=N_{t+1}/N_t$, and the equilibrium condition
$M_{t+1}/(p_{t+1} N_{t+1})= \hat m$, expression \Ep{wall16}
simplifies to
$$
c_1 + {z \over n} c_2 \leq y_1 + {z \over n} y_2 + (z-1) \hat m.
                                                      \EQN wall17
$$

To prove the statement about necessity, Figure \Fg{graph16f} %Figure 8.6
depicts the two curves
\Ep{wall15} and \Ep{wall17} when condition $z \leq 1$ fails to hold,
i.e., we assume that $z>1$. The point that maximizes utility subject
to \Ep{wall15} is denoted $(\overline c_1, \overline c_2)$.
Transitivity of preferences  and the fact that the
slope of budget line \Ep{wall17} is flatter than that of \Ep{wall15}
imply that $(\hat c_1, \hat c_2)$ lies southeast of
$(\overline c_1, \overline c_2)$. By revealed preference, then,
$(\overline c_1, \overline c_2)$ is preferred to
$(\hat c_1, \hat c_2)$ and all generations born in period $t \geq 1$
are better off under the allocation $\overline C$.
The initial old generation can also be made better off under this
alternative allocation since it is feasible to strictly
increase their consumption,
$$
\overline c^0_1 = y_2 + n(y_1 - \overline c^1_1)
                > y_2 + n(y_1 - \hat c^1_1) = \hat c^0_1.
$$
Thus, we have established that $z \leq 1$ is necessary for the
optimality of the stationary monetary equilibrium $\hat m$.

To prove sufficiency, note that \Ep{wall5}, \Ep{wall6} and $z\leq 1$
imply that
$$
\theta(\hat c_1, \hat c_2) = { n \over z} \geq n.
$$
We can then construct an argument that is analogous to the sufficiency
part of the proof to the preceding proposition. \qed

%%%%%%%%%%
%$$
%\grafone{graph16.eps,height=2.5in}{{\bf Figure 8.6} The feasibility
%line \Ep{wall15} and the budget line \Ep{wall17} when $z>1$.
%The consumption allocation in
%the monetary equilibrium is $(\hat c_1, \hat c_2)$, and the point
%that maximizes utility subject to the feasibility line
%is denoted $(\overline c_1, \overline c_2)$.}
%$
%%%%%%%%

\midfigure{graph16f}
\centerline{\epsfxsize=3truein\epsffile{graph16.eps}}
\caption{The feasibility line \Ep{wall15} and the budget line \Ep{wall17}
when $z>1$.  The consumption allocation in the monetary equilibrium is
$(\hat c_1, \hat c_2)$, and the point that maximizes utility subject to
the feasibility line is denoted $(\overline c_1, \overline c_2)$.}
\infiglist{graph16f}
\endfigure

As pointed out by Wallace (1980), the proposition implies no connection
between the path of the price level in an $\hat m$ equilibrium and the
optimality of that equilibrium.  Thus, there may be an optimal monetary
equilibrium with positive inflation, for example, if
$\theta_{\rm aut} < n < z \leq 1$; and there may be a nonoptimal
monetary equilibrium with a constant price level,  for example,
if $z = n > 1 > \theta_{\rm aut}$.  What counts is the nominal
quantity of fiat money. The proposition suggests that the quantity
of money should not be increased. In particular, if $z \leq 1$,
then an optimal $\hat m$ equilibrium exists whenever the nonmonetary
equilibrium is nonoptimal.

\auth{Balasko, Yves} \auth{Shell, Karl}\auth{Cass, David}
\subsection{Balasko-Shell criterion for optimality}
For the case of constant population, Balasko and Shell (1980) have
established a convenient general criterion for testing whether
allocations are optimal.\NFootnote{Balasko and Shell credit David
Cass (1971) with having authored a version of their criterion.}
Balasko and Shell permit diversity among agents in terms of
endowments $[w^{th}_t, w^{th}_{t+1}]$ and utility functions
$u^{th}(c^{th}_t, c^{th}_{t+1})$, where $w^{th}_s$ is the time $s$
endowment of an agent named $h$ who is born at $t$ and $c^{th}_s$
is the time $s$ consumption of agent named $h$ born at $t$.
Balasko and Shell
 assume  fixed populations of types $h$ over time. They
impose several kinds of technical
conditions that serve to rule out possible pathologies.  The two main
ones are these.
First, they assume that indifference curves have neither flat parts nor kinks,
and they also rule out indifference curves with flat parts or kinks as limits
of sequences of indifference curves for given $h$ as $t\to\infty$.  Second,
they assume that the aggregate endowments $\sum_h (w^{th}_t + w^{t-1,h}_t)$
are uniformly bounded from above and that there exists an $\epsilon>0$
 such that
$w^{sh}_t>\epsilon$ for all $s,h$, and for $t\in\{s,s+1\}$.  They consider
consumption allocations
uniformly bounded away from the axes.  With these conditions, Balasko and Shell
consider the class of allocations in which all young agents at $t$ share a
common marginal rate of substitution $1+r_t
\equiv u^{th}_1(c^{th}_t,c^{th}_{t+1})
/u^{th}_2(c^{th}_t,c^{th}_{t+1})$ and in which all of the endowments
are consumed.  Then Balasko and Shell show that an allocation is Pareto
optimal if and only if
$$\sum_{t=1}^\infty\ \prod_{s=1}^t  [1+r_s]=+\infty,
\EQN balshell$$
that is, if and only if the infinite sum of $t$-period gross interest rates,
$\prod_{s=1}^t [1+r_s]$, diverges.

The Balasko-Shell criterion for optimality succinctly summarizes the sense in
which low-interest-rate economies are not optimal.  We have already encountered
repeated examples of the situation that, before an equilibrium with valued
currency can exist, the equilibrium without valued currency must be a
low-interest-rate economy in just the sense identified by Balasko and Shell's
criterion, \Ep{balshell}.  Furthermore, by applying the
Balasko-Shell criterion,
\Ep{balshell},
 or  generalizations of it that allow for a positive net growth
rate of population $n$, it can be shown that, among equilibria with valued
currency, only equilibria with high rates of return on currency are optimal.

\auth{Smith, Bruce D.}  \auth{Wallace, Neil}
\section{Within-generation heterogeneity}
This section describes an overlapping generations
model having within-generation heterogeneity of
endowments.  We shall follow Sargent and Wallace  (1982)
and Smith (1988)
and use this model as a vehicle for talking about some
issues  in monetary theory that  require  a setting
in which government-issued  currency coexists with and is a more-or-less
 good substitute for
private IOUs.

We now assume  that
within each generation born at $t  \geq 1$,
there are $J$ groups of agents. There
is a constant number $N_j$ of group $j$ agents.  Agents of group
$j$ are endowed with $w_1(j)$ when young and
$w_2(j)$ when old.  The saving function
of a household of group $j$ born at time $t$
solves the time $t$ version of  problem
\Ep{savingsprob}.    We denote this savings
function $f(R_t,j)$. If we  assume that all
households of generation $t$  have
preferences  $U^t(c^t) = \ln c_t^t + \ln c^t_{t+1}$,
the saving function is
$$ f(R_t,j) = .5\left( w_1(j) - {w_2(j)  \over R_t} \right).$$
At $t=1$, there are old people who are endowed in
the aggregate with $H = H(0)$ units of an inconvertible currency.

For example, assume that $J=2$, that $(w_1(1), w_2(1))  =
 (\alpha, 0)$ and that $
(w_1(2), w_2(2))   = (0,  \beta)$, where $\alpha >0, \beta >0$.
The type 1 people are lenders, while the type 2 are borrowers.
For the case of log preference we have the savings
functions
$f(R,1) = \alpha /2, f(R,2) = - \beta/(2R)$.

\subsection{Nonmonetary equilibrium}

 A nonmonetary equilibrium consists of sequences $(R, s_j)$ of rates of return $R$
and savings rates for  $j= 1, \ldots , J$  and $t \geq 1$ that satisfy
(1)$ s_{tj} = f(R_t, j)$, and (2)
$ \sum_{j=1}^J N_j f(R_t, j) =0   .$
Condition (1) builds in household optimization; condition
(2) says that aggregate net savings equals zero (borrowing
equals lending).

 For the case in which the endowments, preferences, and group
sizes are constant across time, the interest rate is determined
at the intersection of the aggregate savings function
with the $R$ axis, depicted as $R_1$ in Figure \Fg{savings1f}. %Figure 8.7.
 No intergenerational   transfers occur in the nonmonetary
equilibrium.  The equilibrium consists of a sequence of separate
 two-period pure consumption loan economies of a type
analyzed by Irving Fisher (1907).
\auth{Fisher, Irving}

\subsection{Monetary equilibrium}

In an equilibrium with valued fiat currency, at each
date $t \geq 1$ the old receive goods from the young
in exchange for the currency stock $H$.
 For any variable $x$,  $\vec x = \{x_t\}_{t=1}^\infty$.
  An equilibrium with valued fiat money is a set of
sequences $\vec R, \vec p, \vec s$
such that (1) $\vec p$ is a positive sequence, (2) $R_t =p_t / p_{t+1}$,
(3) $s_{jt} = f( R_t,j)$,  and  (4)
$ \sum_{j=1}^J N_j f(R_t, j) = {H \over p_t}.$
Condition (1) states that currency is valued at all dates.
Condition (2) states  that currency and consumption loans
are perfect substitutes.  Condition (3) requires that saving
decisions are optimal. Condition (4) equates the
net saving of the young (the left side) to the net dissaving
of the old (the right side). The old supply currency inelastically.

We can determine a stationary equilibrium graphically.  A stationary
equilibrium satisfies $p_t =p$ for all $t$, which implies
$R =1$ for all $t$.  Thus, if it exists, a stationary equilibrium
solves
$$ \sum_{j=1}^J N_j f(1, j) = {H \over p}  \EQN equil4
 $$
for a positive price level. (See Figure \Fg{savings1f}.) %Figure 8.7.
Evidently, a stationary monetary equilibrium exists if the net
savings of the young are positive for $R=1$.

%%%%%%%%
%\topinsert{
%$$  \grafone{savings1.eps,height=2.5in}{{\bf Figure 8.7}
%The intersection of the aggregate savings function
%with a horizontal line at $R=1$ determines  a stationary equlibrium value of
%the price level, if positive.}
%$$
%}\endinsert
%%%%%%%%%%%%%%%

\midfigure{savings1f}
\centerline{\epsfxsize=3truein\epsffile{savings1.eps}}
\caption{The intersection of the aggregate savings function with a horizontal
line at $R=1$ determines  a stationary equilibrium value of the price level,
if positive.}
\infiglist{savings1f}
\endfigure

For the special case of logarithmic preferences and two
classes of young people, the aggregate savings function of the young is
time invariant and equal to
$$ \sum_j f(R,j) = .5(N_1 \alpha - N_2{\beta  \over R}). $$
Note that the equilibrium condition \Ep{equil4} can be written
$$ .5 N_1 \alpha = .5 N_2 {\beta  \over R} + {H  \over p}. $$
The left side is the demand for savings or the demand
for ``currency'' while the right side is the supply, consisting
of privately issued IOU's (the first term) and
government-issued currency.  The right side is thus
an abstract version of what is called  M1, which is a sum of privately
issued IOUs (demand deposits) and government-issued reserves and
currency.\index{M1}%

\subsection{Nonstationary equilibria}

Mathematically, the equilibrium conditions for the
 model with log preferences and
two groups have the same structure as the model analyzed
previously in equations \Ep{samdiff1} and \Ep{samsol1},
 with simple reinterpretations of
parameters.  We leave it to the reader here and in an exercise
to show that if there exists a stationary equilibrium with
valued fiat currency, then there exists a continuum of
equilibria with valued fiat currency, all but one of which
have the real value of government currency approaching
zero asymptotically.  A linear difference equation
like \Ep{samdiff1} supports this conclusion.

\subsection{The real bills doctrine}

In nineteenth-century Europe and the early days of the Federal Reserve system
in the United States, central banks conducted open market operations not by
purchasing government securities but by purchasing safe (risk-free)
short-term private IOUs.
 We now analyze this old-fashioned type of
open market operation. We allow the government to issue additional
currency each period.  It uses the proceeds exclusively to purchase
private IOUs (make loans to private agents) in the amount $L_t$ at time $t$.
 Such open market
operations are subject to the sequence of restrictions
$$L_t = R_{t-1} L_{t-1} + {H_t - H_{t-1} \over p_t} \EQN omo1  $$
for $t \geq 1$ and $H_0 = H>0$ given, $L_0 = 0$.
Here $L_t$ is the amount of the time $t$ consumption good that the government
lends to the private sector from period $t$ to period $t+1$.
Equation \Ep{omo1} states that the government finances these loans in two ways: first,
by rolling over the  proceeds
$R_{t-1} L_{t-1}$ from the repayment of last period's loans, and second, by injecting new  currency  in the
amount $H_t-H_{t-1}$.  With the government injecting new currency and purchasing loans in this
way each period, the equilibrium condition in the loan market becomes
$$ \sum_{j=1}^J N_j f(R_t, j) + L_t = {H_{t-1}\over p_t}
    + {H_{t}- H_{t-1} \over p_t}   \EQN equil5
$$
where the first term on the right is the real dissaving of the
old at $t$ (their real balances) and the second term
is the real value of the new money printed by the monetary
authority to finance purchases of private IOUs issued
by the young at $t$.  The left side is the net savings of
the young plus the savings of the government.
\index{open market operation!in private securities}
\index{real bills doctrine}
\auth{Smith, Adam}

 Under several guises, the effects of open market operations
like this have concerned monetary economists for centuries.\NFootnote{One
issue concerned the effects on the price
level of allowing banks to issue private bank notes.  Nothing in our model  makes us
take seriously that the notes $H_t$ are issued by the government.
We can also think of them as being issued by a private bank.}
We  state the following proposition:
\medskip
\noindent{\sc Irrelevance of Open Market Operations:}   Open market
operations are irrelevant: all positive sequences $\{L_t, H_t\}_{t=0}^\infty$
 that satisfy the constraint
\Ep{omo1}  are associated with the same equilibrium allocation,
interest rate, and price level sequences.

\medskip
\noindent{\sc Proof:}
Evidently, we can write the equilibrium condition \Ep{equil5} as
$$ \sum_{j=1}^J N_j f(R_t, j) + L_t = {H_{t}\over p_t}. \EQN equil2 $$
For $t \geq 1$,
iterating \Ep{omo1} once  and using $R_{t-1} = {p_{t-1} \over p_t}$
gives
$$ L_t = R_{t-1} R_{t-2} L_{t-2} + {H_t - H_{t-2} \over p_t} .$$
Iterating back to time $0$ and using $L_0 = 0$ gives
$$ L_t = {H_t - H_0\over p_t }. \EQN omo2 $$
Substituting \Ep{omo2} into \Ep{equil2} gives
$$ \sum_{j=1}^J N_j f(R_t, j)  = {H_{0}\over p_t}. \EQN equil3 $$
This is the same equilibrium condition in the economy with
no open market operations, i.e., the economy
with $L_t \equiv 0$ for all $t\geq 1$.   Any price level  and rate of
return sequence that solves \Ep{equil3} also solves \Ep{equil5} for
any $L_t$ sequence satisfying \Ep{omo1}. \qed

\medskip
This proposition captures  the spirit of Adam Smith's
real bills doctrine, which states that if the
government issues bank notes to purchase safe evidences
of private indebtedness, it is not inflationary.
Sargent and Wallace (1982) go on to analyze
settings in which the money market is initially separated from
the credit market by some legal restrictions that inhibit
intermediation.  Then open market operations are no longer
irrelevant because they can  partially  undo the
legal restrictions.  Sargent and Wallace show how those
legal restrictions can help stabilize the
price level at a cost in terms of economic efficiency.
Kahn and Roberds (1998) extend the Sargent and Wallace model  to study
 issues about regulating electronic payments systems.
\auth{Roberds, William}
\auth{Kahn, Charles}
\auth{Wallace, Neil}
\auth{Smith, Lones}
\auth{Kandori, Michihiro}
\section{Gift-giving equilibrium}
Michihiro Kandori (1992) and Lones Smith (1992) have used ideas
from the literature on reputation (see chapter \use{credible})
to study whether there exist history-dependent sequences of gifts that
support an optimal allocation. \index{gift-giving game!with overlapping
generations}
 Their idea is to set up the economy as a game played with a sequence
of players.  We briefly describe a gift-giving game for an overlapping
generations economy in which voluntary intergenerational gifts support
an optimal allocation.
Suppose that the consumption of an initial old person is
$$  c^0_1 = y_1^0 + s_1$$
 and the utility of each young agent is
$$ u(y^i_i - s_i) + u(y^i_{i+1} + s_{i+1}), \quad i \geq 1
  \EQN utility1 $$
where $s_i\geq 0$ is the gift from a young person
at $i$ to an old person at $i$.  Suppose that the endowment pattern is
$y^i_i = 1-\epsilon, y^i_{i+1} = \epsilon$, where
$\epsilon \in (0,.5)$.

Consider the following system of expectations, to which a young
person chooses whether to conform:
$$\EQNalign{ s_i& = \cases{ .5 - \epsilon & if $v_i = \overline v$; \cr
                             0 &otherwise.\cr } \EQN folk1;a  \cr
             v_{i+1}& = \cases{ \overline v & if $v_i = \overline v$ and
                                       $s_i = .5 - \epsilon$; \cr
                               \underline v & otherwise.\cr} \EQN folk1;b \cr}$$
Here we  are free  to take
$ \overline v = 2 u(.5)$ and $\underline v= u(1-\epsilon) +
u(\epsilon)$.    These are ``promised utilities.'' We make them serve
as ``state variables'' that summarize the history of intergenerational
gift giving.   To start, we need an initial value
$v_1$.   Equations   \Ep{folk1} act as the transition
laws that young agents face in choosing $s_i$ in
\Ep{utility1}.

  An initial condition $v_1$ and the rule \Ep{folk1} form a system
of expectations that tells the young person of each generation
what he is  expected to give.  His gift is  immediately handed over
to an old person.
A system of expectations is called an {\it equilibrium}
if for each $i \geq 1$, each young   agent chooses
to conform.

  We can immediately compute two equilibrium systems of
expectations.  The first  is
the ``autarky'' equilibrium: give nothing yourself and
expect  all future generations to give nothing. To  represent
this equilibrium within equations \Ep{folk1}, set $v_1 \neq \overline v$.
It is easy to verify that each young person will
confirm what is expected of him in this equilibrium.
Given that future generations will not give,  each young
person chooses not to give.

  For the second equilibrium, set $v_1 = \overline v$.   Here
each household chooses to give the expected amount, because
failure to do so causes  the next generation of
young people not to give; whereas affirming the expectation
to give passes that expectation along to the next generation,
which affirms it in turn.
Each of these equilibria is credible, in the sense
of subgame perfection, to be studied extensively in chapter
\use{credible}.


\auth{Kocherlakota, Narayana R.}
 Narayana Kocherlakota (1998) has  compared gift giving and monetary
equilibria in a variety of environments and has used the comparison to
provide a  precise sense in which ``money'' substitutes for ``memory''.

\auth{Mankiw, Gregory}
\auth{Summers, Lawrence}
\auth{Zeckhauser, Richard}
\auth{Abel, Andrew}
\auth{Diamond, Peter A.}
\section{Concluding remarks}
The overlapping generations model is a workhorse in
analyses of  public finance, welfare economics,
and demographics.  Diamond (1965) studied
some fiscal policy issues within a version of the model with a neoclassical production.
He showed that, depending on preference and productivity
parameters, equilibria of the model can
 have too much capital, and that such capital
overaccumulation can be corrected by having the government
issue and perpetually roll over unbacked debt.\NFootnote{Abel,
Mankiw, Summers, and Zeckhauser (1989) propose an empirical
test of whether there is capital overaccumulation in the U.S.
economy, and conclude that there is not.}
Auerbach  and  Kotlikoff (1987)  formulated a long-lived overlapping
generations model with capital, labor, production, and
various kinds of taxes. They used the model to study
a host of fiscal issues. Rios-Rull (1994a) used a calibrated
overlapping generations growth model to examine
the quantitative importance of market incompleteness for
insuring against aggregate risk. See Attanasio (2000) for
a  review of theories and evidence
about consumption within life-cycle models.
\auth{Attanasio, Orazio}

Several authors in a 1980 volume edited by John Kareken
and Neil Wallace argued through example that the overlapping
generations model is useful for analyzing a variety of issues in
monetary economics.  We refer to that volume, McCandless
and Wallace (1992), Champ and Freeman (1994), Brock (1990),
and Sargent (1987b) for a variety of applications
of the overlapping generations model to issues in monetary
economics.
\auth{Wallace, Neil}
\auth{Kareken, John}
\auth{Champ, Bruce}
\auth{Freeman, Scott}
\auth{Brock, William A.}
\auth{McCandless, George T.}

%\section{Exercises}
\showchaptIDfalse
\showsectIDfalse
\section{Exercises}
\showchaptIDtrue
\showsectIDtrue
\noindent{\it Exercise \the\chapternum.1} \ \  At each date $t \geq 1$, an  economy
consists of  overlapping generations of  a constant
number $N$ of two-period-lived agents.  Young
agents born in $t$ have preferences over consumption streams of
a single good that are ordered by
$ u(c_t^t) + u(c^t_{t+1}) $, where
$u(c) = c^{1 - \gamma} / (1-\gamma)$, and where $c^i_t$ is the
consumption of an agent born at $i$ in time $t$. It is understood
that $\gamma >0$, and that when $\gamma = 1$, $u(c) = \ln c$.   Each
young agent born at $t \geq 1$ has identical preferences and endowment
pattern
$(w_1,w_2)$, where $w_1$  is the endowment when young and  $w_2$
 is the endowment when old.
Assume $0 < w_2 < w_1$.
 In addition,
there are some initial old agents at time $1$ who
are endowed with $w_2$ of the time $1$ consumption good, and
who order consumption streams by $c^0_1$.   The initial old
(i.e., the old  at $t=1$)
are also endowed with $M$ units of unbacked fiat currency.  The
stock of currency is constant over time.
\medskip
\noindent{\bf a.}    Find the saving function of a young agent.
\medskip
\noindent{\bf b.}  Define an equilibrium with valued fiat
currency.
\medskip
\noindent{\bf c.}   Define a stationary equilibrium with valued
fiat currency.
\medskip
\noindent{\bf d.}  Compute a stationary equilibrium with valued
fiat currency.
\medskip
\noindent{\bf e.}  Describe how many equilibria with
valued fiat currency there are.  (You are not being asked to
compute them.)
\medskip
\noindent{\bf f.}  Compute the limiting value as $t \rightarrow + \infty$
of the rate of return on currency in each of the nonstationary
equilibria with valued fiat currency.  Justify your calculations.

\bigskip
\noindent{\it Exercise \the\chapternum.2} \ \
  Consider an economy with overlapping generations
of a constant population of an even number  $N$ of  two-period-lived agents.
New young agents are born at each date $t\geq 1$.
Half of the young agents are endowed with
$w_1$ when young and $0$ when old.  The other half are
endowed with $0$ when young and  $w_2$ when old.
Assume $0 < w_2 < w_1$.
Preferences of all young agents are as in problem
1, with $\gamma =1$.   Half of the $N$ initial old are endowed
with $w_2$  units of the consumption good and half are endowed
with nothing.  Each old  person orders consumption streams by
$c^0_1$.  Each old person at $t=1$ is endowed with $M$ units
of unbacked fiat currency. No other generation is endowed with
fiat currency.  The stock of fiat currency is fixed over time.

\medskip
\noindent{\bf a.}  Find the saving function of each of the two
types of young person for $t\geq 1$.
\medskip\noindent {\bf b.}  Define an equilibrium without valued fiat currency.
Compute  all such equilibria.
\medskip
\noindent{\bf c.}  Define an equilibrium with valued fiat currency.
\medskip
\noindent{\bf d.}  Compute all the (nonstochastic) equilibria
with valued fiat currency.
\medskip
\noindent{\bf e.}  Argue that there is a unique stationary equilibrium
with valued fiat currency.
\medskip
\noindent{\bf f.}  How are the various equilibria  with valued
fiat currency ranked by
the Pareto criterion?

\bigskip
\noindent{\it Exercise \the\chapternum.3} \ \   Take the economy of exercise
{\it \the\chapternum.1\/}, but make
one change.    Endow the initial old with a tree that
 yields a constant  dividend of $d >0$ units of the consumption
good for each $t \geq 1$.
\medskip
\noindent{\bf a.}   Compute all  the equilibria with valued
fiat currency.
\medskip
\noindent{\bf b.}  Compute all  the equilibria without valued
fiat currency.
\medskip
\noindent{\bf c.}  If you want,  you can  answer both parts
of this question in the context of the following
particular  numerical
example:  $w_1 =10,   w_2 =5, d=.000001$.


\bigskip
\noindent{\it Exercise  \the\chapternum.4} \ \  Take the economy of exercise {\it \the\chapternum.1\/} and make
the following two changes. First, assume
that $\gamma=1$. Second, assume that the number of young
agents born at $t$ is $N(t) =n N(t-1)$, where $N(0) >0$ is
given and $n \geq 1$.    Everything else about the economy remains
the same.

\medskip
\noindent{\bf a.}  Compute   an equilibrium without valued
fiat money.
\medskip
\noindent{\bf b.}  Compute a stationary equilibrium  with valued
fiat money.
\bigskip
\noindent{\it Exercise \the\chapternum.5} \ \  Consider an economy consisting of overlapping
generations of two-period-lived consumers.  At each date
$t\geq 1$ there are born $N(t)$ identical young people each of whom is
endowed with $w_1 >0$ units of a single consumption good when young
and $w_2 >0$ units of the consumption good when old.   Assume
that $w_2 < w_1$.    The consumption good is not storable.
The population of young people is described by
$N(t) = n N(t-1)$, where $n > 0$.   Young people
born at $t$ rank utility streams according to
$\ln(c^t_t) + \ln(c^t_{t+1})$ where $c^i_t$ is the consumption
of the time $t$ good of an agent born in $i$.    In addition,
there are $N(0)$ old people at time $1$, each of whom is
endowed with $w_2$ units of the time $1$ consumption good.
The old at $t=1$ are also endowed with one unit of  unbacked
pieces of infinitely durable but  intrinsically worthless pieces
of paper called fiat money.
\medskip
\noindent {\bf a.}  Define an equilibrium without valued fiat
currency.   Compute such an equilibrium.

\medskip
\noindent{\bf b.}  Define an equilibrium with valued fiat currency.
\medskip
\noindent{\bf c.}  Compute all equilibria with valued fiat currency.
\medskip
\noindent{\bf d.}  Find the limiting rates of return on currency
as $t \rightarrow + \infty$ in
each of the equilibria that you found in part   c.  Compare them
with the one-period  interest rate in the
equilibrium in part a.
\medskip
\noindent{\bf e.}  Are the equilibria in part c ranked according
to the Pareto criterion?


\medskip
\noindent{\it Exercise \the\chapternum.6} \ \  {\bf Exchange rate determinacy}
\medskip
\noindent   The world consists of two economies, named $i=1,2$, which
except for their governments' policies are ``copies'' of
one another.  At each date $t\geq 1$,
there is a single consumption good, which is storable, but
only  for rich people.  Each economy
consists of overlapping generations of two-period-lived agents.
For each $t \geq 1$, in economy $i$, $N$ poor people and
$N$ rich people are born.
Let $c_t^h(s),y_t^h(s) $
be the  time $s$ (consumption, endowment) of a type $h$ agent born
at $t$.  Poor agents are endowed with $[y_t^h(t), y_t^h(t+1)] = (\alpha,0)$;
rich agents are endowed
 with $[y_t^h(t), y_t^h(t+1)] = (\beta,0)$, where $\beta > > \alpha$.
In each country, there are $2N$ initial old who are
endowed in the aggregate
with $H_i(0)$ units of an unbacked currency and with
$2 N \epsilon$ units of the time $1$ consumption good.
For the rich people, storing $k$ units of the time $t$
 consumption good produces $R k$ units of the time $t+1$
 consumption good, where $R >1$ is a fixed gross
rate of return  on storage.  Rich people can earn
the rate of return $R$ either by storing goods or by lending to either
government by means of indexed bonds.
      We assume that poor people are prevented from storing
capital or holding indexed government debt by the sort
of denomination and intermediation restrictions
described by Sargent and Wallace (1982).

 For each $t\geq 1$, all young agents
order consumption streams according to
$\ln c_t^h(t) + \ln c_t^h(t+1)$.

   For $t \geq 1$, the government of country $i$ finances
a stream of purchases (to be thrown into the ocean)
of $G_i(t)$ subject to the following budget constraint:
$$ G_i(t) + R B_i(t-1) = B_i(t) + {H_i(t) - H_i(t-1)\over p_i(t)}
                        + T_i(t), \leqno(1)$$
where $B_i(0)=0$;
$p_i(t)$ is the price level in country $i$;
 $T_i(t)$ are lump-sum taxes levied by the government
on the {\it rich} young people at time $t$; $H_i(t)$ is
the stock of $i$'s fiat currency at the end
of period $t$; $B_i(t)$ is the stock of indexed government
interest-bearing debt (held by the rich of either country).   The government
does not explicitly tax poor people, but might
tax through an inflation tax.  Each government levies
a lump-sum tax of $T_i(t)/N$ on each young rich citizen of its
own country.


  Poor people in both countries  are free to hold whichever
currency they prefer.  Rich people can hold debt of either
government and can also store; storage and both government debts
bear a  constant gross rate of return $R$.
\medskip

\noindent{\bf a.}  Define an {\it equilibrium\/} with valued fiat
currencies (in both countries).
\medskip
\noindent{\bf b.}  In a nonstochastic equilibrium, verify the following
proposition: if an equilibrium exists in which both fiat currencies
are valued, the exchange rate between the two currencies must be
constant over time.
\medskip


\noindent{\bf c.}  Suppose that government policy   in each
country is characterized by specified (exogenous)
levels $G_i(t) = G_i, T_i(t) = T_i$,
$B_i(t) = 0, \forall t \geq 1$. (The remaining elements of government policy
adjust to   satisfy the government budget constraints.)
Assume that the exogenous components of policy have
been set so that an equilibrium with two valued fiat
currencies exists.    Under this
description of policy, show that the equilibrium exchange rate
is indeterminate.

\medskip
\noindent{\bf d.}  Suppose that government policy in each country is
described as follows: $G_i(t)=G_i, T_i(t)=T_i, H_i(t+1)=H_i(1),
B_i(t)=B_i(1) \
 \forall t \geq 1$.
Show that if there exists an equilibrium with two valued fiat
currencies, the exchange rate is determinate.

\medskip

\noindent{\bf e.}  Suppose that government policy in country
$1$ is specified in terms of exogenous levels of
$s_1 = [H_1(t) - H_1(t-1)]/p_1(t)  \ \forall t \geq 2$,
and $G_1(t)=G_1 \ \forall t \geq 1$.  For country $2$,
government policy consists of exogenous levels
of $B_2(t)= B_2(1), G_2(t) = G_2 \forall t \geq 1$.  Show that if there
exists an equilibrium with two valued fiat currencies, then
the exchange rate is determinate.


\medskip

\noindent{\it Exercise \the\chapternum.7}\quad {\bf Credit controls}
\medskip
\noindent Consider the following overlapping generations model.  At each date $t\ge 1$
there appear $N$ two-period-lived young people, said to be of generation $t$,
who live and consume during periods $t$ and $(\toone)$.  At time $t=1$ there
exist $N$ old people who are endowed with $H(0)$ units of paper ``dollars,''
which they offer to supply inelastically to the young of generation 1 in
exchange for goods.  Let $p(t)$ be the price of the one good in the model,
measured in dollars per time $t$ good.  For each $t\ge 1$, $N/2$ members of
generation $t$ are endowed with $y>0$ units of the good at $t$ and 0 units at
$(\tone)$, whereas the remaining $N/2$ members of generation $t$ are endowed
with 0 units of the good at $t$ and $y>0$ units when they are old.  All members
of all generations have the same utility function:
$$u[c_t^h(t), c_t^h(\tone)]=\ln c_t^h(t) +\ln c_t^h(\tone),$$
where $c_t^h(s)$ is the consumption of agent $h$ of generation $t$ in period
$s$.  The old at  $t=1$ simply maximize $c_0^h(1)$.  The consumption good is
nonstorable.  The currency supply is constant through time, so $H(t)=H(0)$,
$t\ge 1$.
\medskip
\noindent{\bf a.} Define a competitive equilibrium without valued currency for this
model.  Who trades what with whom?
\medskip
\noindent{\bf b.} In the equilibrium without valued fiat currency,
 compute   competitive equilibrium values of the
gross return on consumption loans, the consumption allocation of the old at
$t=1$, and that of the ``borrowers'' and ``lenders'' for $t\ge 1$.
\medskip
\noindent{\bf c.} Define a competitive equilibrium with valued currency.  Who trades
what with whom?\medskip
\noindent{\bf d.} Prove that for this economy there does not exist a competitive
equilibrium with valued currency.\medskip
\noindent{\bf e.} Now suppose that the government imposes the restriction that
$l_t^h(t) [1+r(t)]\ge -y/4$, where $l_t^h(t)[1+r(t)]$ represents claims on
$(\tone)$--period consumption purchased (if positive) or sold (if negative) by
household $h$ of generation $t$.  This is a restriction on the amount of
borrowing.  For an equilibrium without valued currency, compute the consumption
allocation and the gross rate of return on consumption loans.
\medskip
\noindent{\bf f.} In the setup of part  e,
 show that there exists an equilibrium with
valued currency in which the price level obeys the quantity theory equation
$p(t)=qH(0)/N$.  Find a formula for the undetermined coefficient $q$.  Compute
the consumption allocation and the equilibrium rate of return on consumption
loans.\medskip
\noindent{\bf g.} Are lenders better off in economy b or economy f?  What about
borrowers?  What about the old of period 1 (generation 0)?
\medskip
\noindent{\it Exercise \the\chapternum.8}\quad {\bf Inside money and real bills}
\medskip

\noindent Consider the following overlapping generations model of two-period-lived
people.  At each date $t\ge 1$ there are born $N_1$ individuals of type 1 who
are endowed with $y>0$ units of the consumption good when they are young and
zero units when they are old; there are also born $N_2$ individuals of type 2
who are endowed with zero units of the consumption good when they are young and
$Y>0$ units when they are old.  The consumption good is nonstorable.  At time
$t=1$, there are $N$ old people, all of the same type, each endowed with zero
units of the consumption good and $H_0/N$ units of unbacked paper called ``fiat
currency.'' The populations of type 1 and 2 individuals, $N_1$ and $N_2$,
remain constant for all $t\ge 1$.  The young of each generation are identical
in preferences and maximize the utility function $\ln c_t^h(t) +\ln
c_t^h(\tone)$ where $c_t^h(s)$ is consumption in the $s$th period of a member
$h$ of generation $t$.
\medskip
\noindent{\bf a.} Consider the equilibrium without valued currency (that is, the
equilibrium in which there is no trade between generations).  Let $[1+r(t)]$ be
the gross rate of return on consumption loans.  Find a formula for $[1+r(t)]$
as a function of $N_1, N_2,y$, and $Y$.\medskip
\noindent{\bf b.} Suppose that $N_1,N_2,y$,
 and $Y$ are such that $[1+r(t)]>1$ in the
equilibrium without valued currency.  Then prove that there can
exist no quantity-theory-style equilibrium where fiat currency is
valued and where the price level $p(t)$ obeys the quantity theory
equation $p(t)=q\cdot H_0$, where $q$ is a positive constant and
$p(t)$ is measured in units of currency per unit good.
\medskip
\noindent{\bf c.} Suppose that $N_1,N_2,y$, and $Y$ are such that in the
nonvalued-currency equilibrium $1+r(t)<1$.  Prove that there
exists an equilibrium in which fiat currency is valued and that
there obtains the quantity theory equation $p(t)=q\cdot H_0$,
where $q$ is a constant.  Construct an argument to show that the
equilibrium with valued currency is not Pareto superior to the
nonvalued-currency equilibrium.
\medskip \noindent{\bf d.} Suppose that
$N_1,N_2,y$, and $Y$ are such that, in the preceding
nonvalued-currency economy, $[1+r(t)]<1$, there exists an
equilibrium in  which fiat currency is valued.  Let $\bar p$ be
the stationary equilibrium price level in that economy.  Now
consider an alternative economy, identical with the preceding one
in all respects except for the following feature: a government
each period purchases a constant amount $L_g$ of consumption loans
and pays for them by issuing debt on itself, called ``inside
money'' $M_I$, in the amount $M_I(t)=L_g\cdot p(t)$.  The
government never retires the inside money, using the proceeds of
the loans to finance new purchases of consumption loans in
subsequent periods.  The quantity of outside money, or currency,
remains $H_0$, whereas the ``total high-power money'' is now
$H_0+M_I(t)$. \item{(i)} Show that in this economy there
exists a valued-currency equilibrium in which the price level is
constant over time at $p(t)=\bar p$, or equivalently, with $\bar
p=qH_0$ where $q$ is defined in part c. \item{(ii)} Explain
why government purchases of private debt are not inflationary in
this economy. \item{(iii)} In many models,
once-and-for-all government open-market operations in private debt
normally affect real variables and/or price level.  What accounts
for the difference between those models and the one in this
exercise?
\medskip
\noindent{\it Exercise \the\chapternum.9}\quad {\bf Social security and the price level}
\medskip
\noindent Consider an economy (``economy I'') that consists of overlapping generations of
two-period-lived people.  At each date $t\ge 1$ there is born a constant
number $N$ of young people, who desire to consume both when they are young, at
$t$, and when they are old, at $(\tone)$.  Each young person has the utility
function $\ln c_t(t)+\ln c_t(\tone)$, where $c_s(t)$ is time $t$ consumption of
an agent born at $s$.  For all dates $t\ge 1$, young people are endowed with
$y>0$ units of a single nonstorable consumption good when they are young and
zero units when they are old.  In addition, at time $t=1$ there are $N$ old
people endowed in the aggregate with $H$ units of unbacked fiat currency.  Let
$p(t)$ be the nominal price level at $t$, denominated in dollars per time $t$
good.
\medskip\noindent{\bf a.}
 Define and compute an equilibrium with valued fiat currency for this
economy.  Argue that it exists and is unique.  Now consider a second economy
(``economy II'') that is identical to economy I except that economy II
possesses a social security system.  In particular, at each date $t\ge 1$, the
government taxes $\tau>0$ units of the time $t$ consumption good away from each
young person and at the same time gives $\tau$ units of the time $t$
consumption good to each old person then alive.
\medskip\noindent{\bf b.} Does economy II possess an equilibrium with valued fiat currency?
Describe the restrictions on the parameter $\tau$, if any, that are needed to
ensure the existence of such an equilibrium.
\medskip\noindent{\bf c.} If an equilibrium with valued fiat currency exists, is it unique?
\medskip\noindent{\bf d.}
 Consider the {\it stationary\/} equilibrium with valued fiat currency.
Is it unique? Describe how the value of currency or price level would vary
across economies with differences in the size of the social security system, as
measured by $\tau$.
\medskip
\noindent{\it Exercise \the\chapternum.10}\quad {\bf Seignorage}
\medskip
\noindent Consider an economy consisting of overlapping generations of two-period-lived
agents.  At each date $t\ge 1$, there are born $N_1$ ``lenders'' who are
endowed with $\a>0$ units of the single consumption good when they are young
and zero units when they are old.  At each date $t\ge 1$, there are also born
$N_2$ ``borrowers'' who are endowed with zero units of the consumption good
when they are young and $\be>0$ units when they are old.  The good is
nonstorable, and $N_1$ and $N_2$ are constant through time.  The economy starts
at time 1, at which time there are $N$ old people who are in the aggregate
endowed with $H(0)$ units of unbacked, intrinsically worthless pieces of paper
called dollars.  Assume that $\a,\be,N_1$, and $N_2$ are such that
$${N_2\be\over N_1\a} <1.$$
Assume that everyone has preferences
$$u[c_t^h(t), c_t^h(\tone)]=\ln c_t^h(t) +\ln c_t^h(\tone),$$
where $c_t^h(s)$ is consumption of time $s$ good of agent $h$ born at time $t$.
\medskip\noindent{\bf a.} Compute the equilibrium interest rate on consumption loans in the
equilibrium without valued currency.
\medskip\noindent{\bf b.} Construct a {\it brief\/} argument to establish whether or not the
equilibrium without valued currency is Pareto optimal.

The economy also contains a government that purchases and destroys $G_t$ units
of the good in period $t$, $t\ge 1$.  The government finances its purchases
entirely by currency creation.  That is, at time $t$,
$$G_t ={H(t)-H(t-1)\over p(t)},$$
where $[H(t)-H(t-1)]$ is the additional dollars printed by the government at
$t$ and $p(t)$ is the price level at $t$.  The government is assumed to
increase $H(t)$ according to
$$H(t)=zH(t-1),\qquad z\ge 1,$$
where $z$ is a constant for all time $t\ge 1$.

At time $t$, old people who carried $H(t-1)$ dollars between $(t-1)$ and
$t$ offer these $H(t-1)$ dollars in exchange for time $t$ goods.  Also at $t$
the government offers $H(t)-H(t-1)$ dollars for goods, so that $H(t)$ is the
total supply of dollars at time $t$, to be carried over by the young into time
$(\tone)$.
\medskip\noindent{\bf c.} Assume that $1/z>N_2\be/N_1\a$.  Show that under this assumption
there exists a continuum of equilibria with valued currency.
\medskip\noindent{\bf d.}
 Display the unique stationary equilibrium with valued currency in the
form of a ``quantity theory'' equation.  Compute the equilibrium rate of return
on currency and consumption loans.
\medskip\noindent{\bf e.} Argue that if $1/z<N_2\be/N_1\a$, then there exists no
valued-currency equilibrium.  Interpret this result.  ({\it Hint:}
 Look at the rate
of return on consumption loans in the equilibrium without valued currency.)
\medskip\noindent{\bf f.}
 Find the value of $z$ that {\it maximizes\/} the government's $G_t$ in
a stationary equilibrium.  Compare this with the largest value of $z$ that is
compatible with the existence of a valued-currency equilibrium.
\medskip
\noindent{\it Exercise \the\chapternum.11} \quad {\bf Unpleasant monetarist arithmetic}
\medskip
\noindent Consider an economy in which the aggregate demand for government currency for
$t\ge 1$ is given by $[M(t)p(t)]^d =g[R_1(t)]$, where $R_1(t)$ is the gross
rate of return on currency between $t$ and $(\tone)$, $M(t)$ is the stock of
currency at $t$, and $p(t)$ is the value of currency in terms of goods at $t$
(the reciprocal of the price level).  The function $g(R)$ satisfies
$$g(R) (1-R)  =h(R)>0\qquad {\rm for}\ \ R\in (\underline{R},1),
\leqno(1) $$
where $h(R)  \le 0$  for $R<\underline{R}, R\ge 1,
\underline{R}>0$ and
where
$h'(R)<0$ for $R>R_m$, $ h'(R)>0$ for $R<R_m$
$h(R_m)>D$, where $D$ is a positive number to be defined
shortly.
The government faces an infinitely elastic demand for its interest-bearing
bonds at a constant-over-time gross rate of return $R_2>1$.  The government
finances a budget deficit $D$, defined as government purchases minus explicit
taxes, that is constant over time.  The government's budget constraint is
$$D=p(t) [M(t)-M(t-1)] +B(t) -B(t-1) R_2,\qquad t\ge 1,\leqno(2)$$
subject to $B(0)=0, M(0)>0$.  In equilibrium,
$$M(t) p(t)=g[R_1(t)].\leqno(3)$$
The government is free to choose paths of $M(t)$ and $B(t)$, subject to
equations (2) and (3).
\medskip\noindent{\bf a.}
 Prove that, for $B(t)=0$, for all $t>0$, there exist two stationary
equilibria for this model.
\medskip\noindent{\bf b.}
 Show that there exist values of $B>0$, such that there exist
stationary equilibria with $B(t)=B$, $M(t)p(t)=Mp$.
\medskip\noindent{\bf c.} Prove a version of the following proposition: among stationary
equilibria, the lower the value of $B$, the lower the stationary rate of
inflation consistent with equilibrium. (You will have to make an assumption
about Laffer curve effects to obtain such a proposition.)

This problem displays some of the ideas used by Sargent and Wallace (1981).
They argue that, under assumptions like those leading to the
proposition stated in part c, the ``looser'' money is today [that is, the
higher $M(1)$ and the lower $B(1)$], the lower the stationary inflation rate.
\medskip
\noindent{\it Exercise \the\chapternum.12}\quad {\bf Grandmont-Hall}
\medskip
\noindent Consider a nonstochastic, one-good overlapping generations model consisting of
two-period-lived young people born in each $t\ge 1$ and  an initial group of
old people at $t=1$ who are endowed with $H(0)>0$ units of unbacked currency at
the beginning of period 1.  The one good in the model is not storable.  Let the
aggregate first-period saving function of the young be time-invariant and be
denoted $f[1+r(t)]$ where $[1+r(t)]$ is the gross rate of return on consumption
loans between $t$ and $(\tone)$.  The saving function is assumed to satisfy
$f(0)=-\infty$, $f'(1+r)>0$, $f(1)>0$.

Let the government pay interest on currency, starting in period 2 (to holders
of currency between periods 1 and 2).  The government pays interest on currency
at a nominal rate of $[1+r(t)] p(\tone)/\bar p$, where $[1+r(t)]$ is the real
gross rate of return on consumption loans, $p(t)$ is the price level at $t$,
and $\bar p$ is a target price level chosen to satisfy
$$\bar p=H(0)/f(1).$$
The government finances its interest payments by printing new money, so that
the government's budget constraint is
$$H(\tone)-H(t) =\left\{ [1+r(t)] {p(\tone)\over \bar p} -1\right\} H(t),\qquad
t\ge 1,$$
given $H(1)=H(0)>0$.  The gross rate of return on consumption loans in this
economy is $1+r(t)$.  In equilibrium, $[1+r(t)]$ must be at least
as great as the real rate of return on currency
$$1+r(t) \geq [1+r(t)] p(t)/\bar p =[1+r(t)] {p(\tone)\over \bar p} {p(t)\over p(\tone)}$$
with equality if currency is valued,
$$1+r(t) = [1+r(t)]p(t)/\bar p,\qquad 0<p(t)<\infty.$$
The loan market-clearing condition in this economy is
$$f[1+r(t)]=H(t)/p(t).$$
\medskip\noindent{\bf a.} Define an equilibrium.
\medskip\noindent{\bf b.}
 Prove that there exists a unique monetary equilibrium in this economy
and compute it.
\medskip
\noindent{\it Exercise \the\chapternum.13}\quad {\bf Bryant-Keynes-Wallace}
\medskip
\noindent Consider an economy consisting of overlapping generations of two-period-lived
agents.  There is a constant population of $N$ young agents born at each date
$t\ge 1$.  There is a single consumption good that is not storable.  Each agent
born in $t\ge 1$ is endowed with $w_1$ units of the consumption good when young
and with $w_2$ units when old, where $0<w_2<w_1$.  Each agent born at $t\ge 1$
has identical preferences $\ln c_t^h(t) +\ln c_t^h (\tone)$, where $c_t^h(s)$
is time $s$ consumption of agent $h$ born at time $t$.  In addition, at time 1,
there are alive $N$ old people who are endowed with $H(0)$ units of unbacked
paper currency and who want to maximize their consumption of the time 1 good.

A government attempts to finance a constant level of government purchases
$G(t)=G>0$ for $t\ge 1$ by printing new base money.  The government's budget
constraint is
$$G=[H(t)-H(t-1)]/p(t),$$
where $p(t)$ is the price level at $t$, and $H(t)$ is the stock of currency
carried over from $t$ to $(\tone)$ by agents born in $t$.  Let $g=G/N$ be
government purchases per young person.  Assume that purchases $G(t)$ yield no
utility to private agents.
\medskip\noindent{\bf a.} Define a stationary equilibrium with valued fiat currency.
\medskip\noindent{\bf b.} Prove that, for $g$ sufficiently small, there exists a stationary
equilibrium with valued fiat currency.
\medskip\noindent{\bf c.} Prove that, in general, if there exists one stationary equilibrium
with valued fiat currency, with rate of return on currency $1+r(t)=1+r_1$, then
there exists at least one other stationary equilibrium with valued currency
with $1+r(t)=1+r_2 \not=1+r_1$.
\medskip\noindent{\bf d.} Tell whether the equilibria described in parts  b and c
 are Pareto
optimal, among allocations among private agents of what is
left after the government takes $G(t)=G$ each period. (A proof is not required
here: an informal argument will suffice.)

Now let the government institute a forced saving program of the following form.
At time 1, the government redeems the outstanding stock of currency $H(0)$,
exchanging it for government bonds.  For $t\ge 1$, the government offers each
young consumer the option of saving at least $F$ worth of time $t$ goods in the
form of bonds bearing a constant rate of return $(1+r_2)$.  A legal prohibition
against private intermediation is instituted that prevents two or more private
agents from sharing one of these bonds.  The government's budget constraint for
$t\ge 2$ is
$$G/N =B(t) -B(t-1)(1+r_2),$$
where $B(t) \ge F$.  Here $B(t)$ is the saving of a young agent at $t$.  At
time $t=1$, the government's budget constraint is
$$G/N =B(1) -{H(0)\over Np (1)},$$
where $p(1)$ is the price level at which the initial currency stock is redeemed
at $t=1$.  The government sets $F$ and $r_2$.

Consider stationary equilibria with $B(t)=B$ for $t\ge 1$ and $r_2$ and $F$
constant.

\medskip\noindent{\bf e.}
 Prove that if $g$ is small enough for an equilibrium of the type
described in part a to
exist, then a stationary equilibrium with forced saving exists. (Either a
graphical argument or an algebraic argument is sufficient.)
\medskip\noindent{\bf f.} Given $g$, find the values of $F$ and $r_2$ that maximize the utility
of a representative young agent for $t\ge 1$.
\medskip\noindent{\bf g.} Is the equilibrium
 allocation associated with the values of $F$ and
$(1+r_2)$ found in part f optimal among those allocations that give $G(t)=G$ to
the government for all $t\ge 1$?  (Here an informal argument will suffice.)
