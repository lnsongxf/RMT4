
\input grafinp3
\input psfig
%\eqnotracetrue

%  see figures commit1.eps and commit2.eps

%\input form1

%\input psfig
%\input grafinput8

%Ay\c se \.Imrohoro\u glu and Selahattin \.Imrohoro\u glu


\overfullrule=0pt
\parindent=.5truecm
\def\refshape{\parshape 2 0truecm 14.765truecm 1truecm
14.765truecm\noindent}

\def\book#1#2#3{\refshape #1.  {\sl #2}. #3.}

\def\article#1#2#3#4{\refshape #1. `#2'. {\sl #3},
#4.}

\def\contribution#1#2#3#4#5{\refshape #1. `#2'. In #3
(eds.), {\sl #4}. #5.}

\def\ocontribution#1#2#3#4#5{\refshape #1. `#2'. In #3
(ed.), {\sl #4} #5.}

\def\mimeo#1#2#3{\refshape #1. `#2'.  Mimeo. #3.}

\def\dissertation#1#2#3{\refshape #1. `#2'. Ph.D.\
dissertation, #3.}

\def\bull{\vrule height .9ex width .8ex depth -.1ex}
\def\bh{\penalty-100}



%
\def\frac#1/#2{\leavevmode\kern.1em
 \raise.5ex\hbox{\the\scriptfont0 #1}\kern-.1em
 /\kern-.15em\lower.25ex\hbox{\the\scriptfont0 #2}}
%
\showchaptIDtrue
\def\@chaptID{16.}

%Blackboard Bold R
\def\bbR{{I\kern-0.3em R}}

%\input grafinpu
%\input vpsfig

%\hbox{}
\footnum=0
\chapter{Credible Government Policies: I\label{credible}}

\section{Introduction}

%The timing of actions can matter.
Kydland and Prescott (1977)
opened the modern discussion of \idx{time consistency} in macroeconomics
with some examples that show how outcomes differ in otherwise
identical economies when the assumptions about the timing of
government policy choices are altered.\NFootnote{Consider two extensive-form
versions of the ``battle of the sexes'' game described by Kreps (1990), one
in which the man chooses first, the other in which the woman
chooses first.  Backward induction recovers different outcomes in
these two different games.  Though they share the same choice sets
and payoffs, these are different games.}   In particular, they compared
a timing protocol in which a government chooses its (possibly
history-contingent) policies once and for all at the beginning of
time with one in which the government chooses sequentially.
Because outcomes are worse when the government chooses sequentially,
Kydland and Prescott's examples illustrate the superiority of the once-and-for-all choice timing protocol
for the government.%
\index{state-contingent policies}%
\auth{Kreps, David M.}%
\auth{Kydland, Finn E.}%
\auth{Prescott, Edward C.}%

Subsequent work on \idx{time consistency}
focused on how
a \idx{reputation} can improve outcomes  when a government chooses sequentially.\NFootnote{Barro
and Gordon (1983a, 1983b) are early contributors to this literature.
See Kenneth Rogoff (1989) for a survey.}  The issue is whether
constraints confronting the government and private sector expectations %incentives and expectations
can be arranged so that a government
adheres to an expected
pattern of behavior because it would worsen its  reputation
if it did not.%
%The
%literature seeks to sustain rational
%expectations that
%induce a government through fear of loss of a good reputation to
%behave as it would under a commitment technology,
%even when it chooses sequentially.
\index{commitment!technology}%
\auth{Barro, Robert J.}%
  \auth{Gordon, David B.}  \auth{Rogoff, Kenneth}

A ``folk theorem'' from game theory states that if there is no discounting of future payoffs,
then repetition of many  first-period payoff can be sustained by a reputational
equilibrium.  A main purpose of this chapter is to study how discounting of future payoffs
affects the set of  outcomes that are attainable with a reputational
mechanism.
\index{folk theorem}%

Modern formulations of reputational models of government policy use and extend ideas
from \idx{dynamic programming}.   Each period, a government faces choices whose
consequences include a first-period return and a \idx{reputation}
to pass on to next period. Under rational expectations, any reputation that
the government carries into next period must be one that it will want to confirm.
We shall study the set of possible values that the government can attain with
reputations that it could conceivably want to confirm.
\auth{Abreu, Dilip}%
\auth{Stacchetti, Ennio}%
\auth{Pearce, David}%
\auth{Chari, V.V.}%
\auth{Kehoe, Patrick J.}%
\auth{Prescott, Edward C.}%
 \auth{Stokey, Nancy L.}%

This   chapter and chapter \use{chang} apply an apparatus of Abreu, Pearce, and
Stacchetti (1986, 1990) (APS) to reputational equilibria in a
class of macroeconomic models.
APS use ideas from dynamic
programming.\NFootnote{This chapter closely follows Stacchetti (1991),
who applies Abreu, Pearce, and Stacchetti (1986, 1990)  to a
more general class of models than that treated here. Stacchetti
also studies a class of setups in which the private sector
observes only a noise-ridden signal of the government's actions.}
  Their work exploits the
insight that it is more convenient to work with
the set of continuation values associated  with equilibrium
strategies than it is to work directly with the
 set of equilibrium strategies.  We use an economic model
like those  of
Chari, Kehoe, and Prescott (1989) and Stokey (1989, 1991)
to exhibit what Chari and Kehoe (1990) call sustainable
government policies and  what         \index{public policies!sustainable}
 Stokey calls credible public policies.    \index{public policies!credible}
The literature on
sustainable or credible government policies in macroeconomics adapts
ideas from the literature on repeated games so that they can
be applied in contexts in which a single agent (a government) behaves
strategically, and in which all other agents' behavior
can be summarized in terms of a system of expectations about government actions together with    competitive equilibrium outcomes that respond
 to the government's actions.\NFootnote{For descriptions
of theories
of credible government policy, see Chari and Kehoe (1990),
Stokey (1989, 1991), Rogoff (1989), and Chari, Kehoe,
and Prescott (1989).  For applications of the framework
of Abreu, Pearce, and Stacchetti, see Chang (1998) and Phelan
and Stacchetti (1999).}
\auth{Kehoe, Patrick J.}%
\auth{Chari, V.V.}%
\auth{Rogoff, Kenneth}%
\auth{Prescott, Edward C.}%
\auth{Chang, Roberto}%
\auth{Stacchetti, Ennio}%
\auth{Phelan, Christopher}%
\auth{Kydland, Finn E.}%

\subsection{Diverse sources of history dependence}

The theory of credible government policy uses particular kinds of history dependence to render credible a  sequence of actions  chosen by  a {\it sequence\/}
of policy makers. Here {\it credible\/} means an action that the public rationally expects the government to take because it thinks it is in the government's interest  to do so.  Hence, a {\it credible\/}  action  is one that the  government  {\it wants\/} to implement.  By way of contrast, in chapter \use{stackel}, we  encountered a distinct source of history dependence in the policy of a Ramsey planner or Stackelberg leader. There history dependence came from the requirement that it is necessary  to account for  constraints that dynamic aspects of  private sector behavior put on the time $t$ action of a Ramsey planner or Stackelberg leader who at time $0$  makes  once-and-for-all choices of intertemporal sequences.  In that context, history dependence emerges from the requirement that the Ramsey planner's time $t$ action  must confirm private sector expectations
 that the Ramsey planner had chosen  at  time $0$  partly  to influence private sector outcomes in periods $0, \ldots, t-1$.

In settings in which private agents face genuinely dynamic decision problems having their own endogenous state variables like various forms of physical and human capital, {\it both\/} sources of
history dependence influence a credible policy.  It can be subtle  to disentangle the economic forces contributing to history dependence in government
policies in such settings.     However, for  special  examples that  deprive private agents' decision problems of any `natural' state variables, we can isolate the source of history dependence coming from the requirement that a government policy  must be credible.  We consider only such examples  in this chapter
for the avowed purpose of isolating the source of history dependence coming from credibility considerations and distinguishing it from the
chapter \use{stackel}
source that instead comes  from the need to respect substantial  dynamics coming from equilibrium private sector behavior. Having isolated one source
of history dependence in chapter \use{stackel} and another in the present chapter, we proceed in chapter \use{chang} to activate both
sources of history dependence and then to seek a recursive representation for a credible government policy in that more comprehensive setting.



%
%\section{Dynamic programming squared: synopsis}
%
%  Like chapter \use{socialinsurance}, this chapter uses continuation
%values as state variables in terms of which a  Bellman equation is
%cast. Because   the continuation values themselves satisfy another
%Bellman equation, we give the general method the nickname
%``dynamic programming squared'':   one Bellman equation chooses a
%law of motion for a state variable that must itself  satisfy
%another Bellman equation.\NFootnote{Recall also the closely
%related ideas described in chapter \use{stackel}.} \index{dynamic
%programming!squared}
%
%For possible future reference, we outline the main concepts here.  In formulating
%dynamic-programming-squared problems,  we use the following circle of
%ideas about histories, values, and strategy profiles. (Later we shall
% define precisely what we mean by history, value, and strategy profile.)
%A value for each agent in the economy is a discounted sum of
%functions of future outcomes. A {\it history\/} of outcomes
%generates a sequence of profiles of values for the various agents.
%A pure strategy profile is a sequence of functions mapping
%histories  up to $t-1$ into  actions at $t$.  A strategy profile
%generates a history and therefore a sequence of values. A strategy
%profile contains within it  a profile of one-period continuation
%strategies for every  possible value of next period's history.
%Therefore, it also generates  a profile of continuation values for
%each possible one-period continuation history.  The main idea of
%dynamic programming squared is to reorient attention away from
%strategies and toward values, one-period outcomes, and
%continuation values.
%
%  Ordinary dynamic programming  iterates to a fixed point
%on a mapping from continuation
%values to values: $v=T(v)$.  Similarly, dynamic programming squared
%iterates on a mapping from continuation values to values. But now, multiple
%continuation values are required to support a given first-period outcome
%and a given value.  For example, in models with a commitment problem,
%like those in chapter \use{socialinsurance} and in this chapter,
%a decision maker
%receives one continuation value if he does what is expected
%under the contract, and something else if he deviates.    How do we
%generalize to this context the idea of iterating on $v=T(v)$?  Abreu,
%Pearce, and Stacchetti showed that the natural generalization is to
%iterate on an operator that
%maps {\it pairs\/} (and more generally {\it sets\/}) of continuation
%values into {\it sets\/} of values.   They call this  operator
%$B$ and form it in the same spirit that the $T$ operator was constructed:
%it embraces optimal one-period behavior of all decision makers involved,
%assuming arbitrary one-period continuation values.
%
%The reader might want to revisit this synopsis of the structure of
%dynamic programming squared as he or she wades  through various
%technicalities that put content onto this  structure.


\section{One-period economy}

There is a continuum of households, each of which chooses an action
$\xi \in X$.  A government chooses an action $y\in Y$.
The sets $X$ and $Y$ are compact. The average level
of $\xi$ across households is denoted $x \in X$.  The utility
of a particular household
is $u(\xi,x,y)$
when it chooses $\xi$, when the average
household's choice is $x$, and when the government chooses $y$.
  The payoff function $u(\xi,x,y)$ is
strictly concave and continuously differentiable in $\xi$ and $y$.\NFootnote{The discrete-choice
 examples given later violate some of these assumptions in non essential ways.}

\subsection{Competitive equilibrium}

For given levels of $y$ and $x$, the representative household
faces the problem\ $\max_{\xi\in X}\,  u(\xi,x,y)$.  Let the
maximizer be a function $\xi = f(x,y)$.  When a household believes
that the government's choice is $y$ and that the average
level of other households' choices is $x$, it acts to set
$\xi = f(x,y)$.  Because all households are alike, this fact implies
that the actual level of $x$ is $f(x,y)$.  For the representative household's expectations about
the average to be consistent with the average outcome, we require
that $\xi=x$, or $x=f(x,y)$. This makes the representative
agent representative.   We use the following:\NFootnote{See the definition of a
rational expectations equilibrium in chapter \use{recurpe}.}

\medskip\noindent{\sc Definition 1:} A {\it competitive equilibrium\/} or a {\it
rational expectations equilibrium\/} is
an $x\in X$ that satisfies $x=f (x,y)$.
\index{competitive equilibrium}

\medskip\noindent
A competitive equilibrium satisfies $u(x,x,y) = \max_{\xi \in X}
u(\xi,x,y)$.

For each $y\in Y$, let $x=h(y)$ denote the corresponding competitive
equilibrium.  We  adopt:
\medskip\noindent{\sc Definition 2:}  The set of competitive equilibria is
$C=\{ (x,y)\mid u(x,x,y) = \max_{\xi\in X}\,
u(\xi,x,y)\}$, or equivalently $C=\{ (x,y) \mid x= h(y)\}$.

\subsection{Ramsey problem}

 The following timing of actions underlies a {\it Ramsey plan}.
\index{Ramsey plan}
First, the government selects a $y\in Y$.  Then, knowing the
government's choice of  $y$, the aggregate of households responds with
a competitive equilibrium.  The government evaluates policies
$y\in Y$ with the payoff function $u(x,x,y)$; that is,
the government is benevolent.

In choosing $y$, the government has to forecast how
the economy will respond.  We assume that the government correctly forecasts
 that the economy will respond to $y$ with a competitive
equilibrium, $x=h(y)$.  We use these definitions:

\medskip   \index{Ramsey problem}
\noindent{\sc Definition 3:}  The {\it Ramsey problem\/}
is $\max_{y\in Y}\, u[h(y),
h(y), y]$, or equivalently
$\max_{(x,y)\in C}\, u(x,x,y)$.

\medskip   \index{Ramsey outcome}
\noindent{\sc Definition 4:}
 The policy that attains the maximum for the Ramsey problem
is denoted $y^R$.  Let
$x^R=h(y^R)$.  Then
$(y^R,x^R)$ is called the
{\it Ramsey outcome\/} or {\it Ramsey plan}.

\medskip

Two remarks about the Ramsey problem are in order.  First, the Ramsey
outcome is typically inferior to the ``dictatorial outcome'' that
solves the unrestricted problem $\max_{x\in X,\,y\in Y}\, u(x,x,y)$,
because the restriction $(x,y)\in C$ is in general
binding.  Second, the timing of actions is important.
The Ramsey problem assumes that the government chooses first and must  stick with its choice
regardless of how private agents subsequently choose $x \in X$.

If the government were granted the opportunity to reconsider its plan
{\it after\/} households had chosen $x=x^R$, the government would in general want to
deviate from $y^R$ because often there exists
an $\alpha \not= y^R$ for which $u(x^R,x^R,\alpha)> u(x^R,x^R,y^R)$.
The ``time consistency problem'' is the incentive the government  would have
to deviate from
the Ramsey plan if it  were allowed to react {\it after\/} households had set $x=x^R$.
In this one-period setting, to support the Ramsey plan requires
a timing protocol that forces the government to choose first.
\index{time consistency}
   \index{Nash equilibrium}
\subsection{Nash equilibrium}

Consider an alternative timing protocol that confronts  households with  a
forecasting problem because the government chooses after or simultaneously
with the households.   Assume that households forecast that, given $x$, the government
will set $y$ to solve $\max_{y\in Y}\, u(x,x,y)$.  We use:
\medskip\noindent
{\sc Definition 5:}  A {\it Nash equilibrium} $(x^N, y^N)$ satisfies
\smallskip
\noindent (1) $(x^N,y^N)\in C$;
\smallskip
\noindent (2) Given $x^N,\, u(x^N,x^N,y^N)=\max_{\eta\in Y}\,
u(x^N,x^N,\eta)$.
\medskip\noindent
Condition (1) asserts that $x^N = h(y^N)$, or that the economy responds
to $y^N$ with a competitive equilibrium.  Thus,
condition  (1) says that given $(x^N, y^N)$, each
individual household wants to set $\xi=x^N$; that is, the representative household
has no incentive to deviate from $x^N$.  Condition (2) asserts that
given $x^N$, the government chooses a policy $y^N$ from which it has no
incentive to deviate.\NFootnote{Much of the language of this chapter
is borrowed from game theory, but the object under study
is not a game, because we do not specify all
of the objects
that formally define a game.  In particular, we do not
specify the payoffs to all  agents for all feasible choices.  We
only specify the payoffs $u(\xi, x, y)$ where each
private agent chooses the {\it same\/} value of $\xi$.\label{Ftnt:Bassetto}}

We can use the solution of the problem in condition (2) to define
the government's {\it best response\/} function $y=H(x)$.
The definition  of a Nash equilibrium can be phrased
as a pair $(x,y) \in C$ such that $y=H(x)$.


There are two timings of choices for which a Nash equilibrium
is a natural equilibrium concept.  One is where
households choose first, forecasting that the government will
respond to the aggregate outcome $x$ by setting $y=H(x)$.  Another
is where the government and households choose simultaneously,
in which case a Nash equilibrium $(x^N,y^N)$ depicts a situation
in which everyone has rational expectations:  given that
each household expects the aggregate variables to be $(x^N,y^N)$,
each household responds in a way to make $x=x^N$, and given that
the government expects that $x=x^N$, it responds by setting $y=y^N$.

We let the values attained by the government under the Nash and Ramsey outcomes, respectively, be denoted $v^N = u(x^N,x^N,y^N)$ and $v^R= u(x^R,x^R,y^R)$.
%Note that
%$v^N\leq v^R$.
Because of the additional constraint
embedded in the Nash equilibrium, outcomes are ordered according to
$$v^N = \max_{\{(x,y)\in C:\, y=H(x)\}}\, u(x,x,y)
      \leq \max_{(x,y)\in C}\, u(x,x,y) = v^R\ .$$

\section{Nash and Ramsey outcomes}

To illustrate these concepts, we consider two examples:
 taxation within a fully
specified economy, and a black-box model with discrete choice sets.


\subsection{Taxation example}

Each of a continuum of households has preferences over leisure
$\ell$, private consumption $c$, and per capita government expenditures $g$.
The utility function is
$$
U(\ell,c,g) = \ell + \log (\alpha + c) + \log (\alpha + g), \qquad \quad
              \alpha\in (0,\frac1/2).
$$
Each household is endowed with one unit of time that can be devoted to leisure
or labor. The production technology is linear in labor, and the economy's
resource constraint is
$$\overline c + g = 1 - \overline \ell,$$
where $\overline c$ and $\overline \ell$ are
the average levels of private consumption
and leisure, respectively.

A benevolent government wants to maximize the utility of the representative
household.  A benevolent government that is subject only to the constraint imposed by the technology and
  would choose $\ell=0$ and $c=g=\frac1/2$. This ``dictatorial
outcome'' yields welfare $W^d=2 \log (\alpha + \frac1/2)$.

Competitive equilibrium in general  imposes more restrictions on the allocations attainable by
a benevolent government.
Here we will focus on competitive equilibria where the government finances
its expenditures by levying a flat-rate tax $\tau$ on labor income. The
household's budget constraint at equality is $c=(1-\tau)(1-\ell)$. Given a government
policy $(\tau, g)$, an individual household's optimal decision rule for
leisure is
$$
\ell(\tau) = \cases{ {\displaystyle \alpha \over
                  \displaystyle 1-\tau} & if $\tau \in [0,\, 1-\alpha]$; \cr
\noalign{\vskip.2cm}
                      1 & if $\tau > 1-\alpha$. \cr} $$

%%%%%%%%%%%%%%%%%%%
%\midinsert
%$$\grafone{cred1.ps,height=2.5in}{{\bf Figure 16.1}  Welfare
%outcomes in the taxation example.
%The solid portion of the
% curve depicts the set of competitive equilibria, $W^c(\tau)$.
%The set of Nash equilibria is the horizontal portion of the solid curve
%and the equilibrium at $\tau=\frac1/2$.  The Ramsey outcome is
%marked with an asterisk.  The ``time inconsistency problem'' is
%idicated with the triangle showing
%the outcome if the government were able to reset $\tau$  after
%households had chosen the Ramsey labor supply.
%The dashed line describes the welfare level at the unconstrained optimum,
%$W^d$. The graph sets $\alpha=0.3$.}$$
%\endinsert
%%%%%%%%%%%%%%%%%%

\midfigure{cred1f}
\centerline{\epsfxsize=3truein\epsffile{cred1.ps}}
\caption{Welfare outcomes in the taxation example.  The solid  curve depicts the welfare
associated with the set of competitive equilibria, $W^c(\tau)$.  The
set of Nash equilibria is the horizontal portion of the solid curve
and the equilibrium at $\tau=\frac1/2$.  The Ramsey outcome is
marked with an asterisk.  The ``time inconsistency problem'' is
indicated with the triangle showing the outcome if the government were
able to reset $\tau$ after households had chosen the Ramsey labor supply.
The dashed line describes the welfare level at the unconstrained optimum,
$W^d$. The graph sets $\alpha=0.3$.}
\infiglist{cred1f}
\endfigure


\medskip
\noindent
Due to the linear technology and the fact that government expenditures enter
additively in the utility function, the household's decision rule
$\ell(\tau)$ is also the equilibrium value of individual leisure
at a given tax rate $\tau$. Imposing government budget balance,
$g=\tau(1-\ell)$, the representative household's welfare in a
competitive equilibrium can be expressed as a function of  $\tau$ and is equal to

$$W^c(\tau) = \ell(\tau) + \log \bigl\{\alpha + (1-\tau)[1-\ell(\tau)]\bigr\}
            + \log \bigl\{\alpha + \tau [1-\ell(\tau)]\bigr\}.
$$
The Ramsey tax rate and allocation are determined by
 the solution to $\max_\tau W^c(\tau)$. It can be verified that because $\alpha \in (0, .5)$,
 the Ramsey plan sets $\tau < .5$, which produces an allocation in which $c, g$, and $1- \ell$ are all positive.

 By way of contrast, the government's problem in a Nash equilibrium is
$\max_\tau \bigl\{ \ell + \log [\alpha + (1-\tau) ( 1 - \ell)]
+ \log[\alpha + \tau(1 - \ell)]\bigr\}$.  If $\ell <1$, the optimizer
is $\tau = .5$.  There is a continuum of Nash equilibria indexed
by $\tau\in[1-\alpha,\,1]$ where agents choose not to work, and
consequently $c=g=0$. The only Nash equilibrium with production
is $\tau=\frac1/2$ with welfare level $W^c(\frac1/2)$. This
conclusion follows directly from the fact that the government's
best response is $\tau=\frac1/2$ for any $\ell<1$.
These outcomes are illustrated numerically in Figure \Fg{cred1f}. %Figure 16.1.
Here the time inconsistency problem surfaces in the government's
incentive,
if offered the choice,
 to reset  the tax rate $\tau$,
after the household has set its labor supply.


The objects of the general setup in the preceding section can be mapped
into the present taxation example
as follows:  $\xi=\ell$, $x=\overline \ell$, $X=[0,1]$, $y=\tau$, $Y=[0,1]$,
$u(\xi,x,y)=\xi+\log[\alpha + (1-y)(1-\xi)] + \log[\alpha + y(1-x)]$,
$f(x,y)=\ell(y)$, $h(y)=\ell(y)$, and $H(x)=\frac1/2$ if $x<1$; and
$H(x)\in[0,\,1]$ if $x=1$.


\subsection{Black-box example with discrete choice sets}

Consider a black box example with $X=\{x_L,\,x_H\}$ and $Y=\{y_L,\,
y_H\}$, in which $u(x,x,y)$ assume the values given in
Table \the\chapternum.\the\sectionnum.1.  Assume that values of $u(\xi,x,y)$ for $\xi \neq x$ are such
that the values with asterisks for $\xi =x$ are competitive
equilibria. In particular, we might assume that

$$ \eqalign{u(\xi,x_i,y_j)&=0 \quad {\rm when} \ \xi \neq x_i \ {\rm and} \
                      i=j , \cr
            u(\xi,x_i,y_j)&= 20 \quad {\rm when} \ \xi \neq x_i \ {\rm and} \
                       i \neq j. \cr}$$
These payoffs imply that $u(x_L, x_L, y_L) > u(x_H, x_L, y_L)$
(i.e., $3 >0$), and $u(x_H,x_H,y_H) > u(x_L,x_H,y_H)$
(i.e., $10 > 0$).  Therefore, $(x_L,x_L,y_L)$ and $(x_H,x_H,y_H)$ are
competitive equilibria. Also, $u(x_H,x_H,y_L) < u(x_L,x_H,y_L)$
(i.e., $12 < 20)$, so the dictatorial outcome cannot be supported
as a competitive equilibrium.

\medskip
%
%\centerline{ {\bf Table \the\chapternum.1}  One-period payoffs $u(x_i,x_i,y_j)$ }
%$$\vbox{\offinterlineskip
%\hrule
%\halign{\strut #\hfil &\quad \hfil# &\quad \hfil# \cr
%& $x_L$ & $x_H$ \cr \noalign{\hrule}
%$y_L$ & $3\rlap{*}$ & $12$ \cr
%$y_H$ & 1 & $10\rlap{*}$ \cr \noalign{\hrule}
%\noalign{\smallskip}
%\hbox{$^\ast$Denotes $(x,y)\in C$.} \cr
%              }}$$



\midtable{junk1}
$$\vbox{\offinterlineskip
\hrule
\halign{\strut #\hfil &\quad \hfil# &\quad \hfil# \cr
& $x_L$ & $x_H$ \cr \noalign{\hrule}
$y_L$ & $3\rlap{*}$ & $12$ \cr
$y_H$ & 1 & $10\rlap{*}$ \cr \noalign{\hrule}
              }}$$
              \caption{One-period payoffs $u(x_i,x_i,y_j)$;
$^\ast$ denotes $(x,y)\in C$; the Ramsey outcome is $(x_H,y_H)$ and the Nash equilibrium outcome is
$(x_L,y_L)$.}
\endtable


Figure \Fg{game1f} depicts a timing of
choices that supports the Ramsey outcome for this example.  The
government chooses first, then walks away.  The Ramsey outcome
$(x_H,y_H)$ is the competitive equilibrium yielding the highest
value of $u(x,x,y)$.


%%%%%%%%%%%%%%%%%%
%$$\grafone{game1.ps,height=3.5in}{{\bf Figure 16.2} Timing of choices
%that supports Ramsey outcome.   Here $P$ and
%$G$ denote nodes at which the public and the government, respectively,
%choose. The government has a commitment technology that
%binds it to ``choose first.''   The government chooses
%the $y \in Y$ that maximizes $u[h(y)$,
%$h(y),y]$, where $x=h(y)$ is the function
%mapping government actions into equilibrium values of $x$.}$$
%%%%%%%%%%%%%%%%%%%%%%

\midfigure{game1f}
\centerline{\epsfxsize=3truein\epsffile{game1.ps}}
\caption{Timing of choices that supports Ramsey outcome.  Here $P$ and
$G$ denote nodes at which the public and the government, respectively,
choose. The government has a commitment technology that binds it to
``choose first.''   The government chooses the $y \in Y$ that maximizes
$u[h(y)$, $h(y),y]$, where $x=h(y)$ is the function mapping government
actions into equilibrium values of $x$.}
\infiglist{game1f}
\endfigure

\medskip

Figure \Fg{game2f} diagrams a timing of choices that supports the
Nash equilibrium.  Recall that by definition, every Nash
equilibrium  outcome has to be a competitive equilibrium outcome.
We denote competitive equilibrium pairs $(x,y)$ with asterisks.
The government sector chooses after knowing that the private
sector has set $x$, and chooses $y$ to maximize
$u(x,x,y)$.  With this timing, if the private sector chooses
$x=x_H$, the government has an incentive to set $y=y_L$,
a setting of $y$ that does not support $x_H$ as a Nash
equilibrium.  The unique Nash equilibrium is $(x_L,y_L)$, which
gives a lower utility $u(x,x,y)$ than does
the competitive equilibrium $(x_H,y_H)$.

%%%%%%%%%%%%%%%%
%$$\grafone{game2.ps,height=3.5in}{{\bf Figure 16.3} Timing of actions
%in a Nash equilibrium in which the private sector
%acts first.  Here $G$ denotes a node at which
%the government chooses and $P$ denotes
%a node at which the public chooses.
% The private sector sets $x \in X$ before knowing
%the government's setting of $y \in Y$.
%Competitive equilibrium pairs $(x,y)$ are denoted with
%an asterisk.
%%  The unique Nash equilibrium is $(x_L, y_L)$.} $$
%%%%%%%%%%%%%%%%

\midfigure{game2f}
\centerline{\epsfxsize=3truein\epsffile{game2.ps}}
\caption{Timing of actions in a Nash equilibrium in which the private
sector acts first.  Here $G$ denotes a node at which the government
chooses and $P$ denotes a node at which the public chooses.  The private
sector sets $x \in X$ before knowing the government's setting of $y \in Y$.
Competitive equilibrium pairs $(x,y)$ are denoted with an asterisk.  The
unique Nash equilibrium is $(x_L, y_L)$.}
\infiglist{game2f}
\endfigure


\section{Reputational mechanisms: general idea}

 In a finitely
repeated economy, the government will certainly behave opportunistically
the last period, implying that nothing better than a Nash
outcome can be supported the last period.  In a finite
horizon economy with a unique Nash equilibrium, we won't be
able to sustain anything better than a Nash equilibrium
outcome in {\it any\/} earlier period.\NFootnote{If there are multiple
Nash equilibria, it is sometimes possible to sustain a better-than-Nash
 equilibrium outcome for a while in a finite horizon
economy. See exercise {\it \the\chapternum.1\/}, which uses an idea of Benoit and
Krishna (1985).}   \auth{Benoit, Jean-Pierre}  \auth{Krishna, Vijay}


We want to study situations in which a government might sustain a Ramsey
outcome.  Therefore, we shall study economies repeated an infinite number
of times.  Here a system of history-dependent expectations interpretable
as a government reputation might be arranged to sustain something better than
repetition of a Nash outcome.
%The aim of the literature
%has been  to find formulations of reputation that, through the
%influence of the public's expectations,  make the history of the
%government's actions restrict the government's current  and future
We strive  to set things up so that the government so dearly  wants to
confirm a  good reputation that it will not submit to the temptation to
behave opportunistically.   A reputation is
said to be {\it sustainable\/} if it is always in the government's
interests to confirm it.

A state  variable that is capable of encoding a ``reputation'' is peculiar because it is both
``backward looking'' and ``forward looking.''  It is
backward looking  because it remembers salient features of past  behavior.   It is  forward-looking behavior because it
measures something about what private agents expect the government to do in the future.
We are about to study the ingenious machinery of
Abreu, Pearce, and    Stacchetti that astutely exploits
these aspects of a reputational variable by recognizing that the ideal reputational state variable is a ``promised value.''

\subsection{Dynamic programming squared}
\index{dynamic programming!squared}%
A sustainable reputation for the government is one that (a) the public, having rational expectations, wants to believe,
and (b) the government wants to confirm.
Rather than finding all possible sustainable reputations,
Abreu, Pearce, and Stacchetti  (henceforth APS) (1986, 1990) used dynamic
programming to characterize all {\it values\/} for the government that are
attainable  with sustainable reputations.  This section briefly describes
their main ideas, while later sections fill in many details.
\auth{Abreu, Dilip}  \auth{Pearce, David}  \auth{Stacchetti, Ennio}

First we need some language. A {\it strategy profile\/}
 is a pair of  plans, one
each for the private sector and the government.  The time $t$ components of the pair of plans maps the observed
history of the economy into current-period
outcomes $(x,y)$.  A {\it subgame perfect
equilibrium\/} (SPE)
strategy  profile  has a current-period outcome being
a competitive equilibrium
$(x_t,y_t)$ whose $y_t$ component
  the government would want
to confirm
at each $t \geq 1$ and for every
possible history of the economy.

To characterize  SPEs, or at least a very interesting subset of them,
the method of APS is to formulate a Bellman
equation that describes the value to the government of a strategy profile
and that portrays the idea that
the government wants to confirm the
private sector's beliefs about $y$.
 For each $t \geq 1$, the
government's strategy describes its
first-period action $y\in Y$, which,  because the public had expected
it,  determines
an associated first-period competitive equilibrium
$(x,y) \in C$. Furthermore,
the strategy implies two continuation values
 for the government at the beginning of next period,  a
continuation value $v_1$
 if it carries out the first-period choice $y$,
and another continuation value
$v_2$  if for any reason the government
deviates from the expected first-period choice
$y$.
Associated with the government's strategy is
a current value $v$ that obeys the Bellman equation
$$ v = (1-\delta) u(x,x,y) + \delta v_1 ,\EQN cred_dp;a   $$
where $\delta \in (0,1)$ is a discount factor, $(x,y) \in C$, $v_1$ is a continuation value
awarded for confirming the private sector's expectation that the government will choose action $y$ in the current period,  and
 $(y,v_1)$  are constrained to satisfy
the incentive constraint
$$v \geq (1-\delta) u(x,x,\eta) +
 \delta v_2 , \quad
   \forall \eta \in Y, \EQN cred_dp;b  $$
or equivalently
$$v \geq (1-\delta) u\bigl[x,x,H(x)\bigr] +
 \delta v_2 ,$$
where  $H(x) = \argmax_y u(x,x,y)$ is the government's opportunistic one period best policy in response
to $x$.
Here $v_2$ is the continuation value awarded to the government if it fails to confirm
the private sector's expectation that $\eta = y$ this period.
Because it receives the same continuation
value $v_2$ for {\it any\/} deviation from $y$, if it
does deviate, the government  will  choose the most rewarding action, which
is to set $\eta = H(x)$.

  Inequalities \Ep{cred_dp} define a \idx{Bellman equation} that
maps a {\it pair\/} of continuation values $(v_1,v_2)$ into a value $v$ and
first-period outcomes $(x,y)$. %Figure 16.4
Figure \Fg{cred2f} illustrates this
mapping for the infinitely repeated version of the taxation
example. Given a pair $(v_1,v_2)$, the solid curve depicts $v$
in equation \Ep{cred_dp;a}, and the dashed curve describes the right
side of the incentive constraint \Ep{cred_dp;b}. The region in which
the solid
curve is above the dashed curve  identifies
tax rates and competitive equilibria that
satisfy
   \Ep{cred_dp;b}
 at the prescribed continuation values $(v_1,v_2)$. As
can be seen, when $\delta =.8$,
 tax rates below 18 percent cannot be sustained
for the particular $(v_1, v_2)$ pair we have chosen.

%%%%%%%%%%%%%%%%%%
%\midinsert
%$$\grafone{cred2.ps,height=2.5in}{{\bf Figure 16.4} \  Mapping of continuation
%values $(v_1,v_2)$ into values $v$ in the infinitely repeated version of
%the taxation example. The solid curve depicts
%$v=(1-\delta)u[\ell(\tau),\ell(\tau),\tau]+\delta v_1$. The dashed
%curve is the right side of the incentive constraint,
%$v \geq (1-\delta)u\{\ell(\tau),\ell(\tau),H[\ell(\tau)]\}+\delta v_2$,
%where $H$ is the government's best response function. The part of the
%solid curve
%that is above the dashed curve shows competitive equilibrium values that
%are sustainable
%for continuation values $(v_1,v_2)$. The parameterization
%is $\alpha=0.3$ and $\delta=0.8$, and the continuation values are
%set as $(v_1,v_2)=(-0.6,\,-0.63)$.} $$
%\endinsert
%%%%%%%%%%%%%%%%%%%%%%%%%%%%

\midfigure{cred2f}
\centerline{\epsfxsize=3truein\epsffile{cred2.ps}}
\caption{Mapping of continuation values $(v_1,v_2)$ into values $v$ in the
infinitely repeated version of the taxation example. The solid curve depicts
$v=(1-\delta)u[\ell(\tau),\ell(\tau),\tau]+\delta v_1$. The dashed curve is the
right side of the incentive constraint,
$v \geq (1-\delta)u\{\ell(\tau),\ell(\tau),H[\ell(\tau)]\}+\delta v_2$,
where $H$ is the government's best response function. The part of the
solid curve that is above the dashed curve shows competitive equilibrium values
that are sustainable for continuation values $(v_1,v_2)$. The parameterization
is $\alpha=0.3$ and $\delta=0.8$, and the continuation values are set as
$(v_1,v_2)=(-0.6,\,-0.63)$.}
\infiglist{cred2f}
\endfigure


APS calculate the {\it set\/} of equilibrium values by iterating
on the mapping defined by the Bellman equation \Ep{cred_dp}. Let
$W$ be a set of candidate continuation values. As we vary
$(v_1,v_2) \in W \times W$, the Bellman equation traces out a {\it
set\/} of values, say,   $v \in B(W)$. Thus, the Bellman equation
maps {\it sets\/} of continuation values  $W$ (from which we can draw a {\it pair} of
continuation values $(v_1, v_2) \in W \times W$) into sets of current values $v \in   B(W)$.
 To qualify as SPE  values, we require
that $W \subset B(W)$, i.e., the {\it continuation\/} values drawn
from $W$ must themselves be {\it values\/}  that are in turn
supported by continuation values drawn from the same set $W.$  APS
assert that  the largest set for which $W = B(W)$ {\it is\/} the set of all
SPE values. APS show how iterations on the Bellman equation can
determine the set of equilibrium values, provided that one starts
with a big enough but bounded initial set of candidate
continuation values. Furthermore, after that set of values has
been found, APS show how to find a strategy that attains any
equilibrium value in the set.
 The remainder of the chapter describes details of
APS's formulation as applied in our setting.
We shall see   why APS want to get their hands on
the entire set of equilibrium values.

\subsection{Etymology of `dynamic programming squared'}\index{dynamic programming squared!etymology}%
Why do we call it `dynamic programming squared'?  There are two reasons.
\medskip
\item{1.} The construction works by mapping {\it two} continuation values into one, in contrast to  ordinary
dynamic programming, which  maps {\it one} continuation value tomorrow into one value function today.
\item{2.}  A continuation value plays a double role, one as a promised value that summarizes expectations of the rewards associated with future
outcomes, another as a state variable that summarizes the history of past outcomes.  In the present setting, a subgame perfect
equilibrium strategy profile can be represented recursively in terms of an initial value $v_1 \in \bbR$ and  the following 3-tuple of functions:
$$ \eqalign{x_t & = z^h(v_t)  \cr
         y_t & = z^g(v_t) \cr
   v_{t+1} & = {\cal V}(v_t, x_t, \eta_t), \forall \eta_t \in Y  ,\cr
             }
$$
the first two of which map
a promised value into a private sector decision and a government action, while the third maps a promised value and an action pair into a promised value to carry into tomorrow.  By iterating these functions, we can deduce that
the triple of functions $z^h, z^g, {\cal V}$ induces a strategy profile that maps histories of outcomes into sequences of outcomes.  The capacity to represent a subgame perfect equilibrium
recursively affords immense simplifications in terms of the number of functions we must carry.


\medskip
\section{The infinitely repeated economy}

Consider repeating our one-period economy
forever.  At each $t\geq 1$, each household chooses $\xi_t\in X$,
with the result that the average $x_t\in X$; the government
chooses $y_t\in Y$.  We use the notation $(\vec{x},\vec{y}) =
\{(x_t,y_t)\}^\infty_{t=1},\ \vec{\xi} =\{\xi_t\}^\infty_{t=1}$.
To denote a {\it history\/} of $(x_t,y_t)$ up to $t$, we use the
notation $x^t=\{x_s\}^t_{s=1},\ y^t=\{y_s\}^t_{s=1}$.  These
histories live in the spaces $X^t$ and $Y^t$, respectively, where
$X^t=X \times\cdots \times X$, the Cartesian product of $X$ taken
$t$ times, and $Y^t$ is the Cartesian product of $Y$ taken $t$
times.\NFootnote{Marco Bassetto's work (2002, 2005) shows that
this specification, which is common in the literature, excludes
some interesting applications. In particular, it rules out
contexts in which  the set of time $t$ actions available to the
government is influenced by past actions taken by households.
Such excluded examples prevail, for example, in the fiscal theory
of the price level.  To construct
   sustainable plans in those interesting environments,  Bassetto (2002, 2005)
refines the notion of sustainability to include a more complete
theory of the government's behavior off an equilibrium path.}
\auth{Bassetto, Marco}

For the repeated economy, the government
evaluates paths $(\vec{x},\vec{y})$ according to
%$$\EQNalign{
%V_h (\vec{\xi},\vec{x},\vec{y})
% &= {(1-\delta)\over \delta}\ \sum^\infty_{t=1}\, \delta^t
%u(\xi_t,x_t,y_t),                                           \EQN cred_V;a\cr
%V_g (\vec{x},\vec{y}) &= {(1-\delta)\over\delta}\ \sum^\infty_{t=1}\delta^t\,
%r(x_t,y_t),                                                 \EQN cred_V;b\cr}$$
$$
V_g (\vec{x},\vec{y}) = {(1-\delta)\over\delta}\ \sum^\infty_{t=1}\delta^t\,
r(x_t,y_t),                                                 \EQN cred_V $$
where $r(x_t,y_t)\equiv u(x_t,x_t,y_t)$ and $0<\delta<1$.\NFootnote{Note that we have not defined the
government's payoff when $\xi_t \neq x_t$. See footnote \use{Ftnt:Bassetto}.}
  A {\it pure strategy\/} is defined as a sequence of
functions, the $t$th element of which maps the history $(x^{t-1},y^{t-1})$
observed at  the
beginning of $t$ into an action at $t$.  In particular, for the aggregate of
households, a strategy is a sequence $\sigma^h=\{\sigma^h_t\}^\infty_{t=1}$
such that
$$\eqalign{\sigma^h_1 &\in X\cr
\sigma^h_t &: X^{t-1}\times Y^{t-1}\to X\qquad\hbox{for each}\quad t\geq 2
\ .\cr}$$
Similarly, for the government,
a strategy $\sigma^g = \{\sigma_t^g\}_{t=1}^\infty$
is a sequence such that
$$\eqalign{\sigma^g_1 &\in Y\cr
\sigma^g_t &: X^{t-1}\times Y^{t-1}\to Y\qquad\hbox{for each}\quad t\geq 2.
\cr}$$
We call $\sigma = (\sigma^h, \sigma^g)$ a {\it strategy profile}.
We let $\sigma_t = (\sigma^h_t,\,\sigma^g_t)$ be the $t$th component of the
strategy profile.


\subsection{A strategy  profile  implies a history and a value}
APS begin with the insight that
a strategy profile $\sigma=(\sigma^g,\sigma^h)$ evidently recursively
generates a trajectory of
outcomes  $\{[x(\sigma)_t,
y(\sigma)_t]\}^\infty_{t=1}$:
$$\eqalign{\bigl[x(\sigma)_1, y(\sigma)_1\bigr] &= (\sigma^h_1,\sigma_1^g)\cr
\bigl[x(\sigma)_t,\, y(\sigma)_t\bigr] &= \sigma_t \bigl[x(\sigma)^{t-1},\
y(\sigma)^{t-1}\bigr].
\cr}$$
Therefore, a strategy profile also generates a pair of values
for the government and the representative private
agent. In particular,
the value for the government of a strategy profile $\sigma= (\sigma^h,
\sigma^g)$ is the value of the
trajectory that it generates
$$V_g(\sigma) = V_g \bigl[\vec{x}(\sigma),\,\vec{y}(\sigma)\bigr].$$
\index{recursive}
\subsection{Recursive formulation}
A key step toward APS's recursive formulation
comes from defining {\it continuation
stategies\/} and their associated {\it continuation values\/}.
Since the value of a path $(\vec x, \vec y)$ in equation \Ep{cred_V}
is additively separable in its one-period returns, we can express the
value recursively in terms of a one-period economy and a
continuation economy.
In particular, the value to the
government of an outcome sequence $(\vec x, \vec y)$ can be represented
$$ V_g (\vec{x},\vec{y}) = (1-\delta)\, r(x_1, y_1)
+ \delta V_g \bigl(\{x_t\}^\infty_{t=2}\, ,\ \{y_t\}^\infty_{t=2}\bigr)
  \EQN og1$$
and the value for a household can also be represented recursively.
Notice how a strategy profile $\sigma$ induces a strategy profile for
the continuation economy, as follows. Let $\sigma\vert_{(x^t,y^t)}$ denote
the strategy profile for a continuation economy  whose first period is
$t+1$ and that is initiated after history $(x^t,y^t)$ has been observed;
here $(\sigma\vert_{(x^t,y^t)})_s$ is the $s$th component of
$(\sigma\vert_{(x^t,y^t)})$, which for $s \geq 2$ is
a function that maps $X^{s-1} \times Y^{s-1}$
into $X \times Y$, and for $s=1$ is a point in $X \times Y$.
Thus, after a first-period outcome pair $(x_1,y_1)$, strategy $\sigma$ induces
the continuation strategy
$$\eqalign{ (\sigma\vert_{(x_1,y_1)})_{s + 1} \, (\nu^s,\eta^s)
 & = \sigma_{s+2}\, (x_1, \nu_1,\ldots, \nu_s, y_1,
\eta_1,\ldots, \eta_s) \cr
 &  \quad\hbox{for all }\ (\nu^s,\eta^s)\in X^s
\times Y^s\ , \quad \forall s \geq 0.\cr}$$
It might be helpful to write out a few terms for $s = 0, 1, \ldots$:
$$\eqalign{
(\sigma \vert_{(x_1,y_1)})_1 & = \sigma_2(x_1,y_1) =(\nu_1, \eta_1) \cr
(\sigma \vert_{(x_1,y_1)})_2(\nu_1,\eta_1) &= \sigma_3(x_1,\nu_1,
   y_1, \eta_1) = (\nu_2,\eta_2) \cr
(\sigma \vert_{(x_1,y_1)})_3(\nu_1,\nu_2,\eta_1,\eta_2) &=
        \sigma_4(x_1,\nu_1,\nu_2,y_1,\eta_1, \eta_2) = (\nu_3,\eta_3) . \cr}  $$
More generally, define the continuation strategy
$$\eqalign{ (\sigma\vert_{(x^t,y^t)})_1  =& \sigma_{t+1}(x^t,y^t) \cr
(\sigma\vert_{(x^t,y^t)})_{s + 1}  (\nu^s,\eta^s) = &
\sigma_{t+s+1}\,  (x_1,\ldots, x_t, \nu_1,\ldots,\nu_s;
 y_1,\ldots, y_t, \eta_1,\ldots, \eta_s)\cr
&\hbox{for all }\ s\geq 1\quad\hbox{and all}\quad (\nu^s,\eta^s)\in
X^s \times Y^s\ .\cr}$$
Here $(\sigma\vert_{(x^t,y^t)})_{s + 1}
\ (\nu^s,\eta^s)$ is the induced strategy
pair to apply in the ($s +1$)th period
 of the continuation economy.  We attain this strategy by shifting the original strategy
forward $t$ periods and evaluating it at history
$(x_1,\ldots, x_t,\allowbreak \nu_1,\ldots,\nu_s;
 y_1,\ldots, y_t, \eta_1,\ldots, \eta_s)$
for the {\it original\/} economy.
%\Fig{fig3f} depicts the unfolding of choices over time in such
%an economy.

%%%%%%%%%%%%%%%
%$\grafone{fig3.ps,height=3.5in}{{\bf Figure 16.5} \ An infinite horizon
%conomy.  The government
%nd the public choose simultaneously.  Only the first three
%eriods of the economy are shown.  Each period the economy repeats
%tself.
%A {\it continuation strategy} is shown.
%The outcome
%rajectory is \hfil\break
%(x_1, \nu_1, \nu_2; y_1, \eta_1, \eta_2)$.}$$
%%%%%%%%%%%%%%%%%

%\midfigure{fig3f}
%\centerline{\epsfxsize=3truein\epsffile{fig3.ps}}
%\caption{An infinite horizon economy.  The government and the public choose
%simultaneously.  Only the first three periods of the economy are shown.  Each
%period the economy repeats itself.  The outcome trajectory is \hfil\break
%$(x_1, \nu_1, \nu_2; y_1, \eta_1, \eta_2)$.}
%\infiglist{fig3f}
%\endfigure
%

\medskip
In terms of the continuation strategy $\sigma\vert_{(x_1,y_1)}$,
from equation \Ep{og1} we know that $V_g(\sigma)$ can be represented
as
$$ V_g(\sigma) = (1-\delta) r(x_1, y_1) + \delta V_g(\sigma\vert_{(x_1,
     y_1)}).   \EQN og2   $$
Representation \Ep{og2} decomposes the value to the government of strategy
profile $\sigma$ into a one-period return and the continuation value
$V_g(\sigma\vert_{(x_1,y_1)})$ associated with the continuation
strategy $\sigma\vert_{(x_1,y_1)}$.

Any sequence $(\vec{x},\vec{y})$ in equation \Ep{og1} or any strategy profile
$\sigma$ in equation \Ep{og2} can be assigned a value.
We want a notion of an {\it equilibrium\/}  strategy profile.

\index{subgame perfect equilibrium}
\section{Subgame perfect equilibrium (SPE)}
\medskip\noindent{\sc Definition 6:}  A strategy profile $\sigma = (\sigma^h,\sigma^g)$
is a {\it subgame perfect equilibrium} (SPE)
 of the infinitely repeated economy if for each
$t\geq 1$ and each history $(x^{t-1},\, y^{t-1}) \in X^{t-1}\times Y^{t-1}$
\medskip
\noindent(a)  The private sector outcome
 $x_t = \sigma^h_t\, (x^{t-1},y^{t-1})$ is consistent
with   competitive equilibrium when $y_t = \sigma^g_t\,(x^{t-1}, y^{t-1})$;
\medskip
\noindent(b)  For each possible government action $\eta \in  Y$
$$ (1-\delta)  r(x_t, y_t) + \delta V_g (\sigma\vert_{(x^t,
y^t)})
\geq  (1-\delta)\, r(x_t,\eta) +\delta V_g (\sigma\vert_{(x^t;
y^{t-1},\eta)}) .$$
\medskip\noindent
Requirement a
says two things.  It attributes a theory of forecasting government
behavior to members of the public, in particular, that they
use the time $t$ component $\sigma_t^g$ of the government's strategy
and information available at the end of period $t-1$ to
forecast the government's decision at $t$.  Condition a
also asserts that a competitive
equilibrium appropriate to the public's  forecast value for $y_t$ is
the outcome at time $t$.  Requirement b says that
at each point in time and  following each history, the
government has no incentive to deviate from the first-period action called
for by its strategy $\sigma^g$; that is, the government always
 wants
to choose as the public expects.   Notice
how in condition b, the government
{\it contemplates\/} setting its time $t$ choice $\eta_t$ at something
other than the value forecast by the public, but confronts consequences
 that deter it from  choosing an $\eta_t$ that
fails to confirm the public's expectations of it.

In section \use{sec:spe_interpretations}  we'll discuss the following question: who {\it chooses\/} $\sigma^g$,
the government or the public?  This question arises naturally because
$\sigma^g$ is {\it both\/} the government's sequence of policy
functions {\it and\/} the private sector's rule
for forecasting government behavior. Condition b of  definition 6
says that the government chooses to confirm the public's
forecasts.


  Definition 6  implies that for each $t \geq 2$ and each $(x^{t-1},
y^{t-1})
\in X^{t-1} \times Y^{t-1}$, the continuation strategy
$\sigma\vert_{(x^{t-1},y^{t-1})}$ is itself an SPE.
We state this formally for $t=2$.

\medskip\noindent
\medskip\noindent{\sc Proposition 1:} Assume that $\sigma$ is an SPE.
Then for all $(\nu,\eta) \in X \times Y$, $ \sigma\vert_{(\nu,\eta)}$ is
an SPE.
\medskip
\noindent{\sc Proof:} Write out requirements a and b that Definition 6 asserts that  the continuation strategy $\sigma\vert_{(\nu,\eta)}$ must satisfy to
qualify as an SPE.  In particular,
for all $s \geq 1$ and for all $(x^{s-1},y^{s-1})
\in X^{s-1} \times Y^{s-1}$, we require
$$ (x_s, y_s) \in C , \EQN ogi' $$
where $x_s = \sigma^h\vert_{(\nu,\eta)}(x^{s-1},y^{s-1}),
y_s = \sigma^g\vert_{(\nu,\eta)}(x^{s-1},y^{s-1})$.
We also require that for all $\tilde \eta \in Y$,
%$$ \eqalign{ & (1-\delta) r(x_s, y_s) + \delta V_g(\sigma\vert
%       _{(\eta, x^s; \nu, y^s)}) \cr
%       & \geq (1-\delta) r(x_s, \tilde \eta) + \delta
%       V_g(\sigma\vert_{(\nu,x^s; \eta,y^{s-1},\tilde \eta)}) \cr}\EQN ogii'$$
$$ (1-\delta) r(x_s, y_s) + \delta V_g(\sigma\vert
       _{(\eta, x^s; \nu, y^s)})
       \geq (1-\delta) r(x_s, \tilde \eta) + \delta
       V_g(\sigma\vert_{(\nu,x^s; \eta,y^{s-1},\tilde \eta)}) \EQN ogii'$$
Notice that requirements a and b of Definition 6 for $t=2, 3, \ldots $
imply expressions \Ep{ogi'} and \Ep{ogii'} for $s = 1,2, \ldots $. \qed

\medskip
  The statement that
$\sigma\vert_{(\nu,\eta)}$ is an SPE for all $(\nu,\eta) \in
X \times Y$ ensures that $\sigma$ is {\it almost\/} an SPE.
If we know that $\sigma\vert_{(\nu,\eta)}$ is an SPE for
all $(\nu, \eta) \in   (X \times Y)$, we must only add  two
requirements to ensure that $\sigma$ is an SPE:
%The only things that are lacking are (i) and (ii) of the definition
%for $t=1$.
%To assure that $\sigma$ itself is a subgame perfect equilibrium,
%we must add two requirements to the subgame perfection of $\sigma\vert_{(\nu,
%\eta)}$ for all $(\nu,\eta) \in X \times Y$:
first, that
the $t=1$ outcome pair $(x_1, y_1)$ is a competitive equilibrium,
and second, that the government's choice of $y_1$ satisfies the
%time$-1$
time $1$ version of the incentive constraint b in Definition 6.

This reasoning leads  to the following  lemma that is at the heart of the APS analysis:
\medskip\noindent
{\sc Lemma:}  Consider a strategy profile $\sigma$, and let
the associated first-period
outcome be given by $x=\sigma^h_1,y=\sigma^g_1$.  The profile $\sigma$ is
an SPE if and only if
\medskip
\noindent (1) for each $(\nu,\eta) \in X\times Y,\quad
\sigma\vert_{(\nu,\eta)}$ is an SPE;
\smallskip
\noindent (2) $(x,y)$ is a competitive equilibrium;
\smallskip
\noindent (3) $\forall\ \eta\in Y$,
$(1-\delta)\, r(x,y) + \delta\,V_g(\sigma\vert_{(x,y)})\geq (1-\delta)\,
r(x,\eta) + \delta V_g(\sigma\vert_{(x,\eta)})$.
\medskip\noindent
{\sc Proof:} First, prove the ``if'' part.  Property a of
the lemma and properties \Ep{ogi'} and \Ep{ogii'}
of Proposition 1 show that requirements a and b of
Definition 6 are satisfied for $t \geq 2$.  Properties (2) and
(3) of the lemma imply that requirements a and b of Definition 6
hold for $t=1$.

   Second, prove the ``only if'' part.  Part (1) of the lemma follows from
Proposition 1.  Parts (2) and (3) of the lemma follow from requirements
a and b of Definition 6 for $t=1$.
\qed

\medskip
The lemma is very important because it characterizes SPEs in terms of a first-period competitive
equilibrium outcome pair $(x,y)$, and a
{\it pair\/} of continuation
values:  a value $V_g (\sigma\vert_{(x,y)})$ to be awarded to the government
next period if it adheres to the $y$ component of
the first-period  pair $(x,y)$, and a value
$V_g (\sigma\vert_{(x,\eta)}),\, \eta\not= y$, to be awarded to the government if
it deviates from the expected $y$ component.
  Each of these values has to be selected from
a set of  values  $V_g(\sigma)$ that are associated with some
SPE $\sigma$. %Insisting that the continuation values themselves be
%associated with subgame perfect values  embodies the idea that the
%government faces future consequences of its actions today that are
%credible because  in the future it will want to accept those
%consequences. We now illustrate this construction. \vfil\eject
\index{Nash equilibrium!infinite repetition}
\section{Examples of SPE}
\subsection{Infinite repetition of one-period Nash equilibrium}
It is easy to verify that the following strategy profile
$\sigma^N=(\sigma^h, \sigma^g)$ forms an SPE: $\sigma_1^h = x^N,
\sigma_1^g = y^N$ and for $t \geq 2$
$$\eqalign{
\sigma^h_t &= x^N\qquad \forall\ t\ , \quad \forall\ (x^{t-1},
y^{t-1});\cr
\sigma^g_t &= y^N\qquad \forall\ t\ ,\quad \forall\ (x^{t-1},
y^{t-1}).\cr}$$
These strategies instruct the households and the government to choose the
static Nash equilibrium outcomes for all periods for all histories.  Evidently,
for these strategies, $V_g (\sigma^N) = v^N = r(x^N,y^N)$.  Furthermore, for these
strategies the continuation value $V_g (\sigma\vert_{(x^t;y^{t-1},\eta)})=
v^N$ for all outcomes $\eta \in Y$.  These strategies satisfy
requirement a of Definition 6 because $(x^N,y^N)$ is a
competitive equilibrium.  The strategies satisfy requirement
b because $r(x^N,y^N) = \max_{y\in Y} r(x^N,y)$
and because the continuation value $V_g(\sigma) = v^N$ is independent of
the action chosen by the government in the first period.
In this SPE, $\sigma^N_t=\{\sigma^h_t,\,\sigma^g_t\} =
(x^N,y^N)$ for all $t$ and for all $(x^{t-1}, y^{t-1})$, and the value
$V_g(\sigma^N)$ and the continuation values $V_g(\sigma^N\vert_{(x^t,y^t)})$
for each history $(x^t,y^t)$  equal $v^N$.

   It is useful to think about  this SPE in terms
of the lemma.  To verify that $\sigma^N$ is a SPE, we work with the first-period outcome pair $(x^N, y^N)$ and
the pair of values $V_g(\sigma \vert_{(x^N, y^N)})= v^N,
V_g(\sigma \vert_{(x, \eta)})= v^N$, where $v^N = r(x^N, y^N)$.
With these settings, we can verify that $(x^N, y^N)$ and
$v^N$ satisfy requirements (1), (2), and (3)
of the lemma.
\index{trigger strategy}
\subsection{Supporting better outcomes with trigger strategies}\label{sec:trigger1}%
The public can
have a system of expectations about the government's behavior
that induces the government to choose a better-than-Nash outcome
$(\tilde x, \tilde y)\in C$. Thus, suppose that the public expects that so
long as the  government chooses $\tilde y$, it will continue to
do so in the future, but if ever the government deviates from this choice,
 thereafter the public expects that the government will choose $y^N$, prompting
the public (or what we can call  ``the market'') to react with $x^N = h(y^N)$.  This
system of expectations confronts the government with the prospect
of being ``punished by the market's expectations'' if it chooses
to deviate from $\tilde y$.

To formalize this idea,
we shall use the SPE  $\sigma^N$ as a continuation strategy
and the value $v^N$ as a
continuation value on the right side of part (b) of
Definition 6 of an SPE (for $\eta \not=  y_t$); then by working backward one step, we
shall try to construct {\it another\/} SPE $\tilde \sigma$
with first-period outcome $(\tilde x, \tilde y) \neq (x^N, y^N)$.
In particular, for our new SPE  $\tilde \sigma$ we
propose to set
$$\eqalign{\tilde\sigma_1 &= (\tilde x,\tilde y)\cr
\tilde\sigma\vert_{(x,y)} &=
\cases{\tilde \sigma & if $(x,y) = (\tilde x,\tilde y)$\cr
         \sigma^N & if $(x,y)\not= (\tilde x,\tilde y)$\cr} \cr}\EQN og3
$$
where $(\tilde x,\tilde y) \in C$ is a competitive equilibrium that satisfies the
following particular case of part b of Definition 6:
$$\tilde v =(1-\delta)\, r(\tilde x,\tilde y) + \delta \tilde v
\geq (1-\delta)\, r(\tilde x, \eta) + \delta v^N\,,\EQN og4$$
for all $\eta \in Y$.
%where $\tilde v = V_g
%\Bigl( \{x_t = \tilde x\}^\infty_{t=1},\ \{y_t = \tilde y\}^\infty_{t=1}\Bigr)
%= r(\tilde x, \tilde y)$.
  Inequality \Ep{og4} is
equivalent with
$$\max_{\eta\in Y}\, r(\tilde x,\eta) - r(\tilde x,\tilde
y) \leq {\delta\over 1-\delta}\ (\tilde v - v^N) .\EQN og5$$
For any $(\tilde x,\tilde y) \in C$ that satisfies expression \Ep{og5}
with $\tilde v = r(\tilde x,\tilde y)$, strategy \Ep{og3} is an
SPE with value $\tilde v$.
%Figure 16.6 depicts the unfolding of choices in an
%economy following a trigger strategy.

%%%%%%%%%%%%%%%%%%%%
%$$\grafone{fig4.ps,height=3.5in}{{\bf Figure 16.6 A trigger strategy.}
%The public and the government choose simultaneously at each time $t$.
%Only $t=1$ is shown.  If the government sticks to
%the prescribed strategy, the payoff is $(1-\delta) r(x,y) +
%\delta v_1$, where $v_1$ is the value of the infinite subgame
%starting at node $P1$. If the government deviates from the prescribed
%strategy, the payoff will be $(1-\delta) r(x,\eta) + \delta v_2$,
%where $v_2$ is the value of the infinite subgame at node $P2$,
%and $(x,y)$ is the equilibrium outome in the first period;
%$v$ is the value of the subgame perfect outcome path starting at
%node $P0$.}$$
%%%%%%%%%%%%%%%%%%%%%%%%%%

%\midfigure{fig4f}
%\centerline{\epsfxsize=3truein\epsffile{fig4.eps}}
%\caption{The public and the government choose simultaneously at each time $t$.
%Only $t=1$ is shown.  If the government sticks to the prescribed strategy,
%the payoff is $(1-\delta) r(x,y) + \delta v_1$, where $v_1$ is the value
%of the infinite subgame starting at node $P1$. If the government deviates
%from the prescribed strategy, the payoff will be $(1-\delta) r(x,\eta) +
%\delta v_2$, where $v_2$ is the value of the infinite subgame at node $P2$,
%and $(x,y)$ is the equilibrium outome in the first period; $v$ is the value
%of the subgame perfect outcome path starting at node $P0$.}
%\infiglist{fig4f}
%\endfigure


If $(\tilde x,\tilde y) = (x^R,y^R)$ satisfies inequality \Ep{og5}
with $\tilde v = r(x^R,y^R)$, then repetition of the Ramsey outcome
$(x^R,y^R)$ is supportable by a subgame perfect equilibrium of the
form \Ep{og3}.

This construction uses the following objects:
\medskip
\item{1.} A proposed first-period competitive equilibrium $(\tilde x,\tilde y)\in C$.
\item{2.} An SPE $\sigma^2$ with value $V_g
(\sigma^2)$ that is used as the continuation strategy in the
event that the first-period outcome does not equal $(\tilde x,\tilde y)$,
so that $\tilde\sigma\vert_{(x,y)} = \sigma^2$, if $(x,y)\not=
(\tilde x,\tilde y)$.  In the example, $\sigma^2 = \sigma^N$ and
$V_g(\sigma^2) = v^N$.
\item{3.} An SPE $\sigma^1$, with value $V_g
(\sigma^1)$, used to define the continuation value to be assigned after
first-period outcome $(\tilde x,\tilde y)$;  and an associated  continuation strategy
$\tilde\sigma\vert_{(\tilde x,\tilde y)} = \sigma^1$.  In the
example, $\sigma^1 = \tilde \sigma$, which is defined recursively
(and self-referentially) via equation \Ep{og3}.
\item{4.} A candidate for a new  SPE  $\tilde\sigma$ and a corresponding value $V_g(\tilde \sigma)$.  In the example,
$V_g(\tilde \sigma) = r(\tilde x, \tilde y)$.
\medskip
In the example, objects 3 and 4 are equated.

Note how we have used the lemma in verifying that $\tilde \sigma$
is an SPE.  We start with the SPE $\sigma^N$ with associated value
$v^N$.  We guess a first-period outcome pair $(\tilde x, \tilde
y)$ and a value $\tilde v$ for a new SPE, where $\tilde v =
r(\tilde x, \tilde y)$. Then we verify requirements (2) and (3) of
the lemma with $(\tilde v, v^N)$ as continuation values and
$(\tilde x, \tilde y)$ as first-period outcomes.



\subsection{When reversion to Nash is not bad enough}

For discount factors $\delta$ sufficiently close to one, it is typically
 possible to support repetition of the Ramsey outcome
$(x^R, y^R)$ with a section \use{sec:trigger1} trigger strategy of  form  \Ep{og3}. This finding conforms
with a version
of the folk theorem about repeated games.  However,
there exist discount factors $\delta$ so small
that the continuation value associated with infinite   repetition of the one-period Nash
outcome is not low enough to support repetition of Ramsey.
In that case, anticipating that it will revert
to repetition of Nash after a deviation  can at best support
a value for the government that
is less than that associated with repetition of Ramsey outcome,
although perhaps
better than
repetition of the Nash outcome.

It is natural at this point to ask whether in this circumstance  there is a better SPE.
This question inspired  APS to find the
set of values associated with {\it all\/} SPEs.
 To support something {\it better\/} evidently requires finding an SPE
that has a value {\it worse\/} than that
associated with repetition of the one-period Nash outcome.
  Following APS, we shall soon
see that the best and worst equilibrium values are linked.

\auth{Abreu, Dilip}  \auth{Pearce, David}  \auth{Stacchetti, Ennio}
\section{Values of all SPEs}
The role played by the lemma in analyzing our two examples
hints at the central role that it plays in  methods
that APS  developed for describing
and computing  values for
 {\it all\/} the subgame perfect equilibria.  APS build on the way that the lemma
characterizes SPE values in terms of a first-period competitive  equilibrium
outcome, along with a pair of continuation values, each element of
which is itself a value associated with some SPE. The lemma
directs APS's attention away from a {\it set of strategy
profiles\/}  $\sigma$ and toward a {\it set of values\/}
 $V_g(\sigma)$ associated
with those profiles.  They define the set $V$ of values associated with
subgame perfect equilibria:
$$V=\{ V_g\, (\sigma)\mid \sigma\ \hbox{ is an SPE}
\}.$$
Evidently, $V\subset {\bbR}$.  From the lemma, for a given competitive equilibrium $(x,y)\in C$,
there exists an SPE $\sigma$ for which $x=\sigma^h_1,
y=\sigma^g_1$ if and only if there exist two values $(v_1,v_2)\in V\times V$
 such that
$$(1-\delta)\, r(x,y) + \delta v_1\geq (1-\delta)\, r(x,\eta) + \delta
v_2\quad \forall\ \eta\in Y . \EQN og6$$
Let $\sigma^1$ and $\sigma^2$ be subgame perfect equilibria for which $v_1 =
V_g(\sigma^1),\ v_2=V_g(\sigma^2)$.  The SPE $\sigma$
that supports $(x,y) = (\sigma^h_1,\sigma^g_1)$ is completed by specifying the continuation strategies
$\sigma\vert_{(x,y)} = \sigma^1$ and $\sigma\vert_{(\nu,\eta)} = \sigma^2$ if
$(\nu,\eta) \not= (x,y)$.

This construction uses  two continuation values $(v_1,v_2)\in V \times V$ to create an SPE
$\sigma$ with value $v\in V$ given by
$$v = (1-\delta)\, r(x,y) + \delta v_1\ .$$
Thus, the construction maps {\it pairs\/} of continuation values
 $(v_1,v_2)$ into a strategy profile $\sigma$  with
first-period competitive equilibrium outcome $(x,y)$
and a value $v=V_g(\sigma)$.

APS characterize subgame perfect equilibria by studying a mapping
from pairs of continuation
 values $(v_1,v_2)\in V \times V$ into values $v\in V$.  They use the following
definitions:
%define as {\it
%admissible\/} (with respect to $V$) a four tuple $(x,y,\, v_1,v_2)$ such that
%$(x,y)\in C,\ v_1,v_2\in V$, and $(x,y,v_1,v_2)$ satisfies the incentive
%constraint \Ep{og6}.  More generally they use

%\vfill\eject
\medskip\noindent
{\sc Definition 7:}  Let $W\subset  \bbR$.  A 4-tuple $(x,y,w_1,w_2)$ is said to be
{\it admissible with respect to\/} $W$ if $(x,y)\in C,\ (w_1,w_2) \in
    W\times W$, and
$$(1-\delta)\, r(x,y)
+\delta w_1 \geq (1-\delta)\, r(x,\eta) + \delta w_2\,,
\quad\forall\ \eta\in Y . \EQN og7$$

\medskip %\noindent
Notice that when $W\subset V$,
 the admissible 4-tuple $(x,y,w_1,w_2)$
determines an SPE with strategy profile
$$\sigma_1 = (x,y),\ \sigma\vert_{(x,y)} = \sigma^1,\ \sigma\vert_{(\nu,\eta)}
= \sigma^2\ \hbox{ for } (\nu,\eta)\not= (x,y)$$ where $\sigma_1$
is a continuation strategy that yields  value $w_1 = V_g
(\sigma^1)$ and $\sigma_2$ is a strategy that yields
continuation value $w_2 = V_g(\sigma^2)$.   The value of the
SPE is $V_g(\sigma) = w = (1-\delta)\, r(x,y)
 + \delta w_1$.

We want to find the set $V$.

\medskip
\subsection{Basic idea of dynamic programming squared}
 In Definition 7, $W$ serves as a set of candidate
continuation values. The idea is to pick an $(x,y)\in C$,  then to
check whether we can find  $(w_1, w_2)  \in W \times W $
that would make the government want to
adhere to the $y$ component if $w_1$ and  $w_2$ could be used as
continuation values for adhering to and deviating from $y$, respectively. If
the answer is yes, we say that the 4-tuple $(x,y,w_1,w_2)$ is
``admissible with  respect to $W$''.  Because we have
verified that the  incentive constraints are satisfied, a yes answer allows us
to  calculate the {\it value\/} (i.e., the left side of \Ep{og7})
that can be supported with $w_1, w_2$ as continuation values.
Thus, the idea is to use \Ep{og7} to define a mapping  from values
tomorrow to values today, like that used in dynamic programming.
In the next section, we'll define $B(W)$ as the set of possible
values attained with admissible pairs of continuation values drawn from
$W \times W$.  Then we'll view  $B$ as an operator mapping sets of continuation values $W$ into sets of values $B(W)$.
This operator is the counterpart to  the
$T$ operator associated with ordinary dynamic
programming.  % from chapters \use{XXXX}.

To pursue this analogy, recall the Bellman
equation associated with the  McCall model
of chapter \use{search1}:
$$ Q = \int  \max \left\{ {w \over 1 -\beta} , c + \beta Q \right\}
 d \, F(w). $$
Here $Q \in {\bbR}$  is the expected discounted value of an unemployed  worker's
income {\it before\/} he has drawn a wage offer.
The right side defines an operator $T(Q)$, so
that the Bellman equation is $$Q  = T(Q). \EQN mccallbell $$ This equation
can be
 solved by iterating to convergence starting from any initial $Q$.

  Just as the right  side of \Ep{mccallbell} takes a candidate continuation value
 $Q$ for tomorrow
and maps it  into a value $T(Q)$ for today, APS define a mapping
$B(W)$ that, by considering only admissible 4-tuples,  maps a
{\it set\/} of values $W$ tomorrow into a new {\it set\/} $B(W)$ of
values today.
Thus, APS use
admissible 4-tuples to map candidate continuation values tomorrow  into
new candidate values today.
In the next section, we'll iterate to convergence on $B(W)$,
 but as we'll see,
it won't work to start from just any initial set $W$. We have to start
from a big enough set.

\section{APS machinery}

\medskip\noindent
{\sc Definition 8:}  For each set $W\subset \bbR$, let $B(W)$ be the set of
possible values $w=(1-\delta)\, r(x,y) + \delta w_1$ associated with
admissible tuples $(x,y,w_1,w_2)$.

\medskip
 Think of
$W$ as a set of potential continuation values and $B(W)$ as the
set of values that they support. From the
definition of admissibility it immediately follows that the operator $B$ is
{\it monotone}.
\medskip\noindent
{\sc Property} (monotonicity of $B$):  If $W\subseteq W^\prime \subseteq R$,
then $B(W) \subseteq B(W^\prime)$.

\medskip
\noindent{\sc Proof:}
It can be verified directly from the definition of admissible 4-tuples that
if $w\in B(W)$, then $w \in B(W^\prime)$:  simply use the $(w_1,w_2)$
pair that supports $w\in B(W)$ to support $w\in B(W^\prime)$.
\qed

\medskip
It can also be verified that $B(\cdot)$ maps compact sets $W$ into compact
sets $B(W)$.

The self-supporting character of subgame perfect equilibria is referred to in
the following definition:
\medskip\noindent
{\sc Definition 9:}  The set $W$ is said to be {\it self-generating\/} if
$W\subseteq B(W)$.  \index{self-generation}
\medskip
Thus, a set of continuation values  $W$ is said to be self-generating if it is contained
in the set of {\it values\/} $B(W)$
that are generated by pairs of continuation values selected from $W$. This description makes us suspect that if a set of
values is self-generating, it must be a set of SPE
values.    Indeed, notice that
by virtue of the lemma, the set $V$ of SPE values $V_g (\sigma)$ is
self-generating. Thus, we can write
$V\subseteq B(V)$.  APS
show that $V$ is the {\it largest\/} self-generating set.  The key to showing
this point is the following theorem:\NFootnote{The {\it unbounded\/} set
$\bbR$ (the extended real line) is self-generating but not meaningful.
It is self-generating because any value $v \in \bbR$ can be supported
if there are no limits on the continuation values.  It is not
meaningful because most points in $\bbR$ are values that  cannot
be attained with {\it any} strategy profile.}
\medskip\noindent
{\sc Theorem 1} (A self-generating set is a subset of $V$): If $W\subset \bbR$ is bounded and
self-generating, then $B(W)\subseteq V$.
\index{self-generation}

\medskip\noindent
The proof is based on ``forward induction'' and
proceeds  by taking a point $w\in  B(W)$ and
constructing an SPE with value $w$.

\medskip\noindent
{\sc Proof:}  Assume $W\subseteq B(W)$.  Choose an element
$w\in B(W)$ and transform it as follows into a subgame perfect
equilibrium:

{\it Step 1.}  Because $w\in B(W)$, we know that there exist outcomes
$(x,y)$ and values $w_1$ and $w_2$ that satisfy
$$\eqalign{w= (1-\delta)\, &r(x,y) + \delta w_1\geq (1-\delta)\, r(x,\eta)
+ \delta w_2\quad \forall \eta\in Y\cr
(x,y) &\in C\cr
w_1,w_2 &\in W\times W .\cr}$$
Set $\sigma_1 = (x,y)$.

{\it Step 2.}  Since $w_1\in W\subseteq B(W)$, there exist outcomes $(\tilde
x,\tilde y)$ and values $(\tilde w_1,\tilde w_2) \in W$ that satisfy
$$\eqalign{w_1 = (1-\delta)&\, r(\tilde x,\tilde y) + \delta
\tilde w_1\geq (1-\delta)\, r(\tilde x,\eta) + \delta \tilde w_2,\quad
\forall\ \eta \in Y\cr
(\tilde x,\tilde y) &\in C .\cr}$$
Set the first-period outcome in period 2 (the outcome to occur
{\it given\/}
that $y$ was chosen in period 1) equal to $(\tilde x,\tilde y)$; that is,
set $(\sigma\vert_{(x,y)})_1 = (\tilde x, \tilde y)$.

Continuing in this way, for each $w\in B(W)$, we can create a sequence of
continuation values $w_1,\tilde w_1, \tilde {\tilde w}_1,\ldots$ and a
corresponding sequence of first-period outcomes
 $(x,y),\, (\tilde x,\tilde y),\,
(\tilde {\tilde x}, \tilde{\tilde y})$.

At each stage in this construction, policies are {\it unimprovable}, which
means that given the continuation values, one-period deviations from the
prescribed policies are not optimal.  It follows that the strategy profile is
optimal.  By construction $V_g(\sigma) = w$.
\qed
\index{policy improvement algorithm}
\medskip
Collecting results, we know that
\medskip
\noindent{1.} $V\subseteq B(V)$ (by the lemma).
\smallskip
\noindent{2.} If $W\subseteq B(W)$, then $B(W) \subseteq V$ (by theorem 1).
\smallskip
\noindent{3.}  $B$ is monotone and maps compact sets into compact sets.
\medskip
Facts 1 and 2 imply that $V=B(V)$, so that the set of
equilibrium values is a ``fixed point''
 of $B$, in particular, the {\it largest\/} bounded fixed
 point.


  Monotonicity of $B$ and the
fact that it maps compact sets into compact sets provides an
algorithm for computing the set $V,$ namely, to start with a set
$W_0$ for which $V \subseteq B(W_0)  \subseteq W_0$,  and to iterate
 on $B$. In more detail, we use
the following steps:
\medskip

\noindent{1.}  Start with a set $W_0 = [\underline{w}_0,\overline w_0]$ that
we know is bigger than $V$, and for which
$B(W_0) \subseteq W_0$.  It will always work to set $\overline w_0 =
\max_{(x,y)\in C}
 r(x,y)$, $\underline{w}_0=\min_{(x,y)\in C} \, r(x,y)$.
\medskip

\noindent{2.}  Compute the boundaries of the set $B(W_0) = [\underline{w}_1,
\overline w_1]$.  The value $\overline w_1$ solves the problem
$$\overline w_1 = \max_{(x,y)\in C}\, (1-\delta)\, r(x,y) + \delta
\overline w_0$$
subject to
$$(1-\delta)\, r(x,y) + \delta \overline w_0\geq (1-\delta)\, r(x,\eta) +
\delta \underline{w}_0\ \hbox{ for all }\ \eta \in Y.$$
The value $\underline{w}_1$ solves the problem
$$\underline{w}_1 = \min_{(x,y)\in C;\, (w_1,w_2)\in
[\underline{w}_0,\overline w_0]^2}\
(1-\delta)\, r(x,y) + \delta w_1$$
subject to
$$(1-\delta)\, r(x,y) + \delta w_1 \geq (1-\delta)\, r(x,\eta) + \delta
w_2\quad\forall\ \eta\in Y.$$
With $(\underline w_0, \overline w_0)$ chosen as before, it will be
true that $B(W_0) \subseteq W_0$.
\medskip

\noindent{3.}  Having constructed $W_1 = B (W_0) \subseteq W_0$, continue to
iterate, producing a decreasing sequence of compact sets $W_{j+1} = B (W_j)
\subseteq W_j$.  Iterate until the sets converge.


\medskip
In section \use{sec:best_worst}, we will present a direct way to compute the best and worst
SPE values, one that evades  having to iterate  on
the $B$ operator.
\index{subgame perfect equilibrium}
\index{self-enforcing equilibrium}
\section{Self-enforcing SPE}
A subgame perfect equilibrium with a {\it worst\/} value
$v\in V$ has  the remarkable property that it is ``self-enforcing.''
We use the following definition:
\medskip\noindent
{\sc Definition 10:}  A subgame perfect equilibrium $\sigma$ with
first-period outcome $(\tilde x,\tilde y)\in C$ is said to be
{\it self-enforcing\/} if
$$\sigma\vert_{(x,y)} = \sigma\qquad \hbox{if }\ (x,y) \not=(\tilde x,\tilde
y) .\EQN og8$$

\noindent
A strategy profile satisfying equation \Ep{og8} is called self-enforcing
because after a one-shot deviation the consequence is
simply to restart the equilibrium.
\medskip

Recall our earlier characterization
of a competitive equilibrium as a pair $(h(y), y)$,
where $x=h(y)$ is the mapping from the government's
action $y$ to the private sector's equilibrium response.
The value $\underline v$ associated with the worst
subgame perfect equilibrium $\sigma$ satisfies
$$\underline{v} = \min_{y,v}\, \bigl\{ (1-\delta)\, r(h(y),y) + \delta
v\bigr\} = (1-\delta) r(h(\tilde y), \tilde y) + \delta \tilde v , \EQN og9$$
where the minimization is subject to $y\in Y$, $v\in V$, and the
incentive constraint
$$(1-\delta)\, r(h(y),y) + \delta v\geq (1-\delta)\, r(h(y),\eta) +
\delta \underline{v}\quad \hbox{for all }\ \eta\in Y .\EQN og10$$
Let $\tilde v$ be  a continuation value that attains the right
side of equation \Ep{og9}, and let $\sigma_{\tilde v}$ be a subgame
perfect equilibrium that supports continuation value $\tilde v$.  Let
$(\tilde x,\tilde y)$ be the first-period outcome that attains
the right side of equation \Ep{og9}. Thus,
$\underline{v} = (1-\delta) r( \tilde x, \tilde y) + \delta \tilde v$.  Since $\underline v$ is both the
continuation value when first-period outcome $(x,y)\not= (\tilde x,
\tilde y)$ {\it and\/} the value associated with subgame perfect
equilibrium $\sigma$, it follows that
%$$\eqalign{\sigma_1 &= (\tilde x,\tilde y)\cr
%\sigma\vert_{(x,y)} &= \cases{\sigma & if \ $(x,y)\not= (\tilde x,\tilde y)$\cr
%\sigma_{\tilde v} & if \ $(x,y) = (\tilde x,\tilde y)$.\cr} \cr}\EQN og11$$
$$\eqalign{\sigma_1 &= (\tilde x,\tilde y)\cr
\sigma\vert_{(x,y)} &= \cases{\sigma_{\tilde v} & if \ $(x,y) = (\tilde x,\tilde y)$ \cr
   \sigma & if \ $(x,y)\not= (\tilde x,\tilde y).$\cr } \cr}\EQN og11$$
Because of the double role played by $\underline v$, i.e., $\underline v$ is
both the value of equilibrium $\sigma$ and the ``punishment'' continuation
value of the right side of the incentive constraint \Ep{og10}, an equilibrium strategy $\sigma$ that supports $\underline v$ is
self-enforcing.\NFootnote{As we show below,
the structure of the programming problem,
with the double role played by $\underline v$, makes it possible
to compute the worst value directly.}

The preceding argument thus establishes
\medskip\noindent
{\sc Proposition 2:}  A subgame perfect equilibrium $\sigma$
associated with $\underline{v} = \min\{v: v\in V\}$ is self-enforcing.
\medskip


\subsection{The quest for something worse than repetition of Nash outcome}
Notice that the first subgame perfect equilibrium that we computed,
whose outcome was infinite repetition of the one-period Nash
equilibrium, is a self-enforcing equilibrium.  However, in general,
the infinite repetition of the one-period Nash equilibrium is not
the {\it worst\/} subgame perfect equilibrium.
This fact opens  the possibility that even when reversion
to Nash after a deviation is {\it not\/} able to support repetition
of Ramsey as an SPE, we might still
support repetition of the Ramsey
outcome by reverting to a SPE with
a value worse than that associated with repetition of the
 Nash outcome whenever the government deviates from an expected one-period choice.




\section{Recursive strategies}
This section emphasizes similarities between
credible government policies and the recursive
contracts appearing in chapter \use{socialinsurance}.
We will study situations where
 the household's and the  government's strategies  have  recursive representations.
   This
approach substantially restricts the space of strategies
because most history-dependent strategies cannot be
represented recursively. Nevertheless, this class of
strategies excludes no equilibrium payoffs $v\in V$.
We use the following definitions:
\index{credible government policies}
\index{sustainable plans!|see{credible government policies}}

\medskip\noindent{\sc Definition 11:} Households and the
government follow {\it recursive strategies}
if there is a 3-tuple of functions
$\phi = (z^h,z^g, {\cal V})$ and an initial
condition $v_1$ with the following structure:
$$\eqalign{ v_1 & \in \bbR \  \hbox{is given} \cr
  x_t & = z^h(v_t)              \cr
  y_t & = z^g (v_t)  \cr
   v_{t+1} &= {\cal V}(v_t,x_t,y_t),} \EQN old6 $$
where $v_t$ is a
state variable designed to summarize the history of
outcomes before $t$.

\index{history dependence!|see{reputation}}
\index{history dependence!of strategies}

\medskip
This recursive form of strategies operates much
like an autoregression to let time $t$ actions $(x_t,y_t)$ depend on the
history $\{y_s, x_s\}_{s=1}^{t-1}$, as mediated through the state
variable $v_t$.  Representation \Ep{old6} induces history-dependent
government policies, and thereby allows for reputation.  We shall soon
see that beyond its role in keeping track of histories, $v_t$ also
summarizes the  future.\NFootnote{By iterating equations
 \Ep{old6}, we can construct a pair of sequences of functions
indexed by $t \geq 1$
$\{Z^h_{t}(I_t),Z^g_{t}(I_t)\}$, mapping histories that are augmented
by initial conditions
 $I_t=(\{x_s,y_s\}_{s=1}^{t-1},v_1)$ into time $t$ actions
$(x_t,y_t) \in X \times Y$.  Strategies for
the repeated economy are a pair of sequences of such functions
without the restriction that they have a recursive representation.}

\medskip

 A strategy $(\phi, v)$ recursively
generates an outcome path expressed  as $ (\vec{x},\vec{y}) =
(\vec{x},\vec{y}) (\phi, v)$.  By substituting the outcome path
into equation \Ep{og2}, we find that $(\phi,v)$ induces
a value for the government, which we write as
$$\EQNalign{
V^g \bigl[(\vec{x},\vec{y}) (\phi,v)\bigr]
= & \, (1-\delta) \,r\bigl[z^h(v),z^g(v)\bigr]  \cr
& + \delta \, V^g \Bigl((\vec x, \vec y)\bigl\{\phi,
    {\cal V}[v,z^h(v),z^g(v)]\bigr\} \Bigr). \EQN value1 \cr}$$
So far, we have not interpreted the state variable $v$, except
as a particular measure of the history of outcomes.
The theory of credible policy ties past and future together
by making  the state variable $v$
a  promised value, an outcome to be expressed
$$v= V^g \bigl[(\vec{x},\vec{y}) (\phi,v)\bigr] . \EQN fixedpt$$
Equations \Ep{old6}, \Ep{value1},
 and \Ep{fixedpt}  assert a dual role for
$v$. In equation \Ep{old6}, $v$  accounts for past outcomes.
In equations \Ep{value1} and \Ep{fixedpt}, $v$ looks
forward. The state $v_t$
is a discounted future value with which the government enters time $t$
based on past outcomes.
Depending on the outcome $(x,y)$ and the entering promised value $v$,
${\cal V}$ updates the  promised value
with which the government leaves the period.
    In section \use{sec:spe_interpretations}, we shall struggle with
which of two  valid interpretations of the government's
 strategy should be emphasized:
something chosen  by
the government, or
a description of a system of public expectations to which
the government conforms.

Evidently, we have the following:
\medskip
\noindent{\sc Definition 12:} Let $V$ be the set of SPE values.
A recursive strategy $(\phi,v)$
in equation \Ep{old6}
is a {\it subgame perfect equilibrium\/} (SPE) if and only if
$v \in V$ and
\itemitem{(1)}  The outcome $x=z^h(v)$ is a competitive equilibrium,
given that $y=z^g(v)$.
\itemitem{(2)} For each $\eta \in Y$,
${\cal V}(v, z^h(v), \eta)  \in V $.
\itemitem{(3)}  For each $\eta \in Y$,
$$ \eqalign{ v & = (1-\delta) r \bigl[z^h(v),z^g(v)\bigr]
 + \delta {\cal V} \bigl[v,z^h(v),z^g(v)\bigr] \cr
 & \geq (1-\delta) r\bigl[z^h(v),\eta\bigr] + \delta
   {\cal V}\bigl[v, z^h(v), \eta\bigr].     \cr} \EQN defsub $$


Condition (1) asserts that the first-period outcome pair $(x,y)$ is a
competitive equilibrium. Each member of the private
 sector forms an expectation about the government's
action according to  $ y_t = z^g(v_t)$, and the ``market'' responds with
a competitive equilibrium $x_t$,
$$
x_t = h(y_t) = h\bigl[z^g(v_t)\bigr] \equiv z^h(v_t).  \EQN old7
$$
This construction
builds in rational expectations, because
the private sector knows both the state
variable $v_t$ and the government's decision rule $z^g$.


\medskip
Besides the first-period outcome $(x,y)$, conditions (2) and (3)
associate with a subgame perfect
equilibrium three additional objects: a promised value $v$,
 a continuation value $v' = {\cal V}[v, z^h(v),z^g(v)]$
 if the required first-period
 outcome  is observed, and another continuation value
$\tilde v(\eta) = {\cal V}[v,z^h(v),\eta]$
 if the required first-period outcome
is not observed but rather some pair $(x,\eta)$.
All of the continuation values must themselves
be attained as subgame perfect equilibria.
 In terms of these objects, condition (3)
is an incentive constraint inspiring the government
to adhere to the equilibrium
$$ \eqalign{ v & = (1-\delta) r(x, y) + \delta v' \cr
  & \geq (1 - \delta) r(x, \eta) + \delta \tilde v(\eta),
\ \ \forall \eta \in Y. \cr}$$
This formula states that the government receives more if it adheres
to an action called for by its strategy than if it
departs.  To ensure that these values constitute ``credible expectations,''
part (2) of Definition 12 requires that the continuation
values be values for subgame perfect equilibria.  The definition is
circular because members of the same class of objects, namely, equilibrium
values $v$, occur on each side of expression \Ep{defsub}.
Circularity comes with recursivity.

\auth{Abreu, Dilip}  \auth{Pearce, David} \auth{Stacchetti, Ennio}
 One implication of the work of APS (1986, 1990) is that recursive
equilibria of form \Ep{old6} can attain {\it all\/} subgame perfect
equilibrium values.  As we have seen, APS's innovation was
to shift the focus away from the set of equilibrium strategies
and toward the set of values $V$ attainable with subgame perfect
equilibrium strategies.

\section{Examples of SPE with recursive strategies}
Our two earlier examples of subgame perfect equilibria were already of
a recursive nature. But to highlight this property, we recast those
SPE in the present notation for recursive strategies.
Equilibria are constructed by using a guess-and-verify
technique. First, guess  $(v_1,z^h,z^g,{\cal V})$ in equations \Ep{old6}, then
verify parts (1), (2), and (3) of Definition 12.

The examples parallel the historical development of the theory.
(1) The first example is infinite repetition of a  one-period Nash
outcome, which was Kydland and Prescott's (1977) time-consistent
equilibrium. (2) Barro and Gordon (1983a, 1983b) and Stokey (1989)
used the value from infinite repetition of the Nash outcome as a
continuation value to deter deviation from the Ramsey outcome.
For sufficiently high discount factors, the continuation value
associated with repetition of the Nash outcome can deter the
government from deviating from infinite repetition of the Ramsey
outcome.  This is not possible for low discount factors.
\auth{Abreu, Dilip}%
 (3)  Abreu (1988) and Stokey (1991)  showed
that Abreu's ``stick-and-carrot'' strategy induces more severe
consequences than repetition of the Nash outcome. \auth{Barro,
Robert J.} \auth{Gordon, David B.} \auth{Stokey, Nancy L.}
\index{stick and carrot}
%%%%%%
\subsection{Infinite repetition of Nash outcome}
 It is easy to construct an equilibrium whose outcome path
forever repeats the one-period Nash  outcome.
Let $v^N=r(x^N,y^N)$. The proposed equilibrium is
$$\eqalign{v_1 &= v^N , \cr
 z^h(v)&  = x^N \ \forall \ v, \cr
 z^g(v)&  = y^N \ \forall \ v, \ \hbox{and} \cr
 {\cal V}(v, x, y) & = v^N, \ \forall \
(v, x, y).\cr}$$
Here  $v^N$ plays  the roles of all {\it three\/} values
in
condition (3) of Definition 12. Conditions (1) and (2) are satisfied by
construction, and condition (3) collapses to
$$r(x^N, y^N) \geq r\bigl[x^N, H(x^N)\bigr],$$
which is satisfied at equality by the definition of a best
response function.


\subsection{Infinite repetition of a better-than-Nash outcome}
Let $v^b$ be a value associated with
outcome $(x^b,y^b)$ such that $v^b = r(x^b, y^b) > v^N$,
and assume that $(x^b,y^b)$ constitutes a competitive equilibrium.
Suppose further that
$$r\bigl[x^b, H(x^b)\bigr] - r(x^b, y^b) \leq {\delta\over 1-\delta}
(v^b- v^N).  \EQN ** $$
The left side is the one-period return  to the
government from deviating from $y^b$; it is the gain from
deviating. The right side is the
difference in present values associated with conforming to the plan
versus reverting forever to the Nash equilibrium; it is the cost
of deviating.
%present
%value of the difference in promised values associated with
%adhering and deviating.
 When the inequality is satisfied, the
equilibrium presents the government with an incentive not to
deviate from $y^b$.
Then an SPE is
$$ \eqalign{ v_1 & = v^b   \cr
         z^h(v)& = \cases{x^b  &  if  $v=v^b$; \cr
                           x^N   &  otherwise; \cr} \cr
         z^g(v)& = \cases{y^b  &  if  $v=v^b$; \cr
                              y^N   &  otherwise; \cr} \cr
     {\cal V}(v,x,y)& = \cases{ v^b  & if $(v,x,y) = (v^b, x^b, y^b)$; \cr
                               v^N &   otherwise. \cr}  \cr} $$
%$$\eqalign{v_1 & = v^b \cr
% \sigma_1^g (v^b) &= y^b\cr \sigma_2 (v^b, y^b, y^b) &= v^b\cr
%\sigma_1 (v^N) &= y^N  \cr
%\sigma_2 (v,x,y) &= v^N \quad \forall\quad (x,y) \not=
%(y^b, y^b)\cr}$$
This strategy specifies outcome $(x^b, y^b)$ and continuation value
$v^b$ as long as $v^b$ is the value promised at
the beginning of the period.
Any deviation from $y^b$ generates continuation
value $v^N$. Inequality \Ep{**} validates condition (3) of Definition 12.

Barro and Gordon (1983a) considered a version of this equilibrium in which
inequality \Ep{**} is satisfied with $(v^b, x^b, y^b) = (v^R,x^R,y^R)$.
In this case, anticipated reversion to Nash  supports
the Ramsey outcome forever. When inequality \Ep{**}
is {\it not\/} satisfied for $(v^b, x^b, y^b) = (v^R,x^R,y^R)$,
 %the
%incentive constraint \Ep{**} at equality determines the best
%SPE value supportable by infinite reversion to Nash.
% Let
%$\tilde y$ satisfy
we can solve for the best SPE value $v^b$, with associated actions $(x^b,y^b)$,
 supportable by infinite
reversion to Nash  from
$$ v^b = r(x^b, y^b) = (1-\delta) r\bigl[x^b, H(x^b)\bigr]
+ \delta v^N
> v^N.  \EQN best $$
The payoff from following the strategy equals that from deviating
and reverting to Nash.  Any value lower than this can be supported,
but none higher.

%Then we can solve for $\tilde y$ from
%$r(\tilde y, B(\tilde y)) = v^N$; and
%  $v^b = r(\tilde y,\tilde y),\ y^b=\tilde y$.


When $v^b < v^R$,
Abreu (1988) searched for a way to support something
better than $v^b$.
First, one must  construct an equilibrium that yields a
value {\it worse\/} than permanent repetition of the Nash outcome.
The expectation of reverting to this equilibrium
supports  something better
than $v^b$ in equation \Ep{best}.

Somehow the government must be induced temporarily to
take an action $y^\#$ that yields a worse period-by-period
return than
the Nash outcome, meaning that the government in general
would be tempted to deviate.
An equilibrium system of expectations
has to be constructed that makes the government expect to do
better in the future only by conforming to expectations that it
temporarily adheres to the bad policy $y^\#$.

\index{stick and carrot}
\subsection{Something worse: a stick-and-carrot strategy}
To get something worse than repetition of the one-period Nash outcome,
Abreu (1988) proposed a ``stick-and-carrot punishment.''
The ``stick'' part is an outcome $(x^\#,y^\#)\in C$, which
relative to $(x^N, y^N)$
is a bad competitive equilibrium from the government's viewpoint.  The
``carrot'' part is the Ramsey outcome $(x^R,y^R)$, which the
government attains forever after it has accepted the stick in the
first period of its punishment.   \auth{Abreu, Dilip}

 We want
a continuation value $v^*$ for deviating to support the first-period
 outcome $(x^\#, y^\#)\in C$ and attain the value\NFootnote{This is a ``one-period
stick.''  The worst SPE  can require more than one period of
a worse-than-one-period Nash outcome.}
$$\tilde v = (1-\delta) r(x^\#, y^\#) + \delta\ v^R
\geq (1-\delta) r\bigl[ x^\#, H(x^\#)\bigr] + \delta \ v^* . \EQN ++ $$
Abreu proposed to set $v^* = \tilde v$ so that the
continuation value from deviating from the first-period
action equals the original value.  If the stick part is severe
enough, the associated strategy
attains a
value worse than infinite repetition of Nash.
%  by
% promising a continuation value that is better.
The strategy induces the government to accept
the temporarily bad
outcome by promising a high continuation value.

An SPE featuring stick-and-carrot punishments that attains $\tilde v$ is
$$ \eqalign{ v_1 & = \tilde v \cr
             z^h(v) & = \cases{ x^R & if $v= v^R$; \cr
                                     x^\# & otherwise; \cr} \cr
             z^g(v) & = \cases{ y^R & if $v= v^R$; \cr
                                     y^\# & otherwise; \cr} \cr
             {\cal V}(v,x,y) & =
                    \cases{ v^R & if $(x,y)=[z^h(v), z^g(v)]$ ; \cr
                            \tilde v & otherwise. \cr}
                      \cr} \EQN strategy2  $$
When the government deviates from the bad
prescribed first-period action $y^\#$, the consequence  is to
 restart  the equilibrium.  This means that  the equilibrium
is self-enforcing.


\section{Best and  worst SPE values}\label{sec:best_worst}%
The value associated with
Abreu's stick-and-carrot  strategy might still not  be bad enough
to  deter the government from deviating from
repetition of the Ramsey outcome.   We are therefore interested in
finding the worst SPE value.
 We  now display a pair of simple programming problems to find the
best and worst SPE values.
   APS (1990) showed how to find the entire set of equilibrium values
$V$.   In the current setting, their ideas imply the following:

\medskip
\noindent{1.}  The set of equilibrium values $V$  attainable by
the government is a compact subset $[\underline v, \overline v]$
of $[{\rm min}_{(x,y)\in C}\,r(x,y),\,r(x^R,y^R)]$.

\medskip
\noindent{2.} The worst equilibrium value $\underline v$ can be computed from
a simple programming problem.
\medskip
\noindent{3.}  Given the worst equilibrium value $\underline v$, the
best equilibrium value $\overline v$ can be computed from a programming
problem.
\medskip
\noindent{4.}  Given a $v \in [\underline v, \overline v]$, it is easy
to construct an equilibrium  that attains it.

\medskip

Recall from Proposition 2 that the worst equilibrium is self-enforcing,
and here we repeat
versions of equations \Ep{og9} and \Ep{og10},
$$\underline{v} = \min_{y\in Y,\,v_1\in V}\, \bigl\{ (1-\delta)\, r
\bigl[h(y),y\bigr] + \delta
v_1\bigr\}\EQN og9*$$
where the minimization is subject to the
incentive constraint
$$(1-\delta)\, r[h(y),y] + \delta v_1\geq (1-\delta)\, r\bigl\{ h(y),H [h(y)]
\bigr\} + \delta \underline{v} .\EQN og10*$$
In expression \Ep{og10*},
 we use the worst SPE as the continuation value in the event of
a deviation. The minimum will be attained when the constraint is binding,
which implies that
$\underline{v} = r\{h(y), H[h(y)]\}$
for some government action $y$.\NFootnote{An equivalent way to express $\underline v$
is $\underline v = \min_{y \in Y} \max_{\eta \in Y} r(h(y), \eta) $.}
Thus, the problem of finding the worst SPE reduces to solving
$$\underline{v} = \min_{y\in Y} r\bigl\{ h(y), H[h(y)]\bigr\},$$
then computing $v_1$ from $(1-\delta) r[h(\underline y), \underline y]
+ \delta v_1 =
\underline{v}$, where $\underline y = \arg\min r\{h(y)$,
$H[h(y)]\}$, and finally checking that $v_1$ is itself
a value associated with an SPE.
 To check this condition,
we need to know $\overline v$.


The computation of $\overline v$ utilizes the
fact that the best SPE is self-rewarding; that is, the best SPE
has continuation value $\overline v$ when the government follows
the prescribed equilibrium strategy.
Thus, after we have computed a candidate for
the worst SPE  value $\underline v$, we can compute
a candidate for
 the {\it best\/} value $\overline v$ by solving the programming problem

$$\eqalign{\overline v &= \max_{y\in Y} \ r\bigl[h(y),y\bigr]\cr
\hbox{subject to}\quad r\bigl[h(y),y\bigr] &\geq (1-\delta)
r\bigl\{h(y), H[h(y)]\bigr\} + \delta \underline{v}.
\cr}$$
%
Here we are using the fact that $\overline v$ is the maximum
continuation value available to
reward adherence to the policy, so that $\overline v =
(1-\delta) r[h(y),y] + \delta \overline v$. Let $y^b$ be the
maximizing value of $y$.
Once we have computed $\overline v$, we can check that the
continuation value $v_1$ for supporting the worst value is within
our candidate set
 $[\underline v, \overline v]$.  If it is, we have succeeded
in constructing $V$.

\subsection{When $v_1$ is outside the candidate set}
If our candidate  $v_1$ is not within our candidate
set $[\underline v, \overline v]$, we have to seek
a smaller set.     We could find this set  by pursuing
the following line of reasoning.  We know that
$$ \underline v = r\left\{ h(\underline  y), H[ h(\underline y)] \right\}
   \EQN larsworst1  $$
for {\it some\/} $\underline y$, and that for  $\underline y$
the continuation value $v_1$ satisfies
$$ (1-\delta) r[h(\underline y), \underline y] + \delta v_1
  = (1-\delta) r\left\{ h(\underline y), H[h(\underline y)]\right\}
   + \delta \underline v .$$
Solving this equation for $v_1$ gives
$$ v_1 = {1-\delta \over \delta} \Bigl( r\left\{ h(\underline y),
  H[h(\underline y)] \right\} - r[h(\underline y), \underline y]\Bigr)
  + r\left\{ h(\underline y), H[h(\underline y)]\right\}
  \EQN larsworst2 $$

The term in large parentheses on the right measures the one-period
temptation to deviate from $\underline y$.  It is multiplied
by ${1-\delta \over \delta}$, which approaches $+\infty$ as
$\delta \searrow 0$.  Therefore, as $\delta \searrow 0$,
it is necessary that the term in braces approach $0$,
which means that the required $\underline y$ must  approach
$y^N$.

  For discount factors that are so small  that $v_1$
is outside the region of values proposed in
the previous subsection because the implied $v_1$ exceeds
the candidate $\overline v$,  we can proceed in the spirit of
Abreu's stick-and-carrot policy, but instead of using
$v^R$ as the continuation value to reward adherence
(because that is too much to hope for here), we can
simply reward adherence to the worst with $\overline v$, which
we must solve for.
Using $ \overline v = v_1$ as the continuation value
for adherence to the worst leads to the following four
equations to be solved for $\overline v, \underline v,
\overline y, \underline y$:
$$\EQNalign{
\underline v  = & r\left\{ h(\underline  y), H[ h(\underline y)] \right\}
   \EQN larsworst1 \cr
 \overline v
  = & {1-\delta \over \delta} \Bigl( r\left\{ h(\underline y),
  H[h(\underline y)] \right\} - r[h(\underline y), \underline y]\Bigr)
   \cr &+ r\left\{ h(\underline y), H[h(\underline y)]\right\}
\EQN larsworst3   \cr
  \overline v  = & r[h(\overline y), \overline y] \EQN larsworst4 \cr
  \overline v  = & (1-\delta) r\left\{ h(\overline y), H[h(\overline y)]
  \right\} + \delta \underline v. \EQN larsworst5 \cr}$$
In exercise {\it \the\chapternum.3\/}, we ask the reader to solve these equations for
a particular example.

%%%%%%%%%%%%%%%%%%%
%$$
%\grafone{nasheq4.eps,height=3in}{{\bf Figure 6.} Calculation of
%the  worst subgame equilibrium value $\underline v$.}
%$$
%%%%%%%%%%%%%%%%%%

%\midfigure{nasheq4f}
%\centerline{\epsfxsize=3truein\epsffile{nasheq4.eps}}
%\caption{Calculation of the worst subgame equilibrium value $\underline v$.}
%\infiglist{nasheq4f}
%\endfigure

\section{Examples: alternative ways to achieve the worst}\label{sec:altworst}%
We return to the situation envisioned before the last subsection,
so that the candidate $v_1$ belongs to the required candidate
set $[\underline v, \overline v]$. We describe examples
of some equilibria that attain value $\underline v$.


\subsection{Attaining the worst, method 1}
We have seen that to evaluate the best sustainable value
$\overline v$, we want to find the
worst value $\underline v$.  Many SPEs attain the worst
value $\underline v$.  To compute one such SPE strategy,
we can use the following recursive procedure:

\smallskip
\noindent{1.} Set the first-period promised value
$v_0 =\underline{v} = r \{ h(y^\#), H [h(y^\#)] \}$,
where $y^\# = \arg\min r\{ h(y),H [h(y)]\}$.
The competitive equilibrium with the worst one-period
value gives value
$r [h(y^\#), y^\#]$. Given expectations $x^\#=h(y^\#)$, the government
is tempted toward $H(x^\#)$, which yields one-period
utility to the government of
$r \{ h(y^\#), H [h(y^\#)]\}$.
Then use $\underline v$ as continuation value in the
event of a deviation, and construct an increasing sequence of
 continuation values to reward adherence, as follows:
\smallskip
\noindent{2.} Solve
$\underline{v} = (1-\delta) r [h(y^\#),y^\#] + \delta v_2$
for continuation value $v_1$.

\smallskip
\noindent{3.} For $j=1,2,\cdots$, continue solving
$v_j = (1-\delta) r[h(y^\#), y^\#] + \delta v_{j+1}$
for the continuation values $v_{j+1}$ as long as $v_{j+1} \leq \overline v$. If
$v_{j+1}$ threatens to violate this constraint at step $j =
\overline j$, then go to step 4.

\smallskip
\noindent{4.} Use $\overline v$ as the continuation value, and solve
$v_j = (1-\delta) r[h(\tilde y), \tilde y] + \delta \overline v$
for the prescription $\tilde y$ to be followed if promised value
$v_j$ is encountered.

\smallskip
\noindent{5.} Set $v_{j+s} = \overline v$ for $s\geq 1$.

\subsection{Attaining the worst, method 2}
To construct another equilibrium supporting
the worst SPE value, follow steps
1 and 2, and follow step 3 also, except that
we continue solving
$v_j = (1-\delta) r [ h(y^\#), y^\#] + \delta v_{j+1}$
for the continuation values $v_{j+1}$ only so long as $v_{j+1} <  v^N$.
As soon as $v_{j+1} = v^{**} > v^N$, we use $v^{**}$ as both the
promised value and
the continuation value thereafter.  In terms of our recursive
strategy notation, whenever $v^{**}
= r [h(y^{**}),y^{**}]$ is
the promised value, $z^h(v^{**})= h(y^{**})$, $z^g(v^{**})= y^{**}$,
and $v'[v^{**}, z^h(v^{**}), z^g(v^{**})]=v^{**}$.

\subsection{Attaining the worst, method 3}
Here is another subgame perfect equilibrium that supports $\underline v$.
Proceed as in step 1 to find the initial continuation value $v_1$.  Now
set all subsequent values and  continuation values to $v_1$,
with associated first-period outcome $\tilde y$ that solves
$v_1 = r [h(\tilde y),\tilde y]$.  It can be checked that the
incentive constraint is satisfied with $\underline v$ the
continuation value in the event of a deviation.


 % If $\delta$ is close enough to $1$, then the value $0$ associated
%with infinite repetition of the Ramsey outcome
% is an SPE.
%So is the value associated with infinite repetition of the
%Nash equilibrium.


% A simple argument can be used to find the {\it worst} subgame
%perfect equilibrium. The calculation hinges on noting
%that the value associated with the worst equilibrium is used
%as a continuation value following one period deviations from
%the first-period of the worst equilibrium. The compactness
%of $Y$ plays a role in formulating the worst equilibrium value.


%%%%%%%%%%%Maria: the table is down below!!%%%%%%%%%%%%%%%%%%%%%%55


\subsection{Numerical example}
We now illustrate the concepts and arguments using the infinitely
repeated version of the taxation example. To make the problem of
finding $\underline v$ nontrivial, we impose an upper bound on
admissible tax rates given by $\overline \tau=1-\alpha-\epsilon$, where
$\epsilon\in (0,\, 0.5-\alpha)$. Given
$\tau\in Y \equiv [0, \overline \tau]$,
the model exhibits a unique Nash equilibrium with $\tau=0.5$.
For a sufficiently small $\epsilon$, the worst one-period
competitive equilibrium is $[\ell(\overline \tau), \overline \tau]$.

Set $\left[\matrix{\alpha & \delta & \overline \tau \cr}\right]=
\left[\matrix{0.3 & 0.8 & 0.6 \cr}\right]$.
 Compute
$$ \eqalign{& \left[\matrix{\tau^R & \tau^N \cr} \right] =
\left[\matrix{0.3013 &  0.5000\cr}\right], \cr
& \left[\matrix{ v^R & v^N & \underline v & v_{\rm abreu} \cr}\right]=
\left[\matrix{-0.6801 & -0.7863 & -0.9613 & -0.7370 \cr}\right] .
\cr} $$
In this numerical example, Abreu's ``stick-and-carrot'' strategy fails
to attain a value lower than the repeated Nash outcome. The reason
is that the upper bound on tax rates makes the least favorable
one-period return (the ``stick'') not so bad.

%%%%%%%%%%%
%\midinsert{
%$$\grafone{cred3.ps,height=2.5in}{{\bf Figure 16.6}  Continuation
%values (on coordinate axis) of two SPE that attain $\underline v$.}
%$$  }\endinsert
%%%%%%%%%%%%%%%%%%%%

\midfigure{cred3f}
\centerline{\epsfxsize=3truein\epsffile{cred3.ps}}
\caption{Continuation values (on coordinate axis) of two SPE that attain
$\underline v$.}
\infiglist{cred3f}
\endfigure


Figure \Fg{cred3f} describes two SPEs that attain the worst
SPE value  $\underline v$
with the depicted sequences of time $t$ (promised value, tax rate) pairs.
The circles represent the worst SPE attained with
method 1, and the x-marks correspond to method 2. By construction, the
continuation values of method 2 are less than or equal to the continuation
values of method 1. Since both SPEs attain the same promised value
$\underline v$, it follows that method 2 must be associated with higher
one-period returns in some periods. %Figure 16.7
Figure \Fg{cred5f} indicates that method 2
delivers those higher one-period returns around period 20 when the
prescribed tax rates are closer to the Ramsey outcome $\tau^R=0.3013$.


%$$
%\graftwo{cred3.ps,height=3in}{{\bf Figure 6a.}  Continuation
%values (on ordinate axis) of two SPE's that attain $\underline v$.}
%{cred5.ps,height=3in}{{\bf Figure 7b.}  Tax rates
%associated with those continuation values.}
%$$

When varying the discount factor, we find that the cutoff
value of $\delta$ below which reversion to Nash fails
to support Ramsey forever is $0.2194$.

%%%%%%%%%%%%
%$$\grafone{cred5.ps,height=2.5in}{{\bf Figure 16.7}  Tax rates
%associated with the continuation values of Figure 16.6.}$$
%%%%%%%%%%%%%%

\midfigure{cred5f}
\centerline{\epsfxsize=3truein\epsffile{cred5.ps}}
\caption{Tax rates associated with the continuation values of Figure \Fg{cred3f}.}
\infiglist{cred5f}
\endfigure

\vfill\supereject

\section{Interpretations}\label{sec:spe_interpretations}%
  The notion of credibility or sustainability emerges from a ruthless
and complete application of two principles: rational expectations and
self-interest.  At each moment and for each possible history, individuals and
the government act in their own best interests while expecting everyone else
always to act in their best interests.  A credible government policy is one
that it is in the interest of the government to implement on every occasion.

  The structures that we have studied have multiple equilibria that are
indexed by different systems of rational expectations.  Multiple
equilibria are essential because what
sustains a good equilibrium is a system of expectations that
raises the prospect of reverting to a bad equilibrium
if the government chooses to deviate from the good equilibrium.  For
 reversion to the bad equilibrium to be
credible  -- meaning that it is something that the private agents can expect because the
government will want to act accordingly -- the bad equilibrium must itself be an equilibrium.  It must always be in the self-interest of all agents to behave as they
are expected to.  Supporting a Ramsey outcome hinges
on finding an equilibrium with outcomes bad enough to
deter the government from surrendering to a temporary temptation
to deviate.\NFootnote{This statement means that an equilibrium
is supported by beliefs about behavior at prospective
histories of the economy that might never be attained or observed.
Part of the literature on learning in games and dynamic
economies studies situations in which it is not
reasonable to expect ``adaptive'' agents to learn so much.  See
Fudenberg and Kreps (1993), Kreps (1990), and Fudenberg and Levine (1998).  See Sargent (1999, 2008) for macroeconomic counterparts.}
\auth{Kreps, David M.}  \auth{Fudenberg, Drew}   \auth{Levine, David K.}
\auth{Sargent, Thomas J.}


  Is the multiplicity of equilibria a strength or a weakness of such
theories?  Here descriptions of preferences
and technologies, supplemented by the restriction of rational
expectations, don't pin down outcomes. There is an independent role
for expectations not based solely on fundamentals.
  The theory is silent about which
equilibrium will prevail; the theory contains no sense in which the
government {\it chooses\/} among equilibria.

\auth{Christiano, Lawrence J.}
\auth{Chari, V.V.}
\auth{Eichenbaum, Martin}
Depending on the
purpose, the multiplicity of equilibria can be regarded either
as a strength or as a weakness of these theories.
  In inferior equilibria, the government is caught in an
``expectations trap,''\NFootnote{See Chari, Christiano, and Eichenbaum (1998).}
an aspect of the theory that highlights how the government
can be regarded as simply resigning itself to
affirm the public's expectations about it.  Within the theory,
the government's strategy plays a dual role, as it does
in any rational expectations model: one summarizing
the government's choices, the other describing the public's
rule for forecasting the government's behavior.
In inferior equilibria, the government  wishes that it  could
use a different strategy but nevertheless
affirms the public's expectation that
it will adhere to an inferior rule.

\section{Extensions}
In chapter \use{chang}, we shall describe how
Chang (1998) and Phelan and Stacchetti (2001)  extended the
  machinery of this chapter to settings in which  private agents' problems
  have natural state variables like stocks of real balances or
  physical capital so that their best responses to government
  policies satisfy  Euler equations (or costate equations).  This will activate an additional source of history dependence.
  The approach of chapter \use{chang} merges aspects of the method described
  in chapter \use{stackel} and \use{optaxrecur} with those of this chapter.  %They augment the natural state vector
%  and the continuation value with the costate variable of the
%  representative private agent and construct an operator  from sets of pairs of values
%  and costate variables into pairs of values and costate
%  variables. The largest fixed point of that set identifies the
%  set of all sustainable plans.  An equilibrium strategy for the
%  government maps a promised value and a costate into a current
%  decision and next-period promised value and costate.
%  Chang's and Phelan and Stacchetti's work has considerably
%  broadened the class of problems to which the method described in
%  this chapter applies.
  \auth{Stacchetti, Ennio}
  \auth{Phelan, Christopher}
  \auth{Chang, Roberto}
  \auth{Bassetto, Marco}



%\section{Exercises}
\showchaptIDfalse
\showsectIDfalse
\section{Exercises}
\showchaptIDtrue
\showsectIDtrue
\medskip
\smallskip\noindent
{\it Exercise  \the\chapternum.1}\quad
 Consider the following one-period economy.  Let
$(\xi, x, y)$ be the choice variables available to a representative
agent, the market as a whole, and a benevolent government,
respectively.  In a rational expectations equilibrium or
competitive equilibrium,
$\xi = x = h(y)$, where $h(\cdot)$ is the ``equilibrium response''
correspondence that gives competitive equilibrium values of $x$
as a function of $y$; that is, $[h(y),y]$ is a competitive equilibrium.
Let $C$ be the set of competitive equilibria.

Let $X= \{x_M,x_H\}, Y = \{y_M,y_H\}$. For the one-period economy,
when $\xi_i = x_i$, the payoffs to the government and household
are given by the values of $u(x_i,x_i,y_j)$ entered in
the following table:
\vfil\eject

\centerline{One-period payoffs $u(x_i,x_i,y_j)$}
$$\vbox{\offinterlineskip
\hrule
\halign{\strut #\hfil &\quad \hfil# &\quad \hfil# \cr
& $x_M$ & $x_H$ \cr \noalign{\hrule}
$y_M$ & $10\rlap{*}$ & $20$ \cr
$y_H$ & $4$ & $15\rlap{*}$ \cr\noalign{\hrule}
\noalign{\smallskip}
\hbox{$^\ast$Denotes $(x,y)\in C$.}\cr
              }}$$




\noindent
The values of $u(\xi_k,x_i,y_j)$ not reported in the table
are such that the competitive equilibria are the outcome pairs
denoted by an asterisk (*).
\medskip
\noindent{\bf a.}  Find the {\it Nash equilibrium\/} (in pure
strategies) and {\it Ramsey outcome\/} for the one-period economy.
\smallskip
\noindent{\bf b.} Suppose that this economy is repeated twice.  Is it
possible to support the Ramsey outcome in the first period by
reverting to the Nash outcome in the second period
in case of a deviation?
\smallskip
\noindent{\bf c.} Suppose that this economy is repeated three times.  Is
it possible to support the Ramsey outcome in the first period?
In the second period?
\medskip
Consider the following expanded version of the preceding economy.
  $Y=\{y_L, y_M,y_H\}$, $X=\{ x_L,x_M,x_H\}$. When $\xi_i = x_i$,
the payoffs are given by $u(x_i,x_i,y_j)$ entered here:
\smallskip

\centerline{One-period payoffs  $u(x_i,x_i,y_j)$}
$$\vbox{\offinterlineskip
\hrule
\halign{\strut #\hfil &\quad \hfil# &\quad \hfil# &\quad \hfil#\cr
& $x_L$ & $x_M$ & $x_H$\cr \noalign{\hrule}
$y_L$ & $3\rlap{*}$ & $7$ & $9$\cr
$y_M$ & $1$ & $10\rlap{*}$ & 20\cr
$y_H$ & $0$ & $4$ & $15\rlap{*}$\cr \noalign{\hrule}
\noalign{\smallskip}
\hbox{$^\ast$Denotes $(x,y)\in C$.}\cr
         }}$$


\medskip
\noindent{\bf d.}  What are  Nash equilibria in this one-period economy?

\smallskip
\noindent{\bf e.}  Suppose that this economy is repeated twice.  Find a
subgame perfect equilibrium that supports the Ramsey outcome
in the first period.  For what values of $\delta$ will this
equilibrium work?
\smallskip
\noindent{\bf f.}  Suppose that this economy is repeated three times.
Find an SPE that supports the Ramsey
outcome in the first two periods (assume $\delta = 0.8$).  Is it unique?

\medskip

\noindent{\it Exercise \the\chapternum.2} \quad
  Consider a version of the setting studied by Stokey (1989). Let
$(\xi, x, y)$
be the choice variables available to a representative agent, the
market as a whole, and a benevolent government, respectively.  In a
rational expectations or competitive equilibrium, $\xi = x = h(y)$, where
$h(\cdot)$ is the ``equilibrium response'' correspondence that
gives competitive equilibrium values of $x$ as a function of
$y$; that is, $[h(y),y]$ is a competitive equilibrium.  Let $C$ be
the set of competitive equilibria.

Consider the following special case.  Let $X= \{x_L,x_H\}$ and
$Y = \{y_L,y_H\}$. For the one-period economy,
when $\xi_i = x_i$, the payoffs to the government
 are given by the values of $u(x_i,x_i,y_j)$ entered in
the following table:

\smallskip

\centerline{One-period payoffs $u(x_i,x_i,y_j)$}
$$\vbox{\offinterlineskip
\hrule
\halign{\strut #\hfil &\quad \hfil# &\quad \hfil# \cr
& $x_L$ & $x_H$ \cr \noalign{\hrule}
$y_L$ & $0\rlap{*}$ & $20$ \cr
$y_H$ & 1 & $10\rlap{*}$ \cr \noalign{\hrule}
\noalign{\smallskip}
\hbox{$^\ast$ Denotes $(x,y)\in C$.}\cr
              }}$$

\smallskip\noindent
The values of $u(\xi_k,x_i,y_j)$ not reported in the table
are such that the competitive equilibria are the outcome pairs
denoted by an asterisk (*).

\medskip
\noindent{\bf a.}  Define a {\it Ramsey plan\/} and a {\it Ramsey outcome\/}
for the one-period economy.  Find the Ramsey outcome.
\smallskip
\noindent{\bf b.}  Define a {\it Nash equilibrium\/} (in pure strategies) for the
 one-period economy.
\smallskip
\noindent{\bf c.}  Show that there exists no Nash equilibrium (in pure strategies)
for the one-period economy.
\smallskip
\noindent{\bf d.}  Consider the infinitely repeated version of this economy, starting
with $t=1$ and continuing forever.
Define a {\it subgame perfect equilibrium}.
\smallskip
\noindent{\bf e.}  Find the value to the government associated with the
{\it worst\/} subgame perfect equilibrium.
\smallskip
\noindent{\bf f.} Assume that the discount factor is $\delta = .8913
= (1/10)^{1/20}=.1^{.05}$.
 Determine whether infinite repetition of the Ramsey outcome
is sustainable as an SPE.  If it is,
display the associated subgame perfect equilibrium.
\smallskip
\noindent{\bf g.} Find the value to the government associated with the {\it best\/}
subgame perfect equilibrium.
\smallskip
\noindent{\bf h.}  Find the outcome path associated with the {\it worst\/} subgame
perfect equilibrium.
\smallskip
\noindent{\bf i.}  Find the one-period continuation value $v_1$ and
the outcome path associated with
the one-period continuation strategy $\sigma^1$ that induces
adherence to the worst subgame perfect equilibrium.
\smallskip
\noindent{\bf j.}  Find the one-period continuation value $v_2$ and
the outcome path associated with the one-period continuation
strategy $\sigma^2$ that induces adherence to the
first-period outcome of the $\sigma^1$ that you found in part i.
\smallskip
\noindent{\bf k.} Proceeding recursively, define $v_j$ and $\sigma^j$, respectively,
as the one-period continuation value and the continuation strategy
that induces adherence to the first-period outcome of $\sigma^{j-1}$,
where $(v_1, \sigma^1)$ were defined in part i.  Find $v_j$ for
$j= 1, 2, \ldots,$ and find the associated outcome paths.

\smallskip
\noindent{\bf l.}  Find the lowest value for the discount factor for which repetition
of the Ramsey outcome is an SPE.

\medskip
\noindent{\it Exercise \the\chapternum.3} \quad {\bf Finding  worst and best SPEs}
\medskip
\noindent Consider the following model of Kydland and
Prescott (1977). A government
chooses the inflation rate $y$ from a closed  interval $[0, 10]$.
     There is a family of Phillips curves indexed by
the public's expectation of inflation $x$:
$$ U = U^* -  \theta (y-x) \leqno(1) $$
where $U$ is the unemployment rate,  $y$ is the inflation rate
set by the government, and $U^*>0 $  is the natural rate of unemployment
and $\theta >0 $ is the slope of the Phillips curve, and where
$x$ is the average of private agents' setting of
a forecast of $y$, called $\xi$.
Private agents' only decision in this    model is to forecast
inflation.  They choose their forecast $\xi$ to maximize
$$ - .5 (y-\xi)^2. \leqno(2)  $$
Thus, if they know $y$, private agents set $\xi=y$.
All agents choose the same $\xi$, so that
$x = \xi$ in a rational expectations equilibrium.
The government has one-period return function
$$ r(x,y) = -.5(U^2+y^2) = -.5[(U^* - \theta (y-x))^2 + y^2]. \leqno(3) $$
Define a {\it competitive equilibrium\/} as a 3-tuple
$U,x,y$ such that given $y$, private agents solve their forecasting
problem and (1) is satisfied.
\medskip
\noindent{\bf a.} Verify that in a competitive equilibrium,
$x=y$ and   $U=U^*$.
\medskip
\noindent{\bf b.}  Define the government best response function
in the one-period economy. Compute it.
\medskip
\noindent{\bf c.} Define a Nash equilibrium (in the spirit
of Stokey (1989) or the text of this chapter). Compute one.
\medskip
\noindent{\bf d.}  Define  the Ramsey problem for the one-period
economy.  Define the Ramsey outcome. Compute it.
\medskip
\noindent{\bf e.}  Verify that the Ramsey outcome is better than
the Nash outcome.
\medskip
\noindent  Now consider the repeated economy where the  government cares
about
$$ (1-\delta) \sum_{t=1}^\infty \delta^{t-1} r(x_t,y_t), \leqno(4) $$
where $\delta \in (0,1)$.
\medskip
\noindent{\bf f.}  Define a {\it subgame perfect equilibrium\/}.
\medskip
\noindent{\bf g.} Define a {\it recursive\/} subgame perfect
equilibrium.
\medskip
\noindent{\bf h.}    Find a recursive subgame perfect
equilibrium that  sustains infinite repetition of the one-period Nash
equilibrium outcome.
\medskip
\noindent{\bf i.}   For $\delta=.95$,  $U^*=5, \theta=1$,
find the value of (4) associated with the worst subgame perfect
equilibrium.   Carefully and completely
 show your method for computing the
worst subgame perfect equilibrium value.  Also, compute
the values associated with the repeated Ramsey outcome, the Nash equilibrium,
and Abreu's simple stick-and-carrot strategy.
\medskip
\noindent{\bf j.}  Compute a recursive  subgame perfect equilibrium that
attains the worst subgame perfect equilibrium value (4) for the
parameter values in part i.

\medskip
\noindent{\bf k.} For $U^* =5, \theta =1$, find the  cutoff value
$\delta_c$
of the discount factor $\delta$
 below which the Ramsey value $v^R$  cannot
be sustained by reverting to repetition of $v^N$  as a consequence
of deviation from the Ramsey $y$.

\medskip
\noindent{\bf l.}  For the same parameter values as in part k,
find another  cut off value $\tilde \delta_c$ for $\delta$
below which Ramsey cannot be sustained by reverting after a deviation
to an equilibrium attaining the worst subgame   perfect equilibrium
value.  Compute the worst subgame perfect equilibrium value for
$\tilde \delta_c$.
%%Answer: \delta_c=.333; \tilde \delta_c = .125$

\medskip
\noindent{\bf m.}  For $\delta =.08 $, compute
values associated with
the best and worst subgame perfect equilibrium strategies.
%{\it Hint:} Read the section leading up to formulas
%equations 16.29--16.32 in the revised version of red16.ps in the
%teaching seciton on my web page.
