% revised by Francois, May 29, 2012
\input texsis
\book
\input index
\input robustin
\input grafinp3  %  added for the macro book
%\input psfig     % added for the macro book
\SetDate   %  Francois' patch, aug 6, 2001
\ReadAUX
\input epsf.tex

%\input inputs2
\showsectIDtrue



\input txssectsblack2


%\hsize=5.75in
%\vsize=7.5in
%\baselineskip=20pt
%\hoffset=0.375in
%\voffset=.3in
% the above looks good;

%%%%%%%  settings for original version -- looked pretty good
%\hsize=4.25in
%\vsize=7.1in
%\baselineskip=20pt  % original was 20pt
%\hoffset=0.505in
%\voffset=1in
%%%% end of original settings

%%%%%%%%%%  values from MIT book second printing:
%%  The following are Francois's December settings
%\overfullrule=0pt\mathsurround=1pt\hsize=11truecm %
%%
%\vsize=17.75truecm\hoffset=1.1truein \voffset=70pt %\voffset=40pt
%%

%%  The following are Tom's December settings
%\overfullrule=0pt\mathsurround=1pt\hsize=13truecm %
%%
%\vsize=17.75truecm\hoffset=.55truein \voffset=70pt %\voffset=40pt
%%

%  Some new settings following 12 March 04 discussion with FRancois
\overfullrule=0pt\mathsurround=1pt\hsize=12.15truecm %
%
\vsize=17.75truecm\hoffset=.735truein \voffset=72pt %\voffset=40pt
%


%
%\def\Bigskip{\vskip\Bigskipamount}
%\baselineskip=12pt
%
%\overfullrule=0pt
%\parindent=.5truecm
%
%\def\small{\baselineskip=9pt
%  \setbox\strutbox=\hbox{\vrule height7.5pt depth2.5pt
%width0pt}}

%%%%%%%%%%%%%  end of values from MIT press book

\Contentstrue
\elevenpoint
\tenpoint
\normalbaselineskip=14pt
\overfullrule=0pt

\singlespaced

% general macros for mathematics

\def\sci#1#2{$ {#1} \times 10^{{#2}}$}
\def\bmatrix#1{\left[ \matrix{#1} \right]}
\def\vmatrix#1{\left| \matrix{#1} \right|}
\def\abs#1{\left| #1 \right|}

\def\argmax{\mathop{\rm arg\,max}}

\def\prob{{\rm prob}}

\def\a{\alpha}
\def\be{{\beta}}
\def\ga{{\gamma}}
\def\de{{\delta}}
\def\eps{{\epsilon}}
\def\ze{{\zeta}}
\def\th{{\theta}}
\def\ka{{\kappa}}
\def\la{\lambda}
\def\si{{\sigma}}
\def\ups{{\upsilon}}
\def\om{{\omega}}
\def\G{{\Gamma}}
\def\D{{\Delta}}
\def\Th{{\Theta}}
\def\La{{\Lambda}}
\def\Si{{\Sigma}}
\def\Ups{{\Upsilon}}
\def\Om{{\Omega}}
\def\epb{{\overline\epsilon}}
\def\chib{{\overline\chi}}
\def\psib{{\overline\psi}}
\def\mnrst{{\mu\nu\rho\sigma\tau}}
\def\munu{{\mu\nu}}
\def\lbar{{\mathchar'26\mskip-9mu\lambda}}


% various math macros

\def\lim{\mathop{\hbox{lim}}}
\def\max{\mathop{\hbox{max}}}
\def\min{\mathop{\hbox{min}}}
\def\sup{\mathop{\hbox{sup}}}
\def\inf{\mathop{\hbox{inf}}}
\def\limsup{\mathop{\hbox{lim\,sup}}}
\def\liminf{\mathop{\hbox{lim\,inf}}}
\def\log{\mathop{\hbox{log}}}
\def\ln{\mathop{\hbox{ln}}}
\def\qed{\hskip6pt\vrule width4pt height8pt depth1.5pt}

\def\ne{\mathrel{{/\hbox{\hglue-5.6pt =}}}}
\def\nex{\hskip.5pt\mathrel{\scriptstyle
{/\hbox{\hglue-4.1pt \Psix=}}}}



%Blackboard Bold R
\def\bbR{{I\kern-0.3em R}}


\def\sq{{\vbox {\hrule height 0.6pt\hbox{\vrule width 0.6pt\hskip 3pt
\vbox{\vskip 6pt}\hskip 3pt \vrule width 0.6pt}\hrule height 0.6pt}}}

\def\pp{{\prime\prime}}

\def\12{{1\over2}}


\def\part{\partial}
% a macro to help  bibtex

% define some sc fonts

% family sc: small capitals [9]
%\newfam\scfam
%\font\tensc=cmcsc10
%\font\ninesc=cmcsc9
%\font\eightsc=cmcsc8
%\font\sevensc=cmcsc8 at 7pt
%\font\sixsc=cmcsc8 at 6pt

% fix the following
%\font\sc=cmbsy10 at 11pt
 \font\elevensc=cmcsc10 at 8.95pt
\def\mbox#1{\leavevmode\hbox{#1}}

% define special section macro
%% FIX LATER -sc doesn't work
%\def\specsec#1{\medskip{\elevensc \noindent{#1}}}
\def\specsec#1{\medskip{\sc \noindent{#1}}}

\def\bbR{{I\kern-0.3em R}}



\hfuzz=10pt

% setting for TXS macros
\NoTrailingSpaces
\hyphenation{seign-ior-age seign-or-age}

%\def\beginsubsection#1\par{\medbreak\noindent{\it #1}\medskip\par}
\def\ellipsis{\hskip.33em.\hskip.33em.\hskip.33em.\hskip.33em}



%%% Francois' indexing macros


% If you want more than one index I can't guess how you will want to
% use them, so all I can do is tell you how to use the macros above to
% create another (or several other) additional indices:
%
% 1) Create another output file and open it, with something like:
%
       \newwrite\authorindexout
       \immediate\openout\authorindexout=authblk.idx
%
% 2) Define a macro to make the entry to the second index, following
%    the pattern of \index and \@index above, but using the other index
%    file (but don't use @ in a macro name).  For example:
%
       \def\auth{\begingroup     % special characters letters
           \catcode`@=11 \catcode`"=11 \catcode`!=11 \catcode`|=11
               \catcode`-=11
           \Oindex}

       \def\Oindex#1{\endgroup    % special characters return
              \indexwrite\authorindexout{\indexentry{#1}}}

%    Then saying \auth{word} will make an entry into the alternate
%    index for ``word''.


%%  make index for matlab using the above procedure



       \newwrite\matlabindexout
       \immediate\openout\matlabindexout=matlblk.idx
%
       \def\mtlb{\begingroup     % special characters letters
           \catcode`@=11 \catcode`"=11 \catcode`!=11 \catcode`|=11
               \catcode`-=11
           \0index}

       \def\0index#1{\endgroup    % special characters return
              \indexwrite\matlabindexout{\indexentry{#1}}}

%    Then saying \mtlb{word} will make an entry into the alternate
%    index for ``word''.





%%%%% (2)  headline macros

% modify TeXsis macros to have the chapter name on even pages
% and section name on odd pages

%%% NOTE: THE FOLLOWING MACRO BY FRANCOIS "ALMOST" WORKS

% define a parallel set of macros for the left headline
%\def\setLHeadline#1{\@setLHeadline#1\n\endlist}   % set the \HeadText
%\def\@setLHeadline#1\n#2\endlist{% #1 is everything up to first \n
%      \global\edef\LHeadText{#1}%                % No: just use #1
%}
%
%
%\newif\ifrightnom   \rightnomtrue
%\def\HeadLine{%
%   \edef\firstm{{\XA\iffalse\firstmark\fi}}%    % chapternumber of \firstmark
%   \edef\topm{{\XA\iffalse\topmark\fi}}%        % chapternumber of \topmark
%%   \ifRunningHeads                              % print running heads?
%%     \def\He@dText{\HeadText}%                  % define head text
%%   \else\def\He@dText{\relax}\fi                % or nothing
%\ifnum\sectionnum=0\relax\def\LHeadText{\HeadText}% use chapter title
%\else\setLHeadline{\SectionTitle}\fi%
%   \ifbookpagenumbers                           % book-like numbering
%      \ifodd\pageno\rightnomtrue                %   odd numbers right
%      \else\rightnomfalse\fi                    %   even numbers left
%   \else\rightnomtrue\fi                        % or always right
%   \ifx\topm\firstm                             % Marks the same?
%     \ifrightnom                                % number on right?
%{\no\headlinefont\LHeadText\hss\llap{\PageNumber}}% display section title (FRV)
%     \else%                                      % or on left
%{\no\rlap{\PageNumber}\hss\headlinefont \ifnum\chapternum=0\relax
%\else
%  \ifshowchaptID\the\chapternum:\
%\fi%
%\fi%
%\HeadText}%%   display chapter title (FRV)
%      \fi%                                       % for \ifrightnom
%   \else \hfill \fi}%                           % NOTHING ON FIRST PAGE
%
%% this is \FootLine from TXShead.tex
%\def\FootLine{%
%   \edef\firstm{%                               %
%      {\expandafter\iffalse\firstmark\fi}}%     % get first mark
%   \edef\topm{%                                 %
%      {\expandafter\iffalse\topmark\fi}}%       % and top mark
%   \ifx\topm\firstm \hss                        % if not page 1, nothing
%    \else {\hss \FootText \hss} \fi}            % print \FootText if page 1
%
%   \headline={\HeadLine}                        % TeXsis running headlines
%   \footline={\FootLine}                        %    and footlines
%
%   \RunningHeadstrue                    % running headlines
%   \def\FootText{\foliofont--\ \PageNumber\ --}% % for 1st page of chapter
%   \bookpagenumbers                     % page numbers for book binding


%%%% end of Francois' indexing macros

\hyphenation{Na-ra-ya-na}
\nopagenumbers
\newpage
\input blacktit3
\newpage
\input copywrit6
\vfil\eject
\pageno=-5
\pagenumbers\showchaptIDfalse
\Contentsfalse % this must precede \chapter{Contents} !!
\chapter{Contents}
\chapternum=0  % this prevents a chapter number in the running header
\showchaptIDtrue % show the chapter numbers in the TOC itself
\Contents\showchaptIDfalse\chapternum=1
%\vfil\supereject

%\chapter{List of Figures}
%\Contentstrue\addTOC{0}{List of Figures}{\folio}\Contentsfalse
%\ListFigures
%\vfil\supereject

%\chapter{List of Tables}
%\Contentstrue\addTOC{0}{List of Tables}{\folio}\Contentsfalse
%\ListTables
%\vfil\eject

\chapter{Acknowledgments}
% add TOC entry by hand
\Contentstrue\addTOC{0}{Acknowledgements}{\folio}\Contentsfalse
\input blackack
\vfil\eject

\chapter{Preface to the fourth edition}
% add TOC entry by hand
\Contentstrue\addTOC{0}{Preface to the fourth edition}{\folio}\Contentsfalse
\input blackin
\vfil\eject
\pageno=0
\Contentstrue  % turn contents on from now on
\makepart{Part I}{Imperialism of Recursive Methods}
%\hbox{}\newpage
%\hbox{}\nopagenumbers\newpage\pagenumbers
\showchaptIDtrue
%\nopagenumbers\newpage\pagenumbers
%\advance\pageno by 1
\chapternum=0  % reset chapter numbers

%\showchaptIDfalse
%\vfil\eject
%\input blankpage  % added by Tom May 25, 2012  UNSUCCESSFUL
\input black0   % \use{overview}
\vfil\eject
\nopagenumbers\hbox{}\newpage\pagenumbers


\medskip
\TOCwrite{\vfil\eject}
\makepart{Part II}{Tools}
%\pageno=666
%\advance\pageno by -1
\input black1_2017   % %\use{timeseries} nontechnical overview
\input black2  % \use{dynamicdp1} overview of dynamic programming
\input black3   % \use{practical} practical dynamic programming
\input black4  % \use{dplinear}
\vfil\eject
\nopagenumbers\hbox{}\newpage\pagenumbers
\input black5_2010  % \use{seach1}
\vfil\eject
\nopagenumbers\hbox{}\newpage\pagenumbers
\makepart{Part III}{Competitive Equilibria and Applications}
%\advance\pageno by -1
\input black6  %\use{recurpe}
%\input black7before  %\use{recurge}
%\input black7_new%\use{recurge}
\input black7_2016  %\use{recurge}
\input black8  %\use{ogmodels}
\input black9  %\use{ricardian}
\vfil\eject
\nopagenumbers\hbox{}\newpage\pagenumbers
\advance\pageno by -1
\input blackpro_5_2012 %\use{linappro}  -- this is experimental, blackpro_4 is the original
%\input black_keynes
\vfil\eject
\vfil\eject
%\nopagenumbers
\hbox{}\newpage\pagenumbers
\nopagenumbers\hbox{}\newpage\pagenumbers


\input bgrowth2  %\use{growth1}
%\vfil\eject
%\nopagenumbers\hbox{}\newpage\pagenumbers
%\TOCwrite{\vfil\eject}
\input black10_part1 %\use{assetpricing1}
\vfil\eject
\nopagenumbers\hbox{}\newpage\pagenumbers
\input black10_part2_new_ver3 %\use{assetpricing2}
\vfil\eject
\nopagenumbers\hbox{}\newpage\pagenumbers
\input black11  %\use{growth}
\vfil\eject
\nopagenumbers\hbox{}\newpage\pagenumbers
\input black12_9  %\use{optax}
\vfil\eject
\makepart{Part IV}{The Savings Problem and
Bewley Models}
%\advance\pageno by -1

\input black13 %\use{selfinsure}
\vfil\eject
\nopagenumbers\hbox{}\newpage\pagenumbers
\input black14 %\use{incomplete}
%\vfil\eject
\vfil\eject
%\nopagenumbers\hbox{}\newpage\pagenumbers
\makepart{Part V}{Recursive Contracts}
%advance\pageno by -1
\input blkstack_2 %\use{stackel}
\input opt_tax_recur %\use{optaxrecur}
\input black15 %\use{socialinsurance}
\input blk15closed  %\use{socialinsurance2}
\input blinsur %\use{uninsur1}
\vfil\eject
\nopagenumbers\hbox{}\newpage\pagenumbers
\input black16_2010 %\use{credible}
%\vfil\eject
\nopagenumbers\hbox{}\newpage\pagenumbers
\input black_chang %\use{chang}
\vfil\eject
\nopagenumbers\hbox{}\newpage\pagenumbers
\input blktrade  %\use{wldtrade}
\vfil\eject
\nopagenumbers\hbox{}\newpage\pagenumbers
\makepart{Part VI}{Classical Monetary and Labor Economics}

%\advance\pageno by -1
\input black17_tom_4 %\use{fiscalmonetary}
\input black18  %\use{townsend}
\input black19_2011_Dec  %\use{search2}
\input Matching_new_Tom_7 % \use{mechanics_matching}
\vfil\eject
%\input macrolaborI_tom_2  % \use{marcoLaborI}  -- new macro labor chapter
\vfil\eject
\nopagenumbers\hbox{}\newpage\pagenumbers
\input macrolaborII % \use{macrolaborII}

\nopagenumbers\hbox{}\newpage\pagenumbers
\makepart{Part VII}{Technical Appendices}
%\advance\pageno by -1
\input black20  %\use{functional}
\input black21_a  %\use{lincontrol}
\vfil\eject
\vfil\eject
% end material
\showchaptIDfalse
\vfil\eject
\nopagenumbers\hbox{}\newpage\pagenumbers
\chapter{References}
\input blackref
\vfil\supereject
\vfil\eject
\nopagenumbers\hbox{}\newpage\pagenumbers
\showchaptIDfalse
\SetDoubleColumns{0.47\hsize}
\doublecolumns
\chapter{Subject Index}
\setHeadline{Subject Index}
{\eightpoint
\input rmt4.ind
}
\vfil\supereject
\chapter{Author Index}
\setHeadline{Author Index}
{\eightpoint
\input authblk.ind
}
\vfil\supereject
\chapter{Matlab Index}
\setHeadline{Matlab Index}
{\eightpoint
 \input matlblk.ind
 }
% \enddoublecolumns
\vfil\supereject

\bye
