\input grafinp3
%\input grafinput8
\input psfig

%\showchaptIDtrue
%\def\@chaptID{19.}
%\input gayejnl.txt
%\input gayedef.txt
\def\lege{\raise.3ex\hbox{$>$\kern-.75em\lower1ex\hbox{$<$}}}

%\hbox{}
\footnum=0
\chapter{Matching Models Mechanics\label{mechanics_matching}}


\section{Introduction}

%One of the workhorses in macro labor is the framework of
%matching models, as presented in section \use{sec:matchingmodel}
%of chapter \use{search2}.
What are now widely called matching models  have   matching
functions that are designed to represent in congestion externalities concisely.\NFootnote{We encountered these earlier
 in section \use{sec:matchingmodel}. In chapter \use{search1}, %%we applied  the
the word `matching' described  Jovanovic's (1979a)  analysis
of a process in which workers and firms  gradually  learn about match quality.
In macro labor, the term `matching models' has
come instead to mean models that postulate matching functions.} We reserve  the term search models to denote ones in the spirit of McCall (1970), like the search-island
model of Lucas and Prescott (1974) described in
section \use{sec:LucasPrescott}.\NFootnote{Petrongolo and
Pissarides (2001) call
the matching function a black box because it describes outcomes of labor market frictions without
explicitly modeling them.}
This chapter explores some of the mechanics of
matching models, especially those governing the responses of labor market outcomes to productivity shocks.



 To get big responses of unemployment to movements
in productivity, matching models require a high elasticity of market
tightness %-- the ratio of vacancies to unemployment --
with respect to
productivity.  Shimer (2005) pointed out  that
for common calibrations of what was then a standard matching model, the elasticity
of market tightness is too low to explain business cycle fluctuations.
To increase that elasticity, researchers  reconfigured
matching models in various ways:  by elevating the utility
of leisure, by making wages sticky, by assuming alternating-offer wage
bargaining, by introducing costly acquisition of credit, or by
assuming fixed matching costs.  Ljungqvist and Sargent (2017) showed that beneath this apparent
diversity there resides an essential unity: all of these redesigned  matching models increase responses of
unemployment to movements in productivity by diminishing what Ljungqvist and Sargent called the
{\it fundamental surplus} fraction, a name they gave  to an upper bound
on the fraction of a job's output that an invisible hand
can allocate to vacancy creation. Business cycle and
welfare state dynamics of an entire class of reconfigured matching
models  operate through this common channel.


Across a variety of matching models, the fundamental surplus fraction is the
single intermediate channel through which economic forces
generating a high elasticity of market tightness with
respect to productivity must operate.  Differences in the  fundamental surplus
explain  why unemployment responds sensitively to
movements in productivity in some matching models but not in
others. The role of the fundamental surplus in generating that response sensitivity
transcends diverse matching models having
very different outcomes along other dimensions that include the elasticity of
wages with respect to productivity %%the size of match surpluses,
and
whether or not outside values affect bargaining outcomes.




For any model with a matching function,  to arrive at the fundamental surplus take the output of a job, then deduct the
sum of the value of leisure,
the annuitized values of layoff costs and training costs and a worker's
ability to exploit a firm's cost of delay under
alternating-offer wage bargaining, and  any other items that must
be set aside. The fundamental surplus is an
upper bound on what the ``invisible hand'' could allocate
to vacancy creation. If that fundamental surplus constitutes
a small fraction of a job's output, it means that a given change
in productivity translates into a  much larger percentage
change in the fundamental surplus. Because such large movements in the
amount of resources that could potentially be used for vacancy
creation cannot be offset by the invisible hand,
significant variations in market tightness ensue,  causing large
movements in unemployment.

In contrast to search models, matching models are prone to
exhibit externalities. What types of workers -- perhaps
differentiated by education, skill, age -- and what types of jobs
-- perhaps differentiated by skill and strength requirements --
does the analyst make sit within the same matching functions.
Broadly speaking, matching analyses can be divided into those
that highlight and study congestion externalities, and those that
seek to eliminate these externalities, with the objective of
coming up with tractable models of the coexistence of unemployed
workers and unfilled vacancies. For the latter purpose, by
proliferating matching functions and thereby arresting congestion
externalities along some dimensions, we can approach notions of
`directed search' that simplifies the analysis.

%A key dimension along which matching models differs
%is how they envision   a key force captured with a matching function:
%congestion externalities.   What types of workers -- perhaps
%differentiated by education, skill, age --
%and what types of jobs -- perhaps differentiated by skill and strength
%requirements --
%does the analyst make sit within the same matching functions.
%By proliferating  matching functions
%and thereby arresting congestion externalities along some dimensions,
%we can approach notions of 'directed search' in tractable ways.




\auth{Ch\'eron, Arnaud} \auth{Hairault, Jean-Olivier} \auth{Langot, Francois}
\auth{Menzio, Guido} \auth{Telyukova, Irina A.} \auth{Visschers, Ludo}

To illustrate the two types of matching analyses that either
emphasize or eliminate externalities, we turn to aging as one
key source of heterogeneity. Ch\'eron,
Hairault and Langot (2013), and Menzio, Telyukova and Visschers (2016)
study overlapping generations models in which unemployed
workers either enter a single matching function or are assigned
to type-specific matching functions. In a version of the
model of Ch\'eron, Hairault and Langot, we derive
stark policy implications to subsidize the continuing employment
of old workers and to tax that of young workers, in order
to optimally manage the age composition of the unemployed
in a single matching function. When we instead introduce
age-specific matching functions as suggested by
Menzio, Telyukova and Visschers,
those externalities vanish and market forces optimally
generate measures of market tightness decreasing in age.
The latter equilibrium is said to be block recursive by
which agents' value and policy functions depend on the
realization of exogenous shocks but not on the distribution
of agents across unemployment and employment states. This
makes it easy to compute out-of-steady-state dynamics
as well as equilibria with aggregate shocks.


\section{Fundamental surplus}
With exogenous separation,  a comparative steady state analysis decomposes  the elasticity of market tightness with respect to productivity into
two multiplicative factors,  both of which are bounded from below by unity.
In a matching model  of variety $j$, let $\eta^j_{\theta,y}$ be the elasticity of
market tightness $\theta$ with respect to productivity $y$:
$$
 \eta^j_{\theta,y}
\;\equiv\; {d\, \theta \over d\, y} \; {y \over \theta}
\;=\; \Upsilon^j \; \frac{y}{y-x^j} .    \EQN MLI_generic
$$
The first factor $\Upsilon^j$  has  an   upper bound
coming  from a consensus  about values of the elasticity
of matching with respect to unemployment. The second factor
$y/(y-x^j)$ is the inverse of what we  define to be   the
`fundamental surplus fraction'. The fundamental surplus $y-x^j$ equals a quantity that  deducts from productivity $y$ a
value $x^j$ that the `invisible hand' cannot allocate to
vacancy creation, a quantity whose economic interpretation differs across models.  Unlike $\Upsilon^j$, the fraction $y/(y-x^j)$ has no
widely agreed upon upper bound.
To get a high elasticity of market
tightness requires that $y/(y-x^j)$ must  be large, i.e., that  what we call the
fundamental surplus fraction must be small.\NFootnote{We call $y-x$ the fundamental surplus and
$\frac{y-x}{y}$ the fundamental surplus fraction.}
 Across reconfigured matching models, many details
differ, but
what ultimately matters is the fundamental surplus.


\subsection{Sensitivity of unemployment to market tightness}

To set the stage for studying how small changes in productivity
can have large effects on unemployment, we start by computing
the elasticity of unemployment with respect to market tightness.
The derivative of steady-state unemployment in equation
\Ep{unemp} with respect to market tightness is
$$
{{d}\, u \over {d} \, \theta } \;=\;
- { s \left[q(\theta) \,+\, \theta \,q'(\theta) \right] \over
  [s \,+\, \theta \,q(\theta)]^2 }
\;=\; -
\left[ 1 \,+\, { \theta \,q'(\theta) \over q(\theta)}  \right]
{ u\, q(\theta) \over   s \,+\, \theta \,q(\theta) }
\;=\;  - (1-\alpha)\,
{ u \,q(\theta) \over  s \,+\, \theta \,q(\theta) }
             \,,
$$
where the second equality uses equation \Ep{unemp} and factors
$q(\theta)$ from the expression in square brackets of the numerator,
and the third equality is obtained after invoking
the constant elasticity of matching with respect to unemployment,
$\alpha=-q'(\theta)\, \theta / q(\theta)$.
So the elasticity of
unemployment with respect to market tightness is
$$
\eta_{u,\theta}
%\;&\equiv&\; {d\, u \over d\, \theta} \; {\theta \over u}
\;=\; -(1-\alpha)\,
{ \theta \,q(\theta) \over  s \,+\, \theta \,q(\theta) }
\;=\; -(1-\alpha)\,
\left(1\,-\, { s \over  s \,+\, \theta \,q(\theta) } \right)
\;=\;  - (1-\alpha)\,(1-u)\,,                  %%%%%%%%\EQN ML1_unempelast
$$
where the second equality is obtained after adding and subtracting
$s$ to the numerator, and the last third equality invokes
expression \Ep{unemp}.


Thus, to shed light on what contributes to significant volatility
in unemployment, we seek forces that can make market tightness
$\theta$ highly elastic with respect to productivity.


%\subsection{Decomposition of the elasticity of market tightness}
\subsection{Nash bargaining model}
           \label{sec:ML1_Nash_elasticity}
%Our first focus is on the standard matching model with Nash
%bargaining, for which equilibrium expression \Ep{equili}
%%for market tightness
%can be rewritten as
In the standard version of the Nash bargaining model, the equilibrium
expression \Ep{equili} for market tightness
can be rewritten as
$$
{1-\phi \over c} (y-z) \;=\; {r+s \over q(\theta)}\,+\,\phi\,\theta\,.
                                                     \EQN ML1_theta
$$
Implicit differentiation  yields
$$\EQNalign{
{d\, \theta \over d\, y} \;&=\; - {
{\displaystyle  1-\phi \over \displaystyle c} \over
- \left( {\displaystyle - q'(\theta)\,(r+s) \over \displaystyle q(\theta)^2}
\,+\, \phi \right)  }
\;=\; - { \left( {\displaystyle r+s \over \displaystyle q(\theta)}
\,+\, \phi \, \theta \right) \; {\displaystyle 1 \over \displaystyle y-z} \over
- \left( {\displaystyle \alpha (r+s) \over \displaystyle \theta q(\theta)}
\,+\, \phi \right)  }                                   \cr
\noalign{\vskip.2cm}
\;&=\;  {\hfill (r+s)\,+\,\phi\,\theta\,q(\theta)  \over
\alpha (r+s) \,+\, \phi\,\theta\,q(\theta) } \; \; {\theta \over y-z}
\;\equiv\; \Upsilon^{\rm Nash} \; {\theta \over y-z}
\,,                                                 \EQN ML1_thetadiff \cr}
$$
where the second equality is obtained after using
equation \Ep{ML1_theta}
to rearrange the numerator, while in the denominator, we invoke
the constant elasticity of matching with respect to unemployment;
the third equality
follows from multiplying and dividing by $\theta\,q(\theta)$. The
elasticity of market tightness with respect to productivity
is then given by
$$
\eta_{\theta,y}
%%\;\equiv\; {d\, \theta \over d\, y} \; {y \over \theta}
\;=\;
{\hfill (r+s)\,+\,\phi\,\theta\,q(\theta)  \over
\alpha (r+s) \,+\, \phi\,\theta\,q(\theta) } \; \; {y \over y-z}
\;\equiv\;
\Upsilon^{\rm Nash} \; {y \over y-z} \,.    \EQN ML1_thetaelast
$$
This multiplicative decomposition of the elasticity of market
tightness is central to our analysis. Similar decompositions
prevail in all of the reconfigured matching models to be
described below and those in Ljungqvist and Sargent (2017).
The first factor $\Upsilon^{\rm Nash}$
in expression \Ep{ML1_thetaelast},
 has counterparts in other setups.   A consensus
about reasonable parameter values   %%calibrations
bounds its contribution
to the elasticity of market tightness. Hence, the
magnitude of the elasticity of market tightness depends mostly on
the second factor in expression \Ep{ML1_thetaelast}, i.e.,
the inverse of what %in section \ref{sec:intro}
we  define to be the fundamental surplus
fraction.

In
the standard matching model with Nash bargaining, the fundamental surplus
is simply what remains  after deducting  the worker's value
of leisure from productivity; $x=z$ in expression \Ep{MLI_generic}.
To induce  them  to work, workers have to receive at least  the
value of  leisure, so the invisible hand cannot allocate  that value to
vacancy creation.


\subsection{Shimer's  critique}
%As noted above, Shimer's (2005) critique is that for common
%calibrations of the standard matching model,
%the elasticity of market tightness
%%with respect to productivity
%is too low to explain business cycle fluctuations.
Shimer  (2005) observed that  the average job finding rate
$\theta\, q(\theta)$ is large relative to the observed value of
the sum   of the net interest rate and the separation rate $(r+s)$. When
 combined with reasonable parameter values for a worker's
bargaining power $\phi$ and the elasticity of matching with
respect to unemployment $\alpha$, this  implies that the first factor
$\Upsilon^{\rm Nash}$ in expression \Ep{ML1_thetaelast},
is close to
its lower bound of unity. More generally,
the first
factor in \Ep{ML1_thetaelast} is bounded from above by $1/\alpha$.
Because reasonable values of the elasticity $\alpha$ imply an upper bound on
the first factor,  the second factor
$y/(y-z)$ in expression \Ep{ML1_thetaelast} becomes
critical for  generating movements in market tightness.
For values of leisure within a commonly assumed range
well below productivity, the second factor is not large
enough to generate the high volatility of market
tightness associated with observed business cycles. This is  Shimer's critique.


Shimer (2005, pp.\ 39-40)  documented that comparisons of steady states described by   expression
\Ep{ML1_thetaelast}
provide a good approximation to average outcomes from
simulations of an economy subject to aggregate productivity shocks.
% Anticipating that  good  approximations will also  prevail in other
% matching frameworks,
Inspired by his finding, we will
derive  steady states under some alternative specifications. These
will shed light on properties of stochastic simulations
to be reported in section \use{sec:FS_simulations}.     %later sections.





\subsection{Relationship to worker's outside value}
%The match surplus is the capitalized surplus accruing to a firm
%and a worker in a current match.  It is the
%difference between the present value of the match and the sum of
%the worker's outside value and the  firm's outside value.
By rearranging equation \Ep{U_eq} and imposing the first
Nash-bargaining outcome of equations \Ep{split}, $E-U=\phi S$,
the worker's outside value can be expressed as
$$
U \;=\; \frac{z}{1-\beta} \,+\, \frac{\beta}{1-\beta}
\theta q(\theta)\,\phi S
\;=\; \frac{z}{1-\beta}\,+\,\Psi^{\rm m.surplus}_u
                             \,+\,\Psi^{\rm extra}_u\,,
                                                    \EQN U_composition
$$
where the second equality decomposes $U$ into three nonnegative parts: (i) the
capitalized value of choosing leisure in all future periods,
$z(1-\beta)^{-1}$; (ii)  the sum % $\Psi^{\rm m.surplus}_u$
of the discounted values of the worker's share of match
surpluses in his or her as yet unformed future
matches\NFootnote{Let $\Psi^{\rm m.surplus}_n$ be the
analogous capital value of an employed worker's share of all
match surpluses over  lifetime,
including  current
employment. The capital values $\Psi^{\rm m.surplus}_u$
and $\Psi^{\rm m.surplus}_n$ solve the Bellman equations
$$\EQNalign{
\Psi^{\rm m.surplus}_u\;&=\; 0\,+\, \beta \Bigl\{\theta q(\theta)
\Psi^{\rm m.surplus}_n \,+\, \left[1-\theta q(\theta)\right]
\Psi^{\rm m.surplus}_u \Bigr\}\,,
                                                    \cr
\Psi^{\rm m.surplus}_n\;&=\; \psi\,+\, \beta \Bigl\{(1-s)
\Psi^{\rm m.surplus}_n \,+\, s \Psi^{\rm m.surplus}_u \Bigr\}\,,
                                                     \cr}
$$
where $\psi$ is an annuity that, when paid for
the duration of a match, has the same expected present value as
a worker's share of the match surplus, $E-U=\phi S$:
$$
\sum_{t=0}^{\infty} \beta^t (1-s)^t \psi \;=\; \phi S
\qquad \Longrightarrow
\qquad \psi\,=\; (r+s) \beta \phi S  \,.
$$}
$$
\Psi^{\rm m.surplus}_u \;=\;
\frac{r+s}{r} \, \frac{\theta\,q(\theta)}{r+s\,+\,\theta\,q(\theta)}
\, \phi S\,;
                                                 \EQN U_part2
$$
 and, key to our new perspective, (iii)
the  parts %$\Psi^{\rm extra}_u$
of fundamental surpluses
from future employment matches that are not allocated to match
surpluses
$$
\Psi^{\rm extra}_u \;=\; \frac{\theta q(\theta)}{r+s}
\, \Psi^{\rm m.surplus}_u \,,
                                                  \EQN U_part3
$$
which can be deduced from equation \Ep{U_composition}
after replacing $\Psi^{\rm m.surplus}_u$ with
expression \Ep{U_part2}.

We can use decomposition \Ep{U_composition} of a worker's outside
value $U$
%in equation (\ref{U_composition})
to shed light on the activities  of the
`invisible hand' that make the elasticity of market
tightness with respect to productivity be low for
common calibrations of  matching models.
Those parameter settings entail a value
of leisure $z$ well below productivity  and a
significant share $\phi$ of match surpluses being awarded to workers,
which together with  a high job finding
probability $\theta q(\theta)$ imply that the sum
$\Psi^{\rm m.surplus}_u +\Psi^{\rm extra}_u$ in equation
\Ep{U_composition} forms a substantial part of a
worker's outside value. Furthermore, $\Psi^{\rm extra}_u$
is the much larger term in that sum, which follows from
expression \Ep{U_part3} and the assumption that $\theta q(\theta)$
is large relative to $r+s$. That big term
$\Psi^{\rm extra}_u$ makes it easy for the
invisible hand to realign a worker's outside value in a
way that leaves the match surplus almost unchanged when
productivity changes. Offsetting changes
in $\Psi^{\rm extra}_u$ can absorb the impact of productivity
shocks so that resources devoted to vacancy creation can remain
almost unchanged, which in turn explains why unemployment
does not respond sensitively to productivity.

But in Hagedorn and Manovskii's (2008)  calibration with a high value of
leisure,
the fundamental-surplus components of a worker's outside
value are so small that there is little room for the invisible hand to  realign things as we have described,
making the equilibrium amount
of resources allocated to vacancy creation respond
sensitively to variations in productivity.
That  results in a high
elasticity of market tightness with respect to productivity. Put
differently, since the fundamental surplus is a part of
productivity, it follows that a given change
in productivity translates into  a greater percentage change in
the fundamental surplus by a factor of $y/(y-z)$, i.e., the
inverse of the fundamental surplus fraction. Thus, the small
fundamental surplus fraction in  calibrations
like Hagedorn and Manovskii's having high values of leisure imply large percentage
changes in the fundamental surplus. Such large changes in the
amount of resources that could potentially be used for
vacancy creation cannot be offset by the invisible hand
and hence variations in productivity lead to
large variations in vacancy creation, resulting in a high
elasticity of market tightness with respect to
productivity.\NFootnote{It is instructive to consider a
single perturbation, $\phi=0$, to common calibrations of the
standard matching model, for which a worker's outside
value in expression \Ep{U_composition} solely equals
the capitalized value of leisure and the worker receives no
part of fundamental surpluses,
$\Psi^{\rm m.surplus}_u +\Psi^{\rm extra}_u=0$.
What explains  that the elasticity of market tightness with respect to
productivity remains low for such perturbed parameter settings in which
large fundamental surpluses  end up  affecting only  firms' profits that
in  equilibrium are all used for vacancy creation? The answer
lies precisely in the outcome that firms' profits would then be
truly large; therefore, even though variations in productivity then
affect firms' profits directly, the percentage wise impact of
productivity shocks on such huge profits is negligible,  so market
tightness and unemployment  hardly changes. This shows that
decomposition \Ep{U_composition} of a worker's outside value can
only go so far to shed light on the sensitivity of market tightness to
changes in productivity, because what ultimately matters is
evidently the size of the fundamental
surplus fraction in expression \Ep{ML1_thetaelast}.}
%For further
%discussion of profits and fundamental surpluses, see
%section \ref{sec:profits}.}

%Having described  how the fundamental surplus
%supplements the concept of a worker's outside value, we now tell

\subsection{Relationship to match surplus}
How does the fundamental surplus relate to the match surplus?
The fundamental
surplus is an upper bound on resources that the invisible hand
can allocate to vacancy creation.  Its magnitude as a
fraction of output is the prime determinant of the
elasticity of market tightness with respect to productivity.\NFootnote{We express the fundamental
surplus as a flow value while the match surplus is typically a
capitalized value.}
%The smaller the
%fundamental surplus fraction the larger is the elasticity of
%market tightness.
In contrast, although it
%too measures resources that might be
is directly connected to resources that are
devoted to vacancy creation,  match surplus that is small
relative to output
has no direct bearing on the elasticity of market tightness.
Recall that in the standard matching model, the zero-profit
condition for vacancy creation implies that the expected present
value of a firm's share of match surpluses  equals
the average cost of filling a vacancy. Since common
calibrations award firms a significant share of
match surpluses,  and since vacancy cost expenditures are calibrated
to be relatively small,
%relative to the output of a filled job,
it follows that
equilibrium match surpluses must form  small parts of output
across various matching models, regardless of the elasticity
of market tightness in any particular model.

%%From an accounting perspective,
%A fundamental surplus yields both  a match surpluses and firms' profits.
%% emerge from fundamental surpluses.
Fundamental surpluses yield match surpluses, which in turn include
firms' profits.
A small fundamental surplus
fraction necessarily implies small match surpluses and small firms' profits.
But small match surpluses and small firms' profits don't necessarily  imply
small fundamental surpluses. Therefore,   the size of the fundamental surplus fraction
is the only reliable indicator
of  the magnitude
of the elasticity of market tightness with respect to productivity, a situation  conveyed by expression \Ep{ML1_thetaelast}.






%Dynamics that are intermediated through the fundamental
%surplus occur in other popular
%setups, including those with sticky wages, alternative
%bargaining protocols and costly acquisition of credit.
%For example, it  matters little
%if the source of  a diminished fundamental surplus fraction
%is Hagedorn and Manovskii's (\citeyear{HagedornManovskii})
%high value of leisure for workers, Hall's (\citeyear{Hall2005})
%sticky wage, Hall and Milgrom's (\citeyear{HallMilgrom})
%cost of delay for firms that participate in  alternating-offer bargaining,
%or Wasmer and Weil's (\citeyear{WasmerWeil}) upfront cost for firms
%to secure credit.
%A small fundamental surplus fraction
%causes variations in productivity to have large effects on
%resources devoted to vacancy creation either because  workers
% insist on being compensated for their  losses of leisure,
%or because  firms  have to pay the sticky wage, or because workers
%  strategically exploit the firm's cost
%of delay under an alternating-offer bargaining protocol,
%or because firms must bear the cost of acquiring credit.


\subsection{Fixed matching cost}
Pissarides (2009) contributed what for us is  another good laboratory in which to study the pervasive role
of the fundamental surplus  when he argued that fixed
matching costs increase the elasticity of market
tightness with respect to productivity. So in addition to a vacancy posting cost $c$ per period,
we now assume that a firm incurs a fixed cost $H$ when matching
with a worker. Our job is  to verify that the addition of these
 costs diminishes the fundamental surplus fraction.


Under the assumption that a fixed matching cost $H$ is
incurred after the firm and the worker have bargained over the
consummation of a match (e.g., a training cost before work
commences),\NFootnote{For the alternative assumption that
the firm incurs the fixed matching cost before
bargaining with the worker, as well as for analyses of layoff
costs upon separation, see Ljungqvist and Sargent (2017).}
the match surplus $S$ becomes
$$
S \;=\; \left\{\sum_{t=0}^\infty \beta^t (1-s)^t
        \left[ y - (1-\beta) U \right] \right\} \,-\, H\,
  \;=\; \frac{y-(1-\beta)U-(1-\beta(1-s))H}{1-\beta(1-s)}\,.
                                               \EQN fixed_a_S
$$
By Nash bargaining, the firm receives $S_f$ and the worker $S_w$:
% of
%that match surplus, as given by
$$
S_f = (1-\phi)S \hskip1cm \hbox{\rm and} \hskip1cm S_w = \phi S .
                                               \EQN fixed_a_splitS
$$
A worker's value as unemployed is
$$
U \;=\; z \,+\, \beta \left[\theta q(\theta) S_w + U\right],
$$
which by using \Ep{fixed_a_splitS} can be rearranged to
$$
U \;=\; \frac{z \,+\, \beta \theta q(\theta)
         \frac{\displaystyle \phi}{\displaystyle 1-\phi} S_f}
         {1-\beta}\,.                             \EQN fixed_a_U
$$
Equations \Ep{fixed_a_S}, \Ep{fixed_a_splitS} and
\Ep{fixed_a_U} imply that  a firm's match surplus satisfies
$$
S_f \;=\; (1-\phi)\,
          \frac{y-z-\beta(r+s)H}
               {\beta(r+s)+ \beta\theta q(\theta) \phi}\,,
                                                  \EQN fixed_a_Sf
$$
where we have used $\beta=(1+r)^{-1}$ and $1-\beta(1-s)=\beta(r+s)$.

A firm's match surplus must also satisfy the zero profit
condition for vacancy creation:
$$
c\;=\; \beta q(\theta) S_f  \hskip.75cm \Longrightarrow \hskip.75cm
S_f \;=\; \frac{c}{\beta q(\theta)}\,.     \EQN fixed_a_zeroprofit
$$
Expressions \Ep{fixed_a_Sf} and \Ep{fixed_a_zeroprofit}
for a firm's match surplus imply that  the equilibrium
 $\theta$ satisfies
$$
\frac{1-\phi}{c} [y-z -\beta (r+s) H] \;=\;
\frac{r+s}{q(\theta)}\,+\,\phi\,\theta\,.
                                                \EQN fixed_a_theta
$$
Paralleling the steps of implicit differentiation in section
\use{sec:ML1_Nash_elasticity},
we arrive at the elasticity of
market tightness with respect to productivity for the model with a fixed matching cost:
     %%(\ref{fixed_a}).
$$
\eta_{\theta,y}
%% \;\equiv\; {\partial \theta \over \partial y} \; {y \over \theta}
\;=\; \Upsilon^{\rm Nash} \;
        {y \over y-z-\beta (r+s) H} \,.           \EQN fixed_a
$$
The only difference between the elasticity of market tightness with
a fixed matching cost \Ep{fixed_a} and the
earlier expression \Ep{ML1_thetaelast} without such a cost
is the additional term  $\beta (r+s) H$ that is deducted from the fundamental surplus.
So long as the firm continues to operate, this is an annuity payment $a$ having the same expected present
value as the fixed matching cost:
$$
\sum_{t=0}^{\infty} \beta^t (1-s)^t a \;=\; H
\qquad \Longrightarrow
\qquad a\,=\; [1-\beta(1-s)]H \,=\, \beta (r+s) H\,.
$$
%where the flow of annuity payments on the left  side of the first
%equation starts in the first
%period of operating and ceases when the job is destroyed.
The ``invisible hand'' cannot allocate  those resources
to vacancy creation, so it is appropriate to subtract this annuity value
when computing the fundamental surplus.


We have thus  reaffirmed Pissarides's (2009) insight that  the addition of a
fixed matching cost increases the elasticity
of market tightness and shown how the effect works through the fundamental surplus.  In addition,  our analysis thus adds the insight that the
  quantitative effect coming from  that fixed cost is
inversely related to the  size of the fundamental
surplus fraction.




\subsection{Sticky wages}
The standard assumption of Nash
bargaining in matching models is  one way to determine a
wage, but not the only one.  Matching frictions create  a
range of wages that a firm and worker both prefer
 to breaking  a match.  Hall noted that  a constant wage  can
be consistent with no private inefficiencies in
contractual arrangements within a matching model.  That motivated
Hall (2005) to assume sticky wages, in the form of  a constant wage in his main
analysis, as a way of responding to the Shimer critique.   Hall
posited a `wage norm'  $\hat w$
inside the Nash bargaining set that must be paid to workers.
Here we show that an appropriately defined fundamental surplus
fraction determines how does such a constant wage affects the elasticity of market
tightness with respect to productivity.

Given a constant wage $w=\hat w$, an equilibrium is
characterized by the zero-profit condition for vacancy creation
in expression \Ep{wage1} of the standard matching model
$$
\hat w\;=\; y \,-\, {r+s \over q(\theta)} c \,.      \EQN ML1_fixwage
$$
There exists an equilibrium for any constant wage
$\hat w \in [z, y-(r+s)c]$. The lower bound
is a worker's utility of leisure and the upper bound
is determined by the zero-profit condition for vacancy
creation evaluated at the  point where the  probability of a firm filling a vacancy
is at its maximum value of $q(\theta)=1$.
%Rearrange expression \Ep{ML1_fixwage} to get
%$$
%q(\theta)\, (y -\hat w) \;=\; (r+s)\, c
%$$
%and
%implicitly  differentiate to get
%\$$
%{d\, \theta \over d\, y} \;=\; - {
%q(\theta) \over q'(\theta) (y -\hat w) }
%\;=\; {\theta \over \alpha (y -\hat w) }\,,    \EQN ML1_thetadiff2
%$$
%where the second equality follows from invoking
%the constant elasticity of matching with respect to unemployment,
%$\alpha=-q'(\theta)\, \theta / q(\theta)$.
After implicitly differentiating \Ep{ML1_fixwage},
we can compute the elasticity of market tightness as
$$
\eta_{\theta,y}
%% \;\equiv\; {\partial \theta \over \partial y} \; {y \over \theta}
\;=\; { 1 \over \alpha } \;\; {y \over y-\hat w}
\;\equiv\; \Upsilon^{\rm sticky}\; {y \over y-\hat w}\,.
                                                \EQN ML1_thetaelast2
$$
This equation resembles the earlier one for $\eta_{\theta,y}$  in
\Ep{ML1_thetaelast}. Not surprisingly, if the constant wage
 equals the value of leisure, $\hat w = z$, then the
elasticity \Ep{ML1_thetaelast2} is equal to that earlier
elasticity of market tightness in the standard
matching model with Nash bargaining when the worker has a zero
bargaining weight, $\phi=0$. With such lopsided
bargaining power, the equilibrium wage would indeed be the constant
value $z$ of leisure.

This outcome reminds us that the first factor in
expression \Ep{ML1_thetaelast} can play only a limited role in magnifying
the elasticity $\eta_{\theta,y}$ because it is bounded from above by the inverse of
the elasticity of matching with respect to unemployment, $\alpha$.   In
\Ep{ML1_thetaelast2},  the upper bound is attained.
So again it is the second factor, the inverse of the fundamental
surplus fraction, that tells whether the elasticity
of market tightness is high or low. The pertinent definition of
the fundamental surplus is now the difference between productivity and
the stipulated constant wage.

In Hall's (2005) model, all
of the fundamental surplus goes to vacancy creation (as also occurs  in the
standard matching model with Nash
bargaining when  the worker's
bargaining weight is zero). A given percentage
change in productivity is multiplied  by a factor  \hbox{$y/(y-\hat w)$}
to become a larger percentage
change in the fundamental surplus.
Because all of the fundamental surplus now goes to vacancy creation, there
is a correspondingly magnified impact on unemployment. Numerical simulations of economies with aggregate
productivity shocks in section \use{sec:FS_Hall_simul} reaffirm this interpretation.



\subsection{Alternating-offer wage bargaining}
           \label{sec:HallMilgrom}%
Hall and Milgrom (2008) proposed yet another response to the Shimer critique.
Instead of Nash bargaining, a firm and a worker take
turns making  wage offers. The threat is not to break
up and receive  outside values, but instead to continue to bargain
because that choice has a strictly higher payoff than accepting
the outside option. After each unsuccessful bargaining round, the
firm incurs a cost of delay $\gamma > 0$ while the worker enjoys
the value of leisure $z$. There is also a probability $\delta$ that
the job opportunity is exogenously destroyed between bargaining rounds, sending  the
worker to the unemployment pool.

It is optimal for both bargaining  parties  to make  barely
acceptable offers. The firm always offers $w^f$ and
the worker always offers $w^w$. Consequently, in an equilibrium, the
first wage offer is accepted. Hall and Milgrom assume that firms make the first
wage offer.

 Hall and Milgrom (2008, p.~1673)
chose to emphasize that ``the limited influence of unemployment
[the outside value of workers] on the wage results in large
fluctuations in unemployment under plausible movements in
[productivity].'' It is more accurate to emphasize
that  the key
force is actually  that an appropriately defined fundamental surplus fraction has to be
calibrated to be small. Without a small fundamental surplus
fraction, it matters little that the outside value has been
prevented  from influencing  bargaining. To  illustrate this, we  compute the
elasticity of market tightness with respect to productivity and look under the hood.

After a wage agreement, a firm's value of a filled job, $J$, and
the value of an employed worker, $E$,
are still  given
by expressions \Ep{J_eq} and \Ep{E_eq} in the standard
matching model. These  can be rearranged to become
$$\EQNalign{
E \;&=\; {w \,+\, \beta \,s\,U \over 1-\beta(1-s)} \,,
                                               \EQN ML1_employ \cr
\noalign{\vskip.2cm} \cr
J \;&=\; {y - w \over 1-\beta(1-s)} \,,        \EQN {ML1_job}  \cr}
$$
where we have imposed a zero-profit condition  $V=0$ on vacancy
creation in the second expression. Thus, using
expression \Ep{ML1_employ}, the indifference condition for
a worker who has just received a wage offer $w^f$ from the firm and is choosing
whether  to decline  the offer and wait until the next period
to make a counteroffer $w^w$ is
$$
{w^f \,+\, \beta \,s\,U \over 1-\beta(1-s)}
 \;=\; z\,+\, \beta \left[(1-\delta)
       {w^w \,+\, \beta \,s\,U \over 1-\beta(1-s)}
        \,+\, \delta\, U\right].                     \EQN ML1_employW
$$
Using expression \Ep{ML1_job}, the analogous condition for
a firm contemplating a counteroffer from the worker is
$$
{y - w^w \over 1-\beta(1-s)}  \;=\; -\gamma \,+\,
   \beta (1-\delta)\, {y - w^f \over 1-\beta(1-s)} \,.  \EQN ML1_jobW
$$


After collecting and simplifying the terms that involve the
worker's outside value $U$, expression \Ep{ML1_employW} becomes
$$
{w^f \over 1-\beta(1-s)}
 \;=\; z\,+\, \beta (1-\delta)\,{w^w \over 1-\beta(1-s)} \,+\,
 \beta \,{1-\beta \over 1-\beta(1-s) } \,(\delta-s)\,U.
                                               \EQN ML1_employWnew
$$
As emphasized by Hall and Milgrom, the worker's outside value $U$
has a small influence on bargaining: when  $\delta=s$, the
outside value disappears from expression \Ep{ML1_employWnew}. That
is, with  bargaining that  ends   either with an
agreement or with destruction of the job, the outside value will
matter only if  job destruction probabilities differ
before and after reaching an agreement. To strengthen
Hall and Milgrom's (2008) observation that under their bargaining protocol
the outside value has at most a
small influence, we proceed
under the assumption that $\delta=s$, which makes the two indifference
conditions \Ep{ML1_employWnew} and \Ep{ML1_jobW} become
$$\EQNalign{
w^f \;&=\; (1-\tilde \beta)\,z\,+\, \tilde \beta\, w^w
                                              \EQN ML1_employW2 \cr
\noalign{\vskip.2cm}  \cr
y - w^w \;&=\; -(1-\tilde \beta)\,\gamma \,+\,
        \tilde \beta \, (y - w^f)    \,,      \EQN ML1_jobW2   \cr}
$$
where $\tilde \beta \equiv \beta (1-s)$.
Solve for $w^w$ from \Ep{ML1_jobW2} and substitute into
\Ep{ML1_employW2} to get
%%\begin{equation}
%%w^f \;=\; (1-\tilde \beta)\,z \,+\,
%%      \tilde \beta\left[(1-\tilde \beta)(y+\gamma)
%%                      \,+\,\tilde \beta\,w^f\right], \label{ML1_wageF}
%%\end{equation}
%%which can be rearranged to read
$$
w^f \;=\; {(1-\tilde \beta)\left[ z\,+\,\tilde \beta(y+\gamma)\right] \over
           1-\tilde \beta^2 }
\;=\; { z\,+\,\tilde \beta(y+\gamma) \over 1+\tilde \beta }\,.
                                                   \EQN ML1_wageF2
$$
This is the wage that a firm would immediately offer a worker  when first matched;  the offer would
be accepted.\NFootnote{When firms
make the first wage offer, a necessary
condition for an equilibrium is that $w^f$ in expression
\Ep{ML1_wageF2} is less than productivity $y$, i.e., the parameters
must satisfy $z + \tilde \beta \gamma < y$.}
In an equilibrium, this
wage must also be consistent with the no-profit condition for vacancy
creation. Substitution of $w=w^f$ from expression
\Ep{ML1_wageF2} into the no-profit condition \Ep{wage1}
of the standard matching model results in the following
expression for equilibrium market tightness:
$$
{ z\,+\,\tilde \beta(y+\gamma) \over 1+\tilde \beta }  \;=\;
y \,-\, {r + s \over q(\theta)} \, c .       \EQN ML1_altbarg
$$
After implicit differentiation,
we can compute the elasticity of market tightness as
$$
%%\eta_{\theta,y}\Big|_{\rm first\, wage\, offer\, by\, firm}
\eta_{\theta,y}
%% \;\equiv\; {d\, \theta \over d\, y} \; {y \over \theta}
\;=\; {1 \over \alpha} \; \;
      {y \over y-z- \tilde \beta \,\gamma}
\;=\; \Upsilon^{\rm sticky}\;
         {y \over y-z- \tilde \beta \,\gamma}\,,  \EQN ML1_thetaelast3
$$
where the fundamental surplus is the productivity
that remains after making deductions for the value of leisure $z$
and a firm's discounted cost of delay $\tilde \beta \gamma$.
The latter item
captures the worker's prospective gains from his
 ability to exploit the
cost that delay imposes on the firm.  What remains of productivity is the
fundamental surplus that could potentially be allocated by the
`invisible hand'  to  vacancy creation in an equilibrium.

To summarize, the alternative bargaining
protocol of Hall and Milgrom (2008) does suppress the influence of the
worker's outside value during bargaining. But
this outcome would be irrelevant if  Hall and Milgrom had  not
calibrated a small fundamental surplus fraction.







\section{Business cycle simulations}\label{sec:FS_simulations}%
To illustrate that a small fundamental surplus fraction is essential for
generating ample unemployment volatility over the business cycle in
matching models, we use  Hall's (2005) specification
with discrete time and a  random productivity process.
The monthly discount factor $\beta$ corresponds to a 5-percent annual
rate and the value of leisure is $z=0.40$. The elasticity of
matching with respect to unemployment is $\alpha=0.235$, and the
exogenous monthly separation rate is $s=0.034$. Aggregate productivity
takes on five  values $y_s$ uniformly spaced around a mean of
one on the interval $[0.9935, 1.00565]$, and is governed by
a monthly transition probability matrix $\Pi$ with probabilities
that are zero except as follows: $\pi_{1,2}=\pi_{4,5}=2(1-\rho)$,
$\pi_{2,3}=\pi_{3,4}=3(1-\rho)$, with the upper triangle of the
transition matrix symmetrical to the lower triangle and the
diagonal elements equal to one minus the sums of the nondiagonal
elements. The resulting serial correlation of $y$ is $\rho$, which
is parameterized to be $\rho=0.9899$.
To facilitate the
sensitivity analysis, following Ljungqvist and Sargent (2017), we alter
Hall's model period from one month to one day.




\subsection{Hall's (2005) sticky wage}\label{sec:FS_Hall_simul}%
Following  Hall (2005),  we posit a
fixed wage $\hat w = 0.9657$, which  equals the flexible wage that
would prevail at the median productivity level under standard Nash
bargaining (with equal bargaining weights, $\phi=0.5$).
Figure \Fg{figMLIHall1} reproduces Hall's figures 2 and 4
for those two models. The solid line and the upper dotted line
depict unemployment rates at different productivities for the
sticky-wage model and the standard Nash-bargaining model,
respectively.\NFootnote{Unemployment is a state variable
that is not just a function of the current productivity, as are
all of the other variables, but depends on the history of the
economy. But high persistence of productivity and the
high job-finding rates make the unemployment rate that is  observed at
a given productivity level be well approximated by expression
\Ep{unemp}
evaluated at the market tightness $\theta$ prevailing at that
productivity (see Hall (2005, p.~59)).}
Unemployment is almost invariant to productivity under Nash bargaining
but responds sensitively under the sticky wage. These outcomes are
explained by differences in job-finding rates, as shown by the
dashed line and the lower dotted line for the
sticky-wage model and the standard Nash-bargaining model,
respectively, expressed at our daily frequency.\NFootnote{Our
daily job-finding rates are roughly $1/30$ of the monthly
rates in Hall (2005, figures 2 and 4),
confirming our conversion from a monthly to a daily frequency.}
Under the sticky wage, high productivities cause firms to post many
vacancies, making it easy for unemployed workers to find
jobs, while the opposite is true when productivity is low.

\midfigure{figMLIHall1}
\centerline{\epsfxsize=3truein\epsffile{MLI_Hall1.eps}}
\caption{Sticky-wage model. Unemployment rates and daily job-finding
rates at different productivities (given a fixed wage $\hat w = 0.9657$),
where the dotted lines with almost no slopes are counterparts from
a standard Nash-bargaining model.}
\infiglist{figMLIHall1}
\endfigure

%Starting from this verification of our conversion of Hall's
%(2005) model into a daily frequency,
We conduct a sensitivity analysis of the choice of the fixed wage.
The solid line in Figure \Fg{figMLIHall2} shows how
the average unemployment rate varies with the fixed wage $\hat w$.
A small set  of wages spans outcomes ranging from
%nearly zero to very high average unemployment.
             %%REMARK: We truncated the graph so the lowest
             %%unemployment rate is two percent.
 very low to  very high average unemployment rates.
Small variations in a fixed wage close to
productivity generate large changes in the fundamental surplus
fraction, $(y-\hat w)/y$.  Free entry of firms makes that map
directly into the amount of resources devoted to vacancy creation.
The dashed line in Figure \Fg{figMLIHall2}
delineates  implications
for the volatility of unemployment. The standard deviation of
unemployment is nearly zero at the left end of the graph where the
job-finding probability is almost one for all productivity levels.
Unemployment volatility then increases for higher constant wages
until, outside of the graph at the right end, vacancy creation
becomes so unprofitable that average unemployment converges to its
maximum of 100 percent, causing there to be no more fluctuations.



\midfigure{figMLIHall2}
\centerline{\epsfxsize=3truein\epsffile{MLI_Hall2.eps}}
\caption{Sticky-wage model. Average unemployment rate and standard
deviation of unemployment for
different postulated values of the fixed wage.}
\infiglist{figMLIHall2}
\endfigure


At Hall's fixed wage $\hat w = 0.9657$, Figure \Fg{figMLIHall2}
shows a standard deviation of unemployment equal to
1.80 percentage points, which is close to the target of 1.54 to which
Hall (2005) calibrated his model.


\subsection{Hagedorn and Manovskii's (2008) high value of leisure}
           \label{sec:FS_HagMan_simul}%
 It turns out that  by elevating the value of
leisure,
the standard Nash-bargaining model can attain the
same volatility of unemployment as does the sticky wage model of the previous subsection.
To illustrate this, we use  Hall's (2005) parameterized environment
but
now simply assume standard Nash wage bargaining in order to study
 Hagedorn and Manovskii's (2008) analysis of the consequences of
  positing a high value $z=0.960$ of leisure  and a low
bargaining power of workers  $\phi=0.0135$.
%%who focus on a high value
%%of leisure, $z=0.955$, and a low bargaining power of workers,
%%$\phi=0.052$. At such a parameterization,
These parameter values imply a
high standard deviation of 1.4 percentage points for unemployment.
Figure \Fg{figMLINash1}, which depicts outcomes for different
constellations of $z\in [0.4, .99]$ and $\phi\in[0.001, 0.5]$, sheds light
 on the sensitivity of outcomes to  the choice of parameters.
To construct the figure, for each pair $(z,\phi)$, we adjusted the efficiency parameter $A$
of the matching function to make the average unemployment rate stay at
5.5 percent. Because it implies a
a small fundamental surplus fraction, a high value of leisure is essential for obtaining large variations in market tightness and a high volatility of unemployment.

%As detailed in section \ref{sec:HagedornManovskii}
%(and elaborated upon in online appendix \ref{app:wageelasticity}),


To match the elasticity of wages with respect to productivity,
Hagedorn and Manovskii (2008) require a low bargaining power for workers.
%If we use the Hall (2005) parameter values and Hagedorn and Manovskii's
%$(z,\phi)=(0.960, 0.0135)$ in the otherwise same  parametrized
%environment of Hall (2005),
Given the above parameterization with $(z,\phi)=(0.960, 0.0135)$,
we obtain a  wage elasticity
of 0.44, which  is approximately the value that
Hagedorn and Manovskii had targeted. To conduct a sensitivity analysis to variations in $z$ and $\phi$,
Figure \Fg{figMLINash2}
employs   the same computational approach underlying Figure \Fg{figMLINash1}.
  The figure confirms
that  a low $\phi$ is required to obtain a low wage
elasticity.\NFootnote{Note that the
axes in Figure \Fg{figMLINash2} are rotated relative to
Figure \Fg{figMLINash1}, for easy viewing of the relationship.}



Taken together, Figures \Fg{figMLINash1} and \Fg{figMLINash2}
seem to settle a difference of opinions in favor of  Hagedorn and Manovskii (2008, p.~1696), who argued
that ``the volatility of labor market tightness is almost independent
of [$\phi$] and is determined only by the level of $z$.''
Rogerson and Shimer (2011, p.~660) apparently disagreed when they instead  emphasized
that wages are rigid under the calibration of ``Hagedorn and
Manovskii (2008), although it is worth noting that the authors
do not interpret their paper as one with wage rigidities. They
calibrate ... a small value for the workers' bargaining power
[$\phi$]. This significantly amplifies productivity shocks ...''
But Figures \Fg{figMLINash1} and \Fg{figMLINash2} indicate that  the low wage elasticity of
Hagedorn and Manovskii (2008)
is incidental to and neither necessary nor sufficient to obtain
a high volatility of unemployment. We suggest that instead of stressing the importance of a rigid wage, as Rogerson and Shimer
did, what should be concluded is the general principle that
the fundamental surplus fraction must be small in order to amplify business cycle responses to productivity changes.

\midfigure{figMLINash1}
\centerline{\epsfxsize=3truein\epsffile{MLI_Nash1.eps}}
\caption{Nash-bargaining model. Standard deviation of unemployment in
percentage points for different
constellations of the value of leisure $z$, and the bargaining
power of workers $\phi$.}
\infiglist{figMLINash1}
\endfigure

\midfigure{figMLINash2}
\centerline{\epsfxsize=3truein\epsffile{MLI_Nash2.eps}}
\caption{Nash-bargaining model. Wage elasticity with respect to
productivity for different
constellations of the value of leisure $z$, and the bargaining
power of workers $\phi$. (Note that the axes are rotated relative
to Figure \Fg{figMLINash1}.)}
\infiglist{figMLINash2}
\endfigure


\subsection{Hall and Milgrom's (2008) alternating-offer bargaining}
           \label{sec:FS_HallMil_simul}%
 Hall and Milgrom's (2008)
model of alternating-offer wage bargaining is another way to
increase unemployment volatility. Except for
the wage formation process, their environment is  Hall's (2005). But it is  parameterized differently.
One difference between  Hall and Milgrom's parameterization and  Hall's (2005) plays  an especially  important role in setting  the
 fundamental surplus: Hall and Milgrom' raised the value of leisure  to $z=0.71$ from Hall's value of $z= 0.40$.
 Section \use{sec:HallMilgrom} taught us that
the values of leisure and of  the firm's cost of delay in
bargaining $\gamma$ are likely to be critical determinants of
 the elasticity of market tightness with respect
to productivity and hence of the volatility of unemployment.

But that is not what
Hall and Milgrom (2008) chose to emphasize. Instead, they stressed how much the
outside value of unemployment is suppressed in alternating-offer
wage bargaining since disagreement no longer leads to
unemployment but instead to another round of bargaining. So
a key parameter from Hall and Milgrom's perspective
is the exogenous rate $\delta$ at which parties break up
between bargaining rounds. Figure \Fg{figMLIHallMilgrom} shows
how different constellations of $(\gamma, \delta)$ affect the
standard deviation of unemployment.  %% in percentage points.
For each pair $(\gamma,\delta)$, we adjust the efficiency
parameter $A$ of the matching function to make the average
unemployment rate stay at 5.5 percent. Because
Hall and Milgrom (2008) argued that productivity shocks are not
the sole source of unemployment fluctuations and consequently they lower their target  standard deviation
of unemployment to 0.68 percentage points -- a target
attained with their parameterization
$(\gamma, \delta)=(0.27, 0.0055)$
%%$\gamma= 0.27$ and $\delta=0.0055$
and reproduced in Figure \Fg{figMLIHallMilgrom}.

Figure \Fg{figMLIHallMilgrom} supports our earlier finding
that the cost of delay $\gamma$ together with the value of
leisure $z$ are the keys to  generating higher volatility
of unemployment. Without a cost of delay sufficiently high
to reduce the fundamental surplus fraction, the exogenous separation rate between
bargaining rounds  matters little.\NFootnote{To be specific, our formula
\Ep{ML1_thetaelast3} for the steady-state comparative statics
is an approximation of the elasticity of market tightness at
the rear end of Figure \Fg{figMLIHallMilgrom} where
the exogenous rate $\delta$ at which parties break up
between bargaining rounds is equal to Hall and Milgrom's
(2008) assumed job destruction rate of $0.0014$ per day.}

\midfigure{figMLIHallMilgrom}
\centerline{\epsfxsize=3truein\epsffile{MLI_HallMilgrom.eps}}
\caption{Alternating-offer bargaining model. Standard deviation of unemployment in percentage
points for different constellations of firms' cost of delay
$\gamma$ in bargaining  and the exogenous separation rate
$\delta$ while bargaining.}
\infiglist{figMLIHallMilgrom}
\endfigure

While Hall and Milgrom (2008, p.~1670) notice that their ``sum
of $z$ and $\gamma$ is $\ldots$ not very different from the value
of $z$ by itself in $\ldots$ Hagedorn and Manovskii's
calibration'' (as studied in our section \use{sec:FS_HagMan_simul}),
they downplayed this  similarity and chose to emphasize
  differences in mechanisms across Hagedorn and Manovskii's model and theirs.
But   focusing on the  fundamental surplus tells us that it is
their similarity that should
be stressed.   Hall and Milgrom's and Hagedorn and Manovskii's  models are united in requiring a small fundamental
surplus fraction  to generate high unemployment volatility
over the business cycle.


\subsection{Matching and bargaining protocols in a  DSGE model}
Christiano, Eichenbaum and Trabandt (2016) compare consequences
of assuming
alternative-offer bargaining (AOB) and Nash bargaining
in a dynamic stochastic general equilibrium (DSGE)
model with a matching function. They find that,  if they adjust structural parameters across the two models  to  fit the  data,
 models parameters re-estimated under   the two alternative assumptions are  able to account for
the data equally well. That  includes comparable performance in   generating observed  unemployment volatility. 
%They find that  the two
%assumptions about bargaining allow the
%model perform equally well in accounting for patterns in
%the data, including generating unemployment volatility.
The solid line in Figure \Fg{figMLI_CLT}
depicts responses of unemployment to a neutral technology
shock that are virtually identical across the two models.
But  beneath those those nearly identical responses there resides  a substantial difference in estimates a key parameter under
the two assumptions, namely,
the replacement rate from unemployment insurance, a parameter that
corresponds to our value of leisure $z$. They estimate a value of  0.37 under the AOB
model versus 0.88 with the Nash bargaining model.

%Christiano, Eichenbaum and Trabandt (2016) compare the implications
%of assuming alternative-offer bargaining (AOB) and Nash bargaining
%over a wage in  a DSGE framework with a matching function.
%The solid line in Figure \Fg{figMLI_CLT}
%depicts the  response of unemployment to a neutral technology
%shock, which is virtually identical across the two models.
%Though, a key difference in estimated parameter values is
%the replacement rate from unemployment insurance (that
%corresponds to our value of leisure $z$): 0.37 in the AOB
%model versus 0.88 in the Nash bargaining model.

\midfigure{figMLI_CLT}
\centerline{\epsfxsize=3truein\epsffile{MLI_HallMilgrom.eps}}
\caption{Impulse response of unemployment to a neutral
technology shock in the DSGE analyses. The solid line refers
to both estimated models of AOB and Nash bargaining,
respectively. The
dashed line refers to the perturbed models where parameter
values are cut in half for the replacement ratio, as well as
for a firm's cost to make a counteroffer in the AOB model.}
\infiglist{figMLI_CLT}
\endfigure


 Christiano et al.\ (2016, pp.~1551-1552) remark that 
   their high estimate of the value of leisure in the
Nash bargaining model ``$\ldots$  is reminiscent of Hagedorn
and Manovskii's (2008) argument that a high replacement ratio has
the potential to boost the volatility of unemployment''.\NFootnote{See  section \use{sec:FS_HagMan_simul} above.}
To elaborate on this point, Christiano et al. demonstrate  that if they restrict the replacement
rate in the Nash bargaining model to be the same as that of the
AOB model and then recalculate the impulse response functions, there occurs
a dramatic deterioration in the performance of the Nash
bargaining model. Thus,  the dashed line in Figure \Fg{figMLI_CLT}
show how  unemployment becomes much less responsive to
a neutral technology shock under that perturbation in the replacement rate.

Christiano et al.\ (2016, p.~1547) proceed to interpret  their low estimate of the value of leisure in the
AOB model as meaning that ``the replacement ratio does not play a critical role
in the AOB model's ability to account for the data.''
That  reasoning conceals that the fundamental surplus  is really at work once again. Christiano et al.
 generously conducted for us  what can be regarded as a reverse of their perturbation of the AOB model, namely, a cutting in half
of both the replacement rate 0.37 and a firm's cost of delay
in bargaining, where the latter in their model is a firm's
cost to make a counteroffer  calibrated to 0.6 of a firm's
daily revenue per worker.\NFootnote{Christiano et al.\ (2016)
assume that it takes one day for a wage offer to be extended, with a firm and a worker alternating in
making an offer.}
As  sections
\use{sec:HallMilgrom} and \use{sec:FS_HallMil_simul} lead us to expect, this
perturbation of the AOB model also brings a dramatic
deterioration in performance, as bad as that of the perturbed Nash
bargaining model: e.g., the dashed line depicting a
dampened impulse response of unemployment to a neutral
technology shock in Figure  \Fg{figMLI_CLT}
is virtually identical across the two perturbed models.
We conclude from this exercise that the replacement ratio is  critical
in the AOB model too, and that  what is needed to make the fundamental surplus fraction small
in that model
is a combination of very high
values of the replacement rate and a firm's cost of delay in
bargaining.




\section{Overlapping generations in one matching function}


\section{Directed search: age-specific matching functions}


\section{Block recursive equilibrium computations}

