\input grafinp3
%\input grafinput8
\input psfig

%\showchaptIDtrue
%\def\@chaptID{19.}
%\input gayejnl.txt
%\input gayedef.txt
\def\lege{\raise.3ex\hbox{$>$\kern-.75em\lower1ex\hbox{$<$}}}

%\hbox{}
\footnum=0
\chapter{Foundations of Aggregate Labor Supply\label{macrolaborII}}

%%\chapter{Macro labor II: From lotteries to careers\label{macrolaborII}}

\section{Introduction}

The  section \use{Empl_lottery} employment lotteries model   for years % of chapter \use{search2}
served as the foundation of the high aggregate labor supply
elasticity  that  generates big employment fluctuations   in   real business cycle models.
  In the original version of his Nobel prize lecture, Prescott (2005a) highlighted the central role of employment
lotteries  for real business cycle models when he asserted
that ``Rogerson's aggregation result is every bit as important as
the one giving rise to the aggregate production function.''
But Prescott's enthusiasm for employment lotteries has  not been shared universally, especially by
researchers who have studied labor market experiences of individual workers.  For example,
Browning, Hansen, and Heckman (1999)  expressed  doubts about the employment lotteries model
when they asserted  that ``the employment allocation
mechanism strains credibility and is at odds with the micro
evidence on individual employment histories.'' This chapter takes such criticisms of the employment lotteries
 to heart by
investigating how  the aggregate labor supply elasticity would be affected
were we to replace employment lotteries and complete markets for consumption insurance
with the incomplete markets arrangements that seem  more natural  to labor economists. This change reorients attention away from
the fraction of its members that a representative family chooses to send to work at any moment, to  {\it career lengths} chosen by individual workers who self-insure
by saving and dissaving.  We find that abandoning the employment lotteries coupled with
complete consumption insurance claims trading assumed within many real business cycle models and replacing them with individual workers who self-insure by trading
a risk-free bond does not by itself imperil that high aggregate labor supply elasticity championed by Prescott.  The labor supply elasticity
depends on whether shocks and
 government financed social security retirement schemes leave most workers on or off corners with respect to their retirement
decisions, in a model of indivisible labor.
\auth{Prescott, Edward C.}%
\auth{Browning, Martin}
\auth{Hansen, Lars P.}
\auth{Heckman, James J.}

During the last half decade, macroeconomists have mostly abandoned employment lotteries in favor of `time-averaging' and incomplete markets as an `aggregation' theory for aggregate labor supply.
This is undoubtedly a positive development because now researchers who may differ about the size of the aggregate labor supply elasticity can at least talk in terms of a common framework and can focus on their disagreements about the proper quantitative settings for  a commonly agreed on set of
 parameters and constraints.

%In continuous time when the subjective discount rate is equal to
%the market interest rate,
To convey these ideas, we  build on an analysis of  Ljungqvist and Sargent (2007), who in a particular
continuous time model  showed that
the very same aggregate allocation and individual (expected) utilities
that emerge from  a Rogerson-style  complete-market economy with employment lotteries
are also attained in an incomplete-market economy without lotteries.
In the Ljungqvist-Sargent setting, instead of trading  probabilities of working at any point in
time, agents choose fractions of their lifetimes to devote
to work and use a credit market to smooth consumption across
episodes of work and times of retirement.\NFootnote{Larry Jones and Casey Mulligan anticipated aspects of this  equivalence
result.
In the context of indivisible consumption goods, in the original 1988 version of his paper,
Jones (2008)
showed  how timing could replace lotteries when there
is no discounting. In the 2008 published version of his paper, he extended the analysis to cover
 the case of discounting.
In comparing an indivisible-labor complete-market model
and a representative-agent model with divisible labor, Mulligan (2001)
suggested that the elimination of employment lotteries and complete
markets for consumption claims from the former model might not make
much of a quantitative difference;
``The smallest labor supply decision has an infinitesimal effect
on lifetime consumption and the marginal utility of wealth in the
[divisible-labor] model, and a small-but-larger-than-infinitesimal
effect on the marginal utility of wealth in the [indivisible-labor]
model -- as long as the effect on lifetime consumption is a small
fraction of lifetime income {\it or} the marginal utility of wealth
does not diminish too rapidly.'' However, as we shall learn  later in this chapter, these
 qualifications vanish when time is continuous, as well as for infinitely-lived agents in discrete time. As a discussant of Ljungqvist and Sargent (2007), Prescott (2007) endorsed their
incomplete markets, career length model as a model of  aggregate labor supply.
In addition, he reduced his previous stress on the employment lotteries model by adding a new section, ``The life cycle and labor
indivisibility,'' to the final  version of  his Nobel lecture published in America
(Prescott 2006).}
\auth{Prescott, Edward C.}%


This chapter studies  how two camps of
researchers, namely, those who champion high and low labor supply elasticities,
respectively, both came to adopt the same theoretical framework.\NFootnote{This is the theme of Ljungqvist and Sargent (2011).}
The first part  of the chapter revisits equivalence
results between an employment lotteries model and a time-averaging
model, then pursues various extensions to the time-averaging setup as a model of  career
length determination. The second half of the chapter retraces the  steps that led  Chang and
Kim's (2007) to discover a high labor supply elasticity in
simulations of a Bewley with incomplete markets and indivisible
labor. Chang and Kim's agents optimally alternate between periods of work and leisure
(they `time average')   to  allocate consumption
and leisure over their infinite lifespans. The chapter concludes by studying
how Ljungqvist and Sargent's (2007) equivalence result in continuous
time with finitely-lived agents extends to a deterministic version
of Chang and Kim's (2007) discrete-time growth model inhabited by
infinitely-lived agents.

\auth{Chang, Yongsung}%
\auth{Kim, Sun-Bin}%
\auth{Ljungqvist, Lars}
\auth{Sargent, Thomas J.}





\section{Equivalent allocations}\label{sec:LSequi}%
Following Ljungqvist and Sargent (2007), consider an agent who
lives in continuous time with a deterministic
lifespan of unit length, and lifetime preferences given by
$$
\int_0^1 e^{-\rho t} \left[u(c_t) - v(n_t) \right]\,dt\,,
\EQN LSequi_utility
$$
where $c_t\geq 0$ and $n_t\in \{0,1\}$ are consumption and labor supply
at time $t$, respectively, and $\rho$ is his subjective discount rate.
That $n_t\in \{0,1\}$ asserts that labor
supply is indivisible. The instantaneous utility function over
consumption, $u(c)$, is strictly increasing, strictly concave, and
twice continuously differentiable. Since labor is
indivisible, we need  to specify only two points for the disutility
of work $v(n)$, so we normalize $v(0)=0$ and let $v(1)=B > 0$.

Until section \use{sec:ChangKim}, we  assume a given wage rate $w$ and a
given interest rate $r=\rho$.



\subsection{Choosing career length}\label{sec:LS_lifetime}%
At each point in time, an agent can  work at a wage rate
$w$ and can save or dissave at an interest rate $r$. An agent's asset
holdings at time $t$ are denoted by $a_t$ and its time derivative
by $\dot{a}_t$. Initial assets are assumed to be zero, $a_0=0$, and
the budget constraint at time $t$ is
$$
\dot{a}_t = r a_t + w n_t - c_t\,,
\EQN LSequi_bc
$$
with a terminal condition $a_1\geq 0$. This is a no-Ponzi scheme condition.

To solve the agent's optimization problem, we formulate the
current-value \idx{Hamiltonian}
$$
H_t = u(c_t) -B n_t + \lambda_t \left[ r a_t + w n_t - c_t \right]\,,
\EQN LSequi_Hamilton
$$
where $\lambda_t$ is the multiplier on constraint \Ep{LSequi_bc}. It is called
the costate variable associated with the state variable $a_t$.
First-order conditions with respect to $c_t$ and $n_t$, respectively, are:
$$\EQNalign{u'(c_t) - \lambda_t  &= 0\,,      \EQN LSequi_FOC;a \cr
\noalign{\vskip.3cm}
-B + \lambda_t w & \cases{
   < 0 \hskip.5cm &if $n_t=0$;  \cr
   = 0 \hskip.5cm &if indifferent to $n_t\in\{0,1\}$; \cr
   > 0 \hskip.5cm &if $n_t=1$.  \cr } \EQN LSequi_FOC;b \cr
}
$$
Furthermore, the costate variable obeys the differential equation
$$
\dot{\lambda}_t = \lambda_t \rho - {\partial H_t \over \partial a_t}
                = \lambda_t[\rho -r] .
\EQN LSequi_lambda
$$


When $r=\rho$, Ljungqvist and Sargent (2007) show that the solution
to this optimization problem yields the same lifetime utility as if
the agent  had access to employment lotteries and complete
insurance markets (including consumption claims that are contingent
on lottery outcomes). First, we note from equation \Ep{LSequi_lambda}
that when $r=\rho$ the costate variable is constant over time  and
hence, by equation \Ep{LSequi_FOC;a},
the optimal consumption stream is constant over time,
$c_t = \bar c$. Then after  invoking optimality condition
\Ep{LSequi_FOC;b}, there are  three possible cases with respect
to the agent's lifetime labor supply,
$$
-B + u'(\bar c) w \cases{
   < 0 \hskip.5cm &{\it Case 1}: $n_t=0$ for all $t$;  \cr
   = 0 \hskip.5cm &{\it Case 2}: indifference to $n_t\in\{0,1\}$ at any \cr
                  &\hskip1.3cm particular instance in time;          \cr
   > 0 \hskip.5cm &{\it Case 3}: $n_t=1$ for all $t$.  \cr }
                                                  \EQN LSequi_labor
$$
These three cases stand as  analogues to the three cases in
the  section
\use{Empl_lottery}  static model with employment lotteries. % in  of chapter \use{search2}.
The agent finds it
optimal  never to work and always to work in the first and third case,
respectively. The interesting case is the intermediate one in which
the agent is indifferent between work and leisure at any particular
instance in time. At such an interior solution for lifetime labor
supply, optimality condition \Ep{LSequi_labor} at equality determines
the optimal constant consumption stream,
$$
u'(\bar c) = {B \over w}\,. \EQN LSequi_cons
$$
Evidently, this is the counterpart to the consumption outcome
in the  employment lottery model.
When utility is  logarithmic in consumption,
the optimal consumption level in
expression \Ep{LSequi_cons} becomes
$$
\bar c = {w \over B}\,, \hskip1cm \hbox{\rm if   }
u(c)=\log (c)\,.                            \EQN LSequi_conslog
$$

While the agent is indifferent between work and leisure at any
particular instance in time,  he   cares about the integral of his work over his lifetime.  His lifetime labor
supply is determined by the agent's present-value budget constraint
at equality when financing the optimal constant consumption
stream in expression \Ep{LSequi_cons}. The present-value budget
constraint is obtained from budget constraint \Ep{LSequi_bc},
and the initial and terminal conditions for asset holdings,
$a_0=a_1=0$:
$$
w \int_0^1 e^{-rt} n_t\, dt = \bar c \int_0^1 e^{-rt} \, dt.
\EQN LSequi_PVbc
$$
Thus, the optimal plan has the agent working a fraction of
his lifetime, where the associated present value of labor income
is given by expression \Ep{LSequi_PVbc}. Many  streams of lifetime labor supply
 yield the same present value of labor income in expression
\Ep{LSequi_PVbc}. The agent is indifferent among
such alternative lifetime labor profiles because constancy of
the associated present value of labor income implies constancy
of the associated lifetime disutility of work in preference
specification \Ep{LSequi_utility} when $\rho=r$. Hence, the
agent is indeed indifferent about when he supplies his labor, as we
also inferred from the second case of \Ep{LSequi_labor}.

In subsequent sections \use{sec:LStaxsoc}--\use{sec:LSshocks},
we will assume that $r=\rho=0$, i.e., no discounting.
%we will not only assume equality of $\rho$ and $r$ but also that
%they are equal to zero, i.e., no discounting.
Under that assumption, the optimal fraction of a lifetime devoted to
work, as given by present-value budget constraint \Ep{LSequi_PVbc},
is the same regardless of when the agent supplies his labor,
$$
T \equiv \int_0^1 n_t\, dt = {\bar c \over w},        \EQN LSequi_career
$$
where $T$ denotes an agent's choice of career length. When
utility is logarithmic in consumption, equations
\Ep{LSequi_conslog} and \Ep{LSequi_career} determine the
optimal career length at an interior solution,
$$
T ={1 \over B}\,,  \hskip1cm \hbox{\rm if   } u(c)=\log (c)\,;
                                                   \EQN LSequi_careerlog
$$
where  for an interior
solution we require  that $B \geq 1$.

Next, we confirm that a corresponding employment lottery model
yields the same (expected) lifetime utility to an agent and support
the same set of aggregate allocations, i.e., the introduction of
lotteries and complete consumption insurance does not matter in
this economy.

%We will also focus on interior solutions to lifetime
%labor supply by assuming that the necessary parameter
%restrictions are satisified. For example, when utility is logarithmic
%in consumption, the parameter restriction is $B\geq 1$, and by
%equations \Ep{LSequi_conslog} and \Ep{LSequi_career},
%the optimal fraction of a lifetime devoted to work is
%$$
%\int_0^1 n_t\, dt = {1 \over B}\,, \hskip1cm \hbox{\rm if   }
%u(c)=\log (c)\,.                                \EQN LSequi_careerlog
%$$

%But first, we establish an equivalence result for the economy of this
%section. We show that the introduction of employment lotteries and
%complete markets for consumption insurance do not matter. Both the
%time averaging economy of this section and
%the employment lotteries economy of the next section, yield the
%same (expected) lifetime utility to an agent and support the same
%set of aggregate allocations.







\subsection{Employment lotteries}\label{sec:LS_lottery}%
Consider a continuum $j \in [0,1]$ of ex ante identical agents like
those in section \use{sec:LS_lifetime}. When markets are complete
and there are employment lotteries to overcome the nonconvexity in
labor supply, a decentralized market equilibrium is the solution to
a planner problem, in which the planner weights are
equal across all the ex ante identical agents. The planner chooses
a consumption and employment allocation $c_{jt}\geq 0$,
$n_{jt}\in\{0,1\}$ to maximize
$$
\int_0^1 \int_0^1 e^{-\rho t} \left[u(c_{jt}) - B n_{jt}
                                               \right]\,dt\,dj
\EQN LSequi_utility_family
$$
subject to
$$
\int_0^1 \int_0^1 e^{-r t} \Bigl[w n_{jt} - c_{jt}
                                          \Bigr]\,dt\,dj \geq 0.
\EQN LSequi_PVbc_family
$$
Here the planner can borrow and lend at the rate $r$
and send agents to work to earn the wage $w$.

The strict concavity of the utility function $u(\cdot)$
and our assumption that $r=\rho$ imply that  the planner sets a constant
consumption level across agents and across time,
$c_{jt}=\bar c$ for all $j$ and $t$. The planner exposes each
agent at time $t$ to a lottery that sends him to work with
probability $\psi_t\in [0,1]$. The planner chooses $\bar c$
and $\psi_t$ to maximize
$$
\int_0^1 e^{-\rho t} \left[u(\bar c) - B \psi_t \right]\,dt
\EQN LSequi_utility2_family
$$
subject to
$$
\int_0^1 e^{-r t} \Bigl[w \psi_t - \bar c \Bigr]\,dt \geq 0\,.
\EQN LSequi_PVbc2_family
$$
This problem resembles the `time averaging' problem of a
single agent in section \use{sec:LS_lifetime}. At an
interior solution, the optimal
constant consumption stream is once again given by
equation \Ep{LSequi_cons}, $u'(\bar c)=B/w$.
A multitude of employment
lotteries  can satisfy present-value budget constraint
\Ep{LSequi_PVbc2_family} to finance the optimal consumption
choice. Agents would be
indifferent among  all of those alternative lottery designs.
As before, identical present values of labor income for any
two labor supply schemes imply identical (expected)
lifetime disutilities of work for those two schemes since
$\rho = r$.\NFootnote{For example, at the beginnning of
time, the planner can randomize over a constant fraction
of agents $\bar \psi$ who are assigned to work for every
$t\in [0,1]$, and a fraction $1- \bar \psi$ who are asked to
specialize in leisure, where $\bar \psi$ is chosen to satisfy
the planner's intertemporal budget constraint
\Ep{LSequi_PVbc2_family}. An alternative arrangement would be, at
each time $t\in[0,1]$, the planner runs a lottery that sends
a time invariant fraction $\bar \psi$ to work and a fraction
$1- \bar \psi$ to leisure. Agents are indifferent between
these alternative lottery designs since they yield the
same expected lifetime disutility of work. \label{note:LS_lottery}}

This argument suffices to  establish the  equivalence of
aggregate allocations and expected utilities between the
incomplete-market economy in section \use{sec:LS_lifetime}
and the employment-lotteries, complete-market economy of the present section.
An agent's optimal consumption is uniquely determined
and identical across  the two economies. For a given
present-value of aggregate consumption, the same
aggregate present-value of labor income can be attained with
a multitude of intertemporal allocations for the aggregate
measure of employed agents. Each of those
alternative aggregate allocations is associated with
either an incomplete-market economy where individual agents
engage in time averaging or a complete-market economy with one of a variety of appropriate
 lottery designs. Since an
agent's expected disutility of work is the same under the
alternative implementations, it follows
that an agent's expected utility is the same in the two
economies.



\section{Taxation and social security}\label{sec:LStaxsoc}%
We study taxation and social security in a continuous-time
overlapping generations model. At each instance in time, there
is a constant measure of newborn ex ante identical agents like
those in section \use{sec:LS_lifetime} entering the economy.
Thus, the economy's population and age structure stay constant
over time. Our focus is not on the determination of intertemporal
prices in this overlapping generations environment with its
possible dynamic inefficiencies (see chapter \use{ogmodels}),
so we retain our small open economy assumption of an exogenously
given interest rate, which also implies a given
wage rate if the economy's production technology is constant
returns to scale in labor and capital.\NFootnote{In the
case of a constant-returns-to-scale Cobb-Douglas production
function, equation \Ep{LS_CK_wr;b} shows how the interest rate
in international capital markets determines the capital-labor
ratio in a small open economy, which in turn determines the wage
rate in \Ep{LS_CK_wr;a}.}

We assume that utility is logarithmic in consumption,
$u(c) = \log (c)$, and that there is no discounting, $r=\rho=0$.
%and keep that assumption until section \use{sec:ChangKim}.
The assumption of no discounting is inessential for most of
our results, and where it matters we will take note. The analytical
convenience is that the optimal career length is uniquely
determined and does not depend on the timing of an agent's
lifetime labor supply, as shown in expressions
\Ep{LSequi_career} and \Ep{LSequi_careerlog}.


%We let $T$ denote an agent's choice of career length. When
%utility is logarithmic in consumption, equations
%\Ep{LSequi_conslog} and \Ep{LSequi_career} determine the
%optimal career length at an interior solution,
%$$
%T \equiv \int_0^1 n_t\, dt ={1 \over B} ,      \EQN LS_socsec_LF
%$$
%where the implicit parameter restriction for an interior
%solution is that $B \geq 1$.

As emphasized by Prescott (2005), if labor income is taxed
and tax revenues are handed back lump sum to agents, a model
with indivisible labor and employment lotteries exhibits a large
labor supply elasticity. Under the equivalence result in
section \use{sec:LSequi}, we follow Ljungqvist and Sargent (2007)
and demonstrate that the same high labor supply elasticity arises
in the incomplete-market model where career lengths rather
than the odds of working in employment lotteries are
shortened in response to such a tax system.

In the spirit of  Ljungqvist and Sargent (2012), we offer a
qualification to the high labor supply elasticity in a model of
lifetime labor supply. When a government program such as social
security is associated with a large implicit tax on working beyond an
official retirement age, there might not be much of an effect of
taxation on career length for those agents who could be at a corner
solution, strictly preferring to retire at the official
retirement age.



\subsection{Taxation}\label{sec:LS_tax}%
If labor income is taxed at rate $\tau\in [0,1)$ and tax revenues
are not returned to agents as tranfers in any form, there would
be no effect on labor supply, for the same reason that equilibrium
career length \Ep{LSequi_careerlog} does not depend on the level of the
wage $w$. The reason is that income and substitution effects cancel
with variations in the net-of-tax wage rate under the assumption
that preferences are consistent with balanced growth. But if instead
all tax receipts are rebated lump sum to agents, the labor supply
elasticity will be large.

Let $x$ be the present value of lump-sum transfers that each agent
receives over his lifetime, as determined by the government budget
constraint
$$
\tau w T^\star  = x,                             \EQN LS_tax_govbc
$$
where $T^\star$ is the equilibrium career length. Note that given a
zero interest rate and a lifetime of unit length, $x$ is  the
instant-by-instant per capita lump-sum transfer that satisfies the
government's static budget constraint \Ep{LS_tax_govbc} as well as
the present value of total lump-sum transfers paid to an agent over
his lifetime.

As in section \use{sec:LS_lifetime}, an agent again chooses a unique
constant consumption $\bar c$, and is indifferent among alternative labor
supply paths that yield the particular present value of income that is
required to finance his consumption choice. Under the present
assumption of no discounting, all of those alternative labor supply
paths have the same career length, i.e, the same fraction of an
agent's lifetime devoted to work, $T = \int_0^1 n_t\, dt$. Hence,
an agent's optimization problem becomes
$$
\max_{\bar c, T} \Bigl\{ \log(\bar c) - BT \Bigr\}   \EQN LS_tax_utility
$$
subject to
$$\EQNalign{
&\bar c \leq (1-\tau) w T + x,            \EQN LS_tax_bc   \cr
&\bar c \geq 0,\;\; T\in[0,1].  \cr}
$$
Substitute budget constraint \Ep{LS_tax_bc} into the objective
function of \Ep{LS_tax_utility}, then
compute a first-order condition with respect to career length at
an interior solution,
$$
{(1-\tau)w \over (1-\tau)wT + x}-B =0.    \EQN LS_tax_FOC
$$
Substituting \Ep{LS_tax_govbc} into first-order condition
\Ep{LS_tax_FOC} shows that  equilibrium career length is
$$
T^\star(\tau) \equiv {1-\tau \over B}.    \EQN LS_tax_career
$$
We conclude that lifetime labor supply is highly elastic when labor
is indivisible. According to expression \Ep{LS_tax_career}, the
elasticity of lifetime labor supply with respect to the net-of-tax
rate ($1-\tau$) is equal to one.

The reader can verify that a model with employment lotteries yield
the same equilibrium consumption and the same (expected) lifetime utility of
an agent. For example, we can adopt the first example of a lottery design
in footnote \use{note:LS_lottery}, where the planner for each cohort
of newborn agents, administers
a lifetime employment lottery once and for all at the beginning of
life that assigns a fraction $\psi\in [0,1]$ of agents to work
always and a fraction $1-\psi$ always to enjoy leisure. This planner
problem is identical to the time averaging planning problem above, provided that we replace  the choice variable $T$ by $\psi$.



\subsection{Social security}\label{sec:LSsocialsecurity}%
Instead of returning  tax receipts lump sum to agents as in
section~\use{sec:LS_tax}, we now assume that all revenues are
used to finance a social security system in which
agents are eligible to retire and collect benefits after an official
retirement age $R$. All labor earnings are subject to a
flat rate social security tax $\tau\in(0,1)$. Benefits {\it after}
the agent's chosen retirement date $T$, which
may or may not equal $R$, equal  a replacement rate $\rho$
times a worker's average earnings, i.e., $\rho$ times the wage rate $w$.
Agents who choose to retire after $R$  collect no  benefits
until they actually retire.

To construct an equilibrium, we set the two parameters $R$
and $\tau$ of the social security system, and then solve residually for
a replacement rate $\rho$ that is consistent with a balanced government
budget. At an equilibrium career length $\tilde T$, the government
budget constraint is
$$
\tau w \tilde T  = \left(1- \max\{R, \tilde T\} \right) \rho w,  \EQN LS_socsec_bc
$$
where the left side is tax revenues and the right side is social
security benefits. The first (second) argument of the max operator
presumes an equilibrium outcome in which workers retire before (after)
the official retirement age.
Note that the unit length of a lifetime implies that  an age interval
corresponds  both to a fraction of an agent's lifetime and also to a
fraction of the population within that age interval at any point in time.
From budget constraint \Ep{LS_socsec_bc} we can solve for the
replacement rate,
$$
\rho = {\tau \tilde T \over 1- \max\{R, \tilde T\}}\,.  \EQN LS_socsec_rho
$$

An agent's optimal career length solves
$$
\max_{T\in[0,1]} \Bigl\{ \log\Bigl[(1-\tau) w T + \rho w \,
\min\{1-R,\, 1-T\} \Bigr] - BT \Bigr\},                     \EQN LS_socsec_max
$$
where we have substituted the agent's budget constraint into the
utility function, and the arguments of the min operator appear in the
same order as in the max operator of \Ep{LS_socsec_bc}, i.e., the
first (second) argument refers to the case when the agent chooses to
work shorter (longer) than the official retirement age.

\vskip.5cm
\noindent
{\bf Case with $\tilde T \leq R$}
\vskip.25cm

\noindent
In the case of an optimal career length $T \leq R$, the first-order condition
of \Ep{LS_socsec_max} at an interior solution (with respect to $T\leq R$)
becomes
$$
{(1-\tau) w \over (1-\tau) w T + \rho w (1-R)} - B = 0.   \EQN LS_socsec_FOC1
$$
By government budget balance in \Ep{LS_socsec_rho},
$\rho=\tau \tilde T /(1-R)$, which can be substituted into
\Ep{LS_socsec_FOC1} to yield an expression for equilibrium career length,
$$
\tilde T = {1-\tau \over B} \equiv T^{+}\!(\tau).   \EQN LS_socsec_T1
$$


\vskip.5cm
\noindent
{\bf Case with $\tilde T \geq R$}
\vskip.25cm

\noindent
In the case of an optimal career length $T \geq R$, the first-order condition
of \Ep{LS_socsec_max} at an interior solution (with respect to $T\geq R$) becomes
$$
{(1-\tau) w - \rho w \over (1-\tau) w T + \rho w (1-T)} - B \geq 0,
                                                         \EQN LS_socsec_FOC2
$$
which holds with equality except under a binding corner solution with $T=1$.
However, such a corner solution can be ruled out as an equilibrium outcome
because government budget balance in \Ep{LS_socsec_rho} would imply
that the replacement rate goes to infinity; hence, it must be optimal for
a worker to retire prior to the end of his lifetime. After
substituting $\rho=\tau \tilde T /(1-\tilde T)$ into the denominator of
\Ep{LS_socsec_FOC2} at equality, we
obtain an expression for equilibrium career length
$$
\tilde T = {1-\tau - \rho \over B}
         = {1-  {\displaystyle \tau \over \displaystyle 1- \tilde T}  \over B},
                                                         \EQN LS_socsec_T2a
$$
where the second equality follows when we also substitute out for the
second appearance of $\rho$.

Expression \Ep{LS_socsec_T2a} can be rearranged to become
$$
B \tilde T^2 - (1+B) \tilde T + 1-\tau =0.               \EQN LS_socsec_quadratic
$$
The smaller root of this quadratic equation determines the equilibrium
career length:
$$
\tilde T = { 1+B - \sqrt{(1+B)^2 - 4B(1-\tau )} \over 2B }
             \equiv T^{-}\!(\tau),                       \EQN LS_socsec_T2b
$$
where $\tilde T^{-}\!(0)=1/B$, and $T^{-}\!(\tau)$ decreases monotonically
to zero as $\tau$ goes to one.\NFootnote{After setting $\tau=0$ in
quadratic equation \Ep{LS_socsec_quadratic}, the two roots are
$$\EQNalign{
{ 1+B \pm \sqrt{1+2B+B^2-4B} \over 2B}
&={1+B \pm \sqrt{(1-B)^2} \over 2B}                 \cr
\noalign{\vskip.2cm}
&= {1+B \pm \vert 1-B \vert \over 2B}
={1+B \pm (B-1) \over 2B} = \left(1, {1 \over B}\right). \cr} $$
%\cases{
%   1/B; &  \cr
%   1 ;  &  \cr}                                     \cr}
%$$
where we have invoked our parameter restriction $B \geq 1$ to
evaluate the absolute value of $\vert 1-B \vert =B-1$. The smaller
root constitutes the equilibrium career length since it agrees with
the agent's choice in \Ep{LSequi_careerlog}.}

From equation \Ep{LS_socsec_T1} that defines $T^{+}\!(\tau)$ and from
equation \Ep{LS_socsec_T2a} that implicitly defines $T^{-}\!(\tau)$,
it follows immediately that $T^{+}\!(\tau) > T^{-}\!(\tau)$
for $\tau\in(0,1)$.
We can now  state a proposition that describes how the retirement
age $\tilde T$ chosen in equilibrium depends on
the official social security retirement age.
\medskip
\medskip
\noindent{\sc Proposition:} Given an official retirement age $R\in(0,1)$
and a tax rate $\tau\in(0,1)$, the equilibrium career length
$\tilde T(R,\tau)$ is unique and given by

\medskip
\item{  i) }  if $R \leq T^{-}\!(\tau)$, then $\tilde T(R,\tau) = T^{-}\!(\tau)$
      (retire {\it after} the official retirement age);
\item{ ii) }  if $R \geq T^{+}\!(\tau)$, then $\tilde T(R,\tau) = T^{+}\!(\tau)$
      (retire {\it before} the official retirement age);
\item{iii) }  otherwise, $\tilde T(R,\tau) = R$
       (retire {\it at} the official retirement age).
\medskip
\medskip
\noindent
Given $R=0.6$, the solid curve in Figure \Fg{figLSsocsec} displays  equilibrium career
length as a function of $\tau$. Within a range of tax rates
between 16--40 percent,  equilibrium career length does not
respond to changes in the tax rate because agents are at a corner solution and
strictly prefer to retire at the official retirement age $R$. Away
from that corner, career length is highly sensitive to the
social security tax rate $\tau$ in Figure \Fg{figLSsocsec}.

\midfigure{figLSsocsec}
\centerline{\epsfxsize=3truein\epsffile{LS_socsec.ps}}
\caption{Social security. Solid curve depicts equilibrium career
length as a function of a social security tax rate $\tau$,
given an official retirement age $R=0.6$. At low (high)
tax rates, $\tau <0.16$ ($\tau > 0.40$), an agent retires after
(before) the official retirement age, where the actual retirement
age lies along the curve $T^-(\tau)$ ($T^+(\tau)$), given a
disutility of work $B=1$.}
\infiglist{figLSsocsec}
\endfigure

When an equilibrium has agents retiring {\it before} the official
retirement age, $R > \tilde T = T^{+}\!(\tau)$, equilibrium career
length \Ep{LS_socsec_T1} is identical to outcome \Ep{LS_tax_career}
under the Prescott tax system. The reasons are that (a) under
our assumption that average earnings alone determine the replacement
rate without regard to career length,  agents regard their social
security contributions purely as a tax and perceive no extra benefits
accruing to them from paying it, while (b) the present value of future
social security payments operates like a lump sum transfer when optimal
career length falls short of the official retirement age. The
sensitivity of career length to social security taxation is even
larger in an equilibrium that has agents retiring {\it after}
the official retirement age, $R < \tilde T = T^{-}\!(\tau)$,
because the marginal decision about career length is then also
distorted by the loss of benefits incurred from working beyond the
official retirement age, as shown by the first equality in
expression \Ep{LS_socsec_T2a}.



\section{Earnings-experience profiles}\label{sec:LSprofile}%
The equivalence of outcomes across models of employment lotteries and time
averaging breaks down when  human
capital can be accumulated. A human capital accumulation technology typically makes career choice  in effect
induce another indivisibility that will be handled differently by our two types of models. %s the  `mother of all indivisibilities.'
While an agent in a time
averaging model will contemplate  when to terminate a career during which
earnings have increased because of work experience or investments
in human capital, the `invisible hand' in a complete-market
economy with employment lotteries will preside over a dual labor
market in which some agents specialize in work and others in
leisure.
 Here
we adopt  a specification of  earnings-experience profiles  of
Ljungqvist and Sargent (2012).\NFootnote{We defer an analysis of a Ben-Porath's (1967) human
capital technology to section \use{sec:LSbenporath}.} An agent with past employment
spells totaling $h_t = \int_0^t n_s \,ds$ has the opportunity
to earn
$$
w_t = W\, h_t^\phi, \quad \quad W>0, \;\; \phi\in[0,1].  \EQN LSprofile_wage
$$
\auth{Ben-Porath, Yoram}%


\subsection{Time averaging}

Under the assumption of no discounting, an agent is indifferent
about the timing of his labor supply, so we are free to assume
that the agent frontloads work at the beginning of life. The
present value of labor income for someone who works a fraction
$T$ of his lifetime is
$$
\int_0^T W t^\phi \, d\, t = {W\,T^{\phi+1} \over \phi+1} \,.
                                                \EQN LSprofile_PVlabor
$$
As before, since the subjective discount rate equals the market
interest rate, an agent chooses a constant consumption stream
$\bar c$. Hence, an agent's optimization problem becomes
$$
\max_{\bar c, T} \Bigl\{ \log(\bar c) - BT \Bigr\} \EQN LSprofile_utility
$$
subject to
$$\EQNalign{
&\bar c \leq {W\, T^{\phi+1} \over \phi+1}\,,        \EQN LSprofile_bc   \cr
\noalign{\vskip.2cm}
&\bar c \geq 0,\;\; T\in[0,1].  \cr}
$$
We substitute budget constraint \Ep{LSprofile_bc} into the objective
function of \Ep{LSprofile_utility}, and
compute a first-order condition with respect to career length at
an interior solution,
$$
T = {\phi +1 \over B}\,;                         \EQN LSprofile_career
$$
where the implicit parameter restriction for an interior
solution is that $B \geq \phi +1$.

Because preferences are consistent with balanced growth, the
optimal career length \Ep{LSprofile_career} does not depend on the
earnings level parameter $W$. But evidently, career length does
increase with the elasticity parameter $\phi$. The more elastic
the earnings profile is to accumulated working time, the longer
is an agent's career.


\subsection{Employment lotteries}

We make three modifications to the planner problem in section
\use{sec:LS_lottery}. Besides our two specializations of zero
discounting and that the instantaneous utility function over
consumption is logarithmic, there are now agent-specific
wage rates $w_{jt}$ with each agent's earnings increasing in
his past experience as given by \Ep{LSprofile_wage}.

Because an agent's earnings increase with his experience, it
follows immediately that an optimal employment allocation has
a fraction $\psi$ of agents to work always ($n_{jt}=1$ for all
$t\in[0,1]$ for these unlucky people) and a fraction $1-\psi$
always to enjoy leisure ($n_{jt}=0$ for all $t\in[0,1]$ for these
lucky ones). Hence, the indeterminacy in lottery designs is now
gone. An agent who works throughout his lifetime generates
present-value labor income equal to $W/(\phi +1)$, as defined
in \Ep{LSprofile_PVlabor}.

As before, the planner chooses constant consumption $\bar c$
across agents and across time. The planner's problem becomes
$$
\max_{\bar c, \psi} \Bigl\{ \log(\bar c) - B \psi \Bigr\}
                                            \EQN LSprofile_utility_family
$$
subject to
$$\EQNalign{
&\bar c \leq \psi {W \over \phi+1}\,,        \EQN LSprofile_bc_family   \cr
\noalign{\vskip.2cm}
&\bar c \geq 0,\;\; \psi\in[0,1].  \cr}
$$
We substitute budget constraint \Ep{LSprofile_bc_family} into the
objective function of \Ep{LSprofile_utility_family}, and
compute a first-order condition with respect to the fraction of the
population sent to work at an interior solution,
$$
\psi = {1 \over B}\,.                         \EQN LSprofile_fraction
$$

We conclude that agents in a complete-market economy with employment
lotteries on average work less than agents who are left alone to `time average'
in an  incomplete-market economy, as characterized by \Ep{LSprofile_career}.
The latter agents confront a difficult choice between enjoying leisure
and earning additional labor income at the peak of their lifetime
earnings potential. This choice is not faced by  agents who
follow the instructions of the planner who uses lotteries to convexify
the indivisibility brought by careers. Of course, in the special
($\phi=0$) case when work experience does not affect earnings, the
aggregate labor supplies as well as the expected lifetime utilities
are exactly the same across the two economies, as asserted in the equivalence
result of section \use{sec:LSequi}.



\subsection{Prescott tax and transfer scheme}

It is instructive to revisit Prescott's tax analysis in
section \use{sec:LS_tax} for the present environment with
earnings-experience profiles. We invite the readers to verify
that the equilibrium career length in the time averaging
economy is then
$$
T^{\star} = {(1-\tau) (\phi +1) \over B },   \EQN LSprofile_tax_career
$$
and the employment-population fraction in the employment
lotteries economy is
$$
\psi^{\star} = {(1-\tau) \over B }.          \EQN LSprofile_tax_family
$$
While the labor supplies in \Ep{LSprofile_tax_career} and
\Ep{LSprofile_tax_family} differ, we note that the elasticity of
the supply with respect to the net-of-tax rate ($1-\tau$) is the
same and equal to one. This equality is another  reflection of  broad
similarities that typically prevail across incomplete-market and complete-market economies
with indivisible labor.  We shall encounter  another example  in section \use{sec:ChangKim} when we compare the
aggregate labor supply  in a Bewley  incomplete markets
economy with its complete-market counterpart.
\auth{Prescott, Edward C.}%

\subsection{No discounting now matters}

Recall that under a flat earnings-experience profile ($\phi=0$)
in section \use{sec:LS_lifetime}, an agent is indifferent about
the multitude of labor supply paths that yield the same
present-value of labor income in budget constraint
\Ep{LSequi_PVbc}. The reason is that two alternative labor supply
paths with the same present-value of labor income imply
the same lifetime disutility of work when
$\rho = r$. Note that for strictly positive discounting,
$\rho =r >0$, a labor supply path that is tilted toward
the future means that an agent will have to work for
a longer period of time to generate the same present-value
of labor income as compared to a labor supply path that
is tilted toward the present. But that is acceptable to
the agent since future disutilities of work are discounted
at the same rate as labor earnings when the subjective
discount rate is equal to the market discount rate.

But if there is an upward-sloping earnings-experience profile
($\phi >0$), an agent is no longer indifferent to the
described variation in career length associated with the timing
of lifetime labor supply. In particular, when $\rho = r>0$,
an agent strictly prefers to shift his labor supply to the end
of life because at a given lifetime disutility of work, working
later in life would mean spending more total time working.
That would push the worker further up the experience-earnings
profile and thereby increase the present value of lifetime
earnings.

Features  not present in our model would attenuate
 such a desire to postpone labor supply to the end
of life, e.g., borrowing constraints that force an
agent to finance consumption with current labor earnings,
incomplete insurance markets that compel an agent to resolve
career uncertainties earlier, and
forecast declines in dexterity with advances in age.


\section{Intensive margin}\label{sec:LSintensive}%
Prescott et al.\ (2009) extend the analysis of Ljungqvist and
Sargent (2007) in section \use{sec:LSequi} by introducing an
intensive margin in labor supply, i.e., $n_t\in [0,1]$ is now a
continuous rather than a discrete choice variable. However,
to retain the central force of indivisible labor, they postulate
a nonlinear mapping from $n_t$ to effective labor services, in particular, an
increasing mapping that is first convex and then concave. For
expositional simplicity, we let the effective labor services associated with
$n_t$ be $(n_t - \underline{n})$ where $\underline{n}\in(0,1)$.
As noted by Prescott et al.\ (2009) such a mapping can reflect
 costs associated with getting set up in a job, learning about coworkers, and so on.
 \auth{Prescott, Edward C.}%
 \auth{Rogerson, Richard}%
 \auth{Wallenius, Johanna}%
\auth{Ljungqvist, Lars}%
\auth{Sargent, Thomas J.}%

The preferences are the same as those of Ljungqvist and Sargent (2007)
in \Ep{LSequi_utility} but now with no discounting, $\rho = r = 0$.
Under the present assumption that $n_t$ is a continuous
choice variable, we need to make additional assumptions about the
function $v(\cdot)$. The instantaneous disutility function over
work, $v(n)$, is strictly increasing, strictly convex, and
twice continuously differentiable.


\subsection{Employment lotteries}\label{sec:LS_Prescott}%
We begin by solving a complete-market economy with employment
lotteries in a static model. To compute an equilibrium allocation, we posit that a planner chooses
 consumption and employment $c_{j}\geq 0$,
$n_{j}\in [0,1]$ for a continuum of agents $j\in[0,1]$ to maximize
$$
\int_0^1 \left[ u(c_{j}) - v(n_{j}) \right]\,dj
                                         \EQN LSint_utility1_family
$$
subject to
$$
\int_0^1 c_{j} \, dj \leq w \int_0^1 \bigl[ n_j - \underline{n} \bigr]\,dj \,.
\EQN LSint_bc1_family
$$
Strict concavity of $u(c)$ makes it optimal to assign
the same consumption to each agent, $\bar c$. Likewise, because of
strict convexity of $v(n)$, the planner asks for the same labor
supply from each agent who is sent to work, $\bar n$. Conditional on
working, the labor supply $\bar n > \underline{n}$ because
it cannot be optimal to have agents incurring disutility of work
without earning any income. For an agent $j$ who is not working,
$n_j=0$.

Given this characterization of an optimal allocation, the planner's
optimization problem becomes
$$
\max_{\bar c, \bar n, \psi} \Bigl\{ u(\bar c) - \psi v(\bar n) \Bigr\}
                                         \EQN LSint_utility2_family
$$
subject to
$$\EQNalign{
&\bar c \leq w \, (\bar n - \underline{n})\, \psi,
                                             \EQN LSint_bc2_family   \cr
&\bar c \geq 0,\;\;\bar n\in [0,1],\;\;  \psi\in[0,1],               \cr}
$$
where $\psi $ is the fraction of the population that the planner
sends to work,  the same fraction $\psi$ is also the  probability of working in the employment
lottery of the decentralized market economy.

As emphasized by Prescott et al.\ (2009), the interesting case is
the one where the solutions for $\psi$ and $\bar n$ are both
interior. In this case, after substituting budget constraint
\Ep{LSint_bc2_family} at equality into the objective function of
\Ep{LSint_utility2_family}, we obtain the following first-order
conditions with respect to $\psi$ and $\bar n$, respectively,
$$
\EQNalign{
u'\Bigl( \psi\, w\, (\bar n - \underline{n}) \Bigr)\,
          w\, (\bar n -\underline{n}) &= v(\bar n), \EQN LSint_FOC_family;a   \cr
\noalign{\vskip.2cm}
u'\Bigl( \psi\, w\, (\bar n - \underline{n}) \Bigr)\,
          w\, \psi &= \psi \, v'(\bar n). \EQN LSint_FOC_family;b   \cr}
$$
Dividing these equations  gives
$$
v'(\bar n) = {v(\bar n) \over \bar n - \underline{n} }.  \EQN LSint_opt_family
$$
This condition for optimality states that the marginal cost to the
planner to supply additional effective labor services should be
equalized across  intensive and  extensive margins. The marginal
disutility at the intensive margin is $v'(\bar n)$ when employed agents
are asked to increase their hours worked, while the marginal
cost at the extensive margin is $v(\bar n)/(\bar n -\underline{n})$,
i.e., the {\it average} disutility per effective hour of an agent
who is asked to switch from not working to working.

Note that an employed agent's optimal labor supply $\bar n$ can
be computed from \Ep{LSint_opt_family} and depends
on neither $\bar c$ nor $\psi$, except for the supposition of
an interior solution for $\psi$. Given a solution for $\bar n$,
we can then use either \Ep{LSint_FOC_family;a} or
\Ep{LSint_FOC_family;b} to solve for $\psi$.


\subsection{Time averaging}

We now turn to a time averaging economy. An agent's problem is
similar to that in section \use{sec:LS_lifetime} but with the added
intensive margin of Prescott et al.\ (2009) (and no discounting).
An agent chooses lifetime consumption and employment $c_{t}\geq 0$,
$n_{t}\in [0,1]$ for $t\in[0,1]$ to maximize
$$
\int_0^1 \left[ u(c_{t}) - v(n_{t}) \right]\,dt
                                         \EQN LSint_utility1_lifetime
$$
subject to
$$
\int_0^1 c_{t} \, dt \leq w \int_0^1 \bigl[ n_t - \underline{n} \bigr]\,dt \,.
                                         \EQN LSint_bc1_lifetime
$$
It is immediate that this problem is identical to the planner's
problem in the static model of section \use{sec:LS_Prescott},  the
only difference being that we now integrate across time rather than
across agents. Hence, we can  reformulate the agent's
optimization problem to become
$$
\max_{\bar c, \bar n, T} \Bigl\{ u(\bar c) - T v(\bar n) \Bigr\}
                                         \EQN LSint_utility2_lifetime
$$
subject to
$$\EQNalign{
&\bar c \leq w \, (\bar n - \underline{n})\, T,
                                             \EQN LSint_bc2_lifetime \cr
&\bar c \geq 0,\;\;\bar n\in [0,1],\;\;  T\in[0,1],               \cr}
$$
where $T$ is the fraction of an agent's lifetime devoted to work, i.e.,
his career length.


\subsection{Prescott taxation}

To examine effects of taxation where there are both  intensive and  extensive margins, we adapt
the analysis in section \use{sec:LS_tax}. The government
budget constraint becomes
$$
\tau \, w\, (\bar n -\underline{n})\, T^\star = x.  \EQN LSint_tax_govbc
$$
Under the assumption that utility is logarithmic in consumption, an
agent's optimization problem becomes
$$
\max_{\bar c, \bar n, T} \Bigl\{ \log(\bar c) - T v(\bar n) \Bigr\}
                                                    \EQN LSint_log_lifetime
$$
subject to
$$\EQNalign{
&\bar c \leq (1-\tau) w \, (\bar n - \underline{n})\, T +x,
                                             \EQN LSint_tax_lifetimebc \cr
&\bar c \geq 0,\;\;\bar n\in [0,1],\;\;  T\in[0,1].               \cr}
$$
Substitute budget constraint \Ep{LSint_tax_lifetimebc} into the
objective function of \Ep{LSint_log_lifetime}, and
compute the first-order conditions at interior solutions with respect
to $T$ and $\bar n$, respectively,
$$\EQNalign{
{(1-\tau)\, w\, (\bar n -\underline{n}) \over
 (1-\tau)\, w\, (\bar n -\underline{n})\, T + x} -v(\bar n) &=0,
                                             \EQN LSint_FOC_tax;a   \cr
\noalign{\vskip.2cm}
{(1-\tau)\, w\, T \over
 (1-\tau)\, w\, (\bar n -\underline{n})\, T + x} - T\,v'(\bar n) &=0.
                                             \EQN LSint_FOC_tax;b   \cr}
$$
Dividing these equations gives
$$
v'(\bar n) = {v(\bar n) \over \bar n - \underline{n} }.  \EQN LSint_opt_tax
$$
This condition is the same as  expression
\Ep{LSint_opt_family} when there is no taxation and hence the
intensive margin is not affected by taxation. To compute the
equilibrium career length, we substitute \Ep{LSint_tax_govbc} into
first-order condition \Ep{LSint_FOC_tax;a},
$$
T^\star = {1-\tau \over v(\bar n)}.
$$
Along with Prescott et al.\ (2009), we conclude that the effects of taxation
are the same as in Ljungqvist and Sargent (2007), i.e., all the
adjustment of labor supply takes place along the extensive margin, and the
elasticity of aggregate labor supply with respect to the net-of-tax
rate ($1-\tau$) is equal to one.

The reason that none of the adjustment takes place along the intensive
margin is that any changes in labor when already working occur
along an increasing marginal disutility of work, while
adjustment along the extensive margin are made at a constant disutility
of work by varying the fraction of one's lifetime devoted to
work. The constancy of the latter terms of trade between working and
not working was the essential ingredient of  the famous (or, depending on your viewpoint,
infamous) high labor supply elasticity
in models of employment lotteries when labor is indivisible.

Rogerson and Wallenius (2009) break the constancy of the terms
of trade between working and not working by adding a life cycle
earnings profile to the present framework, but in contrast to section
\use{sec:LSprofile}, they  take that earnings profile as exogenously given
rather than having it be determined as a function of an agent's past work experience. In the
Rogerson and Wallenius setup, two results
follow immediately: (a) agents choose to work when their life cycle
earnings profile is highest, namely, when it  exceeds an
optimally chosen reservation level; and (b)
labor supply $n_t$ at a point in time varies positively with the
exogenous earnings level. Taxation in this augmented
framework affects labor supply along both the intensive and extensive
margins. While an increasing marginal disutility of work continues
to frustrate adjustment along the intensive margin, there is now
decreasing earnings when extending the career beyond the heights of the exogenous life cycle earnings profile, which
then also frustrates adjustment along the extensive margin. The
assumed curvatures of the disutility of work at the intensive
margin and that of the exogenous lifecycle earnings
profile determine how much  adjustment occurs
along the intensive and extensive margins.



\section{Ben-Porath human capital}\label{sec:LSbenporath}%
We return to the assumption that labor is strictly indivisible,
$n_t\in \{0,1\}$, and add a Ben-Porath human capital accumulation
technology to the framework of section \use{sec:LSequi}. We take
note of Ben-Porath's (1967, p.~361) observation that if the
technology were to exhibit exact constant returns to scale, the
marginal cost of additional units of human capital would be
constant until all of the agent's current human capital
is devoted to the effort of accumulating human capital and hence,
the optimal rate of investment at any point in time would be
either full specialization or no investment at all. Under our
simplifying assumption of no depreciation of human capital,
it follows that an agent would specialize and make all of
his investment in human capital upfront. Acquiring human capital  can be
thought of as formal education before starting to work.
\auth{Ben-Porath, Yoram}%

To represent the notion of specializing in human capital
investments in a simple way, we assume
that an agent has access to a technology
that can instantaneously determine his human capital through
the investment of $m\geq 0$ units of goods in himself, which
produces a human capital level
$$
h = m^{\gamma},   \hskip1.5cm \gamma\in(0,1),   \EQN LSbenporath_tech
$$
and there is no depreciation of human capital.
It follows trivially that it will be optimal for an agent to
use that technology once and for all before starting to work.
Under our assumption of a perfect credit market, an agent
chooses investment goods $m$ that maximize his present
value labor income, in conjunction with his choice of an
optimal career length $T$.
\auth{Guvenen, Fatih}
\auth{Kuruscu, Burhanettin}
\auth{Ozkan, Serdar}
\auth{Manuelli, Rodolfo}
\auth{Shin, Yongseok}

Papers  by Guvenen et al.\ (2011) and Manuelli et al.\ (2012)
that incorporate  Ben-Porath human capital
technologies in life cycle models inspire our analysis. Those papers
mainly focus on  tax dynamics driven,  not by the force in the Prescott tax
system in section \use{sec:LS_tax}, but instead by wedges that distort
an agent's investment in human capital.
Guvenen et al.\ (2011) postulate progressive labor income taxation
while Manuelli et al.\ (2012) assume that investments in human
capital are not fully tax-deductible. In both cases, the central
force is that the tax rate on returns to human capital is higher
than the rate applied to  labor earnings foregone while
investing in human capital, or the rate at which goods input
to human capital can be deducted from an agent's tax liabilities.

Following Manuelli et al.\ (2012), we assume a flat-rate tax
$\tau \in (0,1)$ on labor income and that only a fraction
$\epsilon \in [0,1]$ of goods input to human capital is
tax-deductible. To isolate the key force at work in Manuelli et al.\ (2012)
as well as in Guvenen et al.\ (2011), we assume no lump sum
transfers of tax revenues to agents. However, at the end of the
section, we will show how lump
sum handovers remain as potent in suppressing the aggregate
labor supply.\NFootnote{Given indivisible labor,
Manuelli et al.\ (2012) disarm the
potentially large effects of lump sum transfers of tax
revenues by modelling social security systems with implicit
tax wedges at an official retirement age, which gives rise to
corner solutions in agents' career decisions as analyzed
in section \use{sec:LSsocialsecurity}. In
Guvenen et al.'s (2011) analysis of divisible labor as
well as in their exploration of indivisible labor in
an earlier working paper,
the sensitivity of career length to lump sum transfers
does not arise because they assume  an
exogenous retirement age,  a common assumption in much  the
overlapping generations literature.} % that effectively
%avoids the challenge about the magnitude of the
%elasticity of lifetime labor supply when adjustment
%takes place at the extensive margin.}


\subsection{Time averaging}\label{sec:LSbenporath_lifetime}%
As mentioned, an agent will find it optimal to invest an amount $m$
in the human capital technology before
starting to work. Equality between the subjective discount rate
and the market interest rate implies that the agent chooses a
constant consumption stream $\bar c$, and that he is indifferent
to the timing of his labor supply. Moreover, because we assume
no discounting so that $\rho =r=0$, the optimal career length $T$ is
unique and does not depend on the timing of the agent's labor
supply. Under the postulated human capital technology
\Ep{LSbenporath_tech} and described tax policy, an agent's
optimization problem becomes
$$
\max_{\bar c, m, T} \Bigl\{ \log(\bar c) - BT \Bigr\}   \EQN LSbenporath_utility
$$
subject to
$$\EQNalign{
&\bar c   \leq (1-\tau) w \, m^{\gamma}\, T
               - (1-\tau \epsilon ) m ,            \EQN LSbenporath_bc   \cr
&\bar c \geq 0,\;\; m \geq 0, \;\; T\in[0,1].  \cr}
$$

We substitute budget constraint \Ep{LSbenporath_bc} into the
objective function  \Ep{LSbenporath_utility}, and compute
first-order conditions with respect to $m$ and, at an interior
solution, $T$. After some manipulations, these first-order
conditions with respect to $m$ and $T$, respectively,
become
$$\EQNalign{
m^{1-\gamma} &= {\gamma (1-\tau) w \over 1-\tau \epsilon } T
                                               \EQN LSbenporath_FOC;a \cr
\noalign{\vskip.2cm}
T &= {1 \over B} + {1-\tau \epsilon \over 1- \tau} \, {m^{1-\gamma} \over w}.
                                               \EQN LSbenporath_FOC;b \cr}
$$
Substituting \Ep{LSbenporath_FOC;a} into \Ep{LSbenporath_FOC;b}
yields
$$
T = {1 \over (1-\gamma) B},                    \EQN LSbenporath_career
$$
where the implicit parameter restriction for an interior solution is
$(1- \gamma ) B \geq 1$.

The optimally chosen career length in \Ep{LSbenporath_career} is
invariant to taxation under our assumption that no tax revenues are
handed back lump sum to agents. Any effect of taxation on the stock
of human capital and hence the level of labor earnings, do not affect
an agent's willingness to work since preferences are consistent with
balanced growth, i.e.,  income and subsitution effects cancel. Given a constant career length in \Ep{LSbenporath_career}, it
follows from expression \Ep{LSbenporath_FOC;a} that human capital
investments would also be invariant to taxation if all these
investments were tax-deductible, i.e., if $\epsilon =1$. But if
$\epsilon < 1$, we see that human capital investments decline in
the tax rate $\tau$ because human capital returns are taxed at a
higher rate than the rate at which rate the goods input to human capital is
tax-deductible.

The severity of the tax distortion depends on the curvature parameter
$\gamma$ of the human capital technology. For example, when human
capital investments are not tax-deductible, $\epsilon =0$, we
can solve for $m$ from equation \Ep{LSbenporath_FOC;a}, and
compute an agent's human capital stock as given by \Ep{LSbenporath_tech}:
$$
h = \Bigl[ \gamma (1-\tau) w\, T\Bigr]^{\gamma \over 1-\gamma }.
$$
Thus, the elasticity of human capital, and for that matter, also
labor earnings $whT$, with respect to net-of-tax rate $(1-\tau )$
is equal to $\gamma / (1-\gamma )$, which becomes arbitrarily large
as $\gamma$ approaches one. A high value of $\gamma$ is associated
with strong output effects of taxation because of reasons similar to
those in an `AK model' (output is linear in a single input, capital). The single input in our
 human capital technology
\Ep{LSbenporath_tech} is reproducible, and exhibits weak diminishing
returns when $\gamma$ is close to one. Likewise, for a standard
formulation of the Ben-Porath technology,
both the input of purchased goods and the input of an agent's current
human capital services are {\it de facto} reproducible, so similarities with
an `AK model' arise if the human capital technology exhibits close to
constant returns to scale (combined with the standard assumption of constant
returns to scale in the goods technology that employs human and physical
capital).



\subsection{Employment lotteries}\label{sec:LSbenporath_family}%
As in the case of an earnings-experience profile in
section \use{sec:LSprofile}, the planner will optimally (and
randomly) assign a fraction $\psi$ of agents to work their
entire lives while a fraction $1-\psi$ will specialize in
leisure. Needless to say, the planner will  invest only in
 human capital for  agents who are sent work. Under the
postulated human capital technology \Ep{LSbenporath_tech} and
described tax policy, the planner's problem becomes
$$
\max_{\bar c, m, \psi} \Bigl\{ \log(\bar c) - B \psi \Bigr\}
                                            \EQN LSbenporath_family
$$
subject to
$$\EQNalign{
&\bar c   \leq (1-\tau) w \, m^{\gamma}\, \psi
     - (1-\tau \epsilon ) m \psi,  \EQN LSbenporath_bc_family   \cr
&\bar c \geq 0,\;\; m \geq 0, \;\; \psi\in[0,1].  \cr}
$$
We substitute budget constraint \Ep{LSbenporath_bc_family} into the
objective function  \Ep{LSbenporath_family}, and compute
first-order conditions with respect to $m$ and, at an interior
solution, $\psi$. After some manipulations, these first-order
conditions with respect to $m$ and $\psi$, respectively,
become
$$\EQNalign{
m^{1-\gamma} &= {\gamma (1-\tau) w \over 1-\tau \epsilon },
                                     \EQN LSbenporath_FOC_family;a \cr
\noalign{\vskip.2cm}
\psi &= {1 \over B}\,.               \EQN LSbenporath_FOC_family;b \cr}
$$

As in the case of an earnings-experience profile, agents in the
employment lotteries economy on average work less than do agents in
the time-averaging economy: compare expression
\Ep{LSbenporath_FOC_family;b} to that of \Ep{LSbenporath_career}. Agents in the time-averaging
economy confront a difficult choice between enjoying leisure
and earning additional labor income derived from their past investment
in human capital. Once again, this difficult choice is not confronted by  agents in the employment lotteries economy where the planner
randomly assigns a fraction $\psi$ of the population to work their
entire lives, and thereby ensures an efficient use of all human
capital. However, the difference in labor supply diminishes as
$\gamma$ approaches zero.  In the limit ($\gamma =0$) when
the technology can no longer be used to augment an agent's human
capital, labor supplies are the same across the two economies and
we are back to our equivalence result in section \use{sec:LSequi}.

Given that agents who are sent to work in the employment
lotteries economy work their entire lives, it is not surprising
that the planner makes a larger human capital investment in
each employed worker as compared to an agent's investment
decision in the time averaging economy: compare expression
\Ep{LSbenporath_FOC_family;a} to that of \Ep{LSbenporath_FOC;a}.
After solving for the planner's choice of $m$ from
\Ep{LSbenporath_FOC_family;a} and substituting into human
capital technology \Ep{LSbenporath_tech}, the human capital
stock per employed worker in the employment lotteries economy
is
$$
h = \Bigl[ \gamma (1-\tau) w\Bigr]^{\gamma \over 1-\gamma }.
$$
Thus, the elasticities  of human capital and
labor earnings $wh\psi $, with respect to net-of-tax rate $(1-\tau )$
are the same as those in the time averaging economy and equal to
$\gamma / (1-\gamma )$.




\subsection{Prescott taxation}\label{sec:LSbenporath_tax}%
We now add Prescott's assumption that tax revenues, net of
any tax deductions on human capital investments, are returned
lump sum to agents. Such handouts remain potent in suppressing the aggregate labor supply in the
time averaging economy as well as in the employment lotteries
economy.


In the time averaging economy, we adapt the analysis of section
\use{sec:LS_tax} as follows. The government budget constraint becomes
$$
\tau\,w\,h^{\star}\, T^{\star} - \tau\, \epsilon \, m^{\star} = x,
                                        \EQN LSbenporath_govbc_prescott
$$
where $T^{\star}$, $h^{\star}$ and $m^{\star}$ are
equilibrium values of career length, human capital stock,
and agents' purchase of goods input to the human capital technology,
respectively. An agent's budget constraint is augmented to include
the lump sum transfer $x$,
$$
\bar c \leq (1-\tau) w \, m^{\gamma}\, T - (1-\tau \epsilon ) m + x.
                                         \EQN LSbenporath_bc_prescott
$$
After substituting budget constraint \Ep{LSbenporath_bc_prescott}
into the objective function  \Ep{LSbenporath_utility}, we can
compute and verify that our earlier first-order condition
\Ep{LSbenporath_FOC;a} with respect to $m$ is unchanged, i.e.,
$$
m^{1-\gamma} = {\gamma (1-\tau) w \over 1-\tau \epsilon } T.
                                         \EQN LSbenporath_FOC_prescott
$$
However, the career length $T$ is no
longer invariant to taxation. Specifically, after substituting
goverment budget constraint \Ep{LSbenporath_govbc_prescott} into
the agent's first-order condition with respect to $T$, at an interior
solution, the result can be rearranged to read
$$
T = {1-\tau \over B} + {m^{1-\gamma } \over w},
                                        \EQN LSbenporath_career_prescott
$$
into which we substitute expression \Ep{LSbenporath_FOC_prescott} to
obtain
$$
T^{\star} = { 1-\tau \over \left[ 1 - \gamma\,
    {\displaystyle 1-\tau \over \displaystyle 1 -\tau \epsilon } \right] B }\,.
                                        \EQN LSbenporath_career2_prescott
$$
Since $\tau \in (0,1)$ and $\epsilon \in [0,1]$, the equilibrium career
length is now shorter than it was in expression \Ep{LSbenporath_career} when
tax revenues, net of any tax deductions on human capital investments,
were not handed back lump sum to agents.

For example, when investments in human capital are fully tax-deductible,
$\epsilon = 1$, we disarm the key distortionary force that is
the focus of Manuelli et al.\ (2012), i.e., the choice of goods
input in expression \Ep{LSbenporath_career2_prescott} is no longer
distorted by the differential tax treatment of investments into  and returns from human capital, but Prescott's tax distortion, arising from the
lump sum handover of tax revenues to agents, is a forceful determinant
of equilibrium career length \Ep{LSbenporath_career2_prescott},
evaluated at $\epsilon =1$,
$$
T^{\star} \Big{\vert}_{\epsilon = 1}
 = { 1-\tau \over (1-\gamma )B }\,.
                                        \EQN LSbenporath_career3_prescott
$$
Hence, career length is no longer invariant to taxation when the
government hands over tax revenues lump sum to agents; in particular,
compare expression \Ep{LSbenporath_career3_prescott} to that in
\Ep{LSbenporath_career}. And once again, the elasticity of aggregate
labor supply with respect to the net-of-tax rate $(1-\tau )$ is equal
to one. But now labor income declines further because of
depressed investments in human capital, as determined by the goods input
from expression \Ep{LSbenporath_FOC_prescott} that varies positively
with career length.

We leave it as an exercise to readers to derive the corresponding
equilibrium outcomes in the employment lotteries model, by following
the same steps as above. It can be verified that the planner's
first-order condition with respect to $m$ remains the same as
in expression \Ep{LSbenporath_FOC_family;a}, while steps
analogous to those above, where government budget constraint is
subtituted into the planner's first-order condition with respect
to $\psi$, yield
$$
\psi^{\star} = {(1-\tau ) w m^{\gamma} - (1-\tau \epsilon ) m \over
                (w m^{\gamma} - m)\, B } < {1 \over B}.
                                        \EQN LSbenporath_family_prescott
$$
The strict inequality is implied by $\tau \in (0,1)$ and
$\epsilon \in [0,1]$, and hence, the equilibrium fraction of
population sent to work is smaller as compared to that found in
section \use{sec:LSbenporath_family}, where  tax
revenues were not handed back lump sum to agents, as given by
\Ep{LSbenporath_FOC_family;b}. It is again instructive to
consider the case when investments in human capital are fully
tax-deductible, $\epsilon = 1$,
$$
\psi^{\star} \Big{\vert}_{\epsilon = 1}
 = { 1-\tau \over B }\,,
                                        \EQN LSbenporath_family2_prescott
$$
where the elasticity of aggregate labor supply with respect to the
net-of-tax rate $(1-\tau )$ is again equal to one.




\section{Earnings shocks}\label{sec:LSshocks}%
Next we study how earnings shocks affect an agent's choice of
career length. Following Ljungqvist and Sargent (2012), we
study an unanticipated permanent earnings shock, which will enable
us to highlight forces that will also be at work in richer
environments. For additional analytical simplicity, we assume a flat
earnings profile, no discounting, and that utility is logarithmic in
consumption. The parameter restriction $B>1$ guarantees a strictly
interior solution to lifetime labor supply (at least prior to
the unanticipated earnings shock).

As in our basic setup in section \use{sec:LS_lifetime}, an agent can
choose to work at the wage rate $w$, and solves a deterministic
lifetime labor supply problem. The optimal solution at an interior
solution is the constant consumption level in
expression \Ep{LSequi_conslog}, $\bar c = w / B$, and a career
length that is given by expression \Ep{LSequi_careerlog}, $1 /B$.
Since an agent is indifferent to the timing of his labor supply,
we are free to assume that he starts to work at time
$t_0 \in (0, \,1-1/B)$,
and continues to work his entire optimally chosen
career length, i.e., the agent is intent on retiring at age
$$
\bar R = t_0 + 1/B.                              \EQN shock_Rfirst
$$
An employed agent pays off debt and accumulates assets for retirement,
with asset holdings at time $t \in [t_0, \bar R]$ as given by
$$
A(t) = \int_0^{t_0} - \bar c \, ds +
      \int_{t_0}^t (w - \bar c)\, ds
    = w (t-t_0 ) - {w  \over B} \,t,              \EQN shock_Afirst
$$
where we have invoked the optimal consumption level, $\bar c = w / B$.
Before starting to work, an agent finances consumption by
borrowing and hence, there is some date $\bar t \in (t_0, \bar R)$
at which an employed has just repaid his debt, $A(\bar t)=0$, and
starts to accumulate assets for retirement, where
$$
A(\bar t) = w (\bar t-t_0 ) - {w  \over B} \,\bar t = 0
\hskip.5cm \Longrightarrow \hskip.5cm \bar t = {B \over B-1} t_0.
                                                 \EQN shock_bar_t
$$

Consider an unanticipated mid-career earnings shock at time
$\hat t \in[t_0, \bar R]$, when the wage rate unexpectedly
jumps from $w$ to $\hat w$ for $t\in [\hat t, 1]$. Subject to
the asset holdings $A(\hat t)$ that were accumulated under the
old optimal plan, the shock prompts the agent to maximize the
remainder of his lifetime utility,
$$
\int_{\hat t}^1 \Bigl[ \log (\hat c_t) - B \hat n_t \Bigr] dt
$$
by choosing new values $\hat c_t \geq 0$ and $\hat n_t \in \{0,1\}$
of consumption and labor supply, respectively, for $t\in [\hat t, 1]$.
The agent's revised optimal plan prescribes a constant consumption
path over the interval $[\hat t, 1]$ and a new retirement
age $\hat R\in [\hat t, 1]$.

For the agent who after the unanticipated wage shock at $\hat t$
chooses to work until $R \in[\hat t, 1]$, the sum of the
financial assets already accumulated at time $\hat t$, $A(\hat t)$,
and the present value of future labor income becomes
$$
A(\hat t) + \int_{\hat t}^R \hat w \, ds
= w (\hat t-t_0 ) - {w  \over B} \,\hat t + \hat w (R -\hat t).
                                                \EQN shock_Asecond
$$
This expression divided by $1-\hat t$ is then the constant
consumption rate over the remaining lifetime $1-\hat t$,
since the time $\hat t$ present value of financial plus
nonfinancial wealth must equal the present value of consumption
over the period $[\hat t, 1]$.

The agent's optimal retirement age thus solves
$$
\max_{R\in [\hat t, 1]} \left\{ (1- \hat t)
\log \left[
{\displaystyle  w (\hat t-t_0 ) - {\displaystyle w  \over \displaystyle B}
\,\hat t + \hat w (R -\hat t) \over \displaystyle  1- \hat t }\right]
- B(R-\hat t) \right\}.
$$
The first-order condition for $R$ is
$$
{ \displaystyle (1-\hat t) \hat w \over \displaystyle
w (\hat t-t_0 ) - {\displaystyle w  \over \displaystyle B}
\,\hat t + \hat w (R -\hat t) }  - B \cases{
   < 0, &corner solution $\hat R = \hat t$;  \cr
   = 0, &interior solution $\hat R \in [\hat t, 1]$; \cr
   > 0, &corner solution $\hat R = 1$;  \cr }
                                                  \EQN shock_FOC
$$
where $\hat R$ is the optimal retirement age after the wage
shock at time $\hat t$. At an interior solution to $\hat R$,
first-order condition \Ep{shock_FOC} holds with equality,
$$
{ (1-\hat t) \over B} = {w \over \hat w} (\hat t-t_0 )
- { w  \over \hat w} \, {1 \over B}\, \hat t + \hat R - \hat t.
$$
After adding $t_0$ to both sides of this equation, and
using expression \Ep{shock_Rfirst} for the original
retirement age, $\bar R = t_0 + 1/B$,
the post-shock retirement age $\hat R$ at an interior solution
relates to the original retirement age $\bar R$ in the following
way:
$$
\hat R = \bar R + {\hat w - w \over \hat w}
\left[ {B-1 \over B} \hat t - t_0 \right]
= \bar R + {\hat w - w \over \hat w}\, (\hat t - \bar t)\, {B-1 \over B}\, ,
                                                          \EQN shock_Rsecond
$$
where the second equality is obtained by using expression
\Ep{shock_bar_t} to substitute out for $t_0$.


Evidently, the sign of the revision $\hat R - \bar R$ to an
unanticipated wage shock depends (i) on whether $\hat w > w$ or
$\hat w < w$, and (ii) on whether $\hat t$ is greater than
or smaller than $\bar t$, where $\bar t$ defined in
\Ep{shock_bar_t} is the point in time when the
asset holdings of an employed agent turns from being negative to
being positive. In response to a {\it negative} earnings shock,
$\hat w < w$, the agent reduces (increases) his lifetime labor
supply if his time $\hat t$ asset holdings are positive (negative),
i.e., if $A(\hat t)>0$ ($A(\hat t)<0$), which
means that the shock occurs at a time $\hat t > \bar t$
($\hat t < \bar t$). In contrast, in response to a {\it positive}
wage shock, $\hat w > w$, the agent increases (decreases) her
lifetime labor supply if her current asset holdings are positive
(negative).

These strong predictions based merely on the signs of the
earnings shock and an agent's asset holdings, follow from the
assumption that preferences are consistent with
balanced growth. Ljungqvist and Sargent (2012) generalize the
result to a larger class of such preferences and allow for
the earnings-experience profile in section \use{sec:LSprofile}.


\subsection{Interpretation of wealth and substitution effects}

For an agent with positive asset holdings at $\hat t$, a {\it negative}
wage shock means that returns to working fall relative to the marginal
value of his wealth.  That induces the agent
 to enjoy more leisure  because doing that has now become
relatively less expensive. But with negative asset holdings at $\hat t$,
a negative wage shock compels the agent to supply more labor both to pay off  time $\hat t$ debt and to  moderate the adverse effect of the shock on  his future consumption.

With a {\it positive} wage shock, leisure becomes more expensive, causing the
agent to substitute away  from leisure and toward consumption. This force makes  lifetime labor supply
increase for an agent with positive wealth. But why does a positive
wage shock lead to a {\it reduction} in life-time labor supply when time $\hat t$ assets are
negative?

In the case of a positive wage shock and negative time $\hat t$ assets,
consider  a hypothetical  asset
path that would have prevailed if the agent had enjoyed the higher wage
rate $\hat w$
from the beginning starting at $t=0$. Along that hypothetical path, the agent would have been even further
in debt at $\hat t$ (since the optimal constant consumption level would have been
equal to $\hat w /B$, as given by \Ep{LSequi_conslog}).
So  at $\hat t$,  the
agent actually finds himself {\it richer} at $\hat t$ than he would have  in our hypothetical  scenario.
Because there
is less debt to be repaid at $\hat t$,  the agent chooses to supply less
labor than he would have in  the hypothetical scenario.
In other words,  it is not optimal to
make up for what would have been  past underconsumption relative to our hypothetical path, so the agent chooses instead to
enjoy more leisure  because he has relatively less debt  at $\hat t$ than he would along the hypothetical path.



\section{Time averaging in a Bewley model}\label{sec:ChangKim}%
In a version of a Bewley model with incomplete markets
(see chapter \use{incomplete}),
Chang and Kim (2007) demonstrate how indivisible labor is
associated with a high labor supply elasticity when the
infinitely-lived agents engage in `time averaging,' i.e.,
alternating between work and leisure. In such an
incomplete-markets model, agents accumulate assets not only
because of the standard precautionary motive to self-insure
against productivity shocks, but also to finance  planned spells of leisure.
\auth{Chang, Yongsung}%
\auth{Kim, Sun-Bin}%

We abstract from aggregate productivity shocks, but otherwise
postulate the same neoclassical growth model (and its
calibration) as that of Chang and Kim (2007). The economy is
populated by a continuum (measure one) of agents who have
identical preferences but experience different idiosyncratic
productivity shocks. An agent's preference specification is
similar to that of section \use{sec:LSequi} except  that
now time is discrete, agents live forever, and there is uncertainty;
$$
E_0 \sum_{t=0}^\infty \beta^t \left[\log (c_t) - B n_t \right],
\EQN LS_CK_utility
$$
where $E_t$ is the expectation operator conditional on information
at time $t$, and $\beta\in(0,1)$ is the agent's subjective discount
factor. An agent who works in period $t$ supplies $z_t$ efficiency
units of labor where the idiosyncratic productivity
level $z_t$ varies exogenously according to a stochastic
process with a transition probability distribution function
$\pi(z ' \vert z) = {\rm Prob}(z_{t+1}\leq z' \vert z_t=z)$,
which has a unique unconditional stationary cumulative
distribution function $G(z)$.\NFootnote{As compared to
Chang and Kim (2007), we let our parameter $B$ in preference
specification \Ep{LS_CK_utility} replace their
composite of three parameters,
$B \bar h^{1+1/\gamma}/(1+1/\gamma)$, since the separate
identification of e.g.\ a curvature parameter $\gamma$ has no
significance under the assumption of indivisible labor.
Likewise, our normalization of time supplied when working,
$n=1$, as compared to their separate parameter $\bar h$ also
lacks significance as long as our disutility parameter $B$
has the same value as their composite of three parameters, and
so long as we properly scale the productivity shocks so that the
implied processes for an agent's efficiency units of labor
are the same.}

The aggregate production function is Cobb-Douglas and exhibits
constant returns to scale,
$$  F(L_t,K_t) = L_t^\alpha K_t^{1-\alpha},
$$
where $L_t$ and $K_t$ are the aggregates of efficiency units
of labor and of physical capital. Capital depreciates at rate
$\delta$ each period.


\subsection{Incomplete markets}

A stationary equilibrium has a constant interest rate $r$
and a constant wage rate $w$ per efficiency unit of labor.
The state variables for an agent's problem are then his
beginning-of-period assets $a$, before receiving interest
earnings, and his productivity $z$. The agent's  value function
 is
$$
V(a,z) = \max_{n, c, a'} \Bigl\{
\log (c) - Bn + \beta \int V(a',z')\, d\pi(z'\vert z)\Bigr\},
                                                     \EQN LS_CK_value
$$
subject to
$$\EQNalign{
& a' = (1+r) a + w z n - c,                      \EQN LS_CK_bc   \cr
& n\in\{0,1\},\;\; c\geq 0,\;\; a' \geq \hat a ,\cr}
$$
where savings must satisfy a borrowing constraint, $a'\geq \hat a$.
The solution to this problem includes a decision rule for labor supply,
$n(a,z)$, consumption, $c(a,z)$, and asset holdings, $a'\!(a,z)$.


After substituting budget constraint \Ep{LS_CK_bc} into the
utility function in \Ep{LS_CK_value}, we take a first-order condition
with respect to $a'$ and obtain an  Euler equation,
$$
{1 \over c(a,z)} = \beta \int V_1(a',z') \, d\pi(z'\vert z)
= \beta (1+r) \int {1 \over c(a',z') } d\pi(z'\vert z),
                                                     \EQN LS_CK_Euler
$$
into which we have substituted the decision rule for consumption,
$c(a,z)$, and applied the Benveniste-Scheinkman formula
$V_1(a',z')=(1+r)\,u'(c(a',z'))$.

Firms' profit maximization ensures that
$$\EQNalign{
w&=F_1(L,K)=(1-\alpha)\left({K \over L}\right)^\alpha, \EQN LS_CK_wr;a \cr
r&=F_2(L,K) - \delta =\alpha \left({K \over L}\right)^{\alpha-1} -\delta.
                                                    \EQN LS_CK_wr;b    \cr}
$$

Associated with  a stationary equilibrium is a time-invariant distribution
of agents across asset holdings and productivities, $J(a,z)$. The invariant
distribution satisfies
$$
J(a^o,z^o) = \int_{a'\leq a^o, z'\leq z^o} \left\{
\int {\bf I}\bigl(a'=a'\!(a,z)\bigr)\, d\pi(z'\vert z)\, dJ(a,z) \right\}
da'\, dz',
$$
where ${\bf I}(\cdot)$ is an indicator function that equals 1 if its argument is true and 0 otherwise.% Note that the
%distribution $J(a,z)$ appears on both sides of the expression,
%that is, the equilibrium distribution of agents remain the same over
%time when applying the endogenous and exogenous laws of motion for
%the state variables $a$ and $z$.

Markets for   labor, capital, and  goods clear:
$$
\EQNalign{
L &= \int z n(a,z) \, dJ(a,z), \cr
K &= \int a \, dJ(a,z), \cr
F(L,K) + (1-\delta) K &= \int \Bigl\{ a'\!(a,z) + c(a,z) \Bigr\} dJ(a,z).
\cr }
$$



\subsection{Complete markets}\label{sec:ChangKim_complete}%
An allocation for an economy with complete markets solves an assignment problem that confronts a
representative family with a continuum of family members. The family tells each
member what to consume and when to work. When preferences
are additively separable in consumption and leisure, optimal
consumption is the same for everyone, regardless
of work status. The family sends
the most productive members to work. In particular,  the representative
family sets a reservation productivity $z^\star$ such that members
with productivities greater than or equal to $z^\star$  work
while the others do not. The value function of the
representative family satisfies
$$
V(a) = \max_{z^\star, c, a'} \Bigl\{
\log (c) - B[1-G(z^\star)] + \beta V(a')\Bigr\},
                                                     \EQN LS_CK_family
$$ where the maximization is
subject to
$$\EQNalign{
& a' = (1+r) a + w \int_{z^\star}^\infty z\, dG(z) - c,
                                              \EQN LS_CK_family_bc \cr
& c\geq 0,\;\; a' \geq \hat a .\cr}
$$
The representative family solves a deterministic
problem because it has a continuum of members. The
{\it ex ante\/} probability that a single member draws from a particular interval of
productivities equals the {\it post\/} fraction of the family's members drawing from that interval.

First-order
conditions with respect to $c$, $a'$ and $z^\star$ at
interior solutions are:
$$
\EQNalign{
&{1 \over c} - \lambda = 0, \EQN LS_CK_familyFOC;a   \cr
\noalign{\vskip.2cm}
& - \lambda + \beta V'(a') = 0, \EQN LS_CK_familyFOC;b   \cr
\noalign{\vskip.2cm}
& B G'(z^{\star}) + \lambda w [-z^\star G'(z^\star)] = 0.
                                   \EQN LS_CK_familyFOC;c  \cr}
$$
In a steady state, consumption is constant over time.
Application of the Benveniste-Scheinkman formula gives
$V'(a)=(1+r) u'(c)$, so it follows from
first-order conditions \Ep{LS_CK_familyFOC;a} and
\Ep{LS_CK_familyFOC;b} that
$$
1+r = {1 \over \beta} = 1-\delta +
         \alpha \left({K \over L}\right)^{\alpha-1},  \EQN LS_CK_family_r
$$
where the second equality invokes profit-maximization
condition \Ep{LS_CK_wr;b}. The optimal consumption level is
obtained from conditions \Ep{LS_CK_familyFOC;a} and
\Ep{LS_CK_familyFOC;c}:
$$
c =  { w z^\star \over B }.                          \EQN LS_CK_family_c
$$
The equilibrium capital stock  held by the representative
family is $K=a=a'$, which together with expressions
\Ep{LS_CK_family_r} and \Ep{LS_CK_family_c} can be substituted
into budget constraint \Ep{LS_CK_family_bc}:
$$
K = {1 \over \beta } K + w \int_{z^\star}^\infty z\, dG(z)
     - { w z^\star \over B }.                      \EQN LS_CK_family_bc2
$$
After dividing expression \Ep{LS_CK_family_bc2} by the integral
in that expression, i.e., by the family's supply of efficiency
units of labor, which in  equilibrium  equals $L$, we obtain
$$
{K \over L} = {1 \over \beta } {K \over L} + w
     - { w z^\star \over B \int_{z^\star}^\infty z\, dG(z)}.
                                                    \EQN LS_CK_family_bc3
$$

We can now solve for a stationary equilibrium in three steps. First, we use
the second equality in expression \Ep{LS_CK_family_r} to determine
the equilibrium capital-labor ratio, $K/L$, in terms of parameters.
Next, given the capital-labor ratio, we can compute the wage rate
from profit-maximization condition \Ep{LS_CK_wr;a}. Finally, with the
capital-labor ratio and the wage rate in hand, expression
\Ep{LS_CK_family_bc3} becomes one equation to be solved for the
equilibrium value of the reservation productivity $z^\star$.





\subsection{Simulations of Prescott taxation}

We adopt the  calibration of Chang and Kim (2007)
except that we shut down  aggregate productivity
shocks. To highlight differences and similarities across our
incomplete- and complete-market versions of the economy, we
compute equilibrium outcomes under Prescott's tax and transfer
scheme in section \use{sec:LS_tax}.
\auth{Chang, Yongsung}%
\auth{Kim, Sun-Bin}%

\midfigure{figLSCKreswage}
\centerline{\epsfxsize=3truein\epsffile{LS_CK_reswage.ps}}
\caption{Reservation productivity as a function of asset
holdings in the economy with incomplete markets (solid curves)
and complete markets (dashed curves), respectively, where the
lower (upper) curve refers to tax rate 0.30 (0.70).}
\infiglist{figLLCKreswage}
\endfigure


For labor tax rates of 0.30 and 0.70, respectively,
reservation productivities as functions of asset holdings are
displayed in Figure \Fg{figLSCKreswage}. In the incomplete-market
economy (solid curves), an agent's reservation productivity
increases in his asset holdings.
A high asset level means that, everything else equal, an agent
is poised to enjoy one of his intermittent spells of leisure,
which will result in asset decumulation and his ultimate return
to work. For an agent with high assets to postpone such a desired
spell of leisure, the agent must experience a relatively high
productivity to be willing to continue to work for a while.
As one would expect, the reservation productivities for the higher
tax rate 0.70 lie well above those for the lower tax
rate 0.30, since Prescott's tax and transfer scheme is very
potent in suppressing agents' labor supply and causing them to
choose more leisure. In the complete-market economy, the single
productivity cutoff (dashed curve) is indicative of a
privately efficient allocation. It is the most productive agents
who work at any point in time.

\midfigure{figLSCKprod}
\centerline{\epsfxsize=3truein\epsffile{LS_CK_prod.ps}}
\caption{Productivity distribution. The upper solid curve is the
population productivity distribution, while the other two
in descending order show the agents thereof who are employed in the
incomplete-market economy given tax rate 0.30
and 0.70, respectively. The corresponding masses of employed agents
in the complete-market economy are the halves of the population
distribution to the right of a vertical dashed line,
where the left (right) dashed line refers to tax rate 0.30 (0.70).}
\infiglist{figLSCKprod}
\endfigure



The top solid curve in Figure \Fg{figLSCKprod} depicts the
stationary distribution of productivities in the population. A dashed
vertical line is the productivity cutoff in the complete-market
economy, where the left (right) one refers to tax rate 0.30 (0.70),
i.e., the same reservation productivity as the corresponding dashed
line in Figure \Fg{figLSCKreswage}. All agents with productivities
to the right of the dashed line work in the complete-market economy,
and hence, the area under that portion of the population
distribution equals the employment-population ratio.
In the incomplete-market economy, the endogenous stationary
distribution of agents across both productivities and asset holdings,
$J(a,z)$, together with the decision rule for whether or not to work,
$n(a,z)$, determine how many agents are at work at different productivity
levels. Those employed workers in the incomplete-market economy are
depicted by the solid curves that lie weakly below the top population
curve, which in descending order refer to tax rates
0.30 and 0.70, respectively. As in the complete-market economy, virtually
all agents with high productivities are working in the incomplete-market
economy. But over a mid-range of productivities, there
are significant differences between the two economies. On the one hand,
 some agents in the incomplete-market do not work but would have been working in the complete-market economy. The reason
is that because their asset holdings are relatively high,  their
shadow value of additional wealth falls below the utility of leisure.
On the other hand, other agents in the incomplete-market
economy  work but would not have worked in the
complete-market economy. These agents have low
asset holdings and so feel compelled to work despite their low
productivities.


The work and asset decisions of individual agents in the incomplete-market
economy determine the distribution of asset holdings, and the
capital stock. For labor tax rates 0.30 and 0.70, respectively, the
solid curves in Figure \Fg{figLSCKassets} depict the cumulative
distribution function for asset holdings in the incomplete-market
economy. At the high tax rate 0.70 (upper solid curve), asset holdings
become concentrated at lower levels. As in the case of the elevated
reservation productivities in
Figure \Fg{figLSCKreswage}, taxation suppresses market
activity in favor of leisure. In the complete-market economy, tax rate
0.70 is associated with a similar large decline in per capita
asset holdings, as depicted by the vertical dashed lines
in Figure \Fg{figLSCKassets} where the left (right) one refers to
tax rate 0.70 (0.30).


\midfigure{figLSCKassets}
\centerline{\epsfxsize=3truein\epsffile{LS_CK_assets.ps}}
\caption{Asset distribution. The lower (upper) solid curve is the
cumulative distribution function for asset holdings in the
incomplete-market economy when the tax rate is 0.30
(0.70). The right (left) vertical dashed line is the
per capita asset holdings in the complete-market
economy when the tax rate is 0.30 (0.70).}
\infiglist{figLSCKassets}
\endfigure


From a production perspective, what matters is the capital stock
relative to the aggregate supply of efficiency units of labor.
In the complete-market economy, that capital-labor ratio
is determined by steady-state relationship \Ep{LS_CK_family_r}
which does not depend on the labor tax rate (but would have depended
on any intertemporal tax wedge such as a tax on capital
income). Since the wage rate is a function of the capital-labor
ratio in \Ep{LS_CK_wr;a}, it follows in Figure \Fg{figLSCKwage}
that the wage rate in the complete-market economy (dashed curve)
is invariant to the labor tax rate. In contrast, the wage
rate in the incomplete-market economy (solid curve) falls with
the labor tax rate and lies above the wage rate of the
complete-market economy. To understand the latter outcome,
we recall that in a Bewley model like ours with infinitely-lived
agents, the interest rate must fall below the subjective rate of
discounting $\beta^{-1}$, which is the steady-state interest
rate in the complete-market economy. Since the equilibrium interest
rate is inversely related to the capital-labor ratio in expression
\Ep{LS_CK_wr;b}, it follows immediately that the capital-labor
ratio is higher in the incomplete-market economy, and therefore by
expression \Ep{LS_CK_wr;a}, so is the wage rate.




\midfigure{figLSCKwage}
\centerline{\epsfxsize=3truein\epsffile{LS_CK_wage.ps}}
\caption{Wage rate per efficiency unit of labor in the
economy with incomplete markets (solid curve) and complete
markets (dashed curve), as a function of the labor tax rate.}
\infiglist{figLSCKwage}
\endfigure




\midfigure{figLSCKfrac}
\centerline{\epsfxsize=3truein\epsffile{LS_CK_frac.ps}}
\caption{Employment-population ratio in the
economy with incomplete markets (solid line) and complete
markets (dashed line), as a function of the labor tax rate.
The dotted line represents the former economy with a less persistent productivity process.}
\infiglist{figLSCKfrac}
\endfigure


Figure \Fg{figLSCKfrac} shows that the fraction of the population employed
is higher in the incomplete-market economy than in the
complete-market economy. As seen in
Figure \Fg{figLSCKprod}, there are  those agents who work and those who
do not work in the incomplete-market economy, but who would have done the opposite
 if they instead had lived in the complete-market economy.
Evidently, the  group of agents who work in the incomplete-market
economy but would not have worked in the complete-market economy is larger.
With no insurance markets, agents on average work more in order to
accumulate precautionary savings in the event of low productivity in the
future.\NFootnote{Marcet et al.\ (2007) conduct an analysis similar to that of
Chang and Kim (2007) but where labor is divisible,
$n_t \in [0,1]$, and the idiosyncratic productivity shock takes on only
two values, $z_t \in \{0,1\}$. In addition to the precautionary savings
effect that tends to increase the capital stock under uncertainty, they
identify an ex post wealth effect on labor supply that can depress
the aggregate hours of work as well as the capital stock in an
incomplete-market economy. (See section \use{sec:self_insure_labor}.)}
\auth{Marcet, Albert}%
\auth{Obiols-Homs, Francesc}%
\auth{Weil,  Philippe}%
\auth{Zhu, Shenghao}%

What makes the employment-population ratio to converge across
the two economies at higher tax rates in Figure \Fg{figLSCKfrac}? A key
reason is that Prescott's tax and transfer scheme  effectively
 insures the agents by collecting  tax revenues and then returning them
 lump sum as equal amounts to all agents. To  explore
how precautionary savings drive the employment wedge
between the incomplete- and complete-market economies at low tax rates,
 consider the following perturbation of the idiosyncratic productivity
process. Specifically, suppose that  agents face a transition
probability distribution function
$$
\tilde \pi(z ' \vert z; \lambda) =
(1-\lambda) \pi(z ' \vert z) + \lambda G(z'),   \EQN LSCK_tildeprob
$$
where $\lambda\in[0,1]$. For $\lambda=0$, the productivity process is
the same as that of Chang and Kim (2007), while for $\lambda=1$,
productivities are independent and identically distributed across agents
and time, with realizations governed by the stationary unconditional
distribution of Chang and Kim's process.
Such perturbations do not affect equilibrium outcomes in the
complete-market economy because they do not affect the constraints of the representative
family. But  agents in the incomplete-market are now
ex ante relieved when they do not have to bear as much of the risk
associated with the persistence of Chang and Kim's productivity
process. The dotted line in Figure \Fg{figLSCKfrac} shows equilibrium outcomes
in the incomplete-market economy for $\lambda=0.1$, where employment is now closer
to that of the complete-market economy.


A striking feature of Figure \Fg{figLSCKfrac} is the high elasticity
of aggregate labor supply to taxation in the complete-market as well
as in the incomplete-market economy. This message is shared with the
first part of this chapter when agents were finitely lived and
at interior solutions with respect to their choices of career length.



\section{L and S equivalence meets C and K's agents}

Krusell et al.\ (2008) conjecture that the equivalence
result of Ljungqvist and Sargent (2007) in continuous time described in section
\use{sec:LSequi} extends to a deterministic version
of Chang and Kim's (2007) discrete-time framework
in section \use{sec:ChangKim}.  Remove  the productivity
shocks and normalize an agent's efficiency units of labor to one
in the latter framework ($z_t=1$ for all $t$). Then  Krusell et al.\ (2008)  show how agents choose the
constant consumption stream in expression \Ep{LSequi_conslog},
$\bar c= w/B$, when the steady-state interest rate is equal to the rate
of subjective discounting, $1+r=1/\beta$. Krusell et al. (2008)
indicate the existence of  stationary equilibrium in which  agents
support that constant consumption stream by  alternating between spells of working  and enjoying leisure (`time averaging').
 They argue that the
aggregate allocation is the same as if markets had been complete and
there had been employment lotteries.
\auth{Krusell, Per}%
\auth{Chang, Yongsung}%
\auth{Kim, Sun-Bin}%
%{Krusell, Per}%
\auth{Mukoyama,  Toshihiko}
\auth{Rogerson,  Richard}%
\auth{Sahin, Ay\c sg\" ul}%

\subsection{Guess the value function}

Since we are removing the productivity shocks from the model of
section \use{sec:ChangKim}, an agent's single state variable
in a stationary equilibrium is his beginning-of-period asset level $a$,
before receiving interest earnings.
We guess, and will then verify, that an agent's value funcion $V(a)$ takes
the form,
$$
V(a) = \cases{
 { \displaystyle \log \left[ {\displaystyle 1-\beta \over \displaystyle \beta }
     \, a + w \right] - B \over \displaystyle 1-\beta }
                                      \hskip.5cm &if $a\leq \underline{a}$; \cr
\noalign{\vskip.3cm}
  { \displaystyle 1 \over \displaystyle 1-\beta }
\left( \log \left[ {\displaystyle w \over \displaystyle B}\right] -1 \right)
    + {\displaystyle B \over \displaystyle \beta w}\, a
                        \hskip.5cm &if $a\in (\underline{a}, \bar a)$;    \cr
   \noalign{\vskip.3cm}  \cr
{ \displaystyle  \log \left[ {\displaystyle 1-\beta \over \displaystyle \beta }
     \, a  \right] - B \over \displaystyle  1-\beta }
                                       \hskip.5cm &if $a\geq \bar a$. \cr }
                                                  \EQN LS_CK_equi2_value
$$

To appreciate what motivates  our guess, we begin with  some
observations. In a stationary equilibrium with
$1+r=1/\beta$ and no uncertainty, an agent's intertemporal Euler
equation \Ep{LS_CK_Euler} implies constant consumption over
time.\NFootnote{Because of the equality between the market interest
rate $1+r$ and the rate of subjective discounting $1/\beta$, any
consumption path that varies over time can be improved upon by
shifting consumption from periods of high to periods of low
consumption. An agent's employment status does not matter since
preference specification \Ep{LS_CK_utility} is additively
separable in consumption and leisure.}
Thus, any asset accumulation or decumulation by an agent can only
be motivated by that agent's desire to engage in time averaging with
respect to his labor supply. For an agent with assets in some range
$(\underline{a}, \bar a)$, we shall  show that time averaging is
indeed optimal because it  enables him to finance an
optimal constant consumption level $\bar c= w/B$. But first we
discuss our guess of the value function outside of this asset range.

If an agent has too little (too much)  assets, he will choose to
work forever (to never work) and to consume the highest affordable
constant consumption level associated with that labor supply plan.
Consider an agent whose beginning-of-period assets
$a \leq \underline{a}$ are so low that if he works forever and
consumes the highest affordable constant consumption, $w+ ra$, that
consumption level will be less than or equal to $\bar c= w/B$.
We can  verify later that such a poor agent will indeed choose
to work forever and to consume $w+ra$ in each period.
After invoking $r=(1/\beta)-1$, the critical asset limit $\underline{a}$
is
$$\EQNalign{
w + {1-\beta \over \beta} \, \underline{a} \hskip.2cm &= \hskip.2cm
{w \over B},  \cr
\underline{a} \hskip.2cm &= \hskip.2cm {\beta w \over 1-\beta}
[B^{-1} -1 ] .
                                                          \EQN LS_CK_alow \cr}
$$
If $n_t=1$ and $c_t=w+ra$ for all $t$, preference specification
\Ep{LS_CK_utility}  yields  lifetime utility  given by the
conjectured value function \Ep{LS_CK_equi2_value} when
$a\leq \underline{a}$.\NFootnote{Under the implicit but necessary
parameter restriction for an equilibrium with time
averaging, $B>1$, note that asset limit
$\underline{a}$ in \Ep{LS_CK_alow} is negative, i.e., only
agents who are initially indebted, $a<0$, could conceivably want
to choose to work forever with constant consumption equal
to $w + ra$.}

Next, consider an agent whose beginning-of-period assets
$a \geq \bar a$ are so high that if he never works and
consumes the highest affordable constant consumption, $ra$, that
consumption level will be greater than or equal to $\bar c= w/B$.
We can later verify that such a rich agent will indeed choose
 never to work and to consume $ra$ in each period.
After invoking $r=(1/\beta)-1$, the critical asset limit $\bar a$ is
$$\EQNalign{
{1-\beta \over \beta} \, \bar a \hskip.2cm &= \hskip.2cm
{w \over B},  \cr
\bar a \hskip.2cm &= \hskip.2cm {\beta w \over 1-\beta} B^{-1}.
                                                       \EQN LS_CK_ahigh \cr}
$$
If $n_t=0$ and $c_t=ra$ for all $t$, preference specification
\Ep{LS_CK_utility} does yield  lifetime utility  given by the
conjectured value function \Ep{LS_CK_equi2_value} when
$a\geq \bar a$.

To complete our guess about the value function,  draw a straight line
between the end point $V(\underline{a})$ and the starting point
$V(\bar a)$ for the  segments with $a \leq \underline{a}$
and $a \geq \bar a$, respectively. This guess
is motivated by the insight from section \use{sec:LSequi} that time
averaging can under some conditions replace employment lotteries and
 attain linear combinations in the space of utilities when employment
is a discrete choice variable. The linear segment for the
value function is  expressed in
\Ep{LS_CK_equi2_value} when $a\in (\underline{a}, \bar a)$. Note that
the slope with respect to assets, $B/(\beta w)$, is the same as the
derivative at the end point of the preceding segment for the value function
($a\leq \underline{a}$) as well as the derivative at the starting point
of the succeeding segment for the value function ($a\geq \bar a$):
$$
\EQNalign{
V'(a)\Big{\vert}_{a\uparrow \underline{a}}
&= {\beta^{-1} \over {\displaystyle 1-\beta \over \displaystyle \beta }
\, \underline{a} + w }  = { B \over \beta w}   \cr
\noalign{\vskip.2cm}
V'(a)\Big{\vert}_{a\downarrow \bar a}
&= {1 \over (1-\beta) \bar a }
 = { B \over \beta w},   \cr }
$$
where we have invoked expression \Ep{LS_CK_alow} and \Ep{LS_CK_ahigh}
for $\underline{a}$ and $\bar a$, respectively.

\subsection{Verify optimality of time averaging}

Without productivity shocks, Bellman equation \Ep{LS_CK_value} can be
simplified to
$$\EQNalign{
V(a) = \max \Biggl\{ &\max_{a'} \Bigl\{
\log \bigl( (1+r)a + w -a' \bigr) - B + \beta V(a') \Bigr\} \;,  \cr
                     &\max_{a'} \Bigl\{
\log \bigl( (1+r)a -a' \bigr) + \beta V(a') \Bigr\}  \Biggr\},
                                      \EQN LS_CK_equi2_value2  \cr}
$$
where the first max operator selects whether to work, and the
budget constraints are substituted into the utility functions.
Given our conjectured value function \Ep{LS_CK_equi2_value},
we  solve the optimization problem on the right side of
\Ep{LS_CK_equi2_value2} to verify that our conjectured value
function $V(a)$ does indeed emerge on left side of
\Ep{LS_CK_equi2_value2}.

Note that the conjectured value function \Ep{LS_CK_equi2_value}
is (weakly) concave so that
the two inner optimization problems (one for working, another for not working) on the
right side of \Ep{LS_CK_equi2_value2} are both concave programming problems. Moreover, since the conjectured value function is continuous and
differentiable everywhere, we can solve each optimization problem (for
working and for not working) one by one, using first-order conditions, and
compare the values. Let $W(a,1)$ and $W(a,0)$ denote the value of working
and not working, respectively, and hence, $V(a) = \max\{W(a,1),\, W(a,0)\}$.

We start by verifying the conjectured value function for
$a\in (\underline{a}, \bar a)$ when time averaging should be an optimal
policy. First, conditional on working, take a first-order
condition with respect to $a'$ in the first inner optimization problem on
the right side of \Ep{LS_CK_equi2_value2}
$$\EQNalign{
{1 \over (1+r) a + w - a'} &= \beta V'(a'), \cr
a' &= {1 \over \beta}a + w - {w \over B}.   \EQN LS_CK_a_work \cr}
$$
Here we have invoked the conjectured steady-state interest rate,
$1+r= \beta^{-1}$, and proceeded as if $a'$ also falls in the
range $(\underline{a}, \bar a)$ where the conjectured
value function \Ep{LS_CK_equi2_value} has
derivative $V'(a')=B/(\beta w)$. Since $a'$ exceeds $a$, it follows
that $a$ must fall below some upper bound $a^{\star} < \bar a$
in order for $a'\in (\underline{a}, \bar a)$, where that upper bound
$a^{\star}$ is given by\NFootnote{Using expression \Ep{LS_CK_a_work}
for $a'$, the upper bound $a^{\star}$ on asset level $a$ that ensures
$a'\leq \bar a$, can be solved from
$$
{1 \over \beta}\,a^{\star} + w - {w \over B} = \bar a.
$$
Multiplying both sides by $\beta$, and subtracting and adding $\bar a$
on the right side, yield
$$
a^{\star} + \beta w - {\beta w \over B} = \beta \bar a - \bar a + \bar a.
$$
After invoking expression \Ep{LS_CK_ahigh} for $\bar a$, we find that
the last term on the left side is equal to the first two terms on the
right side, and hence, we have arrived at the equality in
\Ep{LS_CK_astarhigh}.}
$$a^{\star} = \bar a - \beta w < \bar a.      \EQN LS_CK_astarhigh
$$

Given the optimal choice of $a'$ in expression \Ep{LS_CK_a_work},
we can compute from the budget constraint that the implied consumption
level is $c=w/B$. With choices of both $a'$ and $c$ in hand, we
can compute the value of working,
$$\EQNalign{
W(a,1) &= \log (c) - B + \beta V(a')  \cr
& = \log \left({w \over B}\right) - B
+ \beta \left\{ {1 \over 1-\beta}
\left[ \log \left({w \over B} \right) -1 \right]
+ {B \over \beta w}
\left[ {1 \over \beta} a + w -{w \over B} \right] \right\} \cr
& = {1 \over 1-\beta} \left[ \log \left({w \over B}\right) - 1 \right]
+ {B \over \beta w} \, a = V(a),  \cr }
$$
and hence, we have confirmed that working yields
a value equal to our conjectured value function when
$a\in (\underline{a}, a^{\star})$, where the upper bound
$a^{\star}$ ensures that $a'\in (\underline{a}, \bar a)$.


Next, conditional on not working, the first-order
condition with respect to $a'$ in the second inner optimization problem on
the right side of \Ep{LS_CK_equi2_value2} is
$$\EQNalign{
{1 \over (1+r) a - a'} &= \beta V'(a'), \cr
a' &= {1 \over \beta}a - {w \over B},   \EQN LS_CK_a_notwork \cr}
$$
where we have invoked the conjectured steady-state interest rate,
$1+r= \beta^{-1}$, and proceeded as if $a'$ also falls in the
range $(\underline{a}, \bar a)$ where the conjectured
value function \Ep{LS_CK_equi2_value} has the
derivative $V'(a')=B/(\beta w)$. Since $a'$ falls below $a$, it follows
that $a$ must exceed some lower bound $a_{\star} > \underline{a}$ in
order for $a'\in (\underline{a}, \bar a)$, where that
lower bound $a_{\star}$ is given by\NFootnote{Using expression
\Ep{LS_CK_a_notwork}
for $a'$, the lower bound $a_{\star}$ on asset level $a$ that ensures
$a'\geq \underline{a}$, can be solved from
$$
{1 \over \beta}\,a_{\star} - {w \over B} = \underline{a}.
$$
Multiplying both sides by $\beta$, and subtracting and adding $\underline{a}$
on the right side, yield
$$
a_{\star} - {\beta w \over B} = \beta \underline{a} - \underline{a}
                                                    + \underline{a}.
$$
After invoking expression \Ep{LS_CK_alow} for $\underline{a}$, we find that
the first two terms on the right side is greater than the last term on
the left side by an amount $\beta w$, and hence, we have arrived at the
equality in \Ep{LS_CK_astarlow}.}
$$a_{\star} = \underline{a} + \beta w > \underline{a}. \EQN LS_CK_astarlow
$$

Given the optimal choice of $a'$ in expression \Ep{LS_CK_a_notwork},
we can compute from the budget constraint that the implied consumption
level is $c=w/B$. With choices of both $a'$ and $c$ in hand, we
can compute the value of not working to be
$$\EQNalign{
W(a,0) &= \log (c) + \beta V(a')  \cr
& = \log \left({w \over B}\right)
+ \beta \left\{ {1 \over 1-\beta}
\left[ \log \left({w \over B} \right) -1 \right]
+ {B \over \beta w}
\left[ {1 \over \beta} a -{w \over B} \right] \right\} \cr
& = {1 \over 1-\beta} \left[ \log \left({w \over B}\right) - 1 \right]
+ {B \over \beta w} \, a = V(a),  \cr }
$$
and hence, we have confirmed that not working yields
a value equal to our conjectured value function when
$a\in (a_{\star}, \bar a)$, where the lower bound
$a_{\star}$ ensures that $a'\in (\underline{a}, \bar a)$.


Following steps similar to above, we leave as an exercise for the
reader to complete the verification of conjectured value function
\Ep{LS_CK_equi2_value} (or consult the appendix
of Krusell et al.\ (2008)). In
particular, we can show that an agent with assets
$a\in (\underline{a}, a_{\star})$ ($a\in (a^{\star}, \bar a)$)
strictly prefers to work (not to work)
so that his next period's assets $a'$ do not fall outside of the
asset range $(\underline{a},\, \bar a)$. Thus, we conclude
that agents with assets $a\in (\underline{a},\, \bar a)$ find
it optimal to engage in time averaging, i.e., to alternate between
work and leisure, to finance an optimal consumption level
$\bar c= w/B$ with asset holdings fluctuating within the
range $a\in (\underline{a},\, \bar a)$. Also, we can verify
that an agent with assets $a\leq \underline{a}$
($a\geq \bar a$) strictly prefers to work forever
(never to work) and choose the highest affordable constant
consumption of $c=w+ra$ ($c=ra$).

Krusell et al.\ (2008) assume that
$\beta > 0.5$, which is required for
$a_{\star} < a^{\star}$.\NFootnote{To derive a parameter
restriction that ensures $a_{\star} < a^{\star}$, we substitute
expression \Ep{LS_CK_alow} and \Ep{LS_CK_ahigh}
for $\underline{a}$ and $\bar a$, respectively, into
expression \Ep{LS_CK_astarlow} and \Ep{LS_CK_astarhigh}
for $a_{\star}$ and $a^{\star}$, respectively,
$$
a_{\star} = \underline{a} + \beta w
          = {\beta w \over 1-\beta} [B^{-1} -1 ] + \beta w
\;<\; {\beta w \over 1-\beta} B^{-1} - \beta w
= \bar a - \beta w = a^{\star},
$$
which simplifies to $\beta > 0.5$.}
Together with expressions \Ep{LS_CK_astarhigh} and
\Ep{LS_CK_astarlow}, it then follows that
$\underline{a} < a_{\star} < a^{\star} < \bar a$.
Another implicit assumption of Krusell et al.\ (2008) is
that the preference parameter for the disutility of work
is high enough so that there exist interior solutions to an
agent's lifetime labor supply problem. In our formulation, that
parameter restriction is $B>1$.



\subsection{Equivalence of time averaging and lotteries}

Krusell et al.\ (2008) argue that there exists a stationary
equilibrium for the incomplete-market economy where
all agents engage in time averaging with assets in the range
$(\underline{a}, \bar a)$, and the aggregate values of
$K$ and $L$ are the same as in a corresponding
complete-market economy with employment lotteries.

We have already studied equilibrium outcomes in a more
general version of the complete-market economy in section
\use{sec:ChangKim_complete}. Under our present assumption
that all agents have a constant productivity level
that is normalized to one, equation \Ep{LS_CK_family_c}
shows that the optimal consumption level is $c=w/B$, and
the aggregate labor supply is given by the appropriate
version of equation \Ep{LS_CK_family_bc3}:
$$
{K \over L} = {1 \over \beta } {K \over L} + w
     - { w \over B L},
                                                    \EQN LS_CK_equi2_family
$$
where once again the capital-labor ratio $K/L$ and the wage
$w$ are determined by equations \Ep{LS_CK_family_r} and
\Ep{LS_CK_wr;a}. Hence, we can solve for the aggregate labor
supply $L$ from equation \Ep{LS_CK_equi2_family}.

In the stationary equilibrium of the incomplete-market economy
with time averaging,  agents are indifferent to
alternative lifetime labor supply paths that yield equal  present values of labor income. In a competitive
equilibrium, an `invisible hand'
 arranges  agents' labor and
savings decisions so that at every point in time, the aggregate
labor supply and aggregate asset holdings equal the same
constant aggregates $L$ and $K$ as those in  the complete-markets economy.
An equilibrium interest equal
to $1+r=1/\beta$ makes a constant
consumption $\bar c=w/B$ be the optimal choice for the worker-consumer.


\section{Pillars for high elasticity at the extensive margin}

The high labor supply elasticity at an interior
solution for career lenght rests on two pillars -- indivisibilities
in labor supply and time separable preferences. Labor 
indivisibilities cause workers 
to divide their lifetimes into parts  working and not working.  
Time-separable preferences imply that the marginal choice 
between those two parts occurs at a constant per-period disutility of 
work, giving that  high elasticity of labor supply at an interior 
solution for career length, i.e., at an {\it extensive} margin.

The labor-supply-indivisibility pillar is typically justified
by  the observation  that workers' hours of work
are mostly bunched at a few common values   with the `full-time'  value being  predominant.
Alternative assumptions about technologies and preferences can generate that 
observation. Simple examples would include a setup cost at work and
a fixed disutility of work. The time-separable-preferences pillar
is typically justified as doing a good job of approximating workers' wishes to rest and  refresh 
between periods.

As of now, no pillar has been identified in support
of the assertion that the labor supply elasticity should also be 
high at an {\it intensive} margin.


\section{\dots but not so at the intensive margin}

We will here examine, and ultimately refute the generality of,
the claim by Rogerson and Wallenius (2013, hereafter RW) 
that ``based on existing estimates of the size of nonconvexities 
and measures of full-time work prior to retirement, it is hard 
to rationalize values of the IES [intertemporal elasticity of
substitution for labor at the intensive margin] that are less 
than 0.75.''
 \auth{Rogerson, Richard}%
 \auth{Wallenius, Johanna}%

RW's nonconvexity takes the form of a wage schedule that is 
increasing in hours worked $n$,\NFootnote{The functional 
form \Ep{LSprofile_wage} for the lifetime earnings-experience 
profile in section \use{sec:LSprofile}, is here used to 
describe hourly wages as function of hours worked at the
intensive margin.}
$$
w(n) = w_o n^\theta,    \EQN RW_wage
$$
where $\theta \geq 0$. RW report empirical evidence for choosing
$\theta=0.4$, a value that we adopt throughout our analysis.
Following RW's sequence of analysis, we start by studying workers 
who face a continuous choice of hours $n\in [0,1]$, and then turn to 
what is deemed to be the more realistic case of a finite set of work
options: part-time and full-time work. 


RW assume that the disutility of work is given by 
$$
v^{\rm RW}(n) = \frac{1}{1-\frac{1}{\gamma}} 
                 \Bigl[1-(1-n)^{1-\frac{1}{\gamma}}\Bigr],  
                                                        \EQN RW_pref
$$
which exhibits a constant IES $\gamma$ for {\it leisure},
and the IES for {\it labor} is $\gamma \,(1-n)/n$.
Though, we begin our analysis by exploring an alternative 
functional form with a constant IES  $\phi$ for {\it labor}, 
as used by e.g.\ Prescott et al.\ (2009),
$$
v^{\rm P}(n) = n^{1+\frac{1}{\phi}}.      \EQN PRW_pref
$$
Preference specification \Ep{PRW_pref} is amenable to analytical
solutions that will starkly illuminate RW's findings that the IES 
for labor at the intensive margin must be high in order to support 
outcomes in which workers find it optimal to divide their lifetimes
in one part of working and another part of not working (rather than
smoothing their labor supply over the entire lifetime), and that
this conclusion is not altered by adding a fixed disutility of work,
in the sense that such a fixed disutility of work would have to be
unreasonably large in order to generate retirement for low values 
of the IES.
\auth{Prescott, Edward C.}%



\subsection{A race between elasticities}\label{sec:RW_constantIES}%

Let $\ell_t$ be effective units of labor supplied at time $t$, 
as measured by time-$t$ labor earnings, $\ell_t = w(n_t) n_t$. 
After invoking wage schedule \Ep{RW_wage}, 
we have
$$
\ell_t = w_o\, n_t^{1+\theta} \hskip.1cm \hbox{\rm , i.e., } \hskip.3cm
n_t = \left( \frac{\ell_t}{w_o} \right)^{\frac{1}{1+\theta}},
                                               \EQN PRW_mapp
$$
and we can compute the disutility of effective labor, under
preference specification \Ep{PRW_pref}, 
as
$$
\hat v^{\rm P}(\ell) 
= v^{\rm P}\left(\left[\frac{\ell}{w_o}\right]^{\frac{1}{1+\theta}}\right)
= \left[ \frac{\ell}{w_o} \right]^{\frac{1+ \frac{1}{\phi}}{1+\theta}},
                                                    \EQN PRW_pref2
$$
which is strictly increasing in effective labor, and 
$$
{d^2 \hat v^{\rm P}(\ell) \over d \ell^2 \hfill} =
\left[{1 \over \phi} - \theta \right] 
{1+ {\displaystyle 1 \over \displaystyle \phi} \over ((1+\theta)w_o)^2}
\left[ {\ell \over w_o} \right]^{{{1 \over \phi}-\theta \over 1+\theta}-1} 
\cases{>0 \,,
    \; &if  $\phi < 1 / \theta$; \cr   %${1 \over \phi} > \theta$; \cr
\noalign{\vskip.2cm} \cr
    =0 \,,
    \; &if  $\phi = 1 / \theta$; \cr   %${1 \over \phi} = \theta$; \cr
\noalign{\vskip.2cm} \cr
    <0 \,,
    \; &if  $\phi > 1 / \theta$. \cr}  %${1 \over \phi} < \theta$.\cr}
                                           \EQN PRW_pref2deriv
$$
Whether or not retirement could be an optimal choice depends on 
the IES for labor, as given by $\phi$, in relation to the 
curvature of the wage schedule. At low values of IES, 
$\phi < 1/\theta$, the marginal disutility of effective labor is
increasing according to \Ep{PRW_pref2deriv} and thus, a 
worker would choose to spread her labor supply evenly over the 
entire lifetime without any period of retirement. Therefore, if 
we observe that workers retire, the IES would need to be greater 
than or equal to $1/\theta$. Moreover, if there is an interior 
solution at the intensive margin when working, the IES would 
have to be exactly equal to $1/\theta$, 
which would mean that $IES=2.5$ under the maintained assumption 
of $\theta=0.4$. At that knife-edged parameter condition,
the marginal disutility of effective labor is constant according
to \Ep{PRW_pref2deriv} and hence, the worker would be
indifferent to the allocation of hours worked since the disutility
of work is linear in effective labor, as
illustrated by the solid line in Figure \Fg{PRW_fig1}.
For higher values of the IES, $\phi > 1/\theta$, the labor
supply at the intensive margin becomes determinate because 
the decreasing marginal disutility of effective labor 
in \Ep{PRW_pref2deriv} implies that it
is optimal to allocate all available time to work in periods
with positive labor supply. Thus, if any leisure is desired
over the worker's lifetime, that leisure would have to be 
allocated to periods of retirement when the optimal supply of 
labor is set equal to zero.


%Since preferences are consistent with balanced growth, we can
%without loss of generality proceed by setting $w_0=1$. 

\midfigure{PRW_fig1}
\centerline{\epsfxsize=3truein\epsffile{Disutil_serv.eps}}
\caption{Disutility of work under preference specification 
with a constant IES for labor, 
$\phi\in\{0.25, 1.0, 2.5, 10\}$, 
as a function of effective labor $\ell$ supplied
(under normalization $w_o =1$).}
\infiglist{PRW_fig1}
\endfigure

\subsection{Adding a fixed disutility of work}

To overturn the result that retirement cannot be a rational choice
for low values of IES, we follow in the spirit of RW and introduce a 
fixed disutility $B_o$ of work, which is incurred in any period of 
working and does not vary with the choice of hours worked. At an 
interior solution to the intensive and the extensive margin, and 
when $\phi < 1/\theta$, the optimal choice $\ell^\star$ would 
satisfy
$$
{d \hat v^{\rm P}(\ell) \over d \ell\hfill}\Bigg|_{\ell=\ell^\star} = 
{\hat v^{\rm P}(\ell^\star) + B_o \over \ell^\star},
$$
that is, optimality requires that the marginal and the average 
disutility of effective labor are equal in order to minimize 
lifetime disutility of work. The solution to this condition of
optimality is 
$$
\ell^\star = \left[ {1+\theta \over {\displaystyle 1 
                                  \over \displaystyle \phi}-\theta} B_o 
             \right]^{{1+\theta \over 1+ {1 \over \phi}}} w_o.
$$
Next, we compute how the fixed disutility of work is related to 
the total disutility of work in periods of working,
$$
{B_o \over \hat v^{\rm P}(\ell^\star) +  B_o} 
        = { {1 \over \phi} - \theta  \over
            1+ {1 \over \phi} }.
                                            \EQN fracdisutil
$$
Under the maintained assumption of $\theta=0.4$, RW conclude that 
the fixed disutility of work is unreasonably large relative to 
the total disutility of work at low values of the IES, as 
shown in Figure~\Fg{PRW_fig2}. 


\midfigure{PRW_fig2}
\centerline{\epsfxsize=3truein\epsffile{Disutil_fixedcost.eps}}
\caption{Fixed disutility of work as a fraction of 
total disutility under the so-augmented preference 
specification with a constant IES 
for labor, $\phi\in(0, 2.5]$, given an interior solution
to the intensive and the extensive margin.}
\infiglist{PRW_fig2}
\endfigure


\subsection{Two work options}

As mentioned, RW question the realism of $n$ as a continuous
choice variable and proceed to analyze a finite set of work
options; specifically, in addition to a full-time option $n_f$, 
RW ``consider the case in which there is one part-time option 
available, characterized by hours $n_p = 0.5 n_f$, and that the 
wage per unit of time for this option'' relative to the full-time 
option is $(n_p/n_f)^\theta \approx 0.75$, as given by wage
schedule \Ep{RW_wage} with $\theta=0.4$. 

While inconsequential to preference specification \Ep{PRW_pref},
the specific value of $n_f\in (0,1]$ is central to RW's preference
specification \Ep{RW_pref}. So for the sake of compatibility, we
adopt RW's preferred parameterization of $n_f = 0.385$, with their
motivation that full-time work on an annual basis constitutes 2,000
hours out of an assumed total discretionary time of 5,200 hours.
Figure \Fg{PRW_fig3} zooms in on this relevant range of 
Figure \Fg{PRW_fig1}, where we have marked the effective units of 
labor at $n_p$ and $n_f$ as $\ell_p$ and $\ell_f$, and kept the 
solid line ($\phi=2.5$) and lower dashed line ($\phi=0.25$) from
Figure \Fg{PRW_fig1}. The disutilities of work are rescaled to
equal one at the full-time work option, as marked by a circle.
The other circle in the figure is the hypothetical disutility of 
work ($0.5 \cdot 0.75$) at the part-time option that would imply
indifference between the two work options. In accordance with our
discussion in section \use{sec:RW_constantIES}, the latter circle
does indeed also fall on the solid line, while the lower dashed line 
marked `0.25' is much below that circle, since a low IES makes it 
optimal to smooth labor supply as much as possible over the entire
lifetime, and hence, the part-time work option is preferred to the
full-time work option. 


\midfigure{PRW_fig3}
\centerline{\epsfxsize=3.5truein\epsffile{Disutil_eff_labor.eps}}
\caption{Disutility of work under two preference specifications
with a constant IES for labor, $\phi \in \{2.5, 0.25\}$,
and two alternative specifications RW and LS, where the
latter two exhibit IES for labor equal to 1.18 and 0.24, respectively,
at full-time work. $\ell_p$ and $\ell_f$ show effective units 
of labor at part-time and full-time work, respectively (with
normalization $w_o =1$). All disutilities are rescaled to equal 
one at $\ell_f$.} 
\infiglist{PRW_fig3}
\endfigure


The labor supply implications of preference specification \Ep{PRW_pref} 
with a constant IES for work in Figure \Fg{PRW_fig3} are stark: 
for $\phi < 2.5$, it is optimal to only choose the part-time option 
(unless a worker finds herself working her entire life and values
additional consumption so much that she is compelled to also exercise 
the more costly full-time work option); and for $\phi > 2.5$, it is 
optimal to only choose the full-time option. Since observations
suggest that the full-time work option is the predominant one and
that workers do enter into retirement, we conclude that the IES must 
be at least $2.5$.

While RW do not pursue the stark implications of preference 
specification \Ep{PRW_pref}, Ljungqvist and Sargent (2014) explain how
similar outcomes emerge under RW's assumed preference specification 
\Ep{RW_pref} with a constant IES $\gamma$ for leisure. 
In particular, the IES for work at the full-time work option cannot 
be too low in order to rationalize outcomes by which workers retire 
from full-time work and do not prefer the part-time work option. As a
reference, the upper dashed line marked `RW' in Figure \Fg{PRW_fig3}
depicts the disutility of work for RW's preferred parameterization
(rescaled to equal one at the full-time option),\NFootnote{The
parameterization is RW's favorite on in the sense of implying hours 
worked $n_f=0.385$ in the setup with $n$ as a continuous choice 
variable. \label{foot:RW_param}}
where the IES is 1.18 at the full-time option.
Under preference specification \Ep{RW_pref},
RW show that the IES at the full-time 
option must be greater than $0.75$ in order for the part-time 
option not being preferred to the full-time option.


\subsection{A lack of generality}

RW's finding that the IES for work must be high to match
observations depends sensitively on their decision to assume  
utility function \Ep{RW_pref}, or for that matter, our
assumption above of utility function \Ep{PRW_pref}. To show this, we
blend those two utility functions to construct an alternative one:
$$
v^{\rm LS}(n) = \mu_1 v^{\rm RW}(n) + 
                 \mu_2 v^{\rm P}( \max \{0,\, n-n_p \} ).
                                                   \EQN LS_pref
$$
The utility function \Ep{LS_pref} augments the RW disutility of work, 
$v^{\rm RW}(n)$, 
with extra disutility for hours of work above $n_p$ measured
by the Prescott et al.\ (2009)   disutility 
$v^{\rm P}( \max \{0,\, n-n_p \})$. We set the 
preference parameter $\gamma$ in $v^{\rm RW}(\cdot)$  
to be the favorite value of RW (see footnote \use{foot:RW_param}), 
while we set the preference parameter
$\phi$ in $v^{\rm P}(\cdot)$  to be $0.25$. Next,
we set the weights $\mu_1$ and $\mu_2$  so that 
$v^{\rm LS}(h_f) = 1$ and 
$v^{\rm LS}(h_p) = 0.5 \cdot 0.75$.\NFootnote{The weights $\mu_1$ 
and $\mu_2$ then satisfy 
$$
\mu_1 = \frac{0.5 \cdot 0.75}{v^{\rm RW}(h_p)}  
            \hskip1cm \hbox{\rm and} \hskip1cm
\mu_2 = \frac{1 - \mu_1 v^{\rm RW}(h_f)}{v^{\rm P}(h_f - h_p) }.
$$}
As depicted by the dashed line marked `LS' in Figure \Fg{PRW_fig3}, 
the resulting disutility of work $v^{\rm LS}(n)$ goes through
the two circles and hence, a worker would be indifferent between 
$h_p$ and $h_f$ at an interior solution to career length.
In contrast to RW's utility function with
IES equal to 1.18 at the full-time option, our alternative
utility function exhibits an IES of merely 0.24 at the full-time
option.


Without relying on any other nonconvexity than the one used by
RW, our alternative preference specification questions
RW's (2013, p.~1461) conclusion ``that based on existing estimates 
of the size of nonconvexities and measures of full-time work prior 
to retirement, it is hard to rationalize values of the IES that are 
less than 0.75.'' To the contrary, it is not hard.  We showed this  
by simply blending preferences that were actually used by RW themselves
in closely related contexts.  


\subsection{Intuition behind our counterexample}

The intuition behind our counterexample to RW's claim is as 
follows. First, consider preference specification \Ep{PRW_pref}
with a constant IES $\phi$ for work. When assuming a lower IES, 
the growth in disutility of work is accelerated because at any 
level of supplied hours, less willingness to intertemporally 
substitute must necessarily show up as a relatively larger increase 
in the disutility of work (as compared to a preference specification
with a higher IES). Hence, a lower IES implies a more 
convex shape of the disutility of work, as illustrated by the lower
dashed line labelled `0.25' in Figure \Fg{PRW_fig3}.
Next, a more convex disutility of work implies that `part-time' work 
options tend to be preferred over a `full-time' work option, so as
to enjoy the relatively much lower disutility per hour at
fewer hours of work. In fact, as we learnt above, a low IES makes
it optimal to smooth labor supply as much as possible over 
the entire lifetime. 

Second, regarding preference specification \Ep{RW_pref} with a 
constant IES $\gamma$ for leisure, we note that the IES for work, as 
given by $\gamma\, (1-n)/n$, asymptotes to infinity when hours of work 
goes to zero. Therefore, for some initial range of hours of work, the 
IES for work will necessarily be
high and that heightened willingness to intertemporally substitue,
means that the disutility of work increases almost linearly over
an initial range of hours of work. Given wage schedule 
\Ep{RW_wage}, this translates initially into a concave disutility 
of work as a function of effective labor, as illustrated by the 
upper dashed line labelled `RW' in Figure \Fg{PRW_fig3}.

Third, based on these observations, we can shed light on 
the construction of our counterexample to RW. 
%explain how the blending 
%of preference specifications in \Ep{LS_pref} gives rise to our 
%counterexample. 
Over the first range of hours of work $[0, n_p]$, our preference
specification \Ep{LS_pref} is the same as that of RW, 
$v^{\rm RW}(n)$, for which the initial almost linear growth in 
disutility of work counteracts any attractiveness of the part-time 
work option. Next, by thereafter adding extra disutility of work, 
$v^{\rm P}( \max \{0,\, n-n_p \} )$, as given by 
a specification with a constant IES $\phi$ for
work, and by assuming a small $\phi$, our counterexample
is immediate -- the small $\phi$ causes an 
accelerated growth in disutility of work and hence, this added
component will eventually come to dominate the IES of
preference specification \Ep{LS_pref}, i.e.,  
the IES at the full-time work option will become close to our 
assumed small value of $\phi$.


%the accelerating disutility of work under .....
%means that the IES for work under the mixed preference
%specfication will soon be soon driven solely by the second
%component \Ep{  } that was layered on top RW's disutility of
%work. The first component RW with is high IES (but decreasing)
%IES for work over the intial range of hours worked, serve
%to accumulate disutility of work almost linearly which in
%turn explains why any part-time work option will not be
%attractive to exercise (given the nonconvexity in wage
%function \Ep{   } ).




\section{Concluding remarks}

A high aggregate labor supply elasticity hinges on a substantial
fraction of agents being at an interior solution with respect to their lifetime labor supplies.
 %that gives rise to
%a high aggregate labor supply elasticity, and it is less important
%whether or not there are employment lotteries and complete
%insurance markets. This conclusion can be reached in models of
This finding emerges  from models with
finitely-lived agents who choose career length and also
in Chang and Kim's (2007) model of infinitely-lived agents
who engage in time averaging across periods of work and leisure.

 When agents are finitely lived, two forces  can lower the labor supply
elasticity: (1) government financed
social security retirement schemes that leave agents at a corner
solution with respect to their choices of career lengths,  and (2)  large adverse
labor market shocks towards the end of working lives that
prematurely terminate careers by pushing  the shadow value of additional labor earnings  below the utility
of leisure in early retirement.


It is an occasion to celebrate that two  camps of researchers,
namely, those who have championed high and low labor supply elasticities,
have come together in adopting the same theoretical
framework.  Nevertheless, the serious division  between the two camps about quantitative magnitudes of labor supply elasticities
 persists.
But we see the emergence of agreement over a basic theoretical framework as
 genuine progress relative
to the earlier stalemate when proponents of employment
lotteries used macroeconomic observations to build support for
their aggregation theory, while opponents brought a different
set of microeconomic observations to refute the employment lotteries
allocation mechanism.\NFootnote{It would
be a mistake to regard the abandonment  of a stand-in household with
its employment lotteries as  unconditional surrender to
the other tradition in macroeconomics of overlapping generations
models that has commonly postulated incomplete markets. The reason is that earlier
work in the overlapping generations tradition has often  postulated an exogenous retirement
age, shutting down the key choice focused on in time-averaging models of career choice.  It is the possibility
of interior solutions to lifetime labor supply in combination with
indivisible labor that have led real business cycle researchers like
 Prescott (2006) to embrace lifecycle models of labor supply.}

To illustrate how far we have come, we revisit our own
section \use{sec:lotteries_firms_households} reasoning, where we are concerned about an
asymmetry between idle firms and idle workers in a particular
model. While idle firms are truly indifferent about their operating status because
operating firms are just breaking even without making any profits,
``the aggregation theory behind the stand-in household has an
additional aspect that is not present in the theory that
aggregates over firms, namely, it says how consumption and
leisure are smoothed across households with the help of
an extensive set of contingent claim markets. This market
arrangement and randomization device stands at the center
of the employment lottery model. To us, it seems that they
make the aggregation theory behind the stand-in household
fundamentally different than the well-known aggregation
theory for the firm side.'' Well, we now also can assert   that this
difference is not important for those households who, being  at an interior
solution for lifetime labor supply, are about to choose whether to supply  more of their indivisible labor
by extending their careers  before  retiring.

Having a diverse group of researchers focus on a common set of
observations on lifetime labor supply within a common theoretical
framework bodes well for the eventual arrival of what we hope will be the
``labor supply elasticity accord'' foretold by Ljungqvist and
Sargent (2011).







