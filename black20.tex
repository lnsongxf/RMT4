%  appendix

%\input gayejnl.txt
\input gayedef.txt
%\def\prob{{\rm prob}}
%\showchaptIDtrue
%\def\@chaptID{20.}
%
%\hbox{}\footnum=0
\footnum=0
\chapternum=27
\Appendix{A}{Functional Analysis\label{functional}}
This appendix provides an introduction to the analysis of functional equations
(functional analysis).  It describes the contraction mapping theorem,
a workhorse for studying dynamic programs.

\section{Metric spaces and operators}

We begin with the definition of a metric space, which is a pair of objects, a
set $X$, and a function $d$.\NFootnote{General references on the %
mathematics described in this appendix are Luenberger (1969) and Naylor %
and Sell (1982).}
\medskip
%\specsec{Definition:}
\definition{def20.1}
{\it A metric space is a set $X$ and a function $d$
called a metric, $d$: $X\times X\to R$.  The metric $d(x,y)$ satisfies the
following four properties:}
\medskip
\itemitem{M1.} {\it Positivity: $d(x,y)\ge 0$ for all $x,y\in X$.}
\medskip
\itemitem{M2.} {\it Strict positivity: $d(x,y)=0$ if and only if $x=y$.}
\medskip
\itemitem{M3.} {\it Symmetry: $d(x,y)=d(y,x)$ for all $x,y\in X$.}
\medskip
\itemitem{M4.} {\it Triangle inequality: $d(x,y)\le d(x,z)+d(z,y)$ for all $x,
y$,
and $z\in X$.}
\enddefinition
\medskip

We give some examples of the metric spaces with which we will be working:
\medskip
\noindent
{\bf Example A.1.}\quad  $l_p[0,\infty)$.  We say that $X=l_p[0,\infty)$ is
the set of all sequences of complex numbers $\{x_t\}_{t=0}^\infty$ for which
$\sum_{t=0}^\infty |x_t|^p$ converges, where $1\le p<\infty$.  The function
$d_p(x,y)=(\sum_{t=0}^\infty |x_t-y_t|^p)^{1/p}$ is a metric.  Often, we will
say that $p=2$ and will work in $l_2[0,\infty)$.
\medskip
\noindent{\bf Example A.2.}\quad  $l_\infty[0,\infty)$.  The set
$X=l_\infty[0,\infty)$ is the set of bounded sequences $\{x_t\}_{t=0}^\infty$
of real or complex numbers.  The metric is $d_\infty(x,y) = {\rm
sup}_t|x_t-y_t|$.
\medskip
\noindent{\bf Example A.3.}\quad $l_p(-\infty,\infty)$ is the set
of ``two-sided''
sequences $\{x_t\}_{t=-\infty}^\infty$ such that $\sum_{t=-\infty}^\infty
|x_t|^p<+\infty$, where $1\le p<\infty$.  The associated metric is $d_p(x,y) =
(\sum_{t=-\infty}^\infty |x_t-y_t|^p)^{1/p}$.
\medskip
\noindent{\bf Example A.4.}\quad $l_\infty(-\infty,\infty)$ is the
set of bounded
sequences $\{x_t\}_{t=-\infty}^\infty$ with metric $d_\infty(x,y)={\rm sup}
|x_t-y_t|$.
\medskip
\noindent{\bf Example A.5.}\quad  Let $X=C[0,T]$ be the set of all continuous
functions mapping the interval $[0,T]$ into $R$.  We consider the metric
$$d_p(x,y) = \left[ \int_0^T |x(t) -y(t)|^p dt\right]^{1/p},$$
where the integration is in the Riemann sense.
\medskip
\noindent{\bf Example A.6.}\quad  Let $X=C[0,T]$ be the set of all continuous
functions mapping the interval $[0,T]$ into $R$.  We consider the metric
$$d_\infty(x,y) =\sup\limits_{0\le t\le T} |x(t)-y(t)|.$$
\medskip
We now have the following important definition:
\medskip
%\specsec{Definition:}
\definition{def20.2}
{\it A sequence $\{x_n\}$ in a metric space $(X,d)$
is said to be a Cauchy sequence if for each $\eps>0$ there exists an $N(\eps)$
such that $d(x_n,x_m)<\eps$ for any $n,m\ge N(\eps)$.  Thus a sequence
$\{x_n\}$ is said to be Cauchy if $\lim_{n,m\to\infty} d(x_n,x_m)=0$.}
\enddefinition
\medskip
We also have the following definition of convergence:
\medskip
%\specsec{Definition:}
\definition{def20.3}
{\it A sequence $\{x_n\}$ in a metric space $(X,d)$
is said to converge to a limit $x_0\in X$ if for every $\eps>0$ there exists an
$N(\eps)$ such that $d(x_n,x_0)<\eps$ for $n\ge N(\eps)$.}
\enddefinition
\medskip
The following lemma asserts that every convergent sequence in $(X,d)$ is a
Cauchy sequence:
\medskip
\lemma{lemma20.1}  {\it Let $\{x_n\}$ be a convergent sequence in a metric
space $(X,d)$. Then $\{x_n\}$ is a Cauchy sequence.}\endlemma
\medskip
%\specsec{Proof:}
\proof Fix any $\eps>0$.  Let $x_0 \in X$
 be the limit of $\{x_n\}$.  Then for
all $m,n$ one has
$$d(x_n,x_m)\le d(x_n,x_0)+d(x_m,x_0)$$
by virtue of the triangle inequality.  Because $x_0$ is the limit of $\{x_n\}$,
there exists an $N$ such that $d(x_n,x_0)<\eps/2$ for $n\ge N$.  Together with
the preceding inequality, this statement implies that $d(x_n,x_m)<\eps$
 for $n,m\ge N$.  Therefore, $\{x_n\}$ is a Cauchy sequence.  \endproof %\qed
\medskip

We now consider two examples of sequences in metric spaces.  The examples are
designed to illustrate aspects of the concept of a Cauchy sequence.  We first
consider the metric space $\{C[0,1],d_2(x,y)\}$.  We let $\{x_n\}$ be the
sequence of continuous functions $x_n(t)=t^n$.  Evidently this sequence
converges pointwise to the function
$$x_0(t)=\cases{ 0, & $0\le t<1$\cr
1,& $t=1$.\cr}$$
Now, in $\{C[0,1], d_2(x,y)\}$, the sequence $x_n(t)$ is a Cauchy sequence.  To
verify this claim, calculate
$$d_2(t^m,t^n)^2 =\int_0^1 (t^n-t^m)^2 dt = {1\over 2n+1} +{1\over 2m+1}
-{2\over m+n+1}.$$
Clearly, for any $\eps>0$, it is possible to choose an $N(\eps)$ that makes the
square root of the right side less than $\eps$ whenever $m$ and $n$ both exceed
$N$.  Thus $x_n(t)$ is a Cauchy sequence.  Notice, however, that the limit
point $x_0(t)$ does {\it not\/} belong to $\{C[0,T],d_2(x,y)\}$ because it is
not a continuous function.

As our second example, we consider the space $\{C[0,T],d_\infty(x,y)\}$.  We
consider the sequence $x_n(t)=t^n$.  In $(C[0,1],d_\infty)$, the sequence
$x_n(t)$ is {\it not\/} a Cauchy sequence.  To verify this claim, it is
sufficient to establish that, for any fixed $m>0$, there is a $\de >0$ such
that
$$\sup\limits_{n>0} \sup\limits_{0\le t\le 1} |t^n-t^m|>\de.$$
Direct calculations show that, for fixed $m$,
$$\sup\limits_{n} \sup\limits_{0\le t\le 1} |t^n-t^m|=1.$$
Parenthetically we may note that
$$\eqalign{\sup\limits_{n>0} \sup\limits_{0\le t\le 1} |t^n-t^m| &=
\sup\limits_{0\le t\le 1} \sup\limits_{n>0} |t^n-t^m| =\sup\limits_{0\le t\le
1} \lim_{n\to\infty} |t^n-t^m|\cr
&= \sup\limits_{0\le t\le 1} \lim_{n\to\infty} t^m|t^{n-m} -1|
=\sup\limits_{0\le t\le 1} t^m=1.\cr}$$
Therefore, $\{t^n\}$ is not a Cauchy sequence in $(C[0,1],d_\infty)$.

These examples illustrate the fact that whether a given sequence is Cauchy
depends on the metric space within which it is embedded, in particular on the
metric that is being used.  The sequence $\{t^n\}$ is Cauchy in $(C[0,1],d_2)$,
and more generally in $(C[0,1],d_p)$ for $1\le p<\infty$.  The sequence
$\{t^n\}$, however, is {\it not\/} Cauchy in the metric space
$(C[0,1],d_\infty)$.  The first example also illustrates the fact that a Cauchy
sequence in $(X,d)$ need {\it not\/} converge to a limit point $x_0$ belonging
to the metric space.  The property that Cauchy sequences converge to points
lying in the metric space is desirable in many applications.  We give this
property a name.
\medskip
%\specsec{Definition:}
\definition{def20.4} {\it A metric space $(X,d)$ is said to be complete
if each Cauchy sequence in $(X,d)$ is a convergent sequence in $(X,d)$.  That
is, in a complete metric space, each Cauchy sequence converges to a point
belonging to the metric space.} \enddefinition
\medskip
The following metric spaces are complete:
$$\eqalign{ & (l_p[0,\infty),d_p),\qquad 1\le p<\infty\cr
&(l_\infty [0,\infty),d_\infty)\cr
&(C[0,T],d_\infty). \cr}$$
The following metric spaces are not complete:
$$(C[0,T],d_p),\qquad 1\le p<\infty.$$
\medskip
Proofs that $(l_p[0,\infty),d_p)$ for $1\le p\le \infty$ and
$(C[0,T],d_\infty)$ are complete are contained in Naylor and Sell (1982,
chap. 3).  In effect, we have already shown by counterexample that the space
$(C[0,1],d_2)$ is not complete, because we displayed a Cauchy sequence that did
not converge to a point in the metric space.  A definition may now be stated:
\medskip
%\specsec{Definition:}
\definition{def20.5} {\it A function $f$ mapping a metric space $(X,d)$
into itself is called an operator.}\enddefinition
\medskip
We need a notion of continuity of an operator.
\medskip
%\specsec{Definition:}
\definition{def20.6} {\it Let $f:X\to X$ be an operator on a metric
space $(X,d)$.  The operator $f$ is said to be continuous at a point $x_0\in X$
if for every $\eps>0$ there exists a $\de >0$ such that $d[f(x),f(x_0)]<\eps$
whenever $d(x,x_0)<\de$.  The operator $f$ is said to be continuous if it is
continuous at each point $x\in X$.}
\enddefinition
\medskip
We shall be studying an operator with a particular property,
the application of which to
any two distinct points $x,y\in X$ brings them closer together.
\medskip
%\specsec{Definition:}
\definition{def20.7}{\it Let $(X,d)$ be a metric space and let $f:X\to
X$.  We say that $f$ is a contraction or contraction mapping if there is a real
number $k,0\le k<1$, such that}
$$d[f(x),f(y)]\le kd(x,y)\qquad{\rm for\ all}\quad x,y\in X.$$
It follows directly from the definition that a contraction mapping is a
continuous operator.
\medskip
We now state the following theorem:

%\specsec{Theorem 20.1:}
\theorem{th20.1}
{\it Contraction Mapping}
\medskip
\noindent
{\it Let $(X,d)$ be a complete metric space and let $f:X\to X$ be a
contraction.  Then there is a unique point $x_0\in X$ such that $f(x_0)=x_0$.
Furthermore, if $x$ is any point in $X$ and $\{x_n\}$ is defined inductively
according to $x_1=f(x), x_2=f(x_1),\ldots,x_{n+1}=f(x_n)$, then $\{x_n\}$
converges to $x_0$.}\endtheorem
\medskip
%\specsec{Proof:}
\proof Let $x$ be any point in $X$. Define $x_1=f(x),x_2=f(x_1),\ldots$.
Express this as $x_n=f^n(x)$.  To show that the sequence $x_n$ is Cauchy, first
assume that $n>m$.  Then
$$\eqalign{ d(x_n,x_m) &= d[f^n(x),f^m(x)]=d[f^m(x_{n-m}),f^m(x)]\cr
&\le kd [f^{m-1} (x_{n-m}),f^{m-1}(x)]\cr}$$
By induction, we get
$$d(x_n,x_m) \le k^md(x_{n-m},x).\leqno(*)$$
When we repeatedly use the triangle inequality, the preceding inequality implies
that
$$d(x_n,x_m)\le k^m[d(x_{n-m}, x_{n-m-1})+\ldots + d(x_2,x_1) +d(x_1,x)].$$
Applying ($*$) gives
$$d(x_n,x_m) \le k^m(k^{n-m-1} +\ldots+ k+1)d(x_1,x).$$
Because $0\le k<1$, we have
$$d(x_n,x_m) \le k^m\sum_{i=0}^\infty k^id(x_t,x) ={k^m\over 1-k}
d(x_1,x).\leqno(\dagger)$$
The right side of $(\dagger)$ can be made arbitrarily small by choosing $m$
sufficiently large.  Therefore, $d(x_n,x_m)\to 0$ as $n,m\to \infty$.  Thus
$\{x_n\}$ is a Cauchy sequence.  Because $(X,d)$ is complete, $\{x_n\}$
converges to an element of $(X,d)$.

The limit point $x_0$ of $\{x_n\}=\{f^n(x)\}$ is a fixed point of $f$.  Because
$f$ is continuous, $\lim_{n\to\infty} f(x_n)=f(\lim_{n\to\infty} x_n)$.  Now
$f(\lim_{n\to\infty} x_n)=f(x_0)$ and $\lim_{n\to\infty} f(x_n)
=\lim_{n\to\infty} x_{n+1} =x_0$.  Therefore $x_0=f(x_0)$.

To show that the fixed point $x_0$ is unique, assume the  contrary.  Assume
that $x_0$ and $y_0$, $x_0\not=y_0$, are two fixed points of $f$.  But then
$$0<d(x_0,y_0) =d[f(x_0), f(y_0)]\le kd(x_0,y_0) <d(x_0,y_0),$$
which is a contradiction.  Therefore $f$ has a unique fixed point.  \endproof %\qed
\medskip

 We now restrict ourselves to sets $X$ whose elements are functions.  The
spaces $C[0,T]$ and $l_p[0,\infty)$ for $1\le p\le \infty$ are examples of
spaces of functions.  Let us define the notion of inequality of two functions.

%\specsec{Definition:}
\definition{def20.8} {\it Let $X$ be a space of functions, and let
$x,y\in X$.  Then $x\ge y$ if and only if $x(t)\ge y(t)$ for every $t$ in the
domain of the functions.}\enddefinition
\medskip
Let $X$ be a space of functions.  We use the $d_\infty$ metric, defined as
$d_\infty(x,y)=\sup_t |x(t)-y(t)|$, where the supremum is over the domain of
definition of the function.

A pair of conditions that are sufficient for an operator $T:(X,d_\infty)\to
(X,d_\infty)$ to be a contraction appear in the following
theorem:\NFootnote{See Blackwell's (1965) Theorem 5.  This theorem is %
used extensively by Stokey and Lucas with Prescott (1989).}
\medskip
%\specsec{Theorem 20.2:}
\theorem{thm20.2} {\it Blackwell's Sufficient Conditions for $T$ to
be a Contraction}
\medskip
\noindent
{\it Let $T$ be an operator on a metric space $(X,d_\infty)$, where $X$ is a
space of functions.  Assume that $T$ has the following two
properties:}\hfil\break
\noindent(a) {\it Monotonicity:  For any $x,y\in X$, $x\ge y$ implies $T(x) \ge
T(y)$}.\hfil\break
\noindent(b) {\it Discounting:  Let $c$ denote a function that is constant at the
real value $c$ for all points in the domain of definition of the functions in
$X$.  For any positive real $c$ and every $x\in X$, $T(x+c)\le T(x)+\be c$ for
some $\be$
satisfying $0\le \be <1$.}\hfil\break
\noindent
{\it Then $T$ is a contraction mapping with modulus $\be$.}\endtheorem
\medskip
%\specsec{Proof:}
\proof For all $x,y \in X$, $x \leq y + d(x,y)$.  Applying
properties (a) and  (b) to this inequality gives
$$T(x) \leq T(y+d(x,y)) \leq T(y) + \beta d(x,y). $$
Exchanging the roles of $x$ and $y$ and using the same logic
implies
$$ T(y) \leq T(x) + \beta d(x,y). $$
Combining  these two inequalities gives $|T(x) - T(y)| \leq \beta
d(x,y)$ or
$$ d(T(x), T(y)) \leq \beta d(x,y) .$$\endproof %\qed

\medskip

\section{Discounted dynamic programming}

We study the functional equation associated with a discounted dynamic
programming problem:
$$v(x)=\max_{u\in R^k} \{r(x,u)+\be v(x')\},\qquad x'\le g(x,u),\qquad
0<\be<1.  \EQN A46 $$
We assume that $r(x,u)$ is real valued, continuous, concave, and bounded and
that the set $[x',x,u:x'\le g(x,u),u\in R^k]$ is convex and compact.

We define the operator
$$Tv=\max_{u\in R^k} \{r(x,u)+\be v(x')\},\qquad x'\le g(x,u),\quad x\in X.$$
We work with the space of continuous bounded functions mapping $X$ into the
real line.  We use the %$d_\infty$
 metric
$d_\infty(v,w) =\sup_{x\in X} |v(x)-w(x)|.$
This metric space is complete.

The operator $T$ maps a continuous bounded function $v$ into a continuous
bounded function $Tv$.  (For a proof, see Stokey and Lucas
with Prescott, 1989.)\NFootnote{The assertions in the preceding two
paragraphs are the most difficult  pieces of the argument to prove.}

We now establish that $T$ is a contraction by verifying Blackwell's pair of
sufficient conditions.  First, suppose that $v(x)\ge w(x)$ for all $x\in X$.
Then
$$\eqalign{ Tv&=\max_{u\in R^k} \{r(x,u)+\be v(x')\},\qquad x'\le g(x,u)\cr
&\ge \max_{u\in R^k} \{r(x,u)+\be w(x')\},\qquad x'\le g(x,u)\cr
&=Tw.\cr}$$
Thus, $T$ is monotone.  Next, notice that for any positive constant $c$,
$$\eqalign{ T(v+c) &=\max_{u\in R^k} \{r(x,u)+\be [v(x')+c]\},\qquad x'\le
g(x,u)\cr
&=\max_{u\in R^k} \{r(x,u)+\be v(x')+\be c\},\qquad x'\le g(x,u)\cr
&=Tv+\be c.\cr}$$
Thus, $T$ discounts.  Therefore, $T$ satisfies both of Blackwell's conditions.
It follows that $T$ is a contraction on a complete metric space.  Therefore the
functional equation \Ep{A46}, which can be expressed as $v=Tv$, has a unique
fixed point in the space of bounded continuous functions.  This fixed point is
approached in the limit in the $d_\infty$ metric by iterations $v^n=T^n(v^0)$
starting from any bounded and continuous $v^0$.  Convergence in the $d_\infty$
metric implies uniform convergence of the functions $v^n$.

Stokey and Lucas with Prescott (1989) show that $T$ maps concave functions
into concave functions.  It follows that the solution of $v=Tv$ is a concave
function.


\subsection{Policy improvement algorithm}

For ease of exposition, in this section we shall assume that the
constraint $x' \leq g(x,u)$ holds with equality.  For the
purposes of describing an alternative way to solve dynamic
programming problems, we introduce a new operator.
We use one step of iterating on the Bellman equation to
define the new operator $T_\mu$ as follows:
$$T_\mu (v) = T(v)$$
or
$$T_\mu (v) = r [x,\, \mu(x)] + \beta v \{g [x,\, \mu(x)]\}\ , $$
where $\mu (x)$ is the policy function that attains $T(v) (x)$.
For a fixed $\mu (x)$, $T_\mu$ is an operator that maps bounded
continuous functions into bounded continuous functions. Denote
by $C$ the space of bounded continuous
functions mapping $X$ into $X$.

For any admissible policy function $\mu (x)$,
the operator $T_\mu$ is a contraction mapping.
This fact can be established by verifying
Blackwell's pair of sufficient conditions:

\item{1.}   $T_\mu$ is monotone.
Suppose that $v(x)\geq w(x)$.  Then
$$\eqalign{T_\mu v &= r [x,\, \mu(x)] + \beta v \{g [x,\,\mu(x)]\}\cr
&\geq r [x, \, \mu (x)] + \beta w \{g [x,\,\mu(x)]\} = T_\mu w\ .\cr}$$
\medskip
\item{2.} $T_{\mu}$ discounts.
For any positive constant $c$
$$\eqalign{T_\mu (v + c) &= r (x,\mu) +\beta \left(v \{g [x,\mu(x)] + c\} \right) \cr
&= T_\mu v +\beta c\ .\cr}$$
\medskip
\noindent Because $T_\mu$ is a contraction operator, the functional equation
$$v_\mu (x) = T_\mu [v_\mu (x)]$$
has a unique solution in the space of bounded continuous functions.  This
solution  can be computed as a limit of iterations on $T_\mu$ starting
from any bounded continuous function $v_0 (x) \in C$,
$$v_\mu (x) = \lim_{k\to\infty} T^k_\mu (v_0)\, (x)\ .$$
The function $v_\mu(x)$ is the value of the objective function that would be
attained by using the stationary policy $\mu (x)$ each period.

The following proposition describes the {\it policy iteration\/} or
{\it Howard improvement\/} algorithm.
\medskip\noindent
%\specsec{Proposition:}
\theorem{thm20.3} Let $v_\mu (x) = T_\mu [v_\mu (x)]$.  Define a  new
policy $\bar \mu$ and an associated operator $T_{\bar\mu}$ by
$$T_{\bar\mu} [v_\mu (x)] = T[v_\mu (x)]\ ;$$
that is, $\bar\mu$ is the policy that solves a one-period problem with
$v_\mu (x)$ as the terminal value function.  Compute the fixed point
$$v_{\bar\mu} (x) = T_{\bar\mu} [v_{\bar\mu} (x)]\ .$$
Then $v_{\bar\mu} (x) \geq v_\mu (x)$.  If $\mu (x)$ is not optimal, then
$v_{\bar\mu} (x) > v_\mu (x)$ for at least one $x \in X$. \endtheorem
\medskip\noindent
%\specsec{Proof:}
\proof  From the definition of $\bar\mu$ and $T_{\bar\mu}$, we have
$$\eqalign{T_{\bar\mu} [v_\mu (x)] &=
r [x,\, \bar\mu (x)] + \beta v_\mu \{g[x,\, \bar\mu (x)]\} =\cr
 T(v_\mu) (x) &\geq r [x,\,\mu (x)] +\beta v_\mu \{g[x,\,\mu(x)]\}\cr
\hskip.75in &= T_\mu [v_\mu (x) ] = v_\mu (x)\cr}$$
or
$$T_{\bar\mu} [v_\mu (x)] \geq v_\mu (x)\ .$$
Apply $T_{\bar\mu}$ repeatedly to this inequality and use the monotonicity of
$T_{\bar\mu}$ to conclude
$$v_{\bar\mu}(x) = \lim_{n\to\infty}\, T^n_{\bar\mu}\, [v_\mu (x)] \geq v_\mu (x)\ .$$
This establishes the asserted inequality $v_{\bar\mu}(x)\geq v_\mu (x)$.  If
$v_{\bar\mu} (x) = v_\mu (x)$ for all $x\in X$, then
$$\eqalign{v_\mu (x) &= T_{\bar\mu} [v_\mu (x)]\cr
&= T[v_\mu (x)]\ ,\cr}$$
where the first equality follows because $T_{\bar\mu} [v_{\bar\mu} (x)] =
v_{\bar\mu} (x)$, and the second equality follows from the definitions of
$T_{\bar\mu}$ and $\bar\mu$.  Because $v_\mu (x) = T[v_\mu (x)]$,
 the Bellman equation is satisfied by $v_\mu (x)$. \endproof%\ \qed
\medskip

The {\it policy improvement\/} algorithm starts from an
arbitrary feasible policy and  iterates to convergence
on the two following steps:\NFootnote{A policy $\mu(x)$ is said to
be {\it unimprovable\/} if it is optimal to follow it for the
first period, given a terminal value function $v(x)$.  In effect, the policy
improvement algorithm starts with an arbitrary value function,
then by solving a one-period problem, it generates an improved policy
and an improved value function.  The proposition states that  optimality
is characterized by the features, first, that there is no incentive
to deviate from the policy during the first period, and second,
that the terminal value function is the one associated with
continuing the policy.}

Step 1. For a feasible policy $\mu(x)$, compute
$ v_{\mu} = T_{\mu}(v_{\mu})$.

Step 2. Find ${\bar \mu}$ by computing $T(v_{\mu})$.  Use $\bar \mu$
as the policy in step 1.

\noindent In many applications, this algorithm proves to be much faster
than iterating on the Bellman equation.

\subsection{A search problem}

We now study the functional equation associated with a search problem of
chapter \use{search1}.  The functional equation is
$$v(w)=\max \left\{ {w\over 1-\be}, \be \int v(w') dF(w')\right\}, \qquad 0<\be
<1. \EQN A47 $$
Here, the wage offer drawn at $t$ is $w_t$.  Successive offers $w_t$ are
independently and identically distributed random variables.  We assume that
$w_t$ has cumulative distribution function $\prob\{w_t\le w\}=F(w)$, where
$F(0)=0$ and $F(\bar w) =1$ for some $\bar w<\infty$.  In equation \Ep{A47}, $v(w)$ is the
optimal value function for a currently unemployed worker who has offer $w$ in
hand.  We seek a solution of the functional equation \Ep{A47}.

We work in the space of bounded continuous functions $C[0,\bar w]$ and use the
$d_\infty$ metric
$$d_\infty(x,y) =\sup_{0\le w\le \bar w} |x(w)-y(w)|.$$
The metric space $(C[0,\bar w],d_\infty)$ is complete.

We consider the operator
$$T(z)=\max \Bigl\{ {w\over 1-\be},\be \int z(w') dF(w')\Bigr\}.\EQN A48$$
Evidently the operator $T$ maps functions $z$ in $C[0,\bar w]$ into
functions $T(z)$ in
$C[0,\bar w]$.  We now assert that the operator $T$ defined by
equation \Ep{A48} is a
contraction.  To prove this assertion, we verify Blackwell's sufficient
conditions.  First, assume that $f(w)\ge g(w)$ for all $w\in [0,\bar w]$.  Then
note that
$$\eqalign{ Tg&=\max \left\{ {w\over 1-\be},\be \int g(w') dF(w')\right\}\cr
\noalign{\smallskip}
&\le \max \left\{ {w\over 1-\be} ,\be \int f(w') dF(w')\right\}\cr
&=Tf.\cr}$$
Thus, $T$ is monotone.  Next, note that for any positive constant $c$,
$$\eqalign{ T(f+c) &= \max \left\{ {w\over 1-\be},\be \int [f(w')+c]
dF(w')\right\}\cr
\noalign{\smallskip}
&=\max \left\{ {w\over 1-\be},\be \int f(w') dF(w')+\be c\right\}\cr
\noalign{\smallskip}
&\le \max \left\{ {w\over 1-\be},\be \int f(w') dF(w')\right\}+\be c\cr
\noalign{\smallskip}
&=Tf+\be c.\cr}$$
Thus, $T$ satisfies the discounting property and is therefore a contraction.

Application of the contraction mapping theorem, then, establishes that the
functional equation $Tv=v$ has a unique solution in $C[0,\bar w]$, which is
approached in the limit as $n\to \infty$ by $T^n(v^0)=v^n$, where $v^0$ is any
point in $C[0,\bar w]$.  Because the convergence in the space $C[0,\bar w]$ is
in terms of the metric $d_\infty$, the convergence is uniform.

