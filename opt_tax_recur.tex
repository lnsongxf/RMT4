\input grafinp3
%\input grafinput8
\input psfig

\showchaptIDtrue
\def\@chaptID{12.}

%\eqnotracetrue

%\hbox{}

\def\toone{{t+1}}
\def\ttwo{{t+2}}
\def\tthree{{t+3}}
\def\Tone{{T+1}}
\def\TTT{{T-1}}
\def\rtr{{\rm tr}}
\footnum=0
\chapter{Two  Ramsey Problems Revisited}\label{optaxrecur}
\index{optimal taxation!commitment}%
\medskip
\section{Introduction}\label{Ramsey_appa1}%
This chapter formulates    Ramsey problems recursively for  the Lucas and Stokey (1983) economy with complete markets
  studied in section \use{Lucas-Stokey}  of chapter \use{optax}  and the Aiyagari, Marcet, Sargent, and Sepp\"al\"a (2002) (AMSS)
economy  with only a risk-free bond being traded studied in section \use{sec:AMSS}. As in chapter \use{stackel}, to apply dynamic programming
we   define  state vectors artfully. A key state variable will be a   forward-looking variable that summarizes
optimal responses of private agents to a Ramsey plan.
\auth{Aiyagari, Rao}\auth{Marcet, Albert}%
\auth{Sargent, Thomas J.}\auth{Sepp\" al\" a, Juha}%
\auth{Lucas, Robert E., Jr.} \auth{Stokey, Nancy L.}%

Recursive formulations can  deepen  understandings. We describe  a sense in which
the dimension of the state is lower in the Lucas Stokey model than in the AMSS model.  Accompanying that difference in dimension are  different dynamics of government debt.


\section{The Lucas-Stokey economy}\label{sec:appLS}%
We begin with the Lucas-Stokey (1983) economy.
Throughout this chapter we  assume that  $s$ is governed by a finite state Markov chain with
states $s\in \{1, \ldots, S\} = {\cal S} $, initial distribution $\pi_0(s_0)$,
and  transition matrix $\Pi$, where $\Pi(s'|s) = {\rm Prob}(s_{t+1} = s'| s_t =s)$.
Government purchases $g(s)$ are an exact time-invariant function  of
$s$.
The representative household's preferences are ordered by
$$
\sum_{t=0}^\infty\ \ \sum_{s^t} \beta^t \pi_t(s^t) u[c_t(s^t), \ell_t(s^t)] ,
                                                            \EQN TS_prefr $$
where $\pi_t(s^t)$ is a joint density over $s^t$ induced by $\pi_0, \Pi$.
 Feasibility requires $n_t(s^t) + \ell_t(s^t) = 1 $ and
 $$
c_t(s^t) + g_t(s_t) = n_t(s^t).                        \EQN TSs_techr
$$
The government faces a sequence of budget constraints whose time $t \geq 0$ component is
 $$ g(s_t) =   \tau^n_t(s^t)  n_t(s^t)
              + \sum_{s_{t+1}} p_t(s_{t+1} | s^t) b_{t+1}(s_{t+1} | s^t)
               - b_t(s_t | s^{t-1}),   \EQN TS_govr $$
where  the technology pins down the pre-tax wage rate to unity, $p_t(s_{t+1}|s^t)$ is the competitive equilibrium price of
one-period Arrow state-contingent securities, and $b_t(s_t|s^{t-1})$ is government debt falling due at time $t$, history $s^t$.
The representative household confronts a sequence of budget constraints whose time $t\geq 0$ component is
$$c_t(s^t) + \sum_{s_{t+1}} p_t(s_{t+1} | s^t) b_{t+1}(s_{t+1} | s^t)
=  \left[1-\tau^n_t(s^t)\right]  n_t(s^t) + b_t(s_t | s^{t-1}) .
         \EQN TS_bcr
$$
First-order conditions for the household's  problem imply

$$ (1 - \tau_t^n(s^t)) = {\frac{u_l(s^t)}{u_c(s^t)} }  \EQN LSA_taxr $$
$$ p_{t+1}(s_{t+1}| s^t) = \beta \Pi(s_{t+1} | s_t) {\frac{u_c(s^{t+1})}{u_c({s^t})}} .\EQN LSA_Arrowr $$
% $$\EQNalign{
% { u_\ell(s^t) \over u_c(s^t) } =& [1-\tau^n_t(s^t)] w_t(s^t),  \EQN TS_focXr;a \cr
% p_t(s_{t+1} | s^t) =& \beta { \pi_{t+1}(s^{t+1}) \over \pi_t(s^t) }
%                         { u_c(s^{t+1}) \over u_c(s^{t}) }, \EQN TS_focXr;b \cr}
% $$
% where $E_{t}$ is the mathematical expectation conditional on
% information available at time $t$, i.e., history $s^t$:
The single implementability condition constraining the Ramsey planner's choice of an allocation, equation \Ep{TSs_cham15},
is
$$ \sum_{t=0}^\infty\  \sum_{s^t} \beta^t \pi_t(s^t)
         [u_c(s^t) c_t(s^t) - u_\ell(s^t) n_t(s^t)]
      - u_c(s^0) b_0 = 0.                                    \EQN TSs_cham15r $$
%It is useful to repeat equations \Ep{TSs_tech}, \Ep{LSA_tax}, and \Ep{LSA_Arrow} here:


\subsubsection{Finding the state is an art}
% It is useful to write the household's budget constraint in a sequential version of
% the model in which one-period Arrow securities are traded:
% $$ c_t(s^t) + \sum_{s_{t+1}} p_{t+1}(s_{t+1}|s^t) b_{t+1}(s_{t+1}| s^t) = (1-\tau_t^n(s^t)) n_t(s^t) + b_t(s_t | s^{t-1}). $$
 Define  the level of government debt
scaled by the marginal utility of consumption  today as
$$ x_t(s^t) \equiv u_c(s^t) b_t(s_t | s^{t-1}). $$
  Substituting from \Ep{LSA_taxr},  \Ep{LSA_Arrowr},  and the feasibility condition \Ep{TSs_techr}
into \Ep{TS_bcr} gives
$$   x_t(s^t)  = u_c(s^t) [ n_t(s^t) - g_t(s^t)]  - u_l (s^t) n_t(s^t) + \beta \sum_{s_{t+1}\in {\cal S}} \Pi (s_{t+1}| s_t) x_{t+1}(s^{t+1})
 \EQN LSA_budget1r $$
%$$ \eqalign{ u_c(s^t) [ n_t(s^t) - g_t(s^t)] &+ \beta \sum_{s_{t+1}\in {\cal S}} \Pi (s_{t+1}| s_t)  u_c(s^{t+1}) b_{t+1}(s_{t+1} | s^t)
% \cr & = u_l (s^t) n_t(s^t) + u_c(s^t) b_t(s_t | s^{t-1}) . } \EQN LSA_budget1r $$
  % Notice that $x_t(s^t)$ appears on the right side of \Ep{LSA_budget1} while
%  $\beta$ times the conditional expectation of $x_{t+1}(s^{t+1})$ appears on the left side,
 As noted in chapter \use{optax},   equation \Ep{LSA_budget1r} shares  the structure of
 a simple asset pricing equation with $x_t$ being analogous to the price of an asset at time $t$ that appears to be  a purely ``forward-looking'' variable.
   But using a logic encountered in chapter \use{stackel}, we shall use $x_t$ as a state variable
 that at dates $t \geq 1$ confronts what we shall call a continuation Ramsey planner.  % at dates $t \geq 1$ as part of a Ramsey plan designed at time $t=0$.

We can think of $x_t$ as indexing a competitive equilibrium with distorting taxes.  Equation \Ep{LSA_budget1r} shows implicitly how $x_t$ depends on
on future government policy, via  marginal utilities $u_c$ and $u_l$  linked to government policies through the household's first-order necessary
conditions \Ep{LSA_taxr} and \Ep{LSA_Arrowr}. Roberto Chang (1998) extensively used a counterpart of $x_t$ to index competitive equilibria in this way in a model that we
discuss in chapter \use{chang}.
\auth{Chang, Roberto}%

%  Equation \Ep{LSA_budget1r}  will be an ingredient of a recursive formulation of a Ramsey problem for the % section \use{Lucas-Stokey}
%  Lucas-Stokey (1983) model.


%
% \subsection{Digression on risk-free debt only issued}
%   Assume  that $s_t$ is Markov with transition matrix $\Pi$ and that $g_t$ is a time-invariant function of $s_t$.
% In this case, it  turns out that, as we saw in section  \use{Lucas-Stokey}, the Ramsey planner chooses to make $(c_t,n_t)$
% time invariant functions of $s_t$ for $t \geq 1$.
% Temporarily suppose that the Ramsey planner  happens to choose always to issue  risk-free debt only, which means that it chooses to make $b_{t+1}(s_{t+1}|s^t)$ independent
% of $s_{t+1}$  for all $t, s^t$ and let $b_{t+1}(s_{t+1} | s_t) \equiv \tilde b(s_t)$.
% In this special  case of risk-free debt only, equation \Ep{LSA_budget1} would become
% $$ \eqalign{ u_c(s_t) [ n_t(s_t) - g_t(s_t)] &+ \beta \tilde b(s_t) \sum_{s_{t+1}} \Pi (s_{t+1}| s_t) u_c(s_{t+1})
%   \cr & = u_l (s_t) n_t(s_t) + u_c(s_t) \tilde b(s_{t-1}) } \EQN LSA_budget150 $$
% or
% $$ \tilde b(s_{t-1}) + - g(s_t) = [1 - {\frac{u_l(s_t)}{u_c(s_t)}}] n(s_t) + R(s_t)^{-1} \tilde b(s_t) \EQN RFreebud1 $$
% where
% $$ R(s_t)^{-1} =  {\frac{\beta \sum_{s_{t+1}} \Pi (s_{t+1}| s_t) u_c(s_{t+1})}{u_c(s_t)}} . \EQN RFreerate $$
% The left  side of equation \Ep{RFreebud1} equals the government's total expenditures in state $t$ including repayment of maturing bonds while the right side
% denotes its total revenues that consist of taxes raised on labor and  revenues raised from issuing risk-free debt at the  one-period gross interest rate from
% $t$ to $t+1$ equal  to $R(s_t)$.
% Now let's assume that  $S=2$ and that the Ramsey planner chooses
% to  issue a constant amount of risk-free debt always, so that $\tilde b(s_t) = \tilde b$ independently of $s_t$.  In that case,
% equation \Ep{RFreebud1} becomes
% $$   g(s_t) + \tilde b  = \left(1 - {\frac{u_l(s_t)}{u_c(s_t)}} \right) n(s_t) + R(s_t)^{-1} \tilde b .\EQN RFreebud2 $$
% Here the government rolls over a constant level of  debt (or assets).
% %The left  side of equation \Ep{RFreebud2} equals the government's total expenditures in state $t$ including repayment of maturing bonds while the right side
% %denotes its total revenues that consist of taxes raised on labor and  revenues raised from rolling over its debt at one-period gross interest rate from
% %$t$ to $t+1$ equal  to $R(s_t)$.
%
%  There exist Markov economies with $S=2$ in which the quantities chosen by a Lucas-Stokey Ramsey planner satisfy equation \Ep{RFreebud2} for a
%  unique constant
% risk-free government debt $\tilde b$.  In these economies, the Ramsey planner chooses not to issue or acquire state-contingent debt because  state-contingent
% fluctuations in
% the risk-free interest rate
% $R(s_t)$ can deliver
%  all of the risk-sharing with the private sector that it wants.  In such an equilibrium, $\tilde b <0$, meaning that the government
% perpetually purchases one-period risk-free claims on the private sector.
% When government expenditures are high  today, the risk-free interest rate between today and tomorrow is high.
% Therefore, purchasing the same constant level of cum-interest-rate assets $- \tilde b$  is cheaper when $g$ is high.
% The consequent fluctuations in  interest payments exactly offset the fluctuations in $g$.
%




\subsubsection{Intertemporal delegation}
  To express a Ramsey plan recursively,
we  imagine that  a time $0$ Ramsey planner is followed by a sequence of continuation Ramsey planners at times
$t = 1, 2, \ldots$. A ``continuation Ramsey planner''  has a different objective function and faces
different constraints than a Ramsey planner.
%A continuation Ramsey  planner
% faces different state variables than does a Ramsey planner because
A key step in representing a Ramsey plan recursively is
 to regard the marginal utility scaled government debts $x_t(s^t) = u_c(s^t) b_t(s_t|s^{t-1})$  as  predetermined quantities
 that   continuation Ramsey planners at times $t \geq 1$ are obligated to attain.
 A time $t\geq 1$ continuation Ramsey planner  delivers $x_t$  by choosing a suitable $n_t$ and a list of $s_{t+1}$-contingent
continuation quantities $x_{t+1}$ to impose on   a time $t+1$ continuation Ramsey planner. A time $t \geq 1$ continuation Ramsey
planner faces $x_t, s_t$ as state variables.
A time $0$ Ramsey planner faces $b_0$,
not $x_0$, as a state variable. Furthermore,  the Ramsey planner cares about $(c_0(s_0), \ell_0(s_0))$, while continuation Ramsey planners do not.
The time $0$ Ramsey planner hands $x_1$ as a function of $s_1$  to a time
$1$ continuation Ramsey planner.
%; and each  time $t \geq 1$
%continuation Ramsey planner is authorized to require that the time $t+1$ continuation Ramsey planner deliver the value of $x_{t+1}$ as a function of the  shock $s_{t+1}$.
%The time $0$ Ramsey planner is not a continuation Ramsey planner.
%, a key difference between a Ramsey planner and what we have chosen to call  a continuation Ramsey planner.
These assignments of  preferences, authorities, and responsibilities across time    express the continuation Ramsey planners' obligations to
implement their parts of  a  Ramsey plan that in chapter \use{optax}  we  designed  once-and-for-all  at time $0$.





\subsubsection{Bellman equations}
%  At time $0$, after $s_0$ has been realized, the state variables confronting the
% Ramsey planner
% are $b_0, s_0$, where $b_0$ is an initial level of government debt.
Thus, as in chapter \use{stackel}, we can frame the Ramsey problem posed by Lucas and Stokey in terms of two triples, each of which consists of a Bellman equation, a set of state variables, and a set of choice variables. One triple characterizes the decisions of the Ramsey planner at  $t=0$, while the other  describes the decisions faced by each of a sequence of continuation Ramsey planners at dates $t \geq 1$.
Thus, as in chapter \use{stackel}, we characterize the Ramsey problem recursively in terms of two subproblems.
%
%\medskip
%
% \item{} Subproblem 1: after $s_t$ has been realized at time $t \geq 1$, the state variables confronting the
%time $t$ continuation Ramsey planner  are $(x_t, s_t)$.  Let $V(x, s)$ be the  value of a continuation
%Ramsey plan at $x_t = x, s_t =s$ for $t \geq 1$.
%\medskip
%\item{} Subproblem 2: Let $W(b, s)$ be the value of a Ramsey plan at time $0$ at $b_0=b$ and $s_0 = s$. We now present Bellman equations
%to be satisfied by the continuation Ramsey plan value function  $V(x,s)$ and the Ramsey plan value function $W(b,s)$.
%
%\medskip
%  We work  backwards by   constructing $V(x,s)$ first, then $W(b,s)$.

\subsubsection{Subproblem 1: Continuation Ramsey problem}
 After $s_t$ has been realized at time $t \geq 1$, the state variables confronting the
time $t$ continuation Ramsey planner  are $(x_t, s_t)$.  Let $V(x, s)$ be the  value of a continuation
Ramsey plan at $x_t = x, s_t =s$ for $t \geq 1$.
The  Bellman equation for a time $t \geq 1$ continuation Ramsey planner is
$$ V(x, s) = \max_{n, \{x'(s')\}} u(n-g(s), 1-n) + \beta \sum_{s'\in S} \Pi(s'| s) V(x', s') \EQN LSA_Bellman1 $$
where  maximization over $n$ and the  $S$ elements of $x'(s')$ is subject to the
single implementability constraint for $t \geq 1$
$$ x = u_c(n-g(s)) -  u_l n + \beta \sum_{s' \in {\cal S}} \Pi(s'|s) x'(s')   \EQN LSA_Bellman1cons $$
coming from restriction \Ep{LSA_budget1}.
Here $u_c$ and $u_l$ are today's  marginal utilities.\NFootnote{Equation \Ep{LSA_Bellman1cons} expresses a competitive equilibrium $x$ today
in terms of a current period allocation $(c,n)$ and a continuation competitive equilibrium $x'$ for each state $s'$ tomorrow. See
Chang (1998) and chapter \use{chang} for another application of this idea.}
%$u_{c,t}$ and $u_{l,t}$, respectively, and
%$u_c(n-g)$ again denotes $u_c n - u_c g$, i.e., the marginal utility of consumption times $n$ minus the marginal utility of consumption times $g$.
For each given value of $x, s$, the continuation Ramsey planner chooses $n$ and  one $x'(s')$ for each $s' \in {\cal S}$.
Associated with a value function $V(x,s)$ that solves  Bellman equation \Ep{LSA_Bellman1} are $S+1$ time-invariant  policy functions
$$ \eqalign{ n_t & = f(x_t, s_t), \quad t \geq 1 \cr
             x_{t+1}(s_{t+1}) & = h(s_{t+1}; x_t, s_t), \, s_{t+1} \in  {\cal S}, \, t \geq 1.} \EQN RRpolicyt $$

\subsubsection{Subproblem 2:  Ramsey problem}
 Let $W(b, s)$ be the value of a Ramsey plan at time $0$ at $b_0=b$ and $s_0 = s$.
Let $x'(s_1)$ be the value of $x$ next period when next period's value of $s$ is $s_1$.
The Bellman equation for the time $0$ Ramsey planner is
$$ W(b_0, s_0) = \max_{n_0, \{x'(s_1)\}} u(n_0 - g_0, 1 - n_0) + \beta \sum_{s_1 \in {\cal S}} \Pi(s_1| s_0) V( x'(s_1), s_1)  \EQN LSA_Bellman2 $$
where  maximization over $n_0$ and the $S $ elements of $x'(s_1)$ is subject to the time $0$ implementability constraint
$$   u_{c,0} b_0 = u_{c,0} (n_0 - g_0) - u_{l,0} n_0  + \beta \sum_{s_1\in {\cal S}} \Pi(s_1 | s_0) x'(s_1) \EQN Bellman2cons $$
coming from restriction \Ep{LSA_budget1} at time $0$.
Associated with a value function $W(b_0, n_0)$ that solves Bellman equation \Ep{LSA_Bellman2} are $S +1 $
time $0$ policy functions
$$ \eqalign{ n_0 &= f_0(b_0, s_0) \cr
             x_1(s_1) & = h_0(s_1; b_0, s_0). } \EQN RRpolicy0 $$
Notice the appearance of  state variables $(b_0, s_0)$ in the time $0$ policy functions
\Ep{RRpolicy0} for a Ramsey planner versus  $(x_t, s_t)$ in the policy functions for   time $t \geq 1$ continuation  Ramsey planners.
As we shall discuss below, the presence of these distinct state variables is the device that allows
us to represent and compute a  Ramsey plan  recursively despite its  time inconsistency.\NFootnote{Please recall our discussion
of Edward C.
 Prescott's 1977 paper
entitled ``Should Control  Theory Be Used for Economic Stabilization?'' in chapter \use{overview} and how Prescott's pessimism about the applicability
of recursive approaches to optimal policy design problems
did not survive for long. % eventually led to ``dynamic programming squared'' (to be covered in detail in chapters
% \use{stackel}-% \use{socialinsurance}, \use{socialinsurance2}, \use{uninsur1}, \use{credible}, \use{chang}  and
% \use{wldtrade}).
Kydland and Prescott (1980) corrected Prescott (1977) by
 constructing  a Bellman equation for continuation Ramsey planners at times $t \geq 1$.}
%in the spirit of the ones we are studying here.}
\index{dynamic programming squared}%
\auth{Prescott, Edward C.}%
\auth{Kydland, Finn}%



\specsec{Remark:} The value function $V(x_t, s_t)$ of the time $t$ continuation Ramsey planner   equals
$E_t \sum_{\tau = t}^\infty \beta^{\tau - t} u(c_t, l_t)$, where  consumption and leisure processes are evaluated along the original time $0$ Ramsey plan.





\subsubsection{First-order conditions}

Attach a Lagrange multiplier $\Phi_1(x,s)$ to constraint \Ep{LSA_Bellman1cons} and a Lagrange multiplier
$\Phi_0$ to constraint \Ep{Bellman2cons}.  Working backwards, the first-order conditions for the time $t \geq 1$ constrained maximization
problem on the right side of  the continuation Ramsey planner's Bellman equation \Ep{LSA_Bellman1} are
$$ \beta \Pi(s'|s) V_x (x', s') - \beta \Pi(s' | s) \Phi_1 = 0 \EQN LSARxt $$
for $x'(s')$ and
$$ (1 + \Phi_1) (u_c - u_l ) + \Phi_1 \left[ n (u_{ll} - u_{lc}) + (n-g(s)) (u_{cc} - u_{lc})  \right] = 0  \EQN LSARnt $$
for $n$.
Given $\Phi_1$, equation \Ep{LSARnt} is one equation to be solved for $n$ as a function of $s$ (or of $g(s)$).
Equation \Ep{LSARnt} implies $ V_x(x', s')= \Phi_1$, while an envelope condition is $V_x(x,s) = \Phi_1$, so it follows that
$$ V_x(x', s') = V_x(x,s) = \Phi_1(x,s) . \EQN LSAenv $$


For the time $0$ problem on the right side of the Ramsey planner's Bellman equation \Ep{Bellman2cons}, the first-order conditions
are
$$ V_x(x(s_1), s_1) = \Phi_0 \EQN LSAx0 $$
for $x(s_1), s_1 \in  {\cal S}$, and
%$$  u_{c,0} - u_{n,0}  + \Phi_0 \bigl[ u_{l,0} - u_{c,0} + n_0 (u_{lc,0} - u_{ll,0})  \quad  - (n_0 - g_0) (u_{cc,0} - u_{cl, 0}) + (u_{cc,0} - u_{cl,0} ) b_0  \bigr] =0  \EQN LSAn0
%$$
\offparens
$$ \EQNalign{  (1 + \Phi_0) (u_{c,0} - u_{n,0}) & + \Phi_0 \bigl[ n_0 (u_{ll,0} - u_{lc,0} ) +  (n_0 - g(s_0)) (u_{cc,0} - u_{cl,0}) \Bigr] \cr &
  \quad \quad \quad -  \Phi_0 (u_{cc,0} - u_{cl,0}) b_0  =  0 \EQN LSAn0}  $$
\autoparens
Compare  first-order conditions
\Ep{LSARnt} for $t \geq 1$ and \Ep{LSAn0} for $t=0$.  An additional term is  present in \Ep{LSAn0} except in
the three special cases in which  (a) $b_0 = 0$, or (b) $u_c$ is constant (i.e., preferences are quasi-linear  in consumption),  or
(c) initial government assets are sufficiently large to finance all government purchases with interest earned from those assets, so that $\Phi_0= 0$.
Except in these special cases, %s means that when $b_0  \neq  0$ and $\Phi_0 \neq 0$,
the  allocation  and the labor tax rate as functions
of $s_t$ differ between dates $t=0$ and subsequent dates $t \geq 1$.
The presence of the extra term $\Phi_0 (u_{cc,0} - u_{cl,0}) b_0 $ in the first-order condition at time $0$ expresses the incentive for the Ramsey planner
to manipulate Arrow-Debreu prices in order to affect $u_{c, 0} b_0 = x_0$.


Thankfully, the first order conditions  here   agree with  first-order conditions  \Ep{TSs_foc}  derived
 when we   formulated
a Ramsey plan in the space of sequences in section \use{Lucas-Stokey} of chapter \use{optax}.

\subsubsection{State variable degeneracy}
Equations \Ep{LSAenv} and \Ep{LSAx0}   imply that
$$ \Phi_0 = \Phi_1$$
and that
$$ V_x(x_t, s_t) = \Phi_0   \EQN FONCx $$
for all $t \geq 1$.
% When $V$ is concave in $x$, equations \Ep{LSAenv} and  imply that there is a single function $\psi$ such that $x' = \psi(s')$ and $x = \psi(s)$.
When $V$ is concave in $x$, equation \Ep{FONCx} implies  ``state-variable degeneracy'' along a Ramsey plan in the sense that
 for $t \geq 1$,
$x_t$ will be a time-invariant function of $s_t$.  Given $\Phi_0$, this function mapping $s_t$
into $x_t$  can be expressed as a vector $\vec x$ that solves
equation \Ep{LSA_xsoln} for  $n$ and $c$ as functions of $g$ that are associated
with $\Phi = \Phi_0$.





\subsubsection{Symptom and source  of time inconsistency}

While the marginal utility adjusted level of government debt $x_t$ is  a key state variable for the continuation Ramsey planners at  $t \geq 1$,
it is  not a state variable at time $0$.  The time $0$ Ramsey planner faces $b_0$,  not $x_0 = u_{c,0} b_0$, as a state variable.
The discrepancy in state variables faced by the time $0$ Ramsey planner and the  time $t \geq 1$ continuation
Ramsey planners captures the differing obligations and incentives
faced by the time $0$ Ramsey planner and the time $t \geq 1$ continuation Ramsey planners.  While the time $0$ Ramsey
planner is obligated to honor government  debt $b_0$  measured in time $0$ consumption goods,
its choice of a policy  can alter the marginal utility of time $0$ consumption goods.  Thus, the time $0$ Ramsey planner can manipulate
the {\it value\/} of government debt as measured by $u_{c,0} b_0$.  In contrast, time $t \geq 1$ continuation Ramsey planners
are obligated {\it not\/} to alter  values of  debt, as measured by $u_{c,t} b_t$, that they inherit from an earlier Ramsey planner or continuation Ramsey planner.

When  government expenditures  $g_t$ are  a time invariant function of a Markov state $s_t$, a Ramsey plan and associated Ramsey allocation
feature marginal utilities of consumption  $u_c(s_t)$
that, given $\Phi$, for $t \geq 1$  depend only on $s_t$, but that for $t=0$ depend on $b_0$ as well.  This means that $u_c(s_t)$ will be a time invariant
function of $s_t$ for $ t\geq 1$, but except when $b_0 = 0$, a different function for $t=0$.  This in turn means that prices of one-period Arrow securities
$p_t(s_{t+1} | s_t) = p(s_{t+1}|s_t)$ will be the {\it same\/} time invariant functions of $(s_{t+1}, s_t)$ for $t \geq 1$, but a different function $p_0(s_1|s_0)$
for $t=0$, except when $b_0=0$.  The differences between these time $0$ and time $t \geq 1$ objects reflect the workings of the Ramsey planner's incentive
to manipulate Arrow security prices and, through them, the value of initial government debt $b_0$.

%
%
% \subsubsection{Absence of history dependence}
% %XXXXXX {\bf Tom: check this statement of the logical implication.}
% %Equations \Ep{LSA_xsoln} and \Ep{LSA_bsoln} have the striking implication that for $t \geq 1$,
% Under a Ramsey plan, for $t \geq 1$ the level $b_t$  of government debt
% depends exclusively on $s_t$ and is   independent of the history $s^{t-1}$ of shocks to government expenditures. Of course, as intermediated
% through the Lagrange multiplier $\Phi$ on the time $0$ implementability constraint,
%  $b_t$ does depend on the function  $g(s)$ and on the  probability distributions $\Pi_t(s^t)$ through their influence on  the
%  stochastic process for $g_t$; but it does
%  not depend on the particular history $s^t$ leading to $s_t$.


%\section{Bellmanizing a Ramsey plan}


\section{Recursive formulation of AMSS model}

We now describe a recursive version of  the Aiyagari, Marcet, Sargent, and Sepp\"al\"a (2002)
economy %without capital and with only a risk-free bond being traded,  the model
that we studied in section \use{sec:AMSS}.
\auth{Aiyagari, Rao}\auth{Marcet, Albert}%
\auth{Sargent, Thomas J.}\auth{Sepp\" al\" a, Juha}%
The AMSS economy is identical with the Lucas-Stokey (1983) economy except that instead of trading history-contingent securities or Arrow securities,
the government and household are allowed to trade only a one-period risk-free bond. As we saw in section \use{sec:AMSS}, from the point of view
of the Ramsey planner, the restriction to one-period risk-free securities leaves intact the single implementability constraint on allocations in the
Lucas-Stokey economy, while adding  measurability constraints on functions of tails of  allocations at each
time and history, functions that represent the present values of government surpluses. In this section, we explore how these
measurability  constraints alter the Bellman equations for a time $0$ Ramsey planner and for time $t \geq 1$, history $s^t$ continuation Ramsey planners.


\subsubsection{Recasting state variables}

In the AMSS setting, the government  faces a sequence of  budget constraints
$$   \tau_t^n(s^t) n_t(s^t) + T_t(s^t) +  b_{t+1}(s^t)/ R_t (s^t) =  g_t + b_t(s^{t-1})  ,$$
where $R_t(s^t)$ is the gross risk-free rate of interest between $t$ and $t+1$ at history $s^t$ and $T_t(s^t)$ are nonnegative transfers.
In most of this chapter, we shall set transfers   to zero.
%{\bf XXXXX Tom: add appropriate qualifications and refinements here.}
When $T_t(s^t) \equiv 0$,
the household faces a sequence of budget constraints
$$ b_t(s^{t-1}) + (1-\tau_t(s^t)) n_t(s^t) = c_t(s^t) + b_{t+1}(s^t)/R_t(s^t)  .  \EQN eqn:AMSSapp1  $$
As we saw in section \use{sec:AMSS},  the household's first-order conditions are
$u_{c,t} = \beta R_t E_t u_{c,t+1}$ and $(1-\tau_t^n) u_{c,t} = u_{l,t}$.  Using these to eliminate
$R_t$ and $\tau_t^n$ from \Ep{eqn:AMSSapp1} gives
$$ b_t(s^{t-1}) + { \frac{u_{l,t}(s^t)}{u_{c,t}(s^t)}} n_t(s^t) = c_t(s^t) + {\frac{\beta (E_t u_{c,t+1}) b_{t+1}(s^t)}{u_{c,t}}}
  \EQN eqn:AMSSapp2a $$
or
$$ u_{c,t}(s^t) b_t(s^{t-1}) + u_{l,t}(s^t) n_t(s^t) = u_{c,t}(s^t) c_t(s^t) + \beta (E_t u_{c,t+1}) b_{t+1}(s^t) . \EQN eqn:AMSSapp2 $$
For the purpose of posing a recursive version of the Ramsey problem, define
$$ x_t \equiv \beta b_{t+1}(s^t) E_t u_{c,t+1} = u_{c,t} (s^t) {\frac{b_{t+1}(s^t)}{R_t(s^t)}} \EQN eqn:AMSSapp3 $$
and represent the household's budget constraint \Ep{eqn:AMSSapp2} at time $t$, history $s^t$  as
% the value of government debt
% $ {\frac{b_{t+1}(s^t)}{R_t(s^t)}}$ at time $t$.}
$$ {\frac{u_{c,t} x_{t-1}}{\beta E_{t-1} u_{c,t}}} = u_{c,t} c_t - u_{l,t} n_t + x_t   \EQN eqn:AMSSapp4 $$
for $t \geq 1$.

\subsubsection{Measurability constraints}
% Recall that in the Lucas-Stokey model, we defined
% $ x_t(s^t) = u_{c,t}(s^t) b_t(s^t)$ where the Ramsey planner chooses $b_t(s^t)$ at time $t-1$.
% {\bf XXXXXX: Tom -- the following statement seems wrong. } By way of contrast, for the AMSS model
% we define $x_t = u_{c,t}(s^t) {\frac{b_{t+1}(s^t)}{R_t(s^t)}} $, where here the time $0$ Ramsey planner or time $t \geq 1$ continuation Ramsey
% planner predetermines $x_t$ as a function of variables known at $t-1$, namely, $(x_{t-1}, s_{t-1})$.
Write equation \Ep{eqn:AMSSapp2a} as
$$ b_t(s^{t-1})  = c_t(s^t) -  { \frac{u_{l,t}(s^t)}{u_{c,t}(s^t)}} n_t(s^t) + {\frac{\beta (E_t u_{c,t+1}) b_{t+1}(s^t)}{u_{c,t}}} .
  \EQN eqn:AMSSapp2b $$
The right side of equation \Ep{eqn:AMSSapp2b}
 expresses the time $t$ value of government debt in terms of a linear combination of terms whose individual components are
measurable with respect to $s^t$
but whose sum has to equal $b_t(s^{t-1})$ and so be measurable with respect to $s^{t-1}$.
These are the measurability constraints, described in
section \use{sec:AMSS},   that the AMSS model
adds  to  the single implementation constraint \Ep{LSA_Bellman1cons} imposed  by the Lucas and Stokey model.\NFootnote{These measurability constraints put the equilibrium allocation in a ``marketable subspace'' in the sense of Duffie and Shafer (1985).}
\index{measurability constraints!AMSS model}%


\subsubsection{Bellman  equations}
% We continue to assume that the Markov state $s$ has transition matrix $\Pi(s' | s)$ and that government purchases
% are an exact function of $s$.
Let $V(x_-, s_-)$ be the  continuation value of a continuation
Ramsey plan at $x_{t-1} = x_-, s_{t-1} =s_-$ for $t \geq 1$.
Let $W(b, s)$ be the value of the Ramsey plan at time $0$ at $b_0=b$ and $s_0 = s$.

For $t \geq 1$, the value function for a continuation Ramsey planner satisfies the Bellman equation
$$ V(x_-,s_-) = \max_{\{n(s), x(s)\}} \sum_s \Pi(s|s_-) \left[ u(n(s) - g(s), 1-n(s)) + \beta V(x(s),s) \right] \EQN eqn:AMSSapp5 $$
subject to the following collection of implementability constraints, one for each $s \in {\cal S}$:
$$ {\frac{u_c(s) x_- }{\beta  \sum_{\tilde s} \Pi(\tilde s|s_-) u_c(\tilde s) }} = u_c(s) (n(s) - g(s)) - u_l(s) n(s) + x(s) . \EQN eqn:AMSSapp6 $$
We attach a distinct  Lagrange multiplier $\mu(s)$  to implementability  constraint \Ep{eqn:AMSSapp6} for each $s$.
 A continuation  Ramsey planner solves an {\it ex ante\/} problem.  A continuation  Ramsey planner at $t \geq 1$ takes
$(x_{t-1}, s_{t-1}) = (x_-, s_-)$ as given and chooses $(n_t(s_t), x_t(s_t)) = (n(s), x(s))$ for $s \in  {\cal S}$ before $s_t$ is realized.
%By way of contrast, at $t=0$, the Ramsey planner solves an {\it ex post\/} problem in the sense that he
The Ramsey planner takes
$(b_0, s_0)$ as given and chooses $(n_0, x_0)$.
The Bellman equation for the time $t=0$ Ramsey planner is
$$ W(b_0, s_0) = \max_{n_0, x_0} u(n_0 - g_0, 1-n_0) + \beta V(x_0,s_0) \EQN eqn:AMSSapp100 $$
subject to
$$ u_{c,0} b_0 = u_{c,0} (n_0-g_0) - u_{l,0} n_0 + x_0 . \EQN eqn:AMMSSapp101 $$

% \medskip
%
% \specsec{Remark:} The continuation Ramsey planner in the Lucas Stokey economy faces an {\it ex post\/} problem while the
% continuation Ramsey planner in the AMSS economy  faces an {\it ex ante\/}  problem. {\bf Tom XXXXX: add to, elaborate, and possibly move to
% summary section.}


\subsubsection{Martingale replaces state-variable degeneracy}

Let $\mu(s|s_-) \Pi(s|s_-)$ be a Lagrange multiplier on constraint \Ep{eqn:AMSSapp6} for state $s$.  The continuation Ramsey planner's first-order condition
with respect to $x(s)$ is
$$ \beta V_x(x(s),s) = \mu(s|s_-) .\EQN eqn:AMSSapp7  $$
Applying the envelope theorem to  Bellman equation \Ep{eqn:AMSSapp5} gives
$$ V_x(x_-,s_-) = \sum_s \Pi(s|s_-) \mu(s|s_-) {\frac{u_c(s)}{\beta \sum_{\tilde s} \Pi(\tilde s|s_-) u_c(\tilde s) }} . \EQN eqn:AMSSapp8 $$
Equations \Ep{eqn:AMSSapp7} and \Ep{eqn:AMSSapp8} imply that
$$ V_x(x_-, s_-) = \sum_{s} \left( \Pi(s|s_-) {\frac{u_c(s)}{\sum_{\tilde s} \Pi(\tilde s| s_-) u_c(\tilde s)}} \right) V_x(x(s), s) ,\EQN eqn:AMSSapp9 $$
which states that $V_x(x, s)$ is a `risk-adjusted martingale', meaning that it is  a martingale with respect to the probability distribution over $s^t$ sequences
generated by the twisted transition probability matrix:\NFootnote{It is easy to verify that $\check \Pi(s| s_-)$ is a legitimate Markov transition matrix, in particular,
that the transition probabilities are nonnegative and sum  to $1$ for each $s_-$.}
$$ \check \Pi(s|s_-) \equiv \Pi(s|s_-) {\frac{u_c(s)}{\sum_{\tilde s} \Pi(\tilde s| s_-) u_c(\tilde s)}}.  $$

\medskip

\specsec{Remark:}  Suppose that instead of imposing $T_t = 0$, we impose
a nonnegativity constraint $T_t\geq 0$ on  transfers and consider the special case of quasi-linear preferences,
$u(c,l)= c + H(l)$.   In this case,  $V_x(x,s)\leq 0$  is a non-positive martingale.  By the martingale convergence
 theorem, $V_x(x,s)$ converges almost surely.\NFootnote{For a discussion of the martingale convergence theorem see  the appendix to chapter \use{selfinsure}.}
\index{martingale!convergence theorem}%
When  the Markov chain $\Pi(s| s_-)$ and the function $g(s)$ are such that $g_t$ is perpetually random,
$V_x(x, s)$ almost surely converges to zero.  %\medskip
%\noindent
For  quasi-linear preferences, the first-order condition with respect to $n(s)$ becomes
$$ (1-\mu(s|s_-) ) (1 - u_l(s)) + \mu(s|s_-) n(s) u_{ll}(s) =0  .$$
Since $\mu(s|s_-) = \beta V_x(x(s),x) $ converges to zero, in the limit $u_l(s)= 1 =u_c(s)$, so that $\tau(x(s),s) =0$. In the limit, the government
accumulates sufficient assets
to finance all expenditures from earnings on those assets, returning any excess revenues to the household as nonnegative lump sum transfers.

\medskip


\specsec{Remark:} Along a Ramsey plan, the state variable $x_t = x_t(s^t, b_0)$  becomes a function of the history $s^t$ and also
the initial government debt $b_0$.


\medskip

\specsec{Remark:} In our recursive formulation of the  Lucas-Stokey model in section \use{sec:appLS}, we found that the counterpart to $V_x(x,s)$ is time invariant
and equal to
the Lagrange multiplier on the single time $0$ implementability constraint present in the original version of that model cast in terms of choice of sequences.
We saw that the time invariance of $V_x(x,s)$ in the Lucas-Stokey model is the source of the state variable degeneracy (i.e., $x_t$ is an exact function of
$s_t$), a key feature of the Lucas-Stokey model.
% Here $V_x(x_t,s_t)$, which from equation \Ep{eqn:AMSSapp7} equals a scaled (by transition probabilities
% and discount factor) version of the Lagrange multiplier on the state $(x(s),s)$ Lagrange multiplier confronting the state $(x_-,s_-)$
% continuation Ramsey planner.
That $V_x(x,s)$ varies over time according to a twisted martingale means that there is no state-variable degeneracy in the AMSS model.
Both $x$ and $s$ are needed to describe the state.  This property of the
AMSS model is what transmits a twisted martingale-like component  to consumption, employment, and the tax rate.
\medskip


\subsection{Special case of AMSS model}\label{sec:specialcaseAMSS}%
That the  Ramsey allocation for the AMSS model differs from the Ramsey allocation of the Lucas-Stokey model is a symptom that the measurability
constraints \Ep{eqn:AMSSapp6} bind.
Following Bhandari, Evans, Golosov, and Sargent (2017) (henceforth BEGS), we now consider a special case of the AMSS model in which these constraints don't bind.
Here  the AMSS Ramsey planner chooses not to issue state-contingent debt, though he is free to do so.  The environment is one in which
fluctuations in the risk-free interest rate provide all  the insurance that the Ramsey planner wants.

Following BEGS, we set $S=2$ and assume that
the state $s_t$ is i.i.d., so that the transition matrix $\Pi(s'|s) = \Pi(s')$ for $s=1,2$.
Following BEGS, it is useful to consider  the following special case of the implementability constraints  \Ep{eqn:AMSSapp6} evaluated at the constant
value  of the state variable  $x_-= x(s) = \check x$:
$$ {\frac{u_c(s) \check x}{\beta \sum_{\tilde s} \Pi(\tilde s) u_c(\tilde s) }} = u_c(s) (n(s) - g(s) ) - u_l(s) n(s) + \check x, \ s=1,2. \EQN eqn_AMSS_app_100 $$
We   guess and verify that the scaled Lagrange multiplier $\mu(s)=\mu$ is a constant independent of $s$. At a fixed $x$,  because $V_x(x, s)$ must be independent of $s_-$, equation \Ep{eqn:AMSSapp9} becomes
$ V_x(\check x) = \sum_{s} \left( \Pi(s) {\frac{u_c(s)}{\sum_{\tilde s} \Pi(\tilde s) u_c(\tilde s)}} \right) V_x(\check x) = V_x(\check x) ,$%QN eqn:AMSSapp109 $$
which confirms that equation \Ep{eqn:AMSSapp7} becomes  $\mu = \beta V_x(\check x)$ for a  constant  $\mu$.
For the continuation Ramsey planner facing implementability constraints \Ep{eqn_AMSS_app_100}, the first-order conditions with respect to $n(s)$ become
$$ \EQNalign{ & u_c(s) - u_l(s)  \mu \Bigl\{ {\frac{\check x}{\beta \sum_{\tilde s} \Pi(\tilde s) u_c(\tilde s)}} (u_{cc}(s) - u_{cl}(s) ) - u_c(s) \cr
       &  \ \ - n(s) (u_{cc}(s) - u_{cl}(s)) - u_{lc}(s) n(s) + u_l(s) \Bigr\}  = 0.\EQN eqn:AMSSapp101 }  $$
Equations  \Ep{eqn_AMSS_app_100} and \Ep{eqn:AMSSapp101}  are four equations in the four  unknowns $n(s), s=1,2$, $\check x$, and $\mu$. Under some regularity conditions
on  the period utility function  $u(c,l)$, BEGS show that these equations have a unique solution featuring a negative value of $\check x$.
Consumption $c(s)$ and the flat-rate tax on labor $\tau(s)$ can then be constructed
as history-independent functions of $s$.
\auth{Evans, David G.}%
\auth{Bhandari, Anmol}%
\auth{Golosov, Mikhail}%
\auth{Sargent, Thomas J.}%
In this AMSS economy, $\check x = x(s) = u_c(s) {\frac{b_{t+1}(s)}{R_t(s)}}$.  The risk-free interest rate, the tax rate, and the marginal
utility of consumption fluctuate with $s$, but $x$ does not and neither does  $\mu(s)$. The labor tax rate and
the allocation depend only on the current value of $s$.

For this special AMSS economy to be in a steady state from time $0$ onward, it is necessary that  initial debt $b_0$ satisfy the
time $0$ implementability
constraint \Ep{eqn:AMMSSapp101} at the value $\check x$ and the realized value of $s_0$.  We can solve for this level of $b_0$ by
plugging the $n(s_0)$ and $\check x$
that solve our four equation system  \Ep{eqn_AMSS_app_100}-\Ep{eqn:AMSSapp101}  into
$ u_{c,0} b_0 = u_{c,0} (n_0-g_0) - u_{l,0} n_0 + \check x $ and solving for $b_0$.
This  $b_0$ assures that a steady state $\check x$ prevails from time $0$ on.


\subsubsection{Relationship to a Lucas-Stokey economy}
The constant value $\mu$ of the Lagrange multipliers $\mu(s)$ across states $s$  in the Ramsey plan for our special AMSS economy is a tell tale sign that
the measurability restrictions imposed on the Ramsey allocation by the requirement that government debt must be risk free are slack.
When they bind, those
measurability restrictions  cause the AMSS tax policy and allocation to be history dependent -- that's what activates flucations in the risk-adjusted martingale.
Because  those measurability conditions are  slack in  this special AMSS economy, the same Ramsey allocation is attained in a corresponding complete markets
 a Lucas-Stokey economy that starts
from a particular initial  government debt.  The setting of  $b_0$ for the corresponding Lucas-Stokey
implementability constraint \Ep{Bellman2cons} solves
$ u_{c,0} b_0 = u_{c,0}(n_0 - g_0) - u_{l,0} + \beta \check x$.
In this Lucas-Stokey economy, although the Ramsey planner is free to issue state-contingent debt, it chooses not to and instead  issues  only risk-free debt.
It achieves the
 risk-sharing with the private sector that it wants by altering the amounts of one-period risk-free debt that it issues at each current state,
  while taking advantage of  the fluctuations in the  equilibrium interest
rate that its tax policy induces.

\subsubsection{Convergence to the special case}
In an  i.i.d., $S=2$  AMSS economy in which  the initial $b_0$ does not equal the special value described in the previous subsection, $x$ fluctuates and is history-dependent.
%namely, a value at the implied stationary distribution of risk free government debt
%affiliated with the fixed  $\check x$ determined above,
The Lagrange multiplier $\mu_s(s^t)$ is a non trivial risk-adjusted martingale and the allocation and distorting tax rate
are both  history dependent.  However, BEGS describe conditions  under which such an i.i.d., $S=2$ AMSS economy in which
the representative agent dislikes consumption risk  converges to a
Lucas-Stokey economy in the sense that $x_t \rightarrow \check x$ as $t \rightarrow \infty$.
The  following subsection  displays a numerical example that exhibits convergence.



\subsection{Computed examples}%\label{sec:AMSSexamples}}

Figures \Fg{AMSS1}, \Fg{AMSS2}, and \Fg{AMSS3} report features of a Ramsey plan for an AMSS economy with
one-period utility function
$$ u(c,l) = {\frac{c^{1-\sigma}}{1-\sigma}} - {\frac{l^{1+\gamma}}{1+\gamma}} . $$
We set $\beta = .96, \gamma = 2$, and $\sigma$ equal to either $0$ (this is the
quasi-linear case of AMSS) or $1.5$ (here the representative consumer is averse to
consumption fluctuations).  We set
$S=2$ with $g(s_1) = .05$ and $g(s_2) = .15$ and assume that $s_t$ is an iid process with
the probability of both states equalling $.5$.  We computed the objects shown in figures \Fg{AMSS1}--\Fg{AMSS3}
by numerically solving  Bellman equations \Ep{eqn:AMSSapp5} and \Ep{eqn:AMSSapp6}.


\midfigure{AMSS1}
\centerline{\epsfxsize=3true in\epsffile{AMSS1.eps}}
\caption{Two AMSS economies. Dashed line: $\sigma=0$; solid line: $\sigma = 1.5$. Outcomes converge to those associated with  first-best allocation for the
quasi-linear ($\sigma =0$) economy, while they converge to an allocation associated with a complete markets economy for the economy with risk-aversion ($\sigma=1.5$).}
\infiglist{AMSS1}
\endfigure


\midfigure{AMSS2}
\centerline{\epsfxsize=3true in\epsffile{AMSS2.eps}}
\caption{Decision rule for $x_{t+1}- x_t$ as a function of $x_t$ and $s_t$ for AMSS $\sigma=0$ economy.
Dashed line: $g_t$ is high; solid line, $g_t$  is low.}
\infiglist{AMSS2}
\endfigure



\midfigure{AMSS3}
\centerline{\epsfxsize=3true in\epsffile{AMSS3.eps}}
\caption{Decision rule for $x_{t+1}- x_t$ as a function of $x_t$ and $s_t$ for AMSS $\sigma=1.5$ economy.
Dashed line: $g_t$  is high; solid line, $g_t$  is low. }
\infiglist{AMSS3}
\endfigure

%
%
Figure \Fg{AMSS1} reports outcomes from
long simulations of the two AMSS economies, the quasi-linear ($\sigma=0)$ economy shown in the dashed line and the
 $\sigma=1.5$ economy in the solid line.  As expected from the AMSS results summarized in section \use{sec:AMSS} of chapter \use{optax}, outcomes in the  quasi-linear economy
converge to those
associated with a first-best allocation. The government accumulates sufficient risk-free claims on the private sector  to finance all subsequent government expenditures
with interest earned from those claims. After accumulating enough assets,
the government sets the tax rate to zero and rebates excess earnings
from assets in the form of  lump sum transfers to the representative agent.

Asymptotic outcomes  for the $\sigma = 1.5$ economy display the striking outcome that they too converge, but not to a first-best allocation.  Instead, they converge to the
fixed point in $x$  associated with a Lucas-Stokey complete markets allocation.

Figure \Fg{AMSS2} displays the Ramsey plan decision rule for $x_{t+1}$  for the $\sigma=0$ quasi-linear AMSS economy,  while \Fg{AMSS3} displays
 the decision rule for the $\sigma =1.5$ economy.  The figures display $x_{t+1} - x_t$ as a function of $x_t$ and $g_t$.   Rest points  for $x$ occur where the branch of
the decision rule affiliated with  high $g$ intersects the branch affiliated with low $g$ at  $x_{t+1} - x_t$ equal to zero.
For the quasi-linear economy, the branches intersect at a value of  $x$ and an implied
level of government debt that is sufficient to finance all subsequent government expenditures. Further, when
$x_t$ exceeds this level,  the decision rules on average point downward, imparting a negative drift to $x_t$ until it reaches the level associated with enough
government assets capable of sustaining a first-best allocation.  Notice that after $x_t$ reaches this level,  the tax rate equals zero.


For the figure \Fg{AMSS3} AMSS economy with $\sigma=1.5$, the branches of the decision rule for $x_{t+1} -x_t$ intersect at coordinate  value  zero at an ordinate value of $x$ that,
while negative, exceeds the
level for the quasi-linear  economy.  This $x$ is affiliated with a Lucas-Stokey economy.  After $x_t$ has converged to this level,
the tax rate is positive and constant. That it does not fluctuate
with $g$ is partly a consequence of the specification of the one-period utility function.

\section{Concluding remarks}

The next several chapters  construct Bellman equations for diverse applications in which  implementability conditions inherited from various frictions  require
us to choose state variables artfully.


