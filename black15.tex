%%  QUESTION: \tag \the \pageno commands are in the body of this chapter -- delete?
%
 %% NOTE:  the figures for the two state example are generated by
% the programs kehlev and koch2sa in \books\green3.
%\ReadAUX
\input grafinp3
%\input grafinput8
\input psfig
%\eqnotracetrue
\offparens
%  see figures commit1.eps and commit2.eps

%\input form1
%\input psfig
%\input grafinput8

%\showchaptIDtrue
%\def\@chaptID{15.}

%\hbox{}

\footnum=0
\overfullrule=0pt

\def\bull{\vrule height .9ex width .8ex depth -.1ex}
\def\bh{\penalty-100}
\footnum=0
\chapter{Incentives and Insurance\label{socialinsurance}}%

\section{Insurance with recursive contracts}
 This chapter studies a planner who designs an efficient contract
to supply insurance in the presence of incentive constraints.
%imposed by his limited ability either to enforce contracts or to
%observe households' actions or incomes.
 We pursue two themes, one
substantive, the other technical.  The substantive theme is
 a tension between offering insurance and
providing incentives.  A planner  offers ``stick and carrot'' incentives that adjust
 an
agent's future consumption in ways that provide incentives to adhere to an arrangement at the cost of  providing less
than ideal insurance.
Balancing  incentives against insurance shapes the evolution of
distributions of wealth and consumption.

\index{recursive!contracts} \auth{Spear, Stephen E.}
\auth{Srivastava, Sanjay} \auth{Thomas, Jonathan} \auth{Worrall,
Tim} \auth{Phelan, Christopher} \auth{Townsend, Robert M.}
\auth{Abreu, Dilip}  \auth{Stacchetti, Ennio} \auth{Pearce, David}
  The technical theme is how memory can be
encoded recursively and how  incentive problems can be managed with contracts
that remember  and  promise. Contracts  issue rewards
that depend on the history either of publicly observable outcomes
or of an agent's announcements about his privately observed
outcomes. Histories are large-dimensional objects.  But Spear and
Srivastava (1987), Thomas and Worrall (1988), Abreu, Pearce, and
Stacchetti (1990), and Phelan and Townsend (1991) discovered that
the dimension  can be contained by using an accounting
system cast solely in terms of a ``promised value,'' a
one-dimensional object that  summarizes  enough aspects of an
agent's history.  Working with promised values permits us to
formulate  contract design problems recursively.

\auth{Kocherlakota, Narayana R.}\auth{Thomas, Jonathan} \auth{Worrall, Tim}
\index{promised value!as state variable}
   Three basic models  are set within a single physical
environment but  assume different structures of information,
enforcement, and storage possibilities.  The first adapts a model of
Thomas and Worrall (1988) and Kocherlakota (1996b) that  has all
information being public and focuses
 on commitment or enforcement problems .  The second is a model of Thomas and
Worrall (1990) that has an incentive problem coming from private
information but that assumes away commitment and enforcement
problems. Common to both of these models is that the insurance
contract is assumed to be the {\it only\/} vehicle  for households
to transfer wealth across states of the world and over time. The
third model, created by Allen (1985) and Cole and Kocherlakota (2001), extends Thomas and
Worrall's (1990) model by introducing private storage that cannot
be observed publicly.  Because it lets households
self-insure as in chapters \use{selfinsure} and \use{incomplete}, the possibility of
private storage reduces {\it ex ante\/} welfare by limiting the
amount of social insurance that can be attained when incentive
constraints are present. We shall see that a model with private storage has an interesting connection
to one of the Bewley models discussed in chapter \use{incomplete}. \auth{Cole, Harold L.} \auth{Kocherlakota,
Narayana R.} \auth{Allen, Franklin}

%=====================

%Another model, by Atkeson (1991),
%\auth{Atkeson, Andrew}
%combines both commitment and information
%problems.
%We also study a version of  Shavell
%and Weiss's (1979)
%\auth{Shavell, Steven} \auth{Weiss, Laurence}
%and Hopenhayn and Nicolini's (1997)
%\auth{Hopenhayn, Hugo A.} \auth{Nicolini, Juan Pablo}
%model of unemployment insurance.

%======================

\section{Basic environment}

Imagine a  village with a large number of {\it ex ante\/}
identical households. Each household has preferences over
consumption streams that are ordered by
$$ E_{-1}\sum_{t=0}^\infty \beta^t u(c_t), \EQN pref $$
where $u(c)$ is an increasing, strictly concave, and twice
continuously differentiable  function,
 $\beta \in (0,1)$ is a discount factor, and $E_{-1}$ is the mathematical expectation
 not conditioning on any information available at time $0$ or later.   Each household
receives a stochastic \idx{endowment stream} $\{y_t\}_{t=0}^\infty$,
where for each $t \geq 0$, $y_t$ is independently and
identically distributed according to the discrete
probability distribution ${\rm Prob} (y_t = \overline y_s) = \Pi_s,$
where $s \in \{1, 2, \ldots ,S\}\equiv {\bf S}$ and
$\overline y_{s+1}>\overline y_s$.  The consumption
good is not storable.  At time $t \geq 1$, the
household has received a history of endowments
$h_t = (y_t, y_{t-1}, \ldots, y_0).$
Endowment processes are distributed independently and identically
 both across time and
across households.

In  this setting, if there were a competitive equilibrium with
complete markets as described in chapter \use{recurge}, at date
$0$ households would trade history- and date-contingent claims. Since  households are {\it ex ante\/}
identical, each household would consume the per capita
endowment in every period, and its lifetime utility would be
$$ v_{\rm pool} = \sum_{t=0}^\infty
\beta^t \, u\!\left(\sum_{s=1}^S \Pi_s \overline y_s\right) =
 {1 \over 1-\beta}\, u\!\left(\sum_{s=1}^S \Pi_s \overline y_s\right) .
                                                        \EQN pool_value $$
 Households would thus insure away all
risks from their individual endowment processes. But the
 incentive constraints that we are about to specify make
this allocation unattainable. For each specification of incentive
constraints, we shall solve a planning problem for an efficient
allocation that respects those constraints.

\auth{Green, Edward J.}
   Following a tradition started by
Green (1987),
we assume that a ``moneylender'' or ``planner'' is
the only person in the  village who has access to
a risk-free loan market outside the village.
The moneylender can borrow or lend at a constant one-period
risk-free gross interest rate  $R=\beta^{-1}$.
Households cannot borrow or lend with each other,
and can  trade only with the moneylender.  Furthermore,
we assume that the moneylender is committed   to honor  his
promises.
We will study three alternative types  of incentive constraints.
\medskip
\item{  (a) } Both the money lender and the household observe   the household's  history of endowments at each  time $t$.
Although the moneylender can commit to honor a
contract, households  cannot commit and at any time are
  free to walk away from an arrangement
with the moneylender
and live in perpetual autarky thereafter.  They must be induced not to do so
by the structure of
the contract.
This is a model of ``one-sided commitment'' in which the
contract must be ``self-enforcing''.  That is, it must be structured to induce   the household to  prefer to
conform to it.
\item{ (b) } Households {\it can\/} make commitments  and enter
into enduring and   binding contracts with the moneylender,
but they have private
information about their own incomes. The moneylender
can see neither their income nor their consumption. Instead,
 exchanges between the moneylender and a household must
be based on the household's own reports about income
realizations. An incentive-compatible contract  induces
a household to report its income truthfully.

\item{(c) } The environment is the same as  b except  that now households have access to a storage technology that
cannot be observed by the moneylender.
Households can store nonnegative amounts of goods at a risk-free
gross return of $R$ equal to the interest rate that
the moneylender faces in the outside credit market.
Since the moneylender can both borrow and lend at the interest
rate $R$ outside of the village,
the private storage technology does not change the economy's
aggregate resource constraint, but it does  affect the set of
incentive-compatible contracts between the moneylender and the
households.
%\vskip.2cm


%%%%%  experiment with a two -graph



\midfigure{triKoch}
\centerline{\epsfxsize=2.0truein\epsffile{Seb_1_a_v2.eps}, \epsfxsize=2.0truein\epsffile{Seb_1_b_v2.eps}}
\caption{Left panel: typical consumption path in environment a. Right panel: typical consumption path in environment b.}
\infiglist{triKoch}
\endfigure
\medskip


%%%%%%%%%%%%%%%% GGGGGG  April 2017 problem file  %%%%%%%%%%%%%%%%%%

%\midinsert
%
%\midfigure{triKoch}
%$$\graftwo{Seb_1_a.eps,height=1.7in}{{\bf Figure \Fg{triKoch}.a:} Typical
%consumption path in environment a.}
%{Seb_1_b.eps,height=1.7in}{{\bf Figure \Fg{triKoch}.b:} Typical consumption path in
%environment b.} $$
%\endfigure

\midfigure{triCKf}
\centerline{\epsfxsize=2.5truein\epsffile{Seb_2_v2.eps}}
\caption{Typical consumption path in environment c.}
\infiglist{triCKf}
\endfigure
\medskip


%%%%%%%%%%%%%%%%   April 14 %%%%%%%%%%%%%%%%%%

%%%% old versions are below; disregard
%
%%\midinsert
%\midfigure{triKoch}
%$$\graftwo{triKoch.eps,height=1.7in}{{\bf Figure \Fg{triKoch}.a:} Typical
%consumption path in environment a.}
%{triTW.eps,height=1.7in}{{\bf Figure \Fg{triKoch}.b:} Typical consumption path in
%environment b.} $$
%\endfigure
%%\endinsert

%%%%%%%%%%%%%%%%%
%\midinsert
%$$\grafone{triCK.eps,height=2in}{{\bf Figure 15.1.c} Typical consumption
%path in environment c.}$$
%\endinsert
%%%%%%%%%%%%

%
%
%\midfigure{triCKf}
%\centerline{\epsfxsize=2.5truein\epsffile{triCK.eps}}
%\caption{Typical consumption path in environment c.}
%\infiglist{triCKf}
%\endfigure
%\medskip

When we compute efficient allocations for each of these three
environments, we  find that the dynamics of the implied
consumption allocations differ dramatically. As an indication of the different outcomes that emerge,
 %Figures 15.1.a-15.1.c
Figures \Fg{triKoch} and \Fg{triCKf} depict   consumption streams
that are associated with the same
realization of a random endowment stream for households living
in environments a, b, and c, respectively.\NFootnote{The dotted lines in these figures % the right panel of figure \Fg{triKoch} and  in \Fg{triCKf}
indicate the consumption allocation
under a hypothetical `complete markets'  arrangement that would give each of a continuum of {\it ex ante} identical villagers consumption always equal to   mean income. % We have not plotted mean income in  the left panel of figure \Fg{triKoch} because we initiated the contract to award maximal profits
 % to the money lender in environment {\bf a}, as we explain in  below.
   We thank Sebastian Graves for writing  Python code that
computes optimal value functions and the policy functions that attain them for these three environments.}
For all three of these economies,
 we set $u(c) =-{\gamma}^{-1} \exp(-\gamma c)$
with $\gamma=.7$, $\beta =.8$,
$[\overline y_1, \ldots, \overline y_{10}] = [6, 7, \ldots, 10]$,
and $\Pi_s = {1-\lambda \over 1-\lambda^{10}} \lambda^{s-1}$ with
$\lambda=.4$.
In all three environments, before date $0$, the households have
entered into efficient contracts with the moneylender. We have initiated  values for a villager that allow the money lender just to break even.
 %\NFootnote{For environment {\bf a}, we start the contract with an initial
 %promised value just above the value that the villager receives in autarky. For this setting, his level of consumption never attains the mean level
 %of income depicted in the dotted line in the other two graphs. Please note the different scales on the consumption axes of the three figures.}
% (Note the different
%time scales in the figures.)%
%
%As a benchmark, a horizontal dotted line in each graph
%depicts the constant consumption level that would be attained in
%a complete markets equilibrium where there are no incentive problems. The dynamics
of consumption outcomes evidently differ substantially across the
three environments, increasing monotonically and then flattening out in environment a,
stochastically heading ``south'' in environment b, and stochastically heading ``north'' in
environment c.
%heading `north' in the Thomas-Worrall (1988) and the Cole-Kocherlakota
%(2001) economy, but heading `south' in the Thomas-Worrall (1992) economy.
These sample path properties reflect how the contract copes with the three different frictions that we have put into the environment relative
to the frictionless chapter  \use{recurpe} setting.  This chapter  explains why  sample paths of consumption differ
so much across these three settings.


\section{One-sided no commitment}\label{sec:moneylender1}%: insurance without commitment}
 Our first incentive problem  is a lack of
commitment. A moneylender is committed   to honor  his
promises, but villagers are free to walk away from their
contract with the moneylender at any time.
The moneylender designs a
contract that the villager wants to honor at every moment and
contingency. Such a contract is said to be self-enforcing.
\index{commitment!one-sided lack of}%
 In chapter \use{socialinsurance2}, we shall study another economy in which
 there is no moneylender, only another villager, and when no one is able
to keep prior commitments.
Such a contract design
problem with participation constraints on both sides of an exchange
represents a problem with two-sided lack of commitment, in contrast to
the problem with one-sided lack of commitment treated here.\NFootnote{For an earlier two-period
model of a one-sided
commitment problem, see Holmstr\" om (1983).} \auth{Holmstr\" om, Bengt} \index{self-enforcing contract}

\vskip-.1cm

\subsection{Self-enforcing contract}
A moneylender  can borrow or lend
resources from outside  the village but the villagers cannot.
 A {\it contract\/} is a sequence
of functions
$c_t = f_t(h_t)$
for $t \geq 0$, where  $h_t = (y_t, \ldots, y_0)$.
 The sequence of functions $\{f_t\}$
assigns a history-dependent consumption stream
\index{history dependence!of consumption stream}%
 $c_t = f_t(h_t)$
to the household.  The contract specifies that each period, the
villager contributes his time $t$ endowment $y_t$ to the
moneylender who then returns $c_t$ to the villager.   From this
arrangement, the moneylender earns an {\it ex ante\/} expected present value
$$ P_{-1} = E_{-1} \sum_{t=0}^\infty \beta^t (y_t - f_t(h_t)).  \EQN prof1 $$
By plugging the associated consumption process into expression \Ep{pref},
we find that the contract assigns the villager an expected present value
of $v= E_{-1} \sum_{t=0}^\infty \beta^t u\left(f_t(h_t)\right)$.

  The contract must be self-enforcing.  At any point in time, the
household is free to walk away from the contract and thereafter consume
its endowment stream.  Thus, if the household walks away from the contract,
it must live in autarky evermore.  The {\it ex ante\/} value associated
with consuming the endowment stream, to be called the \idx{autarky value}, is
$$ v_{\rm aut} = E_{-1} \sum_{t=0}^\infty \beta^t u(y_t) =
 {1 \over 1-\beta}\sum_{s=1}^S \Pi_s u(\overline y_s).       \EQN autarky_value $$
\index{history dependence!of contracts}%
At  time $t$, {\it after\/} having observed its current-period endowment,
the household can guarantee itself a present value of utility of
$u(y_t) + \beta v_{\rm aut}$ by consuming its own endowment.  The
moneylender's contract must offer the household at least this utility
at every possible history and every date.  Thus, the contract must
satisfy
$$  u[f_t(h_t)] + \beta E_t \sum_{j=1}^\infty \beta^{j-1} u[f_{t+j} (h_{t+j})]
    \geq u(y_t) + \beta v_{\rm aut} , \EQN incent1 $$
for all $t \geq 0$ and for all histories $h_t$.  Equation
\Ep{incent1} is called the  \index{participation constraint}%
{\it participation constraint\/} for the villager.  A contract that
satisfies equation \Ep{incent1} is said to be {\it sustainable}.
\index{sustainable contract}

\subsection{Recursive formulation and solution}

 A difficulty with constraints like equation \Ep{incent1} is
that there are so many of them: the dimension of the argument
$h_t$ grows  exponentially with $t$. Fortunately,
 there is a recursive way to describe an interesting subset of history-dependent contracts.
In particular, consider the following way of representing a contract $\{f_t\}$
recursively in terms of a state variable $x_t$:
%\auth{Spear, Stephen E.}  \auth{Srivastava, Sanjay}  \auth{Abreu, Dilip} \auth{Pearce, David}
%\auth{Stacchetti, Ennio}
%%
$$\eqalign{ c_t & = g(x_t, y_t), \cr
            x_{t+1} & = \ell(x_t, y_t). \cr}$$
Here $g$ and $\ell$ are time-invariant functions.
Notice that by iterating the $\ell(\cdot)$ function $t$ times
starting from $(x_0, y_0)$, one obtains
$$ x_t = m_t(x_0; y_{t-1}, \ldots , y_0), \quad t \geq 1. $$
Thus, $x_t$ summarizes histories of endowments $h_{t-1}$.  In this sense,
$x_t$ is a ``backward-looking'' variable.

A remarkable fact is  that the appropriate state variable $x_t$ is
 a {\it promised expected discounted future value} $v_t =
E_{t-1} \sum_{j=0}^\infty \beta^j u(c_{t+j})$.
  This ``forward-looking'' variable summarizes a stream of future utilities.
We shall  formulate the contract recursively by having the
moneylender arrive at $t$, before $y_t$ is realized,
 with a previously made promised value $v_t$. He delivers $v_t$ by
letting $c_t$ and the continuation value $v_{t+1}$ both
respond to $y_t$.  In terms of $v_t(h_{t-1})$, the participation constraint
\Ep{incent1} becomes
$$  v_t(h_{t-1}) = u (f_t(h_t)) + \beta v_{t+1}(h_t) \geq  u(y_t) + \beta v_{\rm aut}. $$

We shall treat the promised value $v$ as a {\it state\/}
variable, then  formulate a functional equation for a moneylender. The
moneylender gives a prescribed value $v$ by delivering a
state-dependent current consumption $c$ and a promised value
starting tomorrow, say $v'$, where $c$ and $v'$ each depend on the
current endowment $y$ and the preexisting promise $v$.  The
moneylender chooses $c$ and $v'$ to  provide the promised value $v$ in  a way that maximizes his profits
\Ep{prof1}.
%to be recorded as a function of $v$ in a value
%function $P(v)$.
% Using dynamic
%programming, we can develop a functional equation for $P(v)$.

\index{promised value!as state variable}

%A planner seeks to deliver a promised value $v$ to the household by
%offering him a history dependent consumption plan.

%Let $P(v)$ denote the optimum value function for the planner, to
%be measured as the least discounted expected value of the endowment
% required to deliver value $v$ to the household.



  Each
 period,  the household must be induced to surrender the time $t$ endowment
$y_t$ to the moneylender, who possibly  gives some of it to other households and   invests
the rest
outside the village at a constant risk-free one-period gross interest rate
of $\beta^{-1}$.  In exchange, the moneylender delivers a state-contingent
 consumption
stream to the household that keeps it participating in the
arrangement every period and after every history.  The moneylender
wants to do this in the most efficient way, that is,
the profit-maximizing way. Let $v$ be the  expected discounted future utility previously  promised to a villager.
Let $P(v)$ be the expected present value
of the ``profit stream'' $\{y_t-c_t\}$ for a moneylender who delivers promised
value $v$ in the optimal way.  The optimum value $P(v)$ obeys the
functional equation
$$ P(v) = \max_{\{c_s,w_s\}} \sum_{s=1}^S \Pi_s [ (\overline y_s - c_s)
        + \beta P(w_s) ] \EQN fe $$
where the maximization is subject to the constraints
$$\EQNalign{ \sum_{s=1}^S \Pi_s[u(c_s) + \beta w_s] & \geq v, \EQN con1 \cr
              u(c_s) + \beta w_s &\geq u(\overline y_s) + \beta v_{\rm aut},
   \quad s=1, \ldots, S; \EQN con2
                    \cr
              c_s \in [c_{\rm min},& c_{\rm max}], \EQN con3 \cr
              w_s  \in [v_{\rm aut},& \bar v] .\EQN con4 \cr}$$
Here $w_s$ is the promised value with which the consumer will enter
next period, given that $y=\overline y_s$ this period; $[c_{\rm
min}, c_{\rm max}]$ is a bounded set to which we restrict the
choice of $c_t$ each period. We restrict the continuation value
$w_s$ to be in the set $[v_{\rm aut}, \bar v]$, where $\bar v$ is a
very large number.  Soon we'll compute the highest value that the
moneylender would ever want to set $w_s$.  All we require now is
that $\bar v$ exceed this value. Constraint \Ep{con1} is the
\idx{promise-keeping constraint}. It requires that the contract deliver at least
promised value $v$. Constraints \Ep{con2}, one for
each state $s$, are the \idx{participation constraint}s.
Evidently, $P$  must be  a decreasing function of $v$ because the
higher  the consumption stream of the villager, the lower must
be the profits of the moneylender.

 The constraint set is convex.  The one-period return function
in equation \Ep{fe} is concave.   The value function $P(v)$ that
solves equation \Ep{fe}  is concave.
In fact, $P(v)$ is strictly concave as will become evident from our
characterization of the optimal contract.
Form the Lagrangian
$$ \eqalign{L =& \sum_{s=1}^S \Pi_s[(\overline y_s - c_s) + \beta P(w_s)] \cr
              & + \mu \left\{\sum_{s=1}^S \Pi_s [u(c_s) + \beta w_s] - v
                \right\} \cr
              & + \sum_{s=1}^S  \lambda_s \biggl\{ u(c_s) + \beta w_s -
    [u(\overline y_s) +
                     \beta v_{\rm aut}]\biggr\}. \cr} \EQN lagr$$
For each $v$ and for $s=1, \ldots, S$,
the first-order conditions for maximizing $L$
with respect to $c_s, w_s$, respectively,
are\NFootnote{Please note that
the $\lambda_s$'s depend on the promised value $v$. In particular,
which $\lambda_s$'s are positive and which are zero will depend on $v$, with
more of them being zero when the  promised value $v$ is higher. See figure \Fg{commit1af}.}
$$\EQNalign{ ( \lambda_s  + \mu \Pi_s) u'(c_s) & = \Pi_s,
              \EQN foc1 \cr
    \lambda_s + \mu \Pi_s  & = - \Pi_s P'(w_s). \EQN foc2 \cr} $$
By the envelope theorem,
%%a formula of \idx{Benveniste and Scheinkman},
if $P$ is
differentiable, then $P'(v)=-\mu$. We will proceed under the assumption
that $P$ is differentiable but it will become evident that $P$ is indeed
differentiable when we understand  the optimal contract.
%%% $P(v)$ is evidently
%%%decreasing in $v$, and is concave.  Thus, $P'(v)$ becomes
%%%more and more negative as $v$ increases.
%%\auth{Benveniste, Lawrence}    \auth{Scheinkman, Jose}

   Equations \Ep{foc1} and \Ep{foc2} imply the following relationship
between $c_s$ and $w_s$:
$$ u'(c_s) = -P'(w_s)^{-1}. \EQN{bind2} $$
This condition states that the household's marginal rate of
substitution between $c_s$ and $w_s$, given by $u'(c_s)/\beta$,
should equal the moneylender's marginal rate of transformation
as given by $-[\beta P'(w_s)]^{-1}$.
The concavity of $P$ and $u$ means that equation \Ep{bind2} traces
out a positively sloped curve in the $c, w$ plane,
as depicted in Figure \Fg{commit1af}.
%Figure 15.2.
 We can interpret this condition
as making $c_s$ a function of $w_s$.
 To complete the optimal contract, it will
be enough to find how $w_s$ depends on the promised
value $v$ and the income state $\overline y_s$.

Condition \Ep{foc2} can be written
$$P'(w_s) = P'(v) - \lambda_s/\Pi_s. \EQN motw1 $$
   How $w_s$ varies with $v$ depends on which of
   two mutually exclusive and exhaustive
 sets of states $(s,v)$ falls into after the
realization of $\overline y_s$: those in
which the participation constraint \Ep{con2} binds (i.e., states
in which $\lambda_s > 0$) or those in which it does not (i.e.,
states in which $\lambda_s =0$).

   %We shall analyze what happens in those states
%in which $\lambda_s >0$ and those in which $\lambda_s =0$.

%%%%%%%%%%%%%%%
%\topinsert{
%$$\grafone{commit1a.eps,height=2.5in}{{\bf Figure 15.2}  Determination of (consumption,
%promised utility).  Higher realizations of $\overline y_s$ are associated with higher
%indifference curves $u(c) + \beta w = u(\overline y_s) + \beta v_{\rm aut}$.  For a
%given $v$, there is a threshold level $\bar y(v)$ above which the participation
%constraint is binding and below which the planner awards a constant level of consumption,
%depending on $v$, and maintains the same promised value $w_s =v$.  The cutoff level
%$\bar y(v)$ is determined by the indifference curve going through the intersection of
%a horizontal line at level $v$ with the ``expansion path'' $u'(c_s)P'(w_s)=-1$.}
%%The consumption level denoted $c_s(v)$ is constructed to satisfy
%%$u'(c_s(v))P'(v) = -1$.} $$
%}\endinsert
%%%%%%%%%%%%%%%%

\midfigure{commit1af}
\centerline{\epsfxsize=3truein\epsffile{commit1a.eps}}
\caption{Determination of consumption and promised utility ($c, w$).
Higher realizations of
$\overline y_s$ are associated with higher indifference curves $u(c) + \beta w =
u(\overline y_s) + \beta v_{\rm aut}$.  For a given $v$, there is a threshold level
$\bar y(v)$ above which the participation constraint is binding and below which the
moneylender awards a constant level of consumption, as a function of $v$, and maintains the
same promised value $w =v$.  The cutoff level $\bar y(v)$ is determined by the
indifference curve going through the intersection of a horizontal line at level $v$
with the ``expansion path'' $u'(c)P'(w)=-1$.}
\infiglist{rao1f}
\endfigure

\vskip.5cm
\medskip
\noindent{\bf States where $\lambda_s >0$}
\smallskip

%\subsubsection{States where $\lambda_s >0$}
\noindent
When $\lambda_s>0$, the participation constraint \Ep{con2} holds
with equality.   When $\lambda_s > 0$,
\Ep{motw1} implies
that $P'(w_s) < P'(v)$, which in turn implies, by the concavity
of $P$, that $w_s > v$.  Further, the participation constraint
at equality implies that $c_s < \overline y_s$
(because $w_s > v \geq v_{\rm aut}$).
Together, these results say that when the \idx{participation
constraint} \Ep{con2} binds, the moneylender induces the household to
consume less than its endowment today by raising its
continuation value.

When $\lambda_s > 0$, $c_s$ and $w_s$ solve
the two equations
$$ \EQNalign{ u(c_s) + \beta w_s & = u(\overline y_s) + \beta v_{\rm aut}, \EQN bind1case1 \cr
              u'(c_s) & = - P'(w_s)^{-1}. \EQN bind2case1 \cr }$$
The participation constraint holds with equality.
Notice that these equations are independent of $v$.  This property
is a key to understanding the form of the optimal contract.
It imparts to the contract what Kocherlakota (1996b)  calls
{\it amnesia\/}:  when incomes $y_t$ are realized that
\index{amnesia!of risk-sharing contract}%
cause the participation constraint to bind, the contract disposes
of all history dependence and makes both consumption and the
continuation value depend only on the current income state $y_t$.
We portray amnesia by   denoting the solutions of equations
\Ep{bind1case1} and \Ep{bind2case1} by
$$\EQNalign{c_s &= g_1(\overline y_s), \EQN tent1;a \cr
            w_s & = \ell_1(\overline y_s). \EQN tent1;b \cr} $$
Later, we'll exploit the amnesia property to produce a computational
algorithm.

\vskip.5cm
\medskip
\noindent{\bf States where $\lambda_s =0$}
\smallskip

%\subsubsection{States where $\lambda_s =0$}
\noindent
  When the participation constraint does not bind, $\lambda_s =0$
and first-order condition \Ep{foc2} imply that
$P'(v) =  P'(w_s)$, which implies that $w_s = v$. Therefore, from \Ep{bind2},
 we can write $u'(c_s) = -P'(v)^{-1}$, so that consumption
in state $s$ depends
on promised utility $v$ but not on the endowment in state $s$.  Thus,
when the participation constraint does not bind, the moneylender
awards
$$\EQNalign{c_s &= g_2(v),  \EQN tent2;a \cr
            w_s &=v,       \EQN tent2;b \cr} $$
where
$g_2(v)$ solves $u'[g_2(v)] = - P'(v)^{-1}$.
\vskip.5cm
\medskip
\noindent{\bf The optimal contract}
\smallskip
\noindent
 Combining the branches of the policy functions for the
cases where the participation constraint does and does
not bind, we obtain
$$\EQNalign{c &= \max\{g_1(y),g_2(v)\},  \EQN policy1 \cr
            w&= \max\{\ell_1(y), v\}.  \EQN policy2 \cr }$$
The  optimal policy is displayed graphically
in Figures \Fg{commit1af} and \Fg{commit2af}.
%15.2 and 15.3.
  To interpret the graphs,
   it is useful to study equations \Ep{con2} and \Ep{bind2}
for the case in which $w_s =v$.  By setting $w_s =v$, we can solve
these equations for a ``cutoff value,'' call it $\bar y(v)$, such that
the participation constraint binds  only when $\overline y_s \geq \bar y(v)$.
To find $\bar y(v)$, we first solve equation \Ep{bind2} for
the value $c_s$ associated with $v$ for those states in which
the participation constraint is not binding:
$$ u'[g_2(v)] = - P'(v)^{-1},$$
and then substitute  this value into \Ep{con2} at equality
to solve for $\bar y(v)$:
$$ u[\bar y(v)]= u[g_2(v)] + \beta (v - v_{\rm aut}).\EQN cutoff $$
By the concavity of $P$, the cutoff value $\bar y(v)$ is
increasing in $v$.

%%%%%%%%%%%%
%midinsert{
%$$ \grafone{commit2a.eps,height=2in}{{\bf Figure 15.3}  The shape
% of consumption as a function of realized endowment, when
%the promised initial value is $v$.}
% $$
%}\endinsert
%%%%%%%%%%%%%%%%%

\midfigure{commit2af}
\centerline{\epsfxsize=3truein\epsffile{commit2a.eps}}
\caption{The shape of consumption as a function of realized endowment, when the promised
initial value is $v$.}
\infiglist{commit2a}
\endfigure

 Associated with a given level of $v_t \in
(v_{\rm aut}, \bar v)$, there are two numbers $g_2(v_t)$, $\bar y(v_t)$
such that
 if $y_t \leq \bar y(v_t)$
 the moneylender offers the household $c_t=g_2(v_t)$
and  leaves the promised utility unaltered, $v_{t+1}=v_t$.
The moneylender is thus insuring the villager against the
states $\overline y_s \leq \bar y(v_t)$ at time $t$.
If $y_t > \bar y(v_t)$, the participation constraint  binds,
prompting the moneylender to induce the household
to surrender some of its current-period
endowment in exchange for a raised promised utility
$v_{t+1} > v_t$.  Promised values never decrease.  They stay constant
for low-$y$ states $\overline y_s < \bar y(v_t)$ and
increase in high-endowment states that threaten
to violate the participation constraint.  Consumption stays constant
during periods when the participation constraint fails to bind
and increases  during periods when it threatens to bind. Whenever the participation binds, the household makes a net transfer
to the money lender in return for a higher promised continuation utility.  A household that has ever realized the highest endowment $y_S$ is
permanently awarded the highest consumption level with an
associated promised value $\bar v$ that satisfies
$$u[g_2(\bar v)] + \beta \bar v = u(\overline y_S) + \beta v_{\rm aut}.$$

%%%At time $0$, but before $y_0$ has been realized,
%%% suppose that the planner wants to deliver
%%%$v_{\rm aut}$.    He can deliver $v_{\rm aut}$ while offering
%%%those unlucky households that receive $y_0 = \overline y_1$
%%%an ex-post payoff in that state of only $u(\overline y_1) + \beta v_{\rm aut}$.
%$v=v_{\rm aut}$ and the initial  endowment is
%its lowest value $\overline y_1$,
%%%This offer satisfies the participation constraint
%%%for $s=1$.  However,
%%%notice that $u(\overline y_1) < E u(y)$ implies that
%%%$ u(\overline y_1) + \beta  v_{\rm aut} < v_{\rm aut}$.
%This means that with $v=v_{\rm aut}$, the participation constraint
%is not binding when $y=\overline y_1$.
%Thus the lowest indifference
%curve in Figure  15.1, that
% indexed by $ u(\overline y_1) + \beta  v_{\rm aut}$,
%is {\it below} $(c,w) = [g_2(v_{\rm aut}), v_{\rm aut}]$, which graphically
%expresses that the participation constraint
%is not binding at endowment level
%$\overline y_1$.
% The optimal contract trades
%off consumption against continuation values
%only for sufficiently higher realizations of $y$.
%%%This means that if the planner were to design the contract at time $0$
%%%but {\it after\/} having observed $y_0$, he could push $\overline y_1$-endowment
%%%households' value of
%%%$u(c_0)+  E_0 \sum_{t=1}^\infty \beta^t u(c_t)$
%%%  below $v_{\rm aut}$. To induce those low-endowment households
%%%to adhere to the contract, the planner has only  to offer a contract
%%%that assures them an autarky  continuation value from date $1$ on.

\subsection{Recursive computation of contract}\label{recursive_comp_cont}%
%Suppose that the initial promised value $v_0$ is $v_{\rm aut}$.
%\tag{recurcomp}{\the\pageno}%
As we will now show, a money lender that takes on a villager whose only alternative is to live in autarky will design a profit maximizing contract that delivers an initial promised value $v_0$ equal to $v_{\rm aut}$. Later, we will examine how the optimal contract would be modified if the initial promised value $v_0$ were to be greater than $v_{\rm aut}$.

 We can compute the optimal contract
recursively by using the fact that the villager will ultimately
receive a constant welfare level equal to $u(\overline y_S)+\beta
v_{\rm aut}$
 after ever having
experienced the maximum endowment $\overline y_S$.
We can characterize
the optimal policy in terms of  numbers
$\{\overline c_s,
\overline w_s\}_{s=1}^S \equiv \{g_1(\overline y_s),
\ell_1(\overline y_s)\}_{s=1}^S $ where
$g_1(\overline y_s)$ and $\ell_1(\overline s)$ are given
by \Ep{tent1}. These numbers can
be computed recursively by working backward as follows.
Start with $s=S$ and compute $(\overline c_S, \overline w_S)$ from
the nonlinear equations:
$$\EQNalign{ u(\overline c_S) + \beta \overline w_S
  & = u(\overline y_S) + \beta v_{\rm aut}, \EQN caltech1;a \cr
   \overline w_S & = {u(\overline c_S) \over 1 - \beta } . \EQN caltech1;b
 \cr }$$
%Now roll back to $S-1$ and compute $(\overline c_{S-1},
%\overline w_{S-1})$ from the nonlinear equations:
%$$ \eqalign{ u(\overline c_{S-1}) + \beta \overline w_{S-1}
%   & = u(y_{S-1}) + \beta v_{\rm aut} \cr
%  \overline w_{S-1} &  = \left[ u(\overline c_{S-1}) +
% \beta \overline w_{S-1} \right]
% \sum_{j=1}^{S-1} \Pi_j
%  + \Pi_S \left[ u(\overline c_S) + \beta \overline w_S \right] \cr} $$
Working backward for $j=S-1, \ldots, 1 $,
 compute $\overline c_j, \overline w_j$ from
the two nonlinear equations
$$\EQNalign{ u(\overline c_j) + \beta\overline w_j & = u(\overline y_j)
   + \beta v_{\rm aut}, \EQN caltech2;a \cr
  \overline w_j  =
\left[ u(\overline c_j)
     + \beta \overline w_j \right] &
 \sum_{k=1}^j \Pi_k
 +  \sum_{k=j+1}^S \Pi_k [u(\overline c_k)+\beta \overline w_k] .
 \hskip1cm   \EQN caltech2;b \cr  } $$
These successive iterations yield the optimal contract characterized by
$\{\overline c_s, \overline w_s\}_{s=1}^S$.
  {\it Ex ante\/}, before the time $0$ endowment has been
realized,  the contract offers the household
$$  v_0  =
%\left[ u(\overline c_1)
%     + \beta v_{\rm aut}  \right]
%  \Pi_1 +
 \sum_{k=1}^S \Pi_k [u(\overline c_k)+\beta \overline w_k]
=  \sum_{k=1}^S \Pi_k [u(\overline y_k) + \beta v_{\rm aut}] =
v_{\rm aut} ,
 \hskip1cm \EQN caltech3 $$
where we have used \Ep{caltech2;a} to verify that the contract
indeed delivers $v_0=v_{\rm aut}$.

Some additional manipulations will enable us to express
$\{\overline c_j\}_{j=1}^S$
solely in terms of the utility function and the endowment process.
First, solve for $\overline w_j$ from \Ep{caltech2;b},
$$
  \overline w_j  =
{  u(\overline c_j) \sum_{k=1}^j \Pi_k
 +  \sum_{k=j+1}^S \Pi_k [u(\overline y_k)+\beta v_{\rm aut}] \over
 1 - \beta \sum_{k=1}^j \Pi_k},
  \EQN caltech_ext1
$$
where we have invoked \Ep{caltech2;a} when
replacing $[u(\overline c_k)+\beta \overline w_k]$
by $[u(\overline y_k)+\beta v_{\rm aut}]$. Next, substitute
\Ep{caltech_ext1} into \Ep{caltech2;a} and solve
for $u(\overline c_j)$,
$$\EQNalign{
  u(\overline c_j)  &= \left[1-\beta \sum_{k=1}^{j} \Pi_k \right]
\left[u(\overline y_j) + \beta v_{\rm aut} \right]
- \beta \sum_{k=j+1}^S \Pi_k
\left[u(\overline y_k)+\beta v_{\rm aut} \right] \cr
&= u(\overline y_j) + \beta v_{\rm aut}
-\beta u(\overline y_j) \sum_{k=1}^{j} \Pi_k
- \beta^2 v_{\rm aut}
- \beta \sum_{k=j+1}^S \Pi_k u(\overline y_k) \cr
&= u(\overline y_j) + \beta v_{\rm aut}
-\beta u(\overline y_j) \sum_{k=1}^{j} \Pi_k
- \beta^2 v_{\rm aut}
- \beta \left[ (1-\beta) v_{\rm aut}
              - \sum_{k=1}^j \Pi_k u(\overline y_k) \right] \cr
&= u(\overline y_j) - \beta \sum_{k=1}^j \Pi_k
\left[ u(\overline y_j) - u(\overline y_k) \right]. \EQN caltech_ext3 \cr}
$$
According to \Ep{caltech_ext3}, $u(\overline c_1)= u(\overline
y_1)$ and $u(\overline c_j) < u(\overline y_j)$ for $j\geq 2$.
That is, a household that realizes a record high  endowment of
$\overline y_j$  must surrender some of that endowment to the
moneylender unless the endowment is the lowest possible value
$\overline y_1$. Households are willing to surrender parts of
their endowments in exchange for promises of insurance (i.e.,
future state-contingent transfers) that are encoded in the
associated continuation values, $\{\overline w_j\}_{j=1}^S$. For
those unlucky households that have so far  realized only
endowments equal to $\overline y_1$, the profit-maximizing
contract prescribes that the households retain their endowment,
$\overline c_1 = \overline y_1$ and by \Ep{caltech2;a}, the
associated continuation value is $\overline w_1 = v_{\rm aut}$.
That is, to induce those low-endowment households to adhere to the
contract, the moneylender has only to offer a contract that assures
them an autarky continuation value in the next period.



%%%Continue with these iterations until $j=2$.
%For $j\geq 2$,
%$\overline w_j > v_{\rm aut}$.
%%%For $j=1$,
%define $\overline w_1 =
%v_{\rm aut}$; as the $t=1$ counterpart of \Ep{caltech2}
%%%we evidently have
%%%$$ u(\overline c_1) + \beta \overline w_1
%%%     = u(\overline y_1)
%%%   + \beta v_{\rm aut} .  \EQN caltech3;a $$
%%%We can solve \Ep{caltech3;a} at equality by setting
%%%$$\overline c_1 = \overline y_1, \quad \overline w_1 = v_{\rm aut}.  $$
%This makes sense, because $\overline c_1$ is the lowest
%level of consumption that will not cause the household to
%walk away into autarky when the maximum  of its current
%and past endowment level is $\overline y_1$.

%%%  {\it Ex ante\/}, before the time $0$ endowment has been
%%%realized,  the contract characterized by $\{\overline w_j,
%%%\overline c_j \}_{j=1}^S$ offers the household
%%%$$  v_{\rm aut}  =
%\left[ u(\overline c_1)
%     + \beta v_{\rm aut}  \right]
%  \Pi_1 +
%%% \sum_{k=1}^S \Pi_k [u(\overline c_k)+\beta \overline w_k] .
%%% \hskip1cm \EQN caltech3;b $$
%%%Using %\Ep{caltech1;a} and
%%%\Ep{caltech2;a} shows that \Ep{caltech3;b}
%%%verifies $v_{\rm aut} = (1-\beta)^{-1} \sum_{k=1}^S u(\overline y_k)
%%%\Pi_k$.

%Then use the following equation to solve for
% $\underline c_0$:\NFootnote{Note that $j_{\rm min -1}$ is chosen
%to assure that
%$$ u(\underline c_0) + \beta v_{\rm aut} \geq
%  u(\overline y_{{\rm j}-1} ) + \beta v_{\rm aut}$$
%is satisfied with a strict inequality.}
%$$
%v_{\rm aut} = \left[ u(\underline c_0)
%     + \beta v_{\rm aut} \right]
% \sum_{k=1}^{j_{\rm min}-1 } \Pi_k
% +  \sum_{k=j_{\rm min}}^S \Pi_k [u(\overline c_k)+\beta \overline w_k]  .$$
%%\Pi_{j+1}

%\vskip.5cm
% \vfil\eject
\medskip
\noindent{\bf Contracts  when $v_0 > \overline w_1 = v_{\rm aut}$}
\smallskip

%%\subsubsection{Contracts  when $v_0 > \overline w_1 = v_{\rm aut}$}
\noindent
  We have shown how to compute the optimal contract when
$v_0 = \overline w_1 = v_{\rm aut} $ by computing %%%a double of
quantities $(\overline c_s, \overline w_s)$
for $s=1, \ldots, S$.  Now   suppose that we  want
to construct a contract that assigns initial
value $v_0 \in [\overline w_{k-1}, \overline w_k)$ for $1
< k \leq S$.  %\NFootnote{The arguments
%in this and the next subsection were constructed by
%William Fuchs and Yuliy Sannikov.}
  Given $v_0$, we
can deduce $k$, then solve for $\tilde c$ satisfying
$$ v_0 = \left( \sum_{j=1}^{k-1} \Pi_j \right)
  \left[u(\tilde c) + \beta v_0 \right]
  + \sum_{j=k}^S \Pi_j \left[u(\overline c_j) + \beta \overline w_j \right].
  \EQN caltech4 $$
The optimal  contract promises $(\tilde c, v_0)$ so long
as the maximum $y_t$ to date is less than or equal to
$\overline y_{k -1}$.  When the maximum $y_t$ experienced
to date equals $ \overline y_j$ for $j \geq k$, the contract offers
$(\overline c_j, \overline w_j)$.
%In particular, this is how we
%would compute $v_{\rm aut}$.

It is plausible that a higher initial expected promised value
$v_0> v_{\rm aut}$ can be  delivered  in the most cost-effective way by choosing a
higher consumption level $\tilde c$ for households that experience low endowment
realizations, $\tilde c > \overline c_j$ for $j=1, \ldots, k-1$.  The reason is that
those unlucky households have high marginal utilities of consumption. Therefore,  transferring
resources  to them  minimizes the resources that are needed to increase
 the {\it ex ante\/}
promised expected utility. As for those lucky households that have received
relatively high endowment realizations, the optimal contract prescribes
an unchanged allocation characterized by
$\{\overline c_j, \overline w_j\}_{j=k}^S$.

If we want to construct a contract that assigns initial value
$v_0 \geq \overline w_S$, the efficient solution is simply to
find the constant consumption level $\tilde c$ that delivers lifetime
utility $v_0$:
$$
v_0 = \sum_{j=1}^S \Pi_j \left[u(\tilde c) + \beta v_0 \right]
 \hskip1cm \Longrightarrow \hskip1cm
v_0 = { u(\tilde c) \over 1-\beta}.
$$
This contract trivially satisfies  all participation constraints, and
a constant consumption level maximizes the expected profit of delivering
$v_0$.

%\vskip.5cm
%\medskip
\vfil\eject
\noindent{\bf Summary of optimal contract}
\smallskip



%%\subsubsection{Summary of optimal contract}
\noindent
Define
$$ s(t) = \{ j: \overline y_j  = \max\{y_0, y_1, \ldots, y_t\} \}.   $$
That is, $\overline y_{s(t)}$ is the maximum endowment that the
household has experienced up  until period $t$.

The optimal contract has the following features.
To deliver promised value $v_0 \in [v_{\rm aut}, \overline w_S]$ to the household,
the contract offers stochastic consumption and continuation values,
$\{c_t, v_{t+1}\}_{t=0}^{\infty}$, that satisfy
$$\EQNalign{ c_t & = \max\{ \tilde c, \,\overline c_{s(t)} \}, \EQN optcon1;a \cr
             v_{t+1} & = \max\{ v_0, \,\overline w_{s(t)} \} , \EQN optcon1;b \cr}$$
where $\tilde c$ is given by \Ep{caltech4}.


%%%The profit-maximizing contract has the following features.
%%%To deliver promised value $v_{\rm aut}$ to the consumer,
%%%the contract offers $(\overline c_1, v_{\rm aut})$
%%%until the first period that the household receives
%%%an endowment greater than $\overline y_1.$
%%%Define
%%%$$ s(t) = \{ j: \overline y_j  = \max\{y_1, \ldots, y_t\} \}.   $$
%%%  So long as $s(t) =1 $, the contract offers $(\overline c_1, v_{\rm aut})$.
%%%The first time that $s(t) \geq 2$,   the contract offers
%%%  $(c_t, v_{t+1})$ where
%%%$$\EQNalign{ c_t & = \overline c_{s(t)} \EQN optcon1;a \cr
%%%             v_{t+1} & = \overline w_{s(t)}. \EQN optcon1;b \cr}$$

\subsection{Profits}

We can use \Ep{fe} to compute expected profits
 from offering
continuation value $\overline w_j$, $j= 1, \ldots, S$.
Starting with $P(\overline w_S)$, we work backward
to compute $P(\overline w_k)$, $k= S-1, S-2, \ldots, 1$:
$$ \EQNalign{ P(\overline w_S) & = \sum_{j=1}^S \Pi_j
  \left({\overline  y_j - \overline c_S    \over 1 - \beta}\right),
  \EQN Pvrecur1;a \cr
  P(\overline w_{k})  = & \sum_{j=1}^{k} \Pi_j (\overline y_j
  - \overline c_{k})  + \sum_{j=k+1}^S \Pi_j (\overline y_j - \overline c_j) \cr
  & + \beta\left[ \sum_{j=1}^{k} \Pi_j P(\overline w_{k})
  + \sum_{j=k+1}^S \Pi_j P(\overline w_j) \right]. \hskip1cm   \EQN Pvrecur1;b \cr} $$
%For $k=1$, we have
% $$\eqalign{ P(\overline w_1) & =
%% \sum_{j=1}^{j_{\rm min} -1} \Pi_j
%% (\overline y_j - \underline c_{0})  +
% \sum_{j=2}^S \Pi_j (\overline y_j - \overline c_j) \cr
%  & + \beta\left[ \Pi_1 P(\overline w_1)
%  + \sum_{j=2}^S \Pi_j P(\overline w_j) \right].
% \hskip1cm    \cr} \EQN Pvaut1 $$



\vskip.5cm
\medskip
\noindent{\bf Strictly positive profits for $v_0=v_{\rm aut}$}
\smallskip


%\subsubsection{Strictly positive profits for $v_0=v_{\rm aut}$}
\noindent
We will now demonstrate that a contract that offers an initial
promised value of $v_{\rm aut}$ is associated with strictly
positive expected profits. In order to show that $P(v_{\rm
aut})>0$, let us first examine the expected profit implications of
the following limited obligation. Suppose that a household has
just experienced $\overline y_j$ for the first time and that the
limited obligation amounts to delivering $\overline c_j$ to the
household in that period and in all future periods until the
household realizes an endowment higher than $\overline y_j$. At
the time of such a higher endowment realization in the future, the
limited obligation ceases without any further transfers. Would
such a limited obligation be associated with positive or negative
expected profits? In the case of $\overline y_j=\overline y_1$,
this would entail a deterministic profit equal to zero, since we
have shown above that $\overline c_1=\overline y_1$. But what is
true for other endowment realizations?

To study the expected profit implications of such a limited
obligation for any given $\overline y_j$, we first compute an
upper bound for the obligation's consumption level
$\overline c_j$ by using \Ep{caltech_ext3}:
$$\EQNalign{
  u(\overline c_j)  &=
\left[1-\beta \sum_{k=1}^j \Pi_k \right] u(\overline y_j)
+ \beta \sum_{k=1}^j \Pi_k u(\overline y_k)         \cr
&\leq
u\!\left( \left[1-\beta \sum_{k=1}^j \Pi_k \right] \overline y_j
+ \beta \sum_{k=1}^j \Pi_k \overline y_k \right),  \cr}
$$
where the weak inequality is implied by the strict concavity of
the utility function, and evidently the expression holds with strict inequality
for $j>1$. Therefore, an upper bound for $\overline c_j$ is
$$\EQNalign{
  \overline c_j &\leq
 \left[1-\beta \sum_{k=1}^j \Pi_k \right] \overline y_j
+ \beta \sum_{k=1}^j \Pi_k \overline y_k.     \EQN posprof1 \cr}
$$

We can sort out the financial consequences of the limited
obligation by looking separately at the first period and then at all
future periods. In the first period, the moneylender obtains a nonnegative
profit,
$$\EQNalign{
  \overline y_j - \overline c_j &\geq
 \overline y_j - \left( \left[1-\beta \sum_{k=1}^j \Pi_k \right] \overline y_j
+ \beta \sum_{k=1}^j \Pi_k \overline y_k \right)           \cr
&=  \beta \sum_{k=1}^j \Pi_k \left[ \overline y_j -\overline y_k \right],
                                             \EQN posprof2 \cr}
$$
where we have invoked the upper bound on $\overline c_j$ in
\Ep{posprof1}. After that first period, the moneylender must continue
to deliver $\overline c_j$ for as long as the household does not
realize an endowment greater than $\overline y_j$.  So the
probability that the household remains within the limited
obligation for another $t$ number of periods is $(\sum_{i=1}^j
\Pi_i)^t$. Conditional on remaining within the limited obligation,
the household's average endowment realization is $(\sum_{k=1}^j
\Pi_k \overline y_k ) / (\sum_{k=1}^j \Pi_k)$. Consequently, the
expected discounted profit stream associated with all future
periods of the limited obligation, expressed in first-period
values, is
%\vfil\eject
$$\EQNalign{
\sum_{t=1}^{\infty} \beta^t \left[ \sum_{i=1}^j  \Pi_i \right]^t
\left[ {\sum_{k=1}^j   \Pi_k \overline y_k \over \sum_{k=1}^j   \Pi_k }
- \overline c_j \right]
&= { \left[ \beta  \sum_{i=1}^j  \Pi_i \right]
\over 1 - \beta \sum_{i=1}^j  \Pi_i }
\left[ {\sum_{k=1}^j   \Pi_k \overline y_k \over \sum_{k=1}^j   \Pi_k }
- \overline c_j \right] \cr
& \geq
- \beta \sum_{k=1}^j \Pi_k \left[ \overline y_j -\overline y_k \right],
                                             \EQN posprof3 \cr}
$$
where the inequality is obtained after invoking the upper bound on
$\overline c_j$ in \Ep{posprof1}. Since the sum of \Ep{posprof2} and
\Ep{posprof3} is nonnegative, we conclude that the limited
obligation  at least breaks even in expectation.
In fact, for $\overline y_j > \overline y_1$ we have that
\Ep{posprof2} and \Ep{posprof3} hold with strict inequalities, and
thus, each such limited obligation is associated with strictly positive
profits.

Since the optimal contract with an initial promised value of
$v_{\rm aut}$ can be viewed as a particular constellation of all
of the described limited obligations, it follows immediately that
$P(v_{\rm aut})>0$.


\vskip.5cm
\medskip
\noindent{\bf Contracts with $P(v_0) =0$}
\smallskip


%\subsubsection{Contracts with $P(v_0) =0$}
\noindent
In exercise {\it \the\chapternum.2\/}, you will be asked to compute $v_0$ such that $P(v_0)=0$.
Here is a good way to do this.   After computing the optimal contract for
$v_0 = v_{\rm aut}$, suppose that we can find some $k$ satisfying
$1 < k \leq S$ such that
for $j \geq k,  P(\overline w_j) \leq 0$ and for $j< k$, $P(\overline w_k)
  > 0$.   Use a zero-profit
condition to find an initial $\tilde c$ level:
$$ 0 = \sum_{j=1}^{k-1} \Pi_j (\overline y_j - \tilde c)
  + \sum_{j=k}^S \Pi_j \left[ \overline y_j - \overline c_j
  + \beta P(\overline w_j) \right] .$$
Given $\tilde c$, we can solve
\Ep{caltech4}
for $v_0$.

However, such a $k$ will fail to exist if $P(\overline w_S) > 0$.
In that case, the efficient allocation associated with $P(v_0)=0$
is a trivial one. The moneylender would simply set consumption equal
to the average endowment value. This contract breaks  even on
average, and the household's utility is equal to the first-best
unconstrained outcome, $v_0 = v_{\rm pool}$, as given in
\Ep{pool_value}.


\subsection{$P(v)$ is strictly concave and continuously differentiable}

Consider a promised value $v_0\in[\overline w_{k-1}, \overline w_k)$ for
$1 < k \leq S$. We can then use equation \Ep{caltech4}
to compute the amount of consumption $\tilde c(v_0)$ awarded to a household
with promised value $v_0$, as long as the household is not experiencing
an endowment greater than $\overline y_{k-1}$:
$$ u\!\left[\tilde c(v_0)\right] = {
 \left[ 1- \beta \sum_{j=1}^{k-1} \Pi_j \right]
v_0 - \sum_{j=k}^S \Pi_j \left[u(\overline c_j) + \beta \overline w_j \right]
\over
\sum_{j=1}^{k-1} \Pi_j} \equiv \Phi_k(v_0),           \EQN concdiff0a
$$
that is,
$$
\tilde c(v_0) = u^{-1}\left[\Phi_k(v_0)\right].       \EQN concdiff0b
$$
Since the utility function is strictly concave, it follows that $\tilde c(v_0)$
is strictly convex in the promised value $v_0$:
$$\EQNalign{
\tilde c'(v_0) &= {
 \left[ 1- \beta \sum_{j=1}^{k-1} \Pi_j \right]
\over
\sum_{j=1}^{k-1} \Pi_j}\, {u^{-1}}'\left[\Phi_k(v_0)\right] > 0,  \EQN concdiff1;a \cr
\tilde c''(v_0) &= {
 \left[ 1- \beta \sum_{j=1}^{k-1} \Pi_j \right]^2
\over
 \left[ \sum_{j=1}^{k-1} \Pi_j\right]^2} \,{u^{-1}}''\left[\Phi_k(v_0)\right] > 0.
                                                                \EQN concdiff1;b \cr}
$$
Next, we evaluate the expression for expected profits in \Ep{fe} at the
optimal contract,
$$
\EQNalign{
P(v_0) &= \sum_{j=1}^{k-1} \Pi_j \left[\overline y_j - \tilde c(v_0)
  + \beta P(v_0) \right] + \sum_{j=k}^S \Pi_j \left[ \overline y_j - \overline c_j
  + \beta P(\overline w_j) \right], \cr
\noalign{\hbox{which can be rewritten as }} \cr
P(v_0) &={ \sum_{j=1}^{k-1} \Pi_j \left[\overline y_j - \tilde c(v_0) \right]
  + \sum_{j=k}^S \Pi_j \left[ \overline y_j - \overline c_j
  + \beta P(\overline w_j) \right] \over
  1 - \beta \sum_{j=1}^{k-1} \Pi_j}. \cr}
$$
We can now verify that $P(v_0)$ is strictly concave for
$v_0\in[\overline w_{k-1}, \overline w_k)$,
$$\EQNalign{
P'(v_0) &= - { \sum_{j=1}^{k-1} \Pi_j \over
 1- \beta \sum_{j=1}^{k-1} \Pi_j } \,\tilde c'(v_0)
                = -{u^{-1}}'\left[\Phi_k(v_0)\right] < 0, \hskip1cm \EQN concdiff2;a \cr
P''(v_0) &= - { \sum_{j=1}^{k-1} \Pi_j \over
 1- \beta \sum_{j=1}^{k-1} \Pi_j } \,\tilde c''(v_0) \cr
&\hskip1.2cm = - {
 \left[ 1- \beta \sum_{j=1}^{k-1} \Pi_j \right]
\over
\sum_{j=1}^{k-1} \Pi_j} \,{u^{-1}}''\left[\Phi_k(v_0)\right] < 0,
                                                   \EQN concdiff2;b \cr}
$$
where we have invoked expressions \Ep{concdiff1}.

To shed light on the properties of the value function $P(v_0)$ around
the promised value $\overline w_k$, we can establish that
$$
\lim_{v_0 \uparrow \overline w_k} \Phi_k(v_0)
= \Phi_k(\overline w_k) \; = \; \Phi_{k+1}(\overline w_k),  \EQN concdiff3
$$
where the first equality is a trivial limit of expression \Ep{concdiff0a}
while the second equality can be shown to hold because a rearrangement of
that equality becomes merely a restatement of a version of
expression \Ep{caltech2;b}.
On the basis of \Ep{concdiff3} and \Ep{concdiff0b}, we can conclude that
the consumption level $\tilde c(v_0)$ is continuous in the promised value
which in turn implies continuity of the value function $P(v_0)$. Moreover,
expressions \Ep{concdiff3} and \Ep{concdiff2;a} ensure that the value function
$P(v_0)$ is continuously differentiable in the promised value.


\subsection{Many households}

  Consider a large  village in which
a moneylender faces a continuum of such households.
At the beginning of time $t=0$, before the realization of $y_0$,
the moneylender offers each household $v_{\rm aut}$ (or maybe just a
small amount more).  As time unfolds, the moneylender executes the
contract for each household. A society of such households would
experience a ``fanning out'' of the  \index{fanning out!of wealth distribution}%
distributions of consumption and continuation values across
households for a while, to be followed by an eventual ``fanning
in'' as the cross-sectional distribution of consumption
asymptotically becomes concentrated at the single point  $g_2(\bar
v)$ computed earlier (i.e., the minimum $c$ such that the
participation constraint will never again be binding).  Notice
that early on the moneylender would on average, across villagers,
be collecting money from the villagers,  depositing it in the
bank, and receiving the gross interest rate $\beta^{-1}$ on the
bank balance.  Later he
%%%%would
could be using the interest on his  account outside the village
to finance payments to the villagers. Eventually,
the villagers are completely insured, i.e., they experience no
fluctuations in their consumptions.

For a contract that offers initial promised value
$v_0 \in [v_{\rm aut}, \overline w_S]$,
constructed as above, we can compute the dynamics of the cross-section
 distribution of consumption by appealing to a law of large
numbers of the kind used in chapter
\use{incomplete}.  At time $0$, after the time
$0$ endowments have been realized,   the cross-section
distribution of consumption is evidently
$$ \EQNalign{
 {\rm Prob} \{c_0 = \tilde c\} & =
   \left( \sum_{s=1}^{k-1} \Pi_s \right)  \EQN steadstp0;a \cr
 {\rm Prob} \{c_0 \leq \overline c_j\} & =
   \left( \sum_{s=1}^j \Pi_s \right) , \  j \geq k. \EQN steadstp0;b
     \cr} $$
After $t$ periods,
$$\EQNalign{
 {\rm Prob} \{c_t = \tilde c\} & =
   \left( \sum_{s=1}^{k-1} \Pi_s \right)^{t+1}  \EQN steadstp1;a \cr
 {\rm Prob} \{c_t \leq \overline c_j\} & =
   \left( \sum_{s=1}^j \Pi_s \right)^{t+1},   \ j \geq k.
 \EQN steadstp1;b \cr }$$

  From the cumulative distribution functions \Ep{steadstp0} and
\Ep{steadstp1}, it is easy to compute the corresponding densities
$$ f_{j,t} = {\rm Prob} (c_t = \overline c_j)   \EQN{densityjt} $$
where here we set $\overline c_j = \tilde c$ for all $j < k$.
These densities allow us to compute the evolution over time
of the moneylender's bank balance.  Starting with initial
balance $\beta^{-1} B_{-1} =0$ at time $0$, the moneylender's balance at
the bank evolves according to
$$ B_{t} = \beta^{-1} B_{t-1}
  + \left(\sum_{j=1}^S \Pi_j \overline y_j - \sum_{j=1}^S
  f_{j,t}  \overline c_j\right) \EQN bankbalance $$
for $t \geq 0$, where $B_t$ denotes the end-of-period balance
in period $t$.  Let $\beta^{-1} = 1+r$.
 After the cross-section distribution
of consumption has converged to a distribution
concentrated on $\overline c_S$, the moneylender's bank
balance will obey the difference equation
$$ B_t = (1+r) B_{t-1} + E(y) - \overline c_S , \EQN Bdiff  $$
where $E(y)$ is the mean of $y$.

A convenient formula links $P(v_0)$ to the tail behavior of $B_t$,
in particular, to the behavior of $B_t$  after the consumption
distribution has converged to $\overline c_S$. Here we are once
again appealing to a law of large numbers so that the expected
profits $P(v_0)$ becomes a nonstochastic present value of profits
associated with making a promise $v_0$ to a large number of
households. Since the moneylender lets all surpluses and deficits
 accumulate in the bank account, it follows that $P(v_0)$
is equal to the present value of the sum of any future balances
$B_t$ and the continuation value of the remaining profit stream.
After
all households' promised values have converged to $\overline
w_S$, the continuation value of the remaining profit stream is evidently
equal to $\beta P(\overline w_S)$.  Thus, for $t$ such that the distribution of $c$ has
converged to $\overline c_s$, we deduce that
$$
P(v_0) = {B_t + \beta P(\overline w_S) \over (1+r)^t }.   \EQN
steadst_PV
$$

Since the term $\beta P(\overline w_S)/(1+r)^t$ in expression
\Ep{steadst_PV} will vanish in the limit, the expression implies
that the bank balances $B_t$ will eventually change at the gross
rate of interest. If the initial $v_0$ is set so that
$P(v_0)>0$ ($P(v_0)<0$), then the balances will eventually go to
plus infinity (minus infinity) at an exponential rate. The
asymptotic balances would  be constant only if the initial $v_0$
is set so that $P(v_0)=0$. This has the following implications. First,
recall from our calculations above that there can  exist an
initial promised value $v_0 \in [v_{\rm aut}, \overline w_S]$ such
that $P(v_0)=0$ only if it is true that $P(\overline w_S) \leq 0$,
which by \Ep{Pvrecur1;a} implies that $E(y)\leq\overline c_S$.
After imposing $P(v_0)=0$ and using the expression for
$P(\overline w_S)$ in \Ep{Pvrecur1;a}, equation \Ep{steadst_PV}
becomes $ B_t = - \beta {E(y) - \overline c_S \over 1-\beta}$, or
$$ B_t = {\overline c_S - E(y) \over r} \geq 0,
$$
where we have used the definition $\beta^{-1} = 1+r$. Thus, if the
initial promised value $v_0$ is such that $P(v_0)=0$, then the
balances will converge when all households' promised values
converge to $\overline w_S$. The interest earnings on those
stationary balances will equal  the one-period deficit associated
with delivering $\overline c_S$ to every household while
collecting endowments per capita equal to $E(y) \leq \overline
c_S$.


%%Because $P(v_{\rm aut}) >0 $,  the  moneylender's bank balance
%%will not converge.  If the initial $v$ is set so that $P(v) >0$,
%%then we shall always have
%%$r B_{t-1} +E(y) -\overline c_S >0$. Because the
%%moneylender is not withdrawing funds, his
%%balance will  eventually
%%increase  at the gross rate of interest.


After enough time has passed, all of the villagers will  be perfectly
insured because according to \Ep{steadstp1}, $\lim_{t \rightarrow
+\infty} {\rm Prob}(c_t = \overline c_S) =1$.
  How much time it takes to converge depends
on the distribution $\Pi$. Eventually, everyone will have received
the highest endowment realization sometime in the past, after
which his continuation value remains fixed.   Thus, this is a
model of temporary imperfect insurance, as indicated by the
eventual ``fanning in'' of the distribution of continuation values.


\subsection{An example}

Figures \Fg{tworr1f}
and \Fg{tworr2f}  summarize aspects of the optimal contract for a version
of our economy in which each household has an i.i.d.\ endowment
process that is distributed as
$$ {\rm Prob} (y_t = \overline y_s) = {1-\lambda\over 1-\lambda^S }
 \lambda^{s-1}  $$
where $\lambda \in (0,1)$ and $\overline y_s = s+5$ is the $s$th possible
endowment value, $s=1, \ldots, S$.  The typical household's one-period
utility function is $u(c) = (1-\gamma)^{-1} c^{1-\gamma}$, where $\gamma$
is the household's coefficient of relative risk aversion.  We have assumed
the parameter values $(\beta, S, \gamma, \lambda) = (.5, 20, 2, .95)$.
The initial promised value $v_0$ is set so that $P(v_0)=0$.


%% Exercise \the\chapternum.XXX asks you to compute the optimal contract for these parameter
%%values.  For the computations summarized in Figures 15.4 and 15.5, we have
%%set $v_0$ so that $P(v_0)=0$.
The moneylender's bank balance in Figure \Fg{tworr1f}, panel d, starts at
zero. The moneylender makes money at first, which he deposits in
the bank.  But as time passes, the moneylender's bank balance
converges to the point that he is earning just enough interest on
his balance to finance the extra payments he must make to pay
$\overline c_S$ to each household each period.  These interest
earnings make up for the deficiency of his per capita period
income $E(y)$, which is less than his per period per capita
expenditures $\overline c_S$.

%%%%%%%%%%%%%%%
%\midinsert{
% $$ \grafone{tworr1.eps,height=3in}{{\bf Figure 15.4}
%Optimal contract when $P(v_0)=0$.
%Panel a: $c_s$ as function of maximum $y_s$ experienced.  Panel b:
%$w_s$ as function of maximum $y_s$ experienced to date.
%Panel c: $P(w_s)$ as function of maximum $y_s$ experienced;
%panel d: the moneylender's bank balance.} $$
%}  \endinsert
%%%%%%%%%%%%%%%%

\midfigure{tworr1f}
\centerline{\epsfxsize=4truein\epsffile{tworr1update.ps}}
\caption{Optimal contract when $P(v_0)=0$.  Panel a: $\overline c_s$ as function of
maximum $\overline y_s$ experienced to date.  Panel b: $\overline w_s$ as function of maximum $\overline y_s$
experienced.  Panel c: $P(\overline w_s)$ as function of maximum $\overline y_s$ experienced. Panel d: The
moneylender's bank balance.}
\infiglist{tworr1f}
\endfigure

%%%%%%%%%%%%%%%%
%\midinsert{
% $$ \grafone{tworr2.eps,height=2in}{{\bf Figure 15.5}
%Cumulative distribution functions $F_t(c_t)$ for consumption
%for $t=1, 3, 5, \hfil\break 10, 25, 100$ when $P(v_0)=0$ (later dates have
%c.d.f.s shifted to right).}
% $$ }\endinsert
%%%%%%%%%%%%%%%%

\midfigure{tworr2f}
\centerline{\epsfxsize=3truein\epsffile{tworr2update.ps}}
\caption{Cumulative distribution functions $F_t(c_t)$ for consumption for $t=0, 2, 5,
10, 25, 100$ when $P(v_0)=0$ (later dates have c.d.f.s shifted to right).}
\infiglist{tworr2f}
\endfigure

\section{A Lagrangian method}

  Marcet and Marimon (1992, 1999)
\auth{Marcet, Albert}%
 \auth{Marimon, Ramon}%
have proposed an approach that applies to most of the contract
design problems of this chapter. They form a Lagrangian and use
the Lagrange multipliers on incentive constraints to keep track
of promises.  Their approach  extends the work of Kydland and
Prescott (1980)
 and is related to Hansen, Epple, and Roberds' (1985)
formulation for linear quadratic environments.\NFootnote{Marcet
and Marimon's  method is a variant of the method used to compute
Stackelberg or Ramsey plans in chapter \use{stackel}. See chapter
\use{stackel} for a more extensive review of the history of the ideas
underlying Marcet and Marimon's approach, in particular, some work
from Great Britain in the 1980s by Miller, Salmon, Pearlman, Currie,
and Levine.}
  We can illustrate the method in the context of the preceding model.
  \auth{Kydland, Finn E.} \auth{Prescott, Edward C.}%
  \auth{Hansen, Lars P.} \auth{Epple, Dennis} \auth{Roberds, William}%

 Marcet and Marimon's approach would be to formulate the problem directly
in the space of stochastic processes (i.e., random sequences)
and to  form a Lagrangian
for the moneylender. The contract specifies
a stochastic process for consumption obeying the following constraints:
$$\EQNalign{
& u(c_t) + E_t \sum_{j=1}^\infty \beta^j u(c_{t+j}) \geq u(y_t) +
\beta v_{\rm aut} \ ,
                           \forall t\geq 0,           \EQN MM1;a \cr
& E_{-1} \sum_{t=0}^\infty \beta^t u(c_{t}) \geq v,             \EQN MM1;b \cr}
$$
where $E_{-1}(\cdot)$ denotes the conditional expectation
before $y_0$ has been realized.
 Here $v$ is the initial promised value to be delivered to the villager
starting in period $0$.
Equation \Ep{MM1;a} gives the participation constraints.

\tag{abelsummation}{\the\pageno}
 The moneylender's Lagrangian is
$$\eqalign{ J & = E_{-1} \sum_{t=0}^\infty \beta^t \Bigl\{ (y_t -c_t)
   + \alpha_t\Bigl[E_t \sum_{j=0}^\infty \beta^j u(c_{t+j})
    - [ u(y_t) + \beta v_{\rm aut} ] \Bigr]\Bigr\}   \cr
    &  + \phi \Bigl[ E_{-1} \sum_{t=0}^\infty
      \beta^t u(c_{t}) - v \Bigr] ,  \cr} \EQN MM2
$$
where $\{ \alpha_t\}_{t=0}^\infty$ is a stochastic process
of nonnegative Lagrange multipliers on the participation
constraint of the villager and $\phi$ is
the strictly positive multiplier on the initial promise-keeping constraint that states
that the moneylender must deliver $v$.
 It is useful to transform the Lagrangian by making
use of the following equality, which is a version
of the ``partial summation formula of Abel'' (see
Apostol, \auth{Apostol, Tom}%
1975, p. 194):
$$ \sum_{t=0}^\infty \beta^t \alpha_t \sum_{j=0}^\infty  \beta^j u(c_{t+j})
     = \sum_{t=0}^\infty \beta^t \mu_t u(c_t),  \EQN MM3 $$
where
$$
\mu_t = \mu_{t-1} + \alpha_t, \hskip1.5cm \hbox{\rm with} \hskip.3cm \mu_{-1} = 0.
                                                \EQN MM3i
$$
  Formula \Ep{MM3} can be
verified directly.  If we substitute formula \Ep{MM3} into formula \Ep{MM2}
and use the law of iterated expectations to justify
$E_{-1} E_t (\cdot) = E_{-1} (\cdot)$, we obtain
$$\EQNalign{
J =  E_{-1} \sum_{t=0}^\infty \beta^t &\left\{ (y_t - c_t)
      +  (\mu_t + \phi) u(c_t)                     \right.          \cr
 & \left. - (\mu_t - \mu_{t-1}) \left[u(y_t) + \beta v_{\rm aut} \right]\right\}
    - \phi v.                                                 \EQN MM3a \cr}
$$
For a given value $v$, we seek a saddle point: a maximum
with respect to $\{c_t \}$, a minimum with  respect to
$\{\mu_t \}$ and $\phi$. The first-order condition with respect to
$c_t$ is
$$ u'(c_t) = {1 \over \mu_t + \phi}, \EQN MM4;a
$$
which is a version of equation \Ep{bind2}.  Thus,
$-(\mu_t + \phi)$ equals  $ P'(w)$  from the previous section,
so that the multipliers encode the information contained
in the derivative of the moneylender's value function.
  We also have the  complementary slackness conditions
$$\eqalign{ u(c_t) + E_t  & \sum_{j=1}^\infty \beta^j u(c_{t+j}) -
 \left[u(y_t) + \beta v_{\rm aut}\right]
      \geq 0, \hskip.5cm %%\cr
    =0\ {\rm if} \ \alpha_t >0; \cr}       \EQN MM4;b $$
$$
 E_{-1} \sum_{t=0}^\infty \beta^t u(c_{t}) - v =0.           \EQN MM4;c
$$
Equation \Ep{MM4} together with the transition law \Ep{MM3i}
characterizes the solution of the moneylender's maximization problem.

To explore the time profile of the optimal consumption process, we now
consider some period $t\geq 0$ when $(y_t, \mu_{t-1},\phi)$ are known.
 First, we tentatively try the solution $\alpha_t=0$ (i.e., the participation
constraint is not binding).  Equation \Ep{MM3i} instructs us then to
set $\mu_t = \mu_{t-1}$, which by first-order condition \Ep{MM4;a}
implies that $c_t = c_{t-1}$. If this outcome satisfies participation
constraint \Ep{MM4;b}, we have our solution for period $t$. If not,
it signifies that the participation constraint binds. In other words,
the solution has $\alpha_t>0$ and $c_t>c_{t-1}$. Thus,
equations \Ep{MM3i} and \Ep{MM4;a} immediately show us that $c_t$
is a nondecreasing random sequence, that $c_t$ stays
constant when the participation constraint is not binding,
and that it rises when the participation constraint binds.

The numerical computation of a solution to equation \Ep{MM3a}
is complicated by the fact that slackness conditions \Ep{MM4;b}
and \Ep{MM4;c} involve conditional expectations
of future endogenous variables $\{c_{t+j}\}$. Marcet and
Marimon (1992) handle this complication by resorting to the
parameterized expectation approach; that is, they replace
the conditional expectation by a parameterized function of
the state variables.\NFootnote{For details on the implementation
of the parameterized expectations approach in a simple growth
model, see den Haan and Marcet (1990). The parameterized expectations method was applied  by Krusell and Smith (1998) to compute an approximate equilibrium
of an incomplete markets model with a fluctuating aggregate state variable.  See chapter \use{incomplete}.}
  Marcet and Marimon (1992, 1999) \auth{Marcet, Albert}\auth{Marimon, Ramon}%
describe a variety of other examples using the Lagrangian method.
See Kehoe and Perri (2002)
\auth{Kehoe, Patrick J.}\auth{Perri, Fabrizio}%
for an application to an international trade  model.
\auth{Krusell, Per}%
\auth{Smith, Anthony}%



\section{Insurance with asymmetric information}
\index{asymmetric information}%
\index{commitment!two-sided}%
 The moneylender-villager environment of
section \use{sec:moneylender1} poses a
commitment problem  because  agents are free to choose autarky
each period; but there is  no information problem. We now study a
contract design problem where the incentive problem comes not from
a commitment problem, but instead from asymmetric information. As
before, the moneylender or planner can borrow or lend outside the village at the
constant risk-free gross interest rate of $\beta^{-1}$, and each
household's income $y_t$ is independently and identically
distributed across time and across households. However, now we
assume that the planner and household can  enter
into an enduring and   binding contract. At the beginning of time,
let $v^o$ be the expected lifetime utility that the planner
promises to deliver to a household. The initial promise $v^o$
could presumably not be less than $v_{\rm aut}$, since a household
would not accept a contract that gives a lower utility than he could attain at time $0$ by choosing autarky.
 We defer discussing how $v^o$ is
determined until the end of the section. The other new assumption
here is that households have private information about their own
income, and that the planner can see neither their income nor their
consumption. It follows that any transfers between the
planner and a household must be based on the household's own
reports about income realizations. An incentive-compatible
contract makes households choose to report their incomes
truthfully.

Our analysis follows the work by Thomas and Worrall (1990),
\auth{Thomas, Jonathan}\auth{Worrall, Tim}%
 who
make a few additional assumptions about the preferences in expression \Ep{pref}:
$u:(a, \infty) \rightarrow {\bbR}$ is twice continuously
differentiable with ${\rm sup}\,u(c)<\infty$, ${\rm inf}\,u(c)=-\infty$,
$\lim_{c\rightarrow a}u'(c)=\infty$.
Thomas and Worrall also use the following special assumption:
\medskip
\noindent{\sc Condition A:}  $-u''/u'$ is nonincreasing.
\medskip
\noindent
This is a sufficient condition to make the value function
concave, as we will discuss.  The roles of the other restrictions
on preferences will also be revealed.
\auth{Fernandes, Ana}%
\auth{Phelan, Christopher}%

An  efficient insurance contract  solves a
dynamic programming problem.\NFootnote{It is important that the endowment is independently distributed over time.  See
Fernandes and Phelan (2000) for a related analysis that shows complications that arise when the iid assumption is relaxed}  A planner maximizes expected
discounted profits, $P(v)$, where $v$ is the household's promised
utility from last period. The planner's current payment
to the household, denoted $b$ (repayments from the household register
as  negative numbers),
is a function of the state variable $v$ and the household's reported
current income $y$. Let $b_s$ and $w_s$ be the payment and continuation
utility awarded to the household if it reports income $\overline y_s$.
The optimum value function $P(v)$ obeys the functional equation
$$ P(v) = \max_{\{b_s,w_s\}} \sum_{s=1}^S \Pi_s [ -b_s
        + \beta P(w_s) ] \EQN fe_asym $$
where the maximization is subject to the constraints
%%{\ninepoint
$$\EQNalign{& \sum_{s=1}^S \Pi_s\left[u(\overline y_s + b_s) + \beta w_s\right]
                                                = v \EQN con1_asym \cr
        & C_{s,k} \equiv  u(\overline y_s + b_s) + \beta w_s -
                         \Bigl[u(\overline y_s + b_k) + \beta w_k\Bigr] \geq 0,
                                \ s,k \in {\bf S} \times {\bf S} \hskip1cm \EQN con2_asym \cr
 &  b_s \in [a-\overline y_s,\, \infty] \,, \ s \in {\bf S}    \EQN con3_asym \cr
 &  w_s \in [-\infty,\, v_{\rm max}] \,,\ s \in {\bf S}
                                                           \EQN con4_asym \cr}
$$  %%}%endninept
where $v_{\rm max}={\rm sup}\,u(c)/(1-\beta)$. Equation
\Ep{con1_asym} is the ``promise-keeping'' constraint guaranteeing that
the promised utility $v$ is delivered. Note that our  earlier weak
inequality in \Ep{con1} is replaced by an equality. The planner
cannot award a higher utility than $v$ because that could  violate
an \idx{incentive-compatibility constraint} for telling the truth in
earlier periods.
The set of constraints \Ep{con2_asym} ensures that the households have no
incentive to lie about their endowment realization in each
state $s \in {\bf S}$.
  Here $s$ indexes the actual income state, and
$k$ indexes the reported income state.  We express the
incentive compatibility constraints when the endowment
is in state $s $ as $C_{s,k} \geq 0$ for $k \in {\bf S}$.
Note also that there are no ``participation constraints''
like expression \Ep{con2} from our earlier  model, an absence that
reflects the assumption that both parties are committed to the contract.

It is instructive to establish bounds on the value function $P(v)$.
Consider first a contract that pays a constant amount $\bar b = \bar b(v)$ in all
periods, where $\bar b(v)$ satisfies $\sum_{s=1}^S \Pi_s u(\overline y_s + \bar b)
/(1-\beta) = v$. It is trivially incentive compatible and delivers
the promised utility $v$. Therefore, the discounted profits from this
contract, $- \bar b / (1 -\beta)$,
 provide a lower bound on $P(v)$. In addition, $P(v)$
cannot exceed the value of the unconstrained first-best contract that
pays $\bar c - \overline y_s$ in all periods, where $\bar c$ satisfies
$\sum_{s=1}^S \Pi_s u(\bar c) /(1-\beta) = v$. Thus, the
value function is bounded by
$$
-\bar b(v) / (1-\beta) \;\leq\; P(v) \;\leq\;
             \sum_{s=1}^S \Pi_s [\overline y_s - \bar c(v)]/(1-\beta) \,.  \EQN bounds
$$
The bounds are depicted in Figure %15.6
\Fg{thomworrf}, which also illustrates a few
other properties of $P(v)$. Since $\lim_{c\rightarrow a}u'(c)=\infty$,
it becomes very cheap for the planner to increase the promised
utility when the current promise is very low, that is,
$\lim_{v\rightarrow -\infty}P'(v)=0$. The situation is different
when the household's promised utility is close to the upper
bound $v_{\rm max}$ where the household has a low marginal utility
of additional consumption, which implies that both
$\lim_{v\rightarrow v_{\rm max}}P'(v)=-\infty$ and
$\lim_{v\rightarrow v_{\rm max}}P(v)=-\infty$.

%%%%%%%%%%%%
%\topinsert
%$$\grafone{thomworr.eps,height=2.5in}{{\bf Figure 15.6}  Value
%function $P(v)$ and the two dashed curves depict the bounds
%on the value function. The vertical solid line indicates
%$v_{\rm max}={\rm sup}\,u(c)/(1-\beta)$.}
%$$\endinsert
%%%%%%%%%%%%%

\midfigure{thomworrf}
\centerline{\epsfxsize=3truein\epsffile{thomworr.eps}}
\caption{Value function $P(v)$ and the two dashed curves depict the bounds on the
value function. The vertical solid line indicates $v_{\rm max}={\rm sup}\,u(c)/(1-\beta)$.}
\infiglist{thomworrf}
\endfigure


\subsection{Efficiency implies $b_{s-1} \geq b_s, w_{s-1} \leq w_s$}

An incentive-compatible contract must satisfy $b_{s-1}\geq b_s$ (insurance)  and
$w_{s-1}\leq w_s$ ({\it partial\/} insurance).  This  can be established by adding the ``downward
constraint'' $C_{s,s-1}\geq 0$ and the ``upward constraint''
$C_{s-1,s}\geq 0$ to get
$$
u(\overline y_s + b_s) \,-\, u(\overline y_{s-1} + b_s) \;\geq\;
u(\overline y_s + b_{s-1}) \,-\, u(\overline y_{s-1} + b_{s-1})\,,
$$
where the concavity of $u(c)$ implies $b_s\leq b_{s-1}$.
It then follows directly from
$C_{s,s-1}\geq 0$ that $w_s \geq w_{s-1}$.
Thus, for any $v$, a household reporting a lower income receives a
higher transfer from the planner in exchange for a lower
future utility.

\subsection{Local upward and downward constraints are enough}

Constraint set \Ep{con2_asym} can be simplified. We can show
that if the local downward constraints
$C_{s,s-1}\geq 0$ and upward constraints $C_{s,s+1}\geq 0$ hold for
each $s\in {\bf S}$, then the global constraints $C_{s,k}\geq 0$
hold for each $s,k\in {\bf S}$. The argument goes as follows: Suppose
we know that the downward constraint $C_{s,k}\geq 0$ holds for some
$s>k$,
$$
u(\overline y_s + b_s) + \beta w_s \;\geq\; u(\overline y_s + b_k) + \beta w_k\,.  \EQN step1
$$
From above we know that $b_s\leq b_k$, so the concavity of $u(c)$ implies
$$
u(\overline y_{s+1} + b_s) \,-\, u(\overline y_s + b_s) \;\geq\;
u(\overline y_{s+1} + b_k) \,-\, u(\overline y_s + b_k)\,.                         \EQN step2
$$
By adding expressions \Ep{step1} and \Ep{step2} and using the local
downward constraint $C_{s+1,s}\geq 0$, we arrive at
$$
u(\overline y_{s+1} + b_{s+1}) + \beta w_{s+1} \;\geq\; u(\overline
 y_{s+1} + b_k) + \beta w_k ,
$$
that is, we have shown that the downward constraint $C_{s+1,k}\geq 0$ holds.
In this recursive fashion we can verify that all global downward
constraints are satisfied when  the local downward constraints
hold. A symmetric reasoning applies to the upward constraints. Starting
from any upward constraint $C_{k,s}\geq 0$ with $k<s$, we can show that
the local upward constraint $C_{k-1,k}\geq 0$ implies that the upward
constraint $C_{k-1,s}\geq 0$ must also hold, and so forth.


\subsection{Concavity of $P$}

Thus far, we have not appealed to the concavity of the value function,
but henceforth we shall have to.  Thomas and Worrall
showed that under condition A, $P$ is concave.
\medskip

\noindent{\sc Proposition:}
The value function $P(v)$ is concave.


\medskip\noindent
We recommend just skimming the following proof on first reading:

\medskip
\noindent{\sc Proof:}
 Let $T(P)$ be the operator associated with the right side of
equation \Ep{fe_asym}.  We could compute the optimum value function by
iterating to convergence on $T$.  We want to show that
$T$ maps strictly concave $P$ to strictly concave function
$T(P)$.   Thomas and Worrall use the following argument:

Let $P_{k-1}(v)$  be the $k-1$ iterate on $T$.
Assume that $P_{k-1}(v)$ is strictly concave. We want
to show that $P_k(v)$ is strictly concave.  Consider any $v^o$ and
$v'$ with associated contracts $(b^o_s,w^o_s)_{s \in S}, (b'_s,w'_s)_{s\in S}$.
Let $w^*_s = \delta w^o_s + (1-\delta) w_s'$ and define $b_s^*$ by
$u(b_s^*+\overline y_s)=\delta u(b^o_s+\overline y_s) + (1-\delta) u(b_s'+
\overline y_s)$ where
$\delta \in (0,1)$.   Therefore,  $(b_s^*,w_s^*)_{s\in S}$ gives the
borrower a utility that is the weighted average of the
two utilities, and gives the lender no less than the average  utility
$\delta P_k(v^o)+(1-\delta)P_k(v')$.  Then $C^*_{s,s-1} = \delta
C^o_{s,s-1} + (1-\delta) C'_{s,s-1} + [\delta u(b^o_{s-1} + \overline y_{s})
+(1-\delta) u(b'_{s-1}+\overline y_s) - u(b^*_{s-1} + \overline y_s) ]$.  Because the downward
constraints $C^o_{s,s-1}$ and $C'_{s,s-1}$ are satisfied,
  and because the third term is nonnegative
under condition A, the downward incentive
constraints $C^*_{s,s-1} \geq 0$ are satisfied.  However,
$(b_s^*,w_s^*)_{s\in S}$ may violate the upward incentive constraints.
But Thomas and Worrall use the following argument to construct a new contract from $(b_s^*,w_s^*)_{s\in S}$
that is incentive compatible and that offers both the lender and
the borrower no less utility.  Thus, keep $w_1$ fixed and reduce
$w_2$ until $C_{2,1} =0$ or $w_2=w_1$.  Then reduce $w_3$ in the
same way, and so on.  Add the constant necessary to
leave $\sum_s \Pi_s w_s$ constant. This step will not make the lender
worse off, by the concavity of $P_{k-1}(v)$.
   Now if $w_2 = w_1$, which implies $b_2^* > b_1^*$,
reduce $b_2$ until $C_{2,1} = 0$, and proceed in the same way
for $b_3$, and so on.  Since $b_s + \overline y_s > b_{s-1} + \overline
y_{s-1}$, adding a
constant
to each $b_s$ to leave $\sum_s \Pi_s b_s$ constant cannot make the
borrowers worse off.  So in this new contract, $C_{s,s-1} =0$
and $b_{s-1} \geq b_s$.  Thus, the upward constraints also hold.
Strict concavity of $P_k(v)$ then follows because it is not
possible to have both $b^o_s = b_s'$ and $w^o_s=w'_s$ for all
$s \in S$ and $v^o\not= v'$,    so the contract $(b_s^*, w_s^*)$ yields the
lender strictly more than $\delta P_{k}(v^o) + (1-\delta) P_{k}(v')$.
To complete the induction argument, note that starting
from $P_0(v) =0$, $P_1(v)$ is strictly concave.  Therefore,
$\lim_{k=\infty}P_k(v)$ is concave.
\qed
\medskip

We  now turn to some properties of the optimal allocation that
require strict concavity of the value function. Thomas and
Worrall derive these results for the finite horizon problem with
value function $P_k(v)$, which is strictly concave by the
preceding proposition. In order for us to stay with the infinite
horizon value function $P(v)$, we make the following assumption
about $\lim_{k=\infty}P_k(v)$:\NFootnote{To get  the  main result reported below that all households
become impoverished in the limit, Thomas and Worrall provide a
proof that  requires only  concavity of $P(v)$ as established in
the preceding proposition.}
\medskip
\noindent{\sc Assumption:}  The value function $P(v)$ is strictly concave.
\medskip
\noindent



\subsection{Local downward constraints always bind}

At the optimal solution, the local downward incentive constraints
always bind, while the local upward constraints never do. That is,
a household is always indifferent between reporting the truth and  reporting that its
income is actually a little lower than it is;  but it  never
wants to report that its income is  higher. To see that
the downward constraints must bind, suppose to the contrary
that $C_{k,k-1}> 0$ for some $k \in {\bf S}$. Since $b_k \leq b_{k-1}$,
it must then be the case that $w_k>w_{k-1}$. Consider changing
$\{b_s, w_s; s\in {\bf S}\}$ as follows. Keep $w_1$ fixed, and if
necessary reduce $w_2$ until $C_{2,1}=0$. Next reduce $w_3$ until
$C_{3,2}=0$, and so on, until $C_{s,s-1}= 0$ for all $s\in {\bf S}$.
(Note that any reductions cumulate when moving up the
sequence of constraints.) Thereafter, add the necessary constant to
each $w_s$ to leave the  expected value of all future promises unchanged,
$\sum_{s=1}^S \Pi_s w_s$. The new contract offers the household the
same utility and is incentive compatible because $b_s\leq b_{s-1}$ and
$C_{s,s-1}= 0$ together imply that the local upward constraint
$C_{s-1,s}\geq 0$ does not bind. At the same time,
since the mean of promised values is unchanged and the differences
$(w_s - w_{s-1})$ have either  been left the same
or reduced, the strict concavity
of the value function $P(v)$ implies that the planner's profits have
increased. That is, we have engineered a mean-preserving
{\it decrease\/} in the
spread in the continuation values $w$.  Because
$P(v)$ is strictly concave, $\sum_{s \in {\rm S}} \Pi_s P(w_s)$ rises
and therefore $P(v)$ rises.   Thus, the original contract with
a nonbinding local downward constraint could not have been an
optimal solution.

\index{coinsurance}%
\subsection{Coinsurance}

The optimal contract is characterized by {\it coinsurance}, meaning that
the household's utility and the planner's profits both increase with
a higher income realization:
$$\EQNalign{
u(\overline y_s + b_s) + \beta w_s \;&>\; u(\overline y_{s-1} + b_{s-1}) + \beta w_{s-1}
                                                             \EQN coins1 \cr
-b_s + \beta P(w_s) \;&\geq\; -b_{s-1} + \beta P(w_{s-1})\,. \EQN coins2 \cr}
$$
The higher utility of the household in expression \Ep{coins1} follows trivially
from the downward incentive-compatibility constraint $C_{s,s-1}= 0$.
Concerning the planner's profits in expression \Ep{coins2}, suppose to the
contrary that
$-b_s + \beta P(w_s) < -b_{s-1} + \beta P(w_{s-1})$. Then replacing
$(b_s, w_s)$ in the contract by $(b_{s-1}, w_{s-1})$ raises the
planner's profits but leaves the household's utility unchanged
because $C_{s,s-1}= 0$, and the change is also incentive
compatible. Thus, an optimal contract must be such that the planner's
profits weakly increase in the household's income realization.

\subsection{$P'(v)$ is a martingale}

If we let $\lambda$ and $\mu_s$, $s = 2, \ldots , S$, be Lagrange multipliers
associated with the constraints \Ep{con1_asym} and $C_{s,s-1}\geq 0$,
$s = 2, \ldots , S$, respectively,
the first-order necessary conditions with respect to $b_s$ and
$w_s$, $s\in{\bf S}$, are
%%{\eightpoint
$$\EQNalign{
& \Pi_s\Bigl[1 - \lambda \, u'(\overline y_s+b_s)\Bigr] \; =\;
  \mu_s \, u'(\overline y_s+b_s) \,-\, \mu_{s+1}\, u'(\overline y_{s+1} + b_s),
                         \hskip1cm \EQN foc1_asym \cr
&\Pi_s\Bigl[P'(w_s) \,+\, \lambda \Bigr] \; =\; \mu_{s+1} \,-\, \mu_s \,,
                                                          \EQN foc2_asym \cr}
$$
%%}%endninept
for $s\in {\bf S}$, where $\mu_1=\mu_{S+1}=0$. (There are no constraints
corresponding to $\mu_1$ and $\mu_{S+1}$.) From the envelope condition,
$$
P'(v) \;=\; -\lambda \,.                                 \EQN BS_asym
$$
Summing equation \Ep{foc2_asym} over $s \in {\bf S}$ and using
$\sum_{s=1}^S (\mu_{s+1} - \mu_s) = \mu_{S+1} - \mu_1 = 0$ and
equation \Ep{BS_asym} yields
$$\sum_{s=1}^S \Pi_s \, P'(w_s) \;=\; P'(v)\,.            \EQN martingale
$$
This equation states that $P'$ is a martingale.
\subsection{Comparison to model with commitment problem}

In the model with a commitment problem studied in section \use{sec:moneylender1}, the efficient
allocation had to satisfy equation \Ep{bind2}, i.e., $ u'(\overline
y_s + b_s) = -P'(w_s)^{-1}$. As we  explained then, this condition
sets the household's marginal rate of substitution equal to the
planner's marginal rate of transformation with respect to
transfers in the current period and continuation values in the
next period. This condition fails to hold in the present framework
with incentive-compatibility constraints associated with telling the truth.
The efficient trade-off between current consumption and a
continuation value for a household with income realization
$\overline y_s$ can not be determined without taking into
account the incentives that other households have to  report
$\overline y_s$ untruthfully in order to obtain the corresponding bundle of
current and future transfers from the planner. It is
instructive to note that equation \Ep{bind2} {\it would\/} continue to
hold in the present framework if the incentive-compatibility
constraints for truth telling were not binding. That is, set the
multipliers $\mu_s$, $s=2, \ldots, S$, equal to zero and
substitute first-order condition \Ep{foc2_asym} into
\Ep{foc1_asym} to obtain $ u'(\overline y_s + b_s) =
-P'(w_s)^{-1}$.

\subsection{Spreading continuation values}

An efficient contract requires that the promised future utility
falls (rises) when the household reports the lowest (highest)
income realization, that is, that $w_1 < v < w_S$. To show that
$w_S>v$, suppose to the contrary that $w_S \leq v$. That this assumption leads to a contradiction is established
by the following line of argument.  Since $w_S
\geq w_s$ for all $s \in {\bf S}$ and $P(v)$ is strictly concave,
equation \Ep{martingale} implies that $w_s = v$ for all $s \in
{\bf S}$. Substitution of equation \Ep{BS_asym} into equation
\Ep{foc2_asym} then yields a zero on the left side of equation
\Ep{foc2_asym}. Moreover, the right side of equation
\Ep{foc2_asym} is equal to $\mu_2$ when $s=1$ and $-\mu_S$ when
$s=S$, so we can successively unravel from the constraint set
\Ep{foc2_asym} that $\mu_s=0$ for all $s \in {\bf S}$. Turning to
equation \Ep{foc1_asym}, it follows that the marginal utility of
consumption is equalized across income realizations, $u'(\overline
y_s+b_s)=\lambda^{-1}$ for all $s \in {\bf S}$. Such consumption
smoothing requires $b_{s-1}>b_s$, but from incentive
compatibility, $w_{s-1} = w_s$ implies $b_{s-1} = b_s$, a
contradiction. We conclude that an efficient contract must have
$w_S>v$. A symmetric argument establishes $w_1 < v$.

The planner must spread out promises
to future utility because otherwise it would  be impossible to
provide any insurance in the form of contingent payments today.
Equation \Ep{martingale} describes how the planner balances the delivery of utility today versus tomorrow.
To understand this expression, consider having  the planner increase
the household's promised utility $v$ by one unit. One way of
doing so is to increase every $w_s$ by an increment $1/\beta$
while keeping every $b_s$ constant. Such a change preserves
incentive compatibility at an expected discounted cost to the
planner of $\sum_{s=1}^S \Pi_s P'(w_s)$.
 By the envelope
theorem, locally  this is as good a way to increase $v$ as any
other, and its cost is therefore equal to $P'(v)$; that is, we
obtain expression \Ep{martingale}. In other words,
given a planner's obligation to deliver utility $v$ to the agent,
it is cost-efficient
to  balance  today's contingent deliveries of goods,
$\{b_s\}$, and the bundle of future utilities, $\{w_s\}$, so
that the expected marginal cost of next period's promises,
$\sum_{s=1}^S \Pi_s P'(w_s)$, becomes equal to the marginal cost
of the current obligation, $P'(v)$. No intertemporal price
affects  this trade-off, since any interest earnings on
postponed payments are just sufficient to compensate the agent
for his own subjective rate of discounting, $(1+r)=\beta^{-1}$.

\subsection{Martingale convergence and poverty}
\index{martingale!convergence theorem}%
The martingale property  \Ep{martingale} for $P'(v)$ has an intriguing implication for the
long-run tendency of a household's promised future utility. Recall
that $\lim_{v\rightarrow -\infty}P'(v)=0$ and $\lim_{v\rightarrow
v_{\rm max}}P'(v)=-\infty$, so $P'(v)$ in expression
\Ep{martingale} is a nonpositive martingale. By a theorem of Doob
(1953, p. 324),
\auth{Doob, Joseph L.}%
$P'(v)$ then converges almost surely.  We can show that
$P'(v)$ must converge to 0, so that $v$ converges to $-\infty$.
Suppose to the contrary that $P'(v)$ converges to a nonzero
limit, which implies that $v$ converges to a finite limit. However,
this assumption contradicts our earlier result that  future $w_s$
always spread out to provide  incentives. The contradiction
is  avoided only for $v$ converging to $-\infty$; therefore, the limit
of $P'(v)$ must be zero.

The result that all households become impoverished in the limit can be
understood in terms of  the concavity of $P(v)$. First, if there were no
asymmetric information, the least expensive way of delivering lifetime
utility $v$ would be to assign the household a constant consumption stream,
given by the upper bound on the value function in expression \Ep{bounds}.
The concavity of $P(v)$ and standard intertemporal
considerations favor a time-invariant consumption stream. But the presence
of asymmetric information makes it necessary for the planner to vary
promises of future utility to induce truth telling, which is costly due
to the concavity of $P(v)$. For example,  Thomas and
Worrall pointed out that  if $S=2$, the cost of spreading $w_1$ and $w_2$ an equally small
amount $\epsilon$ on either side of their average value $\bar w$ is
approximately $-0.5 \epsilon^2 P''(\bar w)$.\NFootnote{The expected discounted
profits of providing promised values
$w_1=\bar w - \epsilon$ and
$w_2=\bar w + \epsilon$ with equal probabilities can be approximated
with a Taylor series expansion around $\bar w$
$ \sum_{s=1}^2 {1 \over 2} P(w_s) \approx
\sum_{s=1}^{2} {1 \over 2} \Bigl[ P(\bar w)
    + (w_s - \bar w)  P'(\bar w)
    + {(w_s - \bar w)^2 \over 2 } P''(\bar w) \Bigr]
= P(\bar w) + {\epsilon^2 \over 2} P''(\bar w).
$}
In general, we cannot say how this cost differs for any two
values of $\bar w$, but it follows from the properties of $P(v)$
at its endpoints that
$\lim_{v\rightarrow -\infty}P''(v)=0$, and
$\lim_{v\rightarrow v_{\rm max}}P''(v)=-\infty$.
Thus, the cost of spreading promised values goes to zero at one
endpoint and to infinity at the other endpoint.
Therefore, the concavity of
$P(v)$ and incentive compatibility considerations impart  a downward
drift to future utilities and, consequently, consumption.
That is, with private information the ideal time-invariant consumption level without private information
 is abandoned in favor of random consumption paths that are expected to be  tilted
toward the present.

One possibility is that the initial utility level $v^o$
is determined in competition between insurance
providers. If there are no costs associated with
administering contracts, $v^o$ would then be implicitly determined
by the zero-profit condition, $P(v^o)=0$. Such a contract must be enforceable because, as we have seen,
the household will almost surely eventually wish that it could  revert  to autarky. However, since
the contract is the solution to a dynamic programming
problem where the continuation of the contract is always efficient
at every date, the insurer and
the household will never mutually agree to renegotiate the contract.
\auth{Atkeson, Andrew} \auth{Lucas, Robert E., Jr.}%
\subsection{Extension to general equilibrium}
Atkeson and Lucas (1992)
provide examples of closed economies where the constrained
efficient allocation also has each household's expected utility
converging to the minimum level with probability $1$. Here the
planner chooses the incentive-compatible allocation for all agents
subject to a constraint that the total consumption handed out in
each period to the population of households cannot exceed some
constant endowment level. Households are assumed to experience
unobserved idiosyncratic taste shocks $ \epsilon$ that are i.i.d.\
over time and households. The taste shock enters multiplicatively
into preferences that take either the logarithmic form
$u(c,\epsilon)=\epsilon \log (c)$, the constant relative risk aversion
 (CRRA) form $u(c,\epsilon) = \epsilon c^{\gamma}/\gamma$,
$\gamma<1$, $\gamma \not= 0$, or the constant absolute risk
aversion (CARA) form $u(c,\epsilon) = -\epsilon \exp (-\gamma c)$,
$\gamma>0$. The assumption that the  utility function belongs to
one of these families greatly simplifies the analytics of the
evolution of the wealth distribution.
%%the convenient implication that if we know how best to allocate risk
%%among households at any one wealth level, we can scale this allocation
%%to suit any wealth level, and hence any wealth distribution.
Atkeson and Lucas show that an equilibrium  of this model yields an efficient
allocation that assigns an ever-increasing fraction of resources to an
ever-diminishing fraction of the economy's population.


\subsection{Comparison with self-insurance}

  We have just seen how in the Thomas and Worrall model, the planner
responds to the incentive problem created by
the consumer's private information by putting a
 downward tilt into temporal  consumption profiles.   It is
useful to recall how in the savings problem of
chapters \use{selfinsure} and \use{incomplete}, the
martingale  convergence theorem was used to show that
the consumption profile acquired an upward tilt coming
from the motive of the consumer to self-insure.
\index{self-insurance}
\index{martingale!convergence theorem}


\section{Insurance with unobservable storage}
\index{asymmetric information!hidden income and hidden storage}%
\index{commitment!two-sided} \auth{Allen, Franklin}%
In the spirit of an analysis of  Franklin Allen (1985), we now augment the model of the
previous section by assuming that households have access to a
technology that enables them to store nonnegative amounts of goods
at a risk-free gross return of $R>0$. The planner cannot observe
private storage. The planner can borrow and lend outside the
village at a risk-free gross interest rate that also equals $R$, so
that private and public storage yield identical rates of return.
The planner retains an advantage over households of being the
only one able to {\it borrow\/} outside of the village.

The outcome of our analysis will be to show that allowing households
to store amounts that are not observable to the planner so impedes
the planner's ability to manipulate the household's continuation
valuations that no social insurance can be supplied.  Instead,
the planner helps households overcome the nonnegativity constraint
on  households' storage by in effect allowing them to engage also
in private {\it borrowing\/} at the risk-free rate $R$, subject to natural
borrowing limits. Thus, outcomes share many features of the allocations
studied in chapters \use{selfinsure} and \use{incomplete}.

Our analysis partly follows Cole and Kocherlakota (2001),
\auth{Cole, Harold L.}%
 \auth{Kocherlakota, Narayana R.}%
who assume that a household's utility function $u(\cdot)$ is strictly
concave and twice continuously differentiable over $(0,\infty)$
with $\lim_{c\rightarrow 0}u'(c)=\infty$. The domain of $u$
is the entire real line with $u(c)=-\infty$ for
$c<0$.\NFootnote{Allowing for negative consumption while setting utility
equal to $-\infty$ is a convenient device for avoiding having to deal
with transfers that exceed the household's resources.} They also
assume that $u$ satisfies condition A above. This preference specification
allows Cole and Kocherlakota to characterize an  efficient allocation in
a finite horizon model. Their extension to an infinite horizon
involves a few other assumptions, including upper and lower
bounds on the utility function.

We retain our earlier assumption that the planner
has access to a risk-free loan market outside of the village.
Cole and Kocherlakota (2001) postulate a closed economy where the planner
is constrained to choose nonnegative amounts of storage.
Hence, our concept of feasibility  differs from theirs.



\subsection{Feasibility}

Anticipating that our characterization of efficient outcomes will
be in terms of sequences of quantities, we let the history of
a household's reported income enter as an argument in the
function specifying the planner's transfer scheme.
In period $t$, a household with an earlier
history $h_{t-1}$ and a currently reported income of
$y_t$ receives a transfer  $b_t(\{h_{t-1},y_t\})$ that can
be either positive or negative. If all households report their
incomes truthfully, the planner's time $t$ budget constraint is
$$
K_t + \sum_{h_t} \pi(h_t) b_t(h_t) \leq R K_{t-1},  \EQN CK_gov
$$
where $K_t$ is the planner's end-of-period savings (or, if
negative, borrowing) and
$\pi(h_t)$ is the unconditional probability that a household
experiences history $h_t$, which in the planner's budget
constraint equals the fraction of  households that
experience history $h_t$. Given a finite horizon with a final
period $T$, solvency of the planner requires that $K_T\geq 0$.

We use a household's history $h_t$ to index consumption
and private storage at time $t$; $c_t(h_t)\geq 0$ and $k_t(h_t)\geq 0$. The
household's resource constraint at history $h_t$ at  time $t$  is
$$
c_t(h_t) + k_t(h_t) \leq y_t(h_t) + R k_{t-1}(h_{t-1}) + b_t(h_t), \EQN CK_priv
$$
where the function for current income $y_t(h_t)$ returns the $t$th
element of the household's history $h_t$. We assume that the
household has always reported its income truthfully, so that the transfer
in period $t$ is given by $b_t(h_t)$.

Given initial conditions $K_0=k_0=0$, an allocation
$(c,k,b,K)\equiv\{c_t(h_t),$\hfil$k_t(h_t),\, b_t(h_t),\,K_t\}$
is physically feasible if inequalities \Ep{CK_gov}, \Ep{CK_priv} and $k_t(h_t)\geq 0$
are satisfied for all periods $t$ and all histories $h_t$, and $K_T \geq 0$.


\subsection{Incentive compatibility}

Since  income realizations and private storage are both unobservable,
households are free to  deviate from an allocation $(c,k,b,K)$ in two ways.
First, households can lie about their income and thereby receive
the transfer payments associated with the reported but untrue
income history. Second, households can choose different levels
of storage. Let $\Omega^T$ be the set of reporting and storage
strategies
$(\hat y, \hat k)\equiv\{\hat y_t(h_t),\,\hat k_t(h_t);
                           \hbox{\rm for all } t, h_t\}$,
where $h_t$ denotes the household's true history.

Let $\hat h_t$ denote the history of reported incomes,
$\hat h_t(h_t) = \{\hat y_1(h_1), \,
 \hat y_2(h_2),$\hfil$\ldots, \hat y_t(h_t)\}$.
With some abuse of notation, we let $y$ denote the truth-telling
strategy for which
$\hat y_t(\{h_{t-1},\,y_t\})=y_t$ for all $(t,h_{t-1})$,
and hence for which  $\hat h_t(h_t) = h_t$.

Given a transfer scheme $b$, the expected utility of following
reporting and storage strategy $(\hat y, \hat k)$ is
%{\bf XXXX Lars: we want $\hat y_t(h_t)$ in the second line below instead of
%$y_t(h_t)$?}
$$\EQNalign{
\Gamma(\hat y,\hat k;b) \equiv & \sum_{t=1}^T \beta^{t-1} \sum_{h_t} \pi(h_t) \cr
& \cdot
u\!\left(y_t(h_t) + R \hat k_{t-1}(h_{t-1}) + b_t(\hat h_t(h_t))
-\hat k_t(h_t) \right), \hskip1cm  \EQN CK_payoff  \cr}
$$
given $k_0=0$.
An allocation is incentive compatible if
$$
\Gamma(y,k;b) = \max_{(\hat y, \hat k)\in \Omega^T} \Gamma(\hat y, \hat k;b).   \EQN CK_ic
$$
An allocation that is both incentive compatible and feasible is called
an {\it incentive feasible\/} allocation. The following proposition
asserts that any incentive feasible allocation with
private storage can be attained with an alternative incentive feasible
allocation without private storage.

\medskip\noindent{\sc Proposition 1:}
Given any incentive feasible allocation $(c,k,b,K)$, there exists
another incentive feasible allocation $(c,0, b^o, K^o)$.

\medskip
\noindent{\sc Proof:} We claim that $(c,0, b^o, K^o)$ is incentive feasible
where
$$\EQNalign{
 b^o_t(h_t)  &\equiv b_t(h_t) - k_t(h_t) + R k_{t-1}(h_{t-1}),  \EQN CK_contradict1 \cr
 K^o_t & \equiv \sum_{h_t} \pi(h_t) k_t(h_t) + K_t.             \EQN CK_contradict2 \cr}
$$
Feasibility follows from the assumed feasibility of
$(c,k,b,K)$. Note also that $\Gamma(y,0; b^o) = \Gamma(y, k; b)$.  The
proof of incentive compatibility is by contradiction.  Suppose that
$(c,0, b^o, K^o)$ is not incentive compatible, i.e., that there exists
a reporting and storage strategy $(\hat y, \hat k)\in \Omega^T$ such that
$$
\Gamma(\hat y,\hat k; b^o) > \Gamma(y, 0;b^o) = \Gamma(y, k; b).   \EQN CK_contradict3
$$
After invoking expression \Ep{CK_contradict1} for
transfer payment $b^o_t(\hat h_t(h_t))$, the left side of
inequality \Ep{CK_contradict3} becomes
$$\EQNalign{
\Gamma(\hat y,\hat k;b^o) &= \sum_{t=1}^T \beta^{t-1} \sum_{h_t} \pi(h_t)
\, u\!\Bigl(y_t(h_t) + R \hat k_{t-1}(h_{t-1})
-\hat k_t(h_t)                                                                 \cr
&\hskip.5cm + \Bigl[ b_t(\hat h_t(h_t)) - k_t(\hat h_t(h_t))
         + R k_{t-1}(\hat h_{t-1}(h_{t-1}))\Bigr]\Bigr)   \cr
&=\;\Gamma(\hat y, k^*; b),                           \cr }
$$
where we have defined
$k^*_t(h_t) \equiv \hat k_t(h_t) + k_t(\hat h_t(h_t))$.
Thus, inequality
\Ep{CK_contradict3} implies that
$$
\Gamma(\hat y, k^*; b) > \Gamma(y, k;b),
$$
which contradicts the assumed incentive compatibility of $(c,k,b,K)$.
\qed
\medskip

\subsection{Efficient allocation}

An incentive feasible allocation that maximizes {\it ex ante\/} utility is
called an efficient allocation. It solves the following problem:
$$ {\rm (P1)} \hskip1.5cm
\max_{\{c,k,b,K\}} \sum_{t=1}^T \beta^{t-1} \sum_{h_t} \pi(h_t) u(c_t(h_t))
\hskip2cm \,$$
%%                                                           \EQN CK_objective $$
subject to
$$\EQNalign{&
\Gamma(y,k;b) = \max_{(\hat y, \hat k)\in \Omega^T} \Gamma(\hat y, \hat k;b) \cr
&c_t(h_t) + k_t(h_t) = y_t(h_t) + R k_{t-1}(h_{t-1}) + b_t(h_t),\hskip.5cm
                                                       \forall t, h_t      \cr
&K_t + \sum_{h_t} \pi(h_t) b_t(h_t) \leq R K_{t-1}, \hskip.5cm   \forall t      \cr
 & k_t(h_t) \geq 0,\, \hskip.5cm                   \forall t, h_t  \cr
&K_T \geq 0,  \cr
 & K_0=k_0=0. \,\cr}
$$

The incentive compatibility constraint with unobservable private storage
makes problem (P1) exceedingly difficult to solve.  To find the
efficient allocation we will adopt a guess-and-verify approach. We will
guess that the consumption allocation that solves (P1) coincides with the
optimal consumption allocation in another economic environment. For example,
we might guess that the consumption allocation that solves (P1) is the same
as in a complete markets economy with complete enforcement. A better guess might
 be the autarkic consumption allocation where each household
stores goods only for its own use, behaving according to a version of the chapter \use{selfinsure} model with a no-borrowing constraint.  Our analysis of the model without private
storage in the previous section makes the first guess doubtful.  In fact, both
guesses are wrong. What turns out to be true is the following.

\medskip\noindent{\sc Proposition 2:}
An incentive feasible allocation $(c,k,b,K)$ is efficient if and only if
$c=c^*$, where $c^*$ is the consumption allocation that solves
$$ {\rm (P2)} \hskip1.5cm
\max_{\{c\}} \sum_{t=1}^T \beta^{t-1} \sum_{h_t} \pi(h_t) u(c_t(h_t))
\hskip2cm \,$$
subject to
$$\sum_{t=1}^T R^{1-t} \left[ y_t(h_T) - c_t(h_t(h_T)) \right] \geq 0, \hskip.5cm
                                                       \forall  h_T .
$$
\medskip
%{\bf Note to Sargent: make a homework to that for finite $T$ this is
%the natural borrowing limit.}


The proposition says that the consumption allocation that solves
(P1) is the same as that in an economy where each household can borrow
or lend outside the village at the risk-free gross interest rate
$R$ subject to a solvency requirement.\NFootnote{The solvency requirement
is equivalent to the {\it natural debt limit} discussed in chapters \use{selfinsure}
and \use{incomplete}.}
Below we will provide a proof for the case of two periods ($T=2$).  We
refer readers to Cole and Kocherlakota (2001) for a general proof.

Central to the proof are the first-order conditions of
problem (P2), namely,
$$\EQNalign{
& u'(c_t(h_t)) = \beta R \sum_{s=1}^S \Pi_s
  u'\left( c_{t+1}(\{h_t, \overline y_s\})\right),\hskip.4cm
                            \forall t, h_t  \hskip.5cm  \  \EQN CK_smooth    \cr
& \sum_{t=1}^T R^{1-t} \left[ y_t(h_T) - c_t(h_t(h_T)) \right] = 0,
               \hskip.4cm   \forall h_T.  \hskip.5cm  \    \EQN CK_npv       \cr}
$$
Given the continuous, strictly concave objective function and
the compact, convex constraint set in problem (P2),
the solution $c^*$ is unique and the first-order conditions are
both necessary and sufficient.

In the efficient allocation, the planner chooses transfers that in effect relax
the nonnegativity constraint on a household's storage is
not binding, i.e., consumption smoothing condition \Ep{CK_smooth}
is satisfied. However, the optimal transfer scheme offers no
insurance across households because the present value of transfers
is zero for any history $h_T$, i.e., the net-present value condition
\Ep{CK_npv} is satisfied.

\subsection{The two period case}

In a finite horizon model, an immediate implication of the incentive
constraints is that transfers in the final period $T$ must be independent
of households' reported values of $y_T$.  In the case of two periods, we
can therefore encode permissible transfer schemes
as
$$\EQNalign{
&b_1(\overline y_s) = b_s,   \hskip.4cm   \forall s\in{\bf S},  \cr
&b_2(\{\overline y_s, \overline y_j\}) = e_s,
                       \hskip.4cm   \forall s,j\in{\bf S},  \cr}
$$
where
$b_s$ and $e_s$ denote the transfer in the first and second period,
respectively, when the household reports income $\overline y_s$ in the
first period and income $\overline y_j$ in the second period.

Following Cole and Kocherlakota (2001), we will first characterize
the solution to the modified planner's problem (P3) stated below.  It has
the same objective function as (P1) but
a larger constraint set. In particular, we enlarge the constraint
set by considering a smaller set of reporting strategies for the
households, $\Omega_R^2$. A household strategy
$(\hat y, \hat k)$ is an element of $\Omega_R^2$ if
$$\EQNalign{
\hat y_1(\overline y_s) &\in \{\overline y_{s-1},\, \overline y_s\},
                     \hskip.5cm {\rm for}\; s=2,3,\ldots,S \cr
\hat y_1(\overline y_1) &= \overline y_1.                            \cr}
$$
That is, a household can either tell the truth or lie downward
by one notch in the grid of possible income realizations. There is
no restriction on possible storage strategies.

Given $T=2$, we state problem (P3) as follows. Choose $\{b_s, e_s\}_{s=1}^S$ to maximize
$$ {\rm (P3)} \hskip1.5cm
\sum_{s=1}^S \Pi_s\left[ u(\overline y_s + b_s)
+ \beta \sum_{j=1}^S \Pi_j u(\overline y_j + e_s) \right] \hskip2cm
$$
subject to
$$\EQNalign{&
\Gamma(y,0;b) = \max_{(\hat y, \hat k)\in \Omega_R^2} \Gamma(\hat y, \hat k;b) \cr
&c_t(h_t) = y_t(h_t) + b_t(h_t),        \hskip.5cm  \forall t, h_t      \cr
&k_t(h_t) = 0,                             \hskip.5cm     \forall t, h_t      \cr
&K_t + \sum_{h_t} \pi(h_t) b_t(h_t) \leq R K_{t-1}, \hskip.5cm    \forall t    \cr
&K_2 \geq 0,       \cr
&\hbox{\rm given   }K_0=k_0=0. \,                                            \cr}
$$
Beyond the restricted strategy space $\Omega_R^2$, problem (P3) differs
from (P1) in considering only allocations that have zero private storage.
But by Proposition 1, we know that this is an innocuous restriction that
does not affect the maximized value of the objective.

Here it is useful to explain why we are first studying the contrived
problem (P3) rather than turning immediately to the real problem (P1).
Certainly problem (P3) is easier to
solve because we are exogenously restricting the households' reporting
strategies to either telling the truth or making one specific lie.
But how can knowledge of the solution to problem (P3) help us
understand problem (P1)? Well, suppose it happens that problem (P3) has a
unique solution equal to the optimal consumption allocation $c^*$ from
Proposition 2 (which will in fact turn out to be true). In that
case, it follows that $c^*$ is also the solution
to problem (P1) because of the following argument. First,
it is straightforward to verify that
$c^*$ is incentive compatible with respect to the unrestricted
set $\Omega^2$ of reporting strategies. Second, given that no better
allocation than $c^*$ can be supported with the restricted set
$\Omega_R^2$ of reporting strategies (telling the truth or making
one specific lie),
it is impossible that we can attain better outcomes  by merely
 introducing additional ways of lying.

Let us therefore first study problem (P3). In particular,
using a proof by contradiction, we now show that any allocation
$(c,0,b,K)$ that solves problem (P3) must satisfy three
conditions:\NFootnote{The proof by contradiction goes as follows. Suppose
that an allocation $(c,0,b,K)$ solves problem (P3) but violates one of our
conditions. Then we can show either that  $(c,0,b,K)$ cannot be
incentive feasible with respect to (P3) or that there exists another
incentive feasible allocation $(c^o, 0, b^o, K^o)$ that yields an even
higher {\it ex ante\/} utility than $(c,0,b,K)$.}
\medskip
\noindent (i) The aggregate resource constraint \Ep{CK_gov} holds
with equality in both periods and $K_2=0$;
\vskip.2cm
\noindent(ii) $u'(c_1(\overline y_s)) = \beta R \sum_{j=1}^S \Pi_j
  u'\left( c_{2}(\{\overline y_s, \overline y_j\})\right),\hskip.5cm  \forall s; $
\vskip.2cm
\noindent(iii) $b_s + R^{-1}e_s = 0,\hskip.5cm   \forall s.$
\vskip.3cm

\noindent
Condition (i) is easy to establish given the restricted strategy
space $\Omega^2_R$. Suppose that condition (i) is violated and
hence, some aggregate resources have not been transferred to the
households. In that case, the planner should store all unused resources
until period 2 and give them to any household who reported the highest
income in period 1. Given strategy space $\Omega^2_R$, households are
only allowed to lie downward so the proposed allocation cannot violate
the incentive constraints for truthful reporting. Also, transferring more
consumption in the last period will not lead to any private storage.
We conclude that condition (i) must hold for any solution to
problem (P3).

Next, suppose that condition (ii) is violated, i.e., for some $i\in {\bf S}$,
$$
u'(c_1(\overline y_i)) > \beta R \sum_{s=1}^S \Pi_s
  u'\left( c_{2}(\{\overline y_i, \overline y_s\})\right).       \EQN CK_contradict4
$$
(The reverse inequality is obviously inconsistent with the incentive
constraints since households are free to store goods between periods.)
We can then construct an alternative incentive feasible allocation that yields
higher {\it ex ante\/} utility as follows.
Set $K^o_1=K_1 - \epsilon \Pi_i$, $b^o_i=b_i + \epsilon$,
$e^o_i = e_i - \delta$, and choose $(\epsilon, \delta)$ such that
$$\EQNalign{
u(\overline y_i + &b_i + \epsilon) + \beta \sum_{s=1}^S \Pi_s
  u\left( \overline y_s + e_i - \delta \right)   \cr
&= u(\overline y_i + b_i) + \beta \sum_{s=1}^S \Pi_s
  u\left( \overline y_s + e_i \right),                    \EQN CK_contradict5 \cr}
$$
$$
u'(\overline y_i + b_i + \epsilon) \geq \beta R \sum_{s=1}^S \Pi_s
  u'\left( \overline y_s + e_i - \delta \right).            \EQN CK_contradict6
$$
By the envelope condition, \Ep{CK_contradict4} implies that
$\delta > R \epsilon$, so this alternative allocation frees up
resources that can be used to improve {\it ex ante\/} utility. But
we have to check that the incentive constraints are respected.
For households experiencing $\overline y_i$, the proposed
allocation is clearly incentive compatible, since their payoffs from
reporting truthfully or lying are unchanged, and condition
\Ep{CK_contradict6} ensures that they are not deviating from zero
private storage. It can be verified that a household with the
next higher income shock $\overline y_{i+1}$ would not want to lie
downward because a household with a higher
income $\overline y_{i+1}$ would not want  the proposed
loan against the future at the implied interest rate,
$\delta/\epsilon>R$, at which the lower-income household is
indifferent to the transaction. The following lemma shows this
formally.

\medskip\noindent{\sc Lemma:} Let $\epsilon$, $\delta>0$ satisfy $\delta>R\epsilon$, and
define
$$\EQNalign{
Z(m) &\equiv \max_{k\geq 0} \Bigl[u(m-k) + \beta E_y u(y + R k) \Bigr] \cr
W(m) &\equiv \max_{k\geq 0} \Bigl[u(m-k+\epsilon)
                                  + \beta E_y u (y + R k - \delta) \Bigr], \cr}
$$
where $u$ is a strictly concave function and the expectation $E_y$ is taken
with respect to a random second-period income $y$.
If $Z(m_a)=W(m_a)$ and $m_b>m_a$, then $Z(m_b)>W(m_b)$.

\medskip
\noindent{\sc Proof:} Let the unique, weakly increasing
sequence of maximizers of the savings problems $Z$ and $W$ be
denoted $k_Z(m)$ and $k_W(m)$, respectively, which are guaranteed to
exist by the strict concavity of $u$. The proof of the lemma proceeds
by contradiction. Suppose that $Z(m_b)\leq W(m_b)$.  Then by the mean
value theorem, there exists $m_c\in(m_a,m_b)$ such that
$Z'(m_c)\leq W'(m_c)$.  This implies that
$$
u'(m_c - k_Z(m_c)) \leq u'(m_c - k_W(m_c) + \epsilon).
$$
The concavity of $u$ implies that
$
0 \leq k_Z(m_c) \leq k_W(m_c) - \epsilon.
$
The weak monotonicity of $k_W$ implies that $k_W(m_b)\geq k_W(m_c)$, so
we know that $0 \leq k_W(m_b) - \epsilon$ and we can write
$$\EQNalign{
Z(m_b) &\geq u(m_b - k_W(m_b) + \epsilon) + \beta E_y u(y + R k_w(m_b) - R \epsilon) \cr
       &>    u(m_b - k_W(m_b) + \epsilon) + \beta E_y u(y + R k_w(m_b) - \delta)
        = W(m_b),                                                                \cr}
$$
which is a contradiction.
\qed

\medskip

Finally, suppose that condition (iii) is violated, i.e., for some
$i\in{\bf S}$,
$$
\Psi_s \equiv b_s + R^{-1}e_s \not= b_{s-1} + R^{-1}e_{s-1} \equiv \Psi_{s-1}.
$$
First, we can rule out $\Psi_s < \Psi_{s-1}$ because it would compel
households with income shock $\overline y_s$ in the first period to lie downward.
This is so because our condition (ii) implies that the nonnegative
storage constraint binds for neither  these households nor the
households with the lower income shock $\overline y_{s-1}$. Hence, households
with income shock $\overline y_s$ will only report truthfully if
$Z(\overline y_s + \Psi_s) \geq Z(\overline y_s + \Psi_{s-1})$, where $Z(\cdot)$ is the value of the first savings
problem defined in the lemma above. Thus, we conclude that $\Psi_s \geq \Psi_{s-1}$.

Second, we can rule out $\Psi_s > \Psi_{s-1}$ by constructing an alternative
incentive feasible allocation that attains a higher {\it ex ante\/} utility.
Compute the certainty equivalent $\tilde \Psi$ such that
$$
\Pi_s Z(\overline y_s + \tilde \Psi) + \Pi_{s-1} Z(\overline y_{s-1} + \tilde \Psi) =
\Pi_s Z(\overline y_s + \Psi_s) + \Pi_{s-1} Z(\overline y_{s-1} + \Psi_{s-1}).
$$
Then change the transfer scheme so that households reporting $\overline y_s$
or $\overline y_{s-1}$ get the same present value of transfers equal to
$\tilde \Psi$. Because of the strict concavity of the utility function,
the new scheme frees up resources that can be used to improve
{\it ex ante\/} utility. Also, the new scheme does not violate any incentive
constraints. Households with income shock $\overline y_{s-1}$ are now better
off when reporting truthfully, households with income shock $\overline y_s$
are indifferent to telling the truth, and households with income shock
$\overline y_{s+1}$ will not lie because the present value of the transfers
associated with lying has gone down. Since the planner satisfies
the aggregate resource constraint at equality in our condition (i),
we conclude that all households receive the same present value of
transfers equal to zero.


\vskip.3cm

By establishing conditions (i)--(iii), we have
in effect shown that any solution to (P3)
must satisfy equations \Ep{CK_smooth} and \Ep{CK_npv}. Thus, problem
(P3) has a unique solution $(c^*,0,b^*,K^*)$, where $c^*$ is given by
Proposition 2 and
$$\EQNalign{
 b^*_t(h_t) & = c^*_t(h_t) - y_t(h_t),  \cr
 K^*_t & = - \sum_{h_t} \pi(h_t) \sum_{j=1}^t R^{t-1} b^*_j(h_j(h_t)). \cr}
$$
Moreover, $(c^*,0,b^*,K^*)$
is incentive compatible with respect to the unrestricted strategy set
$\Omega^2$. If a household tells the truth, its consumption is optimally
smoothed. Hence, households weakly prefer to tell the truth and not store.

The proof of Proposition 2 for $T=2$ is  completed by noting that by
construction, if some allocation $(c^*,0,b^*,K^*)$
solves
(P3), and $(c^*,0,b^*,K^*)$ is incentive compatible with respect to
$\Omega^2$, then $(c^*,0,b^*,K^*)$ solves (P1). Also, since
equations \Ep{CK_smooth} and \Ep{CK_npv} fully characterize the
consumption allocation $c^*$, we have uniqueness with
respect to $c^*$ (but there exists a multitude of storage and
transfer schemes that the planner can use to implement $c^*$ in problem (P1)).


\subsection{Role of the planner}

Proposition 2 states that any allocation $(c,k,b,K)$ that solves
the planner's problem (P1) has the same consumption outcome $c=c^*$
as the solution to (P2), i.e., the market outcome when each household
can lend {\it or\/} borrow at the risk-free interest rate $R$.  This
result has both positive and negative messages about the role of the
planner.  Because households have access only to a storage technology,
the planner implements the efficient allocation by designing an elaborate
transfer scheme that effectively undoes each household's nonnegativity
constraint on storage while respecting solvency requirements.  In this
sense, the planner has an important role to play.  However, the optimal
transfer scheme offers no insurance across households and  implements only
a self-insurance scheme tantamount to a borrowing-and-lending
outcome for each household.  Thus, the planner's accomplishments as an
insurance provider are very limited.

If we had assumed that households themselves have direct access to the
credit market outside of the village, it would follow immediately that
the planner would be irrelevant, since the households could then
implement the efficient allocation themselves. Allen (1985)  first made this observation.
 Given any transfer scheme, he showed that
all households would choose to report the income that yields the highest
present value of transfers regardless of what the actual income is. In our
setting where the planner has no resources of his own, we get the zero
net present value condition for the stream of transfers to any
individual household.


\subsection{Decentralization in a closed economy}

Suppose that consumption allocation $c^*$ in Proposition 2
satisfies
$$
 \sum_{h_t} \pi(h_t) \sum_{j=1}^t R^{t-j} \left[ y_j(h_t) - c^*_j(h_j(h_t)) \right]
        \geq 0,     \hskip.4cm   \forall t.      \EQN CK_closed
$$
That is, aggregate storage is nonnegative at all dates. It follows
that the efficient allocation in Proposition 2 would then also be
the solution to a closed system where the planner has no access to
outside borrowing. Moreover,
$c^*$ can then be decentralized as the equilibrium
outcome in an incomplete markets economy where households competitively
trade consumption and risk-free one-period bonds that are available in zero
net supply in each period. Here we are assuming complete enforcement so
that households must pay off their debts in every state of the world, and
they cannot end their lives in debt.

In the decentralized equilibrium,
let $a_t(h_t)$ and $k^d_t(h_t)$ denote bond holdings and storage,
respectively, of a household indexed by its
history $h_t$. The gross interest rate on bonds between periods
$t$ and $t+1$ is denoted $1+r_t$. We claim that the efficient allocation
$(c^*,0,b^*,K^*)$ can be decentralized by recursively defining
$$\EQNalign{
r_t &\equiv R-1,                                             \EQN CK_dec1 \cr
k_t^d(h_t) &\equiv K^*_t,                                    \EQN CK_dec2 \cr
a_t(h_t) &\equiv y_t(h_t) - c^*_t(h_t) - K^*_t + R K^*_{t-1}
                 + R a_{t-1}(h_{t-1}),     \hskip1cm \        \EQN CK_dec3   \cr}
$$
with $a_0=0$. First, we verify that households are
behaving optimally. Note that we have chosen the interest rate so that
households are indifferent between lending and storing. Because we also know
that the household's consumption is smoothed at $c^*$,
we need only to check that households' budget constraints
hold with equality. By substituting \Ep{CK_dec2} into \Ep{CK_dec3},
we obtain the household's one-period budget constraint. The consolidation
of all one-period budget constraints yields
$$\EQNalign{
a_T(h_T) = & -k^d_T(h_T)
+ \sum_{t=1}^T R^{T-t} \left[ y_t(h_T) - c^*_t(h_t(h_T)) \right]\cr
&+ R^{T-1} ( k^d_0 + a_0)
= 0  \cr}
$$
where the last equality is implied by $K^*_T=K_0=a_0=0$ and
\Ep{CK_npv}. Second, we verify that the bond market clears
by summing all households' one-period budget constraints,
$$\EQNalign{
\sum_{h_t} \pi(h_t) a_t(h_t) =
\sum_{h_t} &\pi(h_t) \Bigl[y_t(h_t) - c^*_t(h_t)
 - k^d_t(h_t) \cr
&+ R k^d_{t-1}(h_{t-1}(h_t))  + R a_{t-1}(h_{t-1}(h_t))\Bigr].\cr}
$$
After invoking \Ep{CK_dec2} and the fact that $b^*_t(h_t) = c^*_t(h_t) - y_t(h_t)$,
we can rewrite this expression as
$$\EQNalign{
\sum_{h_t} \pi(h_t) a_t(h_t) = & - K^*_t + R K^*_{t-1} \cr
&-\sum_{h_t} \pi(h_t) \Bigl[b^*_t(h_t)  - R a_{t-1}(h_{t-1}(h_t))\Bigr]  \cr
= & R\, \sum_{h_{t-1}} \pi(h_{t-1}) a_{t-1}(h_{t-1})
 = 0\,, \cr}
$$
where the second equality is implied by \Ep{CK_gov} holding with equality at
the allocation $(c^*,0,b^*,K^*)$, and the last equality follows
from successive substitutions leading back to the initial condition $a_0=0$.

It is straightforward to make the reverse argument and show that if
$1+r_t=R$ for all $t$ in our incomplete markets equilibrium, then the
equilibrium consumption allocation is efficient and equal to $c^*$,
as given in Proposition 2.

Cole and Kocherlakota note that seemingly  ad hoc restrictions on
the securities available for trade are consistent with the implementation of
the efficient allocation in this setting, and they argue that
their framework provides an explicit micro foundation for
incomplete markets models such as Aiyagari's (1994) model that we studied
in chapter \use{incomplete}.

%However, recall that the competitive market
%outcome in Aiyagari's model is not efficient due to an overaccumulation
%of capital. A crucial difference between these two frameworks is that
%Cole and Kocherlakota consider a linear technology with constant returns
%to capital while Aiyagari postulates a standard production function
%with a diminishing marginal product of capital.


\section{Concluding remarks}

  The idea of using promised values as a state variable has
made it possible to use dynamic programming  to study
problems with history dependence. In this chapter we have
studied how using a promised value as a state variable helps
to study optimal risk-sharing arrangements when there are incentive
problems coming from limited enforcement or limited information.  The next several
chapters apply and extend this idea in   other contexts.
Chapter \use{socialinsurance2} discusses how to build a closed-economy, or general equilibrium,
version of our model with imperfect enforcement.
Chapter \use{uninsur1} discusses ways of designing  unemployment insurance that optimally
compromise between supplying insurance and providing incentives for unemployed workers to  search diligently.
Chapter \use{credible} uses a continuation value as a state variable to encode a government's
reputation.  Chapter
\use{wldtrade} discusses some models of contracts and government policies that have been applied to
some enforcement problems in international trade.






 %We have referred to various papers, such as Zhao (1999), that extend
%some of the models  studied in this chapter.  We also recommend Bond
%and Park (1998), who study the gradualism of some trade liberalization
%agreements between large and small countries in terms of a model where
%the agreement manipulates a country's  continuation valuation to manage
%incentive problems.   Krueger (1999) uses a model with participation
%constraints and Arrow securities to confront some empirical puzzles
%about consumption.  Chapter \use{credible} uses dynamic programming with
%promised values as state variables to study ``reputational  macroeconomics.''
%\auth{Bond, Eric W.}
%\auth{Park, Jee-Hyeong}
%\auth{Krueger, Dirk}


% The second describes
%a possible  computational strategy   for Atkeson's model.






\appendix{A}{Historical development}

\subsection{Spear and Srivastava}
Spear and Srivastava (1987) introduced the following
recursive formulation of an infinitely repeated, discounted  repeated
principal-agent problem:  A {\it principal\/} owns a technology
that produces output $q_t$ at time $t$, where $q_t$ is determined
by a family of c.d.f.'s  $F(q_t\vert a_t)$, and $a_t$ is an
action taken at the beginning of $t$ by an {\it agent\/} who
operates the technology.  The principal has access to an outside
loan market with constant risk-free gross interest rate $\beta^{-1}$.
The agent has preferences over consumption streams ordered by
\index{repeated principal-agent problem}%
$E_0 \sum^\infty_{t=0} \beta^t u(c_t, a_t).$
The principal is risk neutral and offers a contract to
the agent designed to maximize
$E_0 \sum^\infty_{t=0} \beta^t \{q_t - c_t\}$
where $c_t$ is the principal's payment to the agent at $t$.

\auth{Spear, Stephen E.}   \auth{Srivastava, Sanjay}
\subsection{Timing}
Let $w$ denote the discounted utility promised to the agent
at the beginning of the period.  Given $w$, the principal
selects three functions $a(w)$, $c(w,q)$, and $\tilde w(w,q)$
determining the current action $a_t=a(w_t)$,
the current consumption $c=c(w_t, q_t)$, and a promised
utility $w_{t+1} = \tilde w (w_t, q_t)$.
The choice of the three functions $a(w)$, $c(w,q)$, and $\tilde w (w,q)$
must satisfy the following two sets of constraints:
$$w = \int \{ u[c(w,q), a(w)] + \beta \tilde w(w,q)\}\
dF[q\vert a(w)] \EQN 1$$
and
$$\EQNalign{\int &\{ u[c(w,q), a(w)] + \beta\tilde w (w,q)\}\
dF[q\vert a(w)]\cr &\geq \int \{u [c(w,q),\hat a] + \beta\tilde w
(w,q)\} dF(q\vert\hat a)\,, \hskip.5cm \forall\; \hat a \in A. \EQN 2\cr}$$
Equation \Ep{1} requires the contract to deliver the promised
level of discounted utility. Equation \Ep{2} is the {\it incentive
compatibility} constraint requiring the agent to want to
deliver the amount of effort called for in the
contract. \index{incentive-compatibility constraint}%
Let $v(w)$ be the value to the principal associated with promising discounted utility $w$ to the agent.  The principal's Bellman equation is
$$v(w) =\max_{a,c,\tilde w}\ \{q-c(w,q)+\beta \ v[\tilde w(w,q)]\}\
dF[q\vert a(w)]\EQN 3$$
where the maximization is over functions $a(w)$, $c(w,q)$, and $\tilde w(w,q)$
and is subject to the constraints \Ep{1} and \Ep{2}.
This value function $v(w)$ and the associated optimum policy functions
are to be solved by iterating on the Bellman equation \Ep{3}.

\subsection{Use of lotteries}
In various implementations of this approach,
a difficulty can be that the constraint set fails to
be convex as a consequence of the structure of the incentive
constraints.  This problem has been overcome by Phelan and
Townsend (1991) by convexifying the constraint set through \idx{randomization}.
Thus, Phelan and Townsend simplify the problem by extending the
principal's choice to the space of lotteries
over actions $a$ and outcomes $c,w'$.
To introduce Phelan and Townsend's formulation, let $P(q\vert a)$ be
a family of discrete  probability distributions
over discrete spaces of outputs and actions $Q,A$, and
imagine that consumption and values are also constrained to
lie in discrete spaces $C,W$, respectively.
Phelan and Townsend instruct  the principal to
choose a probability distribution
 $\Pi(a,q,c,w^\prime)$ subject first to the constraint
 that for all fixed $(\bar a, \bar q)$
%
\auth{Phelan, Christopher}  \auth{Townsend, Robert M.}%
%
$$\sum_{C\times W} \Pi (\bar a, \bar q, c, w^\prime) = P (\bar q\vert \bar a)
\sum_{Q\times C\times W}\ \Pi(\bar a, q,c,w') \EQN town1;a$$
$$\Pi(a,q,c,w')\geq 0 \EQN town1;b$$
$$\sum_{A\times Q\times C\times W}\ \Pi(a,q,c,w^\prime)=1 .\EQN town1;c$$
Equation \Ep{town1;a} simply states that
${\rm Prob} (\bar a, \bar q) = {\rm Prob}(\bar q \vert \bar a)
{\rm Prob}(\bar a)$.
 The remaining pieces of \Ep{town1} just
require that ``probabilities are probabilities.''
The counterpart of Spear-Srivastava's equation \Ep{1} is
$$w=\sum_{A\times Q\times C\times W}\ \{u(c,a) +\beta w^\prime\}\
\Pi(a,q,c,w^\prime) . \EQN 1'$$
The counterpart to Spear-Srivastava's equation \Ep{2}  for each
$a,\hat a$ is
$$\eqalign{\sum_{Q\times C\times W}\ &\{u(c,a) + \beta w^\prime\}\
\Pi (c,w^\prime\vert q, a) P(q\vert a)\cr &\geq \sum_{Q\times C\times W}
\ \{u(c,\hat a) + \beta w^\prime\}\ \Pi(c,w^\prime\vert q,a) P(q\vert\hat a).\cr}$$
%
Here $\Pi(c,w^\prime\vert q,a) P(q\vert \hat a)$ is the probability of $(c,w^\prime, q)$ if the agent claims to be working $a$ but is actually
working $\hat a$.  Express
$$\eqalign{\Pi(c,w^\prime\vert q,a) P(q\vert\hat a)&=\cr\noalign{\smallskip}
\Pi(c,w^\prime\vert q,a) P(q\vert a)\ {P(q\vert\hat a)\over P(q\vert a)} &=
\Pi(c,w^\prime,q\vert a)\ \cdot\ {P(q\vert\hat a)\over P(q\vert a)}.\cr}$$
To write the incentive constraint as
$$\eqalign{\sum_{Q\times C\times W}\ &\{u(c,a)
 +\beta w^\prime\} \Pi(c,w^\prime, q\vert a)\cr &\geq
\sum_{Q\times C\times W}\ \{u(c,\hat a) +\beta w^\prime\}\
\Pi(c,w^\prime, q\vert \hat a)\
    \cdot\ {P(q\vert \hat a)\over P(q\vert a)}.\cr}$$
Multiplying both sides by the unconditional probability $P(a)$ gives expression \Ep{2'}.
$$\EQNalign{& \sum_{Q\times C\times W}\ \{u(c,a)+\beta w^\prime\}\
\Pi(a,q,c,w^\prime)\cr \noalign{\smallskip}
&\geq \sum_{Q\times C\times W}\ \{u(c,\hat a) + \beta w^\prime\} \
{P(q\vert\hat a)\over P(q\vert a)}\ \Pi (a,q,c,w^\prime)
\EQN 2'\cr}$$
The Bellman equation for the principal's problem is
$$ v(w) =\max_{\Pi} \{(q -c) +
     \beta v(w')\} \Pi(a,q,c,w') , \EQN  bell2 $$
where the maximization is over the probabilities $\Pi(a,q,c,w')$
subject to equations \Ep{town1}, \Ep{1'}, and \Ep{2'}.
The problem on the right side of equation \Ep{bell2} is a linear
programming problem.  Think of each of $(a,q,c,w')$
being constrained to a discrete grid of points.  Then, for example,
the term $(q-c)+\beta v(w')$ on the right side of equation \Ep{bell2}
can be represented as a {\it fixed} vector that multiplies a vectorized
version of  the
probabilities $\Pi(a,q,c,w')$.  Similarly, each of the
constraints \Ep{town1}, \Ep{1'}, and \Ep{2'} can be represented
as a linear inequality in the choice variables, the
probabilities $\Pi$.   Phelan and Townsend compute solutions
of these linear programs to
iterate on the Bellman equation \Ep{bell2}.   Note that
at each step of the iteration on the  Bellman equation,
there is  one linear program to be solved for each point
$w$ in the space of grid values for $W$.

   In practice, Phelan and Townsend have found that
lotteries are often redundant in the sense that most of the
$\Pi(a,q,c,w')$'s  are  zero, and a few are $1$.


%\section{Exercises}
\showchaptIDfalse
\showsectIDfalse
\section{Exercises}
\showchaptIDtrue
\showsectIDtrue

\medskip
\noindent{\it Exercise  \the\chapternum.1}  \quad {\bf Thomas and Worrall meet Markov}
\medskip
\noindent  A  household orders sequences $\{c_t\}_{t=0}^\infty$
 by
$$E \sum_{t=0}^\infty \beta^t u(c_t), \hskip1.5cm \beta \in (0,1) $$ where
$u$ is strictly increasing, twice continuously differentiable, and
strictly concave with $u'(0) = +\infty$. The good is nondurable. The household
receives an endowment of the   consumption good of
$y_t$ that obeys a discrete-state Markov chain with
$P_{ij} = {\rm Prob}(y_{t+1} = \overline y_j | y_t =
\overline y_i )$, where the endowment $y_t$ can take one of the
$I$ values $[\overline y_1, \ldots, \overline y_I]$.

\medskip
\noindent{\bf a.}  Conditional on having  observed the time $t$ value
of the household's endowment, a social insurer  wants to deliver expected
discounted utility $v$ to the household in the least costly way.  The
insurer observes $y_t$ at the beginning of every period,  and
contingent on the observed history of those endowments, can make a transfer
$\tau_t$ to the household.  The transfer can be positive or negative
and can be enforced without cost.   Let $C(v,i)$ be the minimum
expected discounted
cost to the insurance
agency of delivering promised discounted utility   $v$ when the household
has just received endowment $\overline y_i$.  (Let the insurer discount
with factor $\beta$.)
Write a Bellman equation for $C(v,i)$.
\medskip
\noindent{\bf b.}  Characterize the    consumption plan and the transfer
plan that attains $C(v,i)$;  find an associated law of motion
for promised discounted value.
\medskip
\noindent{\bf c.}  Now assume that the household is isolated and has
no access to insurance.  Let $v^a(i)$ be the expected discounted
value of utility for a household in autarky, conditional
on current income being $\overline y_i$.  Formulate  Bellman
equations for $v^a(i), i=1, \ldots, I$.
\medskip
\noindent{\bf d.}  Now return to the problem of the insurer mentioned
in part b, but assume that the insurer cannot enforce  transfers because
each period the consumer is free to walk away from the insurer and
live in autarky thereafter.  The insurer must structure a history-dependent
transfer  scheme that prevents the  household from ever exercising the
option to revert to autarky.  Again, let $C(v,i)$ be
the minimum cost for an insurer that wants to deliver promised
discounted utility $v$  to a household with current endowment
$i$.  Formulate  Bellman equations for $C(v,i), i=1, \ldots, I$.
Briefly discuss the form of the law of motion for $v$ associated
with the minimum cost insurance scheme.

\medskip
\medskip
\noindent{\it  Exercise \the\chapternum.2} \quad {\bf Wealth dynamics in moneylender
model}
\medskip
\noindent
Consider the model in the text of the village with a moneylender.
The village consists of a large number (e.g., a continuum) of
households, each of which has an i.i.d.\ endowment process
that is distributed as
$$ {\rm Prob} (y_t = \overline y_s) = {1-\lambda\over 1-\lambda^S }
 \lambda^{s-1}  $$
where $\lambda \in (0,1)$ and $\overline y_s = s+5$ is
the $s$th  possible endowment value, $s=1, \ldots, S$.
Let $\beta\in (0,1)$ be the discount factor and
$\beta^{-1}$ the gross rate of return at which the moneylender
 can borrow or lend. The typical household's
one-period utility function is
$u(c) = (1-\gamma)^{-1} c^{1-\gamma}$, where $\gamma$ is the
household's coefficient of relative risk aversion.
Assume the parameter values $(\beta, S, \gamma, \lambda)
= (.5, 20, 2, .95)$.
%{\bf Tom XXXX: I changed $\beta $ from .95 to .5 following Isaac's advice. Double check my matlab programs to verify that $\beta=.5$ in the programs.}


\medskip
\noindent {\it Hint:}  The formulas given in the section \use{recursive_comp_cont}
%%TTTTTTT `Recursive
%computation of the optimal contract'
will be helpful in answering the following questions.
\medskip
\noindent{\bf a.} Using Matlab,    compute the optimal   contract that the
moneylender offers a villager, assuming that the contract
leaves the villager indifferent between refusing and accepting the
contract.
\medskip
\noindent{\bf b.}  Compute the expected profits   that
the moneylender earns by offering this contract for an initial
discounted utility  that equals the one that the household
would receive in autarky.

\medskip
\noindent{\bf c.}  Let the cross-section distribution
of consumption at time $t \geq 0$   be
given by the c.d.f.\ ${\rm Prob ( c_t \leq \overline C)}
= F_t (\overline C)$.  Compute $F_t$.  Plot it for
$t=0$, $t=5$, $t=10$, $t=500$.

\medskip
\noindent{\bf d.} Compute the  moneylender's  savings for $t \geq 0$
and plot it for $t=0, \ldots, 100$.

\medskip
\noindent{\bf e.}  Now adapt your program to find the
initial level of promised utility $v > v_{\rm aut}$ that would
set $P(v)=0$.
 % {\it Hint:} Think of an iterative algorithm
%to solve $P(v)=0$.

\medskip

\medskip
\noindent{\it  Exercise \the\chapternum.3} \quad {\bf Thomas and Worrall (1988)}
\medskip
\noindent There is a competitive spot market for labor always
available to each of a continuum of workers. Each worker is
endowed with one unit of labor each period that he supplies
inelastically to work either permanently for "the company" or each
period in  a new one-period job in the spot labor market. The
worker's  productivity in either the  spot labor market or with
the company is an i.i.d.\ endowment process that is distributed as
$$ {\rm Prob} (w_t = \overline w_s) = {1-\lambda\over 1-\lambda^S }
 \lambda^{s-1}  $$
where $\lambda \in (0,1)$ and $\overline w_s = s+5$ is the $s$th
possible marginal product realization, $s=1, \ldots, S$. In the
spot market, the worker is paid $w_t$.  In the company, the worker
is offered a history-dependent payment $\omega_t = f_t(h_t)$ where
$h_t = w_t, \ldots, w_0$. Let $\beta\in (0,1)$ be the discount
factor and $\beta^{-1}$ the gross rate of return at which the
company can borrow or lend. The worker cannot borrow or lend. The
worker's one-period utility function is $u(\omega) =
(1-\gamma)^{-1} w^{1-\gamma}$ where $\omega$ is the  period wage
from the company, which equals   consumption, and  $\gamma$ is the
worker's coefficient of relative risk aversion. Assume the
parameter values $(\beta, S, \gamma, \lambda) = (.5, 20, 2,
.95)$. % {\bf Tom XXXX: again, I changed $\beta$ from .95 to .5, following Isaac's advice. Check matlab programs.}


The company's discounted expected profits
are
$$ E \sum_{t=0}^\infty \beta^t \left(w_t - \omega_t\right).$$ %% \EQN twprof $$
The worker is free to walk away from the company at the start of
any period, but must then stay in the spot labor market forever.
In the spot labor market, the worker receives continuation value
$$ v_{\rm spot} = {E  u(w) \over 1 -\beta}.  $$
The company designs a history-dependent compensation contract
that must be sustainable (i.e., self-enforcing) in the face of
the worker's freedom to enter the spot labor market at
the beginning of period $t$ {\it after\/} he has observed
$w_t$ but before he receives the $t$ period wage.
\medskip
\noindent {\it Hint:}  Do these questions  ring a bell? See
exercise {\it \the\chapternum.2\/}.

\medskip
\noindent{\bf a.} Using Matlab, compute the optimal contract that the
company offers the worker, assuming that the contract leaves the worker
indifferent between refusing and accepting the contract.

\medskip
\noindent{\bf b.}  Compute the expected profits that the firm earns by
offering this contract for an initial discounted utility that equals the
one that the worker would receive by remaining forever in the spot market.

\medskip
\noindent{\bf c.}  Let the distribution of wages that the firm offers to
its workers  at time $t \geq 0$ be given by the c.d.f.\
${\rm Prob ( \omega_t \leq \overline w)}
= F_t (\overline w)$.  Compute $F_t$.  Plot it for
$t=0$, $t=5$, $t=10$, $t=500$.

\medskip
\noindent{\bf d.}  Plot an expected wage-tenure profile
for a new worker.

\medskip
\noindent{\bf e.}  Now assume that there is competition among
companies and free entry.  New companies enter by competing for
workers by raising initial promised utility with the company.
Adapt your program to find the initial level of promised utility
$v > v_{\rm spot}$ that would set expected profits from the
average worker $P(v)=0$.

%\medskip
%\noindent{\it  Exercise \the\chapternum.4} \quad {\bf Cole and Kocherlakota (2001)}
%\medskip
%\noindent Consider a {\bf closed version} of our two-period model
%($T=2$) based on Cole and Kocherlakota's (2001) framework, where
%the planner has no access to outside borrowing. In this economy,
%suppose that an incomplete markets equilibrium would give rise to
%an interest rate on bonds equal to $1+r>\beta^{-1}$. Show that
%this decentralized outcome is inefficient. That is, show that
%there exists an incentive feasible allocation that yields a higher
%{\it ex ante} utility than the decentralized outcome.

%%YYYYY
\medskip
\medskip
\noindent {\it Exercise \the\chapternum.4}  \quad
{\bf Thomas-Worrall meet Phelan-Townsend }
\medskip
\noindent Consider the Thomas Worrall environment and denote
$\Pi(y)$ the density of the i.i.d.\ endowment process, where $y$ belongs
to the discrete set of endowment levels $Y = [\overline y_1, \ldots, \overline y_S]$.
The one-period utility function is
$ u(c) = (1-\gamma)^{-1} (c-a)^{1-\gamma}$ where
$\gamma > 1$ and $ \overline y_S > a >0$.

Discretize the set of transfers $B$ and the set of continuation values $W$.
We assume that the discrete set $B \subset (a - \overline y_S, \overline b]$.
Notice that with the one-period utility function above, the planner could
never extract more than $a - \overline y_S$ from the agent.
Denote $\Pi^v(b,w | y)$ the joint density over $(b,w)$ that the  planner
offers the agent who reports $y$ and to whom
he has offered beginning-of-period promised value  $v$.
For each $y \in Y$ and each $v \in W$,
the planner chooses a set of conditional probabilities
$\Pi^v(b,w |y)$ to satisfy the Bellman equation
$$ P(v) = \max_{\Pi^v(b,w,y)}  \sum_{B  \times W \times Y}
     \left[ -b  + \beta P(w) \right] \Pi^v(b,w,y) \eqno(1) $$
subject to the following constraints:
$$\eqalignno{& v   = \sum_{B \times W \times Y} \left[u(y+b) + \beta w\right]
      \Pi^v(b,w,y)   & (2) \cr
 &  \sum_{B \times W}  \left[ u(y+b) + \beta w\right] \Pi^v(b,w|y)
  \geq
  \sum_{B \times W} \left[ u(y+b) + \beta w\right] \Pi^v(b,w|\tilde y)
   \cr
  &  \hskip4cm \ \ \forall (y, \tilde y) \in Y \times Y & (3) \cr
 &  \Pi^v(b,w,y)   = \Pi(y) \Pi^v(b,w|y)  \hskip1cm \ \forall (b,w,y)   \in
   B \times W \times Y & (4) \cr
  &  \sum_{B \times W \times Y} \Pi^v(b,w,y)   = 1 .& (5) \cr}$$
%%$$\EQNalign{& v   = \sum_{B \times W \times Y} \left[u(y+b) + \beta w\right]
%%      \Pi^v(b,w,y)   \EQN tw2  \cr
%% &  \sum_{B \times W}  \left[ u(y+b) + \beta w\right] \Pi^v(b,w|y)
%%  \geq
%%  \sum_{B \times W} \left[ u(y+b) + \beta w\right] \Pi^v(b,w|\tilde y)
%%   \cr
%%  &  \ \ \forall (y, \tilde y) \in Y \times Y \EQN tw3 \cr
%% &  \Pi^v(b,w,y)   = \Pi(y) \Pi^v(b,w|y)  \ \ \forall (b,w,y)   \in
%%   B \times W \times Y \EQN tw4 \cr
%%  &  \sum_{B \times W \times Y} \Pi^v(b,w,y)   = 1 .\EQN tw5 \cr}$$
Here (2) is the promise-keeping constraint,
 (3)  are the truth-telling constraints,
and (4), (5) are restrictions  imposed by the laws
of probability.

\medskip
\noindent{\bf a.}  Verify that given $P(w)$,
one step on the Bellman equation is a linear programming problem.

\medskip
\noindent {\bf b.}  Set $\beta=.94, a = 5, \gamma=3$.
  Let $S, N_B,  N_W$ be the number of points in
the grids for $Y, B, W$, respectively.  Set $S=10$,
$N_B = N_W = 25$.  Set $Y =
\left[\matrix{6 & 7 & \ldots 15\cr}\right]$, ${\rm Prob}(y_t =\overline y_s)
= S^{-1}$.   Set $W = [ w_{\rm min}, \ldots, w_{\rm max}]$
  and
$B  = [b_{\rm min}, \ldots, b_{\rm max}]$, where the intermediate
points in $W$ and $B$, respectively, are equally spaced.
Please set $w_{\rm min}= {1 \over 1-\beta} {1 \over 1 - \gamma}
  \left( y_{\rm min}-a \right)^{1-\gamma}$ and $w_{\rm max}
=  w_{\rm min}/20 $ (these are negative numbers, so $w_{\rm min}
  < w_{\rm max}$).   Also set $b_{\rm min} = (1-y_{\rm max} + .33)$
and $b_{\rm max} = y_{\rm max} - y_{\rm min}$.

\medskip

For these parameter values, compute the optimal contract by formulating
a linear program for one step on the Bellman equation,
then iterating to convergence on it.

\medskip
\noindent{\bf c.}  Notice the following probability laws:
$$ \EQNalign{ {\rm Prob}(b_t, w_{t+1}, y_t | w_t ) & \equiv \Pi^{w_t}( b_t,
  w_{t+1},y_t) \cr   %%\EQN tw7;a \cr
  {\rm Prob}(w_{t+1} | w_t) & = \sum_{b\in B, y \in Y } \Pi^{w_t} (b,
  w_{t+1}, y) \cr    %%\EQN tw7;b \cr
  {\rm Prob}(b_t, y_t | w_t) & = \sum_{w_{t+1} \in W}
   \Pi^{w_t}(b_t, w_{t+1}, y_t).
  \cr}  %%\EQN tw7;c \cr }
$$
Please use these and other probability  laws
to compute ${\rm Prob} (w_{t+1} | w_t)$. Show how to compute
$ {\rm Prob} (c_t)$, assuming a given initial promised value $w_0$.
\medskip
\noindent{\bf d.}  Assume that $w_0 \approx -2$.  Compute and plot
$F_t(c) = {\rm Prob}(c_t \leq c)$ for $t=1, 5, 10, 100$.
Qualitatively, how do these distributions  compare with those for
the simple village and moneylender model with no information problem
and one-sided lack of commitment?

\medskip
\noindent{\it Exercise  \the\chapternum.5}
\quad {\bf The IMF}
\medskip\noindent Consider the problem of a government of a small
country  that has to finance an exogenous stream of expenditures
$\{g_t\}$. For time $t \geq 0$, $g_t$ is i.i.d.\ with ${\rm
Prob}(g_t = \overline g_s) = \pi_s$ where $\pi_s >0, \sum_{s=1}^S
\pi_s =1$ and $0 < \overline g_1 < \cdots < \overline g_S$.
Raising revenues by taxation is distorting. In fact, the
government confronts a deadweight loss function $W(T_t)$ that
measures the distortion at time $t$. Assume that $W$ is an
increasing, twice continuously differentiable, strictly convex
function that satisfies $W(0) =0, W'(0) = 0 , W'(T) > 0$ for $T
>0$ and $W''(T)
> 0$ for $T \geq 0$. The government's intertemporal loss function for taxes
 is such that it wants  to minimize
$$ E_{-1} \sum_{t=0}^\infty  \beta^t W(T_t), \quad \beta \in (0,1) $$  %%\leqno(1) $$
where $E_{-1}$ is the mathematical expectation before $g_0$ is
realized. If it cannot borrow or lend, the government's budget
constraint is $g_t = T_t$. In fact, the government is unable to
borrow and lend {\it except\/} through an international coalition
of lenders called the IMF.  If it does not have an arrangement
with the IMF, the country is in autarky and the government's loss
is the value
$$ v_{\rm aut} =E \sum_{t=0}^\infty \beta^t W(g_t).$$

The IMF itself is able to borrow and lend at a constant risk-free
gross rate of interest of $R = \beta^{-1}$.  The IMF offers the
country a contract that gives the country a net transfer of $g_t -
T_t$. A {\it contract\/} is a sequence of functions for $t\geq 0$,
the time $t$ component of which maps the history $g^t$ into a net
transfer $g-T_t$. The IMF has the ability to commit to the
contract. However, the country cannot commit to honor the
contract. Instead, at the beginning of each period, after $g_t$
has been realized but before the net transfer $g_t-T_t$ has been
received, the government can default on the contract, in which
case it receives loss $W(g_t)$ this period and the autarky value
ever after.  A contract is said to be {\it sustainable\/} if it is
immune to the threat of repudiation, i.e., if it  provides the
country with the incentive not to leave the arrangement with the
IMF.  The present value of the contract to the IMF is
$$ E \sum_{t=0}^\infty \beta^t (T_t - g_t)   .$$

\medskip
\noindent{\bf a.}  Write a Bellman equation that can be used to
find an optimal sustainable contract.
\medskip
\noindent{\bf b.}  Characterize an optimal  sustainable contract
that delivers initial promised value $v_{\rm aut}$ to the country
(i.e., a contract that renders the country indifferent between
accepting and not accepting the IMF contract starting from
autarky).

\medskip
\noindent{\bf c.}  Can you say anything about a typical pattern of
government tax collections $T_t$  and distortions $W(T_t)$ over
time for a country in an optimal sustainable contract with the
IMF? What about the average pattern of government surpluses
$T_t-g_t$ across a panel of countries with identical $g_t$
processes and $W$ functions?  Would there be a ``cohort'' effect in
such a panel (i.e., would the calendar date when the country
signed up with the IMF matter)?

\medskip

\noindent{\bf d.} If the optimal sustainable contract gives the
country value $v_{\rm aut}$, can the IMF expect to earn anything
from the contract?



%\vfill \eject
%%%%%%%%%%%%%%%%%%%%%%%%%%%%%%%%%%%%%%%%%%%%%%%%%%%%%%%%%%%
