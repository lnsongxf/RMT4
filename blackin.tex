
%\showchapterIDtrue
%\chapter{Preface\label{preface}}
\footnum=0
\
%\showchapterIDtrue


%\def\@chaptID{1.}
%\chap{-23}{0}{Preface}
%\pagenumbers
%\pageno=-1
%\noindent{\Tbf Preface}
%\chapter{Preface\label{preface}}
%\medskip
\nosechead{Recursive Methods}
\noindent Much of this book is about how to use recursive methods to study dynamic
macro\-economic models. Recursive methods are very important in the
analysis of dynamic systems in economics and other sciences. They
originated after World War II in diverse literatures promoted by
Wald (sequential analysis), Bellman (dynamic programming), and
Kalman (Kalman filtering). \auth{Kalman, Rudolf} \auth{Bellman,
Richard} \auth{Wald, Abraham}
\nosechead{Dynamics}
\noindent Dynamics studies sequences  of vectors of random variables
indexed by time, called {\it time series}.     Time series
are immense objects, with as many components as the
number  of variables times the number of time periods.
A dynamic economic
model  characterizes and
interprets covariations  among   of all of these components
in terms of the purposes and opportunities  of economic agents.
Agents {\it choose\/} components of the time series in light
of their opinions about other components.

Recursive methods   break a dynamic problem
into pieces by forming a sequence  of problems, each one
being a constrained choice between utility today and utility  tomorrow.
The idea is to find a way to describe the position
of the system now, where it might be tomorrow, and
how agents care now about where  it is tomorrow. Thus,
recursive methods study dynamics indirectly by characterizing a pair
of {\it functions}: a transition function mapping the {\it state} today into the state tomorrow,
 and another function mapping the state into
the other endogenous variables of the model.
The {\it state\/} is a vector of variables that characterizes
the system's current position.  Time series
are generated from these objects
by iterating  transition laws.
%Only experience from
%solving practical problems
%fully conveys
%the power of  this construction.     This book provides
%many applications.

%\vfil\eject
%\nosechead{Recursive approach}

Recursive methods  focus on a tradeoff between the current period's
utility and a continuation value for utility in all future periods and
the evolution of
state variables that capture all consequences of today's actions and
events.  Half of
the job  is accomplished
once we choose and understand the roles of suitable  state variables.


Another reason for learning about  recursive methods  is
the increased importance of numerical simulations in
macroeconomics. Many computational algorithms use
recursive methods. When such numerical simulations are called for
in this book, we give some suggestions for how to proceed but rely on other sources to
provide important details.\NFootnote{Judd
(1998) and Miranda and Fackler (2002) provide good treatments of
numerical methods in economics.}\auth{Judd, Kenneth L.}
\auth{Miranda, Mario J.}\auth{Fackler, Paul L.}



\nosechead{Philosophy}\noindent
We think that only experience from
solving practical problems
fully conveys
the power of  the recursive approach.     Therefore, this book provides
many applications.
 The book mixes tools and   applications.      We  present the tools with just enough technical
sophistication
for our applications, but little more.         We aim
to give readers a taste of the power of the methods and
to direct them to sources where  they can learn more.

  Macroeconomic dynamics is now an immense field with
diverse applications.  We do not pretend to survey the field, only
to sample it.  We intend our sample to equip  the reader to
approach much of the field with confidence. Fortunately for us,
  good  books cover parts of the field
that we neglect, for example, Adda and Cooper (2003), Aghion and Howitt (1998), Altug and Labadie
(1994), Azariadis (1993), Barro and
Sala-i-Martin (1995), Benassy (2011),  Blanchard and Fischer (1989),  Christensen and  Kiefer (2009), Canova (2007), Cooley (1995),  Cooper (1999),
DeJong and Dave (2011), Farmer (1993),  Gali (2008), Hansen and Sargent (2013),
Majumdar (2009), Pissarides (1990),
Romer (1996),  Shimer (2010), Stachurski (2009),  Walsh (1998),
and   Woodford (2000).   Bertsekas (1976) and Stokey and  Lucas with
Prescott (1989)  remain standard references
for recursive methods in macro\-economics. %Chapters \use{search1}
%and
Technical  Appendix \use{functional}  in this book revises material
from  chapter 2 of Sargent (1987b). \auth{Farmer, Roger}
\auth{Woodford, Michael} \auth{Azariadis, Costas} \auth{Romer,
David} \auth{Fischer, Stanley} \auth{Blanchard, Olivier J.}
\auth{Cooper, Russell} \auth{Walsh, Carl} \auth{Prescott, Edward
C.} \auth{Stokey, Nancy L.} \auth{Lucas, Robert E.,
Jr.}\auth{Labadie, Pamela} \auth{Altug, Sumru}\auth{Howitt, Peter} \auth{Aghion, Philippe}\auth{Farmer,
Roger} \auth{Cooper, Russell} \auth{Adda, Jerome}
\auth{Benassy, Jean-Pascal}
\auth{Canova, Fabio}
\auth{Shimer, Robert}
\auth{Majumdar, Mukul}
\auth{Christensen, Bent Jasper}
\auth{Kiefer,  Nicholas M.}
\auth{Stachurski, John}
\auth{DeJong, David}
\auth{Dave, Chetan}
\auth{Gali, Jordi}
\auth{Hansen, Lars P.}



\nosechead{Changes in the fourth edition} \noindent
This edition
contains two  new chapters and substantial revisions of many
other chapters from earlier editions.  New to this edition
are chapter
\use{optaxrecur} on recursive formulations of optimal taxation problems and chapter \use{mechanics_matching}
about the structure underlying models with ``matching functions'' that map unemployment and vacancies into
job-finding and job-filling probabilities. Chapter \use{stackel} has been extensively revised and simplified in ways
that closely link its formulation of Stackelberg plans   to  one in our  new chapter \use{optaxrecur}.
% \use{overview}, \use{linappro},
%\use{growth1}, \use{stackel}, \use{socialinsurance2}, \use{uninsur1},
%and
%\use{wldtrade}.
The new chapters and  revisions cover   topics that
 widen and deepen
the message that recursive methods are pervasive and powerful.


\nosechead{New chapters}\noindent
Chapter \use{optaxrecur} applies ``dynamic programming squared''  to two of the optimal taxation models studied in
chapter \use{optax}, namely, Lucas and Stokey's (1983) model of optimal taxation and borrowing in an economy with complete markets, and
an incomplete markets version of that model. Among other insights that  recursive formulations bring to  these models is a sharp
characterization of the time inconsistency of optimal plans that emerges in the form of two value functions for each optimal taxation problem,
one for time $t=0$, another for all times $t \geq 1$. Distinct value functions, state vectors,  decision rules,  and  Bellman equations at times $t=0$  and $t \geq 1$
are tell-tale signs of time-inconsistency.

The ``Shimer puzzle'' is  that standard calibrations of matching models do not generate fluctuations in unemployment rates nearly as large
as those observed over business cycles.\NFootnote{A puzzle is always relative to a  model. A `puzzle' is a feature or  prediction of a model that
 is contradicted by data.} \index{puzzle!definition}%
Chapter \use{mechanics_matching} looks under the hood of a variety of matching models that have been reconfigured in attempts to confront the Shimer puzzle.
The chapter examines a variety of different reconfigurations of the standard model and identifies  a
single common economic force that each  of these diverse specifications activates in order address the Shimer critique.



%\nosechead{Revisions of other chapters}\noindent

\nosechead{Ideas}
\noindent Beyond emphasizing recursive  methods,
     the economics  of this book
revolves around several main  ideas.
\medskip
\item{1.}  The competitive equilibrium model of
a dynamic stochastic economy:
This model contains  complete markets, meaning
that  all commodities at different dates
that are contingent  on 
random events can be traded in a market with a centralized
clearing arrangement.  In one version of the model,
all trades occur at the beginning of time.  In another,
trading in one-period claims occurs sequentially.
The model   is a foundation for  asset-pricing theory, growth
theory, real business cycle theory, and normative
public finance.    There is no room for fiat money in
the standard competitive equilibrium model, so we
shall have to alter the model to make room for  fiat money.
\medskip
\item{2.} A class of incomplete markets models with heterogeneous
agents:  These models arbitrarily restrict the types of assets
that can be traded, thereby possibly igniting a precautionary
motive    for agents to hold those assets.    Such models
have been used to study the
distribution of wealth  and the evolution of an individual or
family's wealth over time.   One model in this class lets money in.

\medskip
\item{3.}   Several models of
fiat money:    We add a shopping time specification
to a competitive equilibrium model to get a simple vehicle
for explaining ten
  doctrines of monetary economics. These doctrines depend
on the government's intertemporal budget constraint and the demand
for fiat money, aspects that transcend many models.     We also use
Samuelson's overlapping generations model, Bewley's
incomplete markets model,  and Townsend's
turnpike model to perform a variety of policy experiments.

\medskip
\item{4.}  Restrictions on government  policy implied by the arithmetic of
budget sets:  Most of the ten monetary doctrines reflect
properties of the government's budget constraint.  Other
important doctrines do too.
These doctrines, known as Modigliani-Miller
and Ricardian equivalence theorems, have a common structure that come from
identifying an equivalence class of
government policies that produce the same allocations.
We display the structure of such theorems with an eye to
finding 
features whose absence causes them to fail, letting
particular policies matter.
\medskip
\item{5.} Ramsey taxation problems:
What is the optimal tax structure when only distorting taxes are
available? The primal approach to taxation recasts this question as a
problem in which a government chooses  allocations directly and  tax rates only indirectly. Permissible allocations are those that
satisfy resource constraints and implementability constraints, where
the latter are budget constraints in which the consumer
and firm first-order conditions are used to eliminate 
prices and tax rates. We study labor and
capital taxation, and examine the optimality of the inflation
tax prescribed by the Friedman rule.


\medskip
\item{6.}  Social insurance with private information and
enforcement problems: We use the recursive contracts approach to
study a variety of problems in which a benevolent social insurer
 balances providing insurance against
providing incentives.  Applications
include the provision of unemployment insurance
and the design of loan contracts when a lender
has an imperfect capacity to monitor a borrower.
\medskip
\item{7.}   Reputation  models in macroeconomics:   We study
how far reputation can go to overcome a government's
inability to  commit to a policy.   The theory describes
multiple systems of expectations about its behavior
to which a government wants to conform.  The theory
has many applications, including implementing optimal
taxation policies and  making monetary
policy in the presence of a temptation to inflate
offered by a Phillips curve.
\medskip
\item{8.}  Search models: Search theory
makes  assumptions
different from   ones underlying a complete markets competitive
equilibrium model.  It imagines that  there is no
centralized  place where exchanges can be made,
or that there are not standardized commodities.  Buyers
and/or sellers have to devote effort to search
for opportunities to buy or sell good or factors of production, opportunities that might 
arrive randomly. We describe the basic
McCall search model and various applications.
We also describe some equilibrium versions of the McCall model
and compare
them with  models of another  type that postulate
matching functions.
\medskip
\item {9.} Matching models.  A matching function
accepts measures of  job seekers and vacancies as inputs and maps them
 opportunities to  form matches.  Models with matching functions
build in congestion externalities that job searchers impose on  other job searchers
and that vacancy posters impose on other job searchers.  The models study how these externalities contend with each other
 and how they shape job-finding rates, job-filling rates, and unemployment rates.
In the last fifteen years, matching models have been revised in ways intended to make them
fit business cycle facts and welfare state outcomes.
\medskip
\item{10.} Employment lotteries versus career time-averaging.  A model that was popular until recently interpreted
 the aggregate labor supply
as the fraction of people that a planner chooses assigns to work by using  a lottery
in which the losers must work and the winners enjoy leisure.  An  alternative
model instead focuses on an individual worker who  chooses the fraction of his or her
life to work within a life-cycle model.  The two frameworks have strikingly similar
implications about some aggregate outcomes, but not about others.
\medskip
\item{11.} Heterogeneous beliefs.  While  for very good reasons most applied macroeconomic models continue to assume  rational expectations,
it is useful to study  frameworks in which there are multiple beliefs either across people or, in models of ``ambiguity'' and
``robustness'', within the mind of one decision maker.  Parts of chapters \use{recurge} and \use{assetpricing2} study
such models.



\nosechead{Theory and evidence}

\noindent Though this book aims to give the reader the tools
to read about applications, we spend little time
on empirical applications.
  However, the empirical failures of one class of models have been a main force
prompting development of another class of models.
Thus,  the  perceived empirical failures of the standard
complete markets general equilibrium model stimulated
the development of the incomplete markets and recursive
contracts models.   For example, the complete markets model
forms  a standard benchmark model  or point of departure
for theories and empirical work  on
consumption and asset pricing.   The complete markets model
has these empirical problems:  (1)  there is too much
correlation between individual income and consumption growth
in micro data (e.g., Cochrane, 1991 and Attanasio and Davis, 1995);
(2)  the equity premium is larger in the data than
is implied by a representative agent  asset-pricing model with reasonable
risk-aversion parameter (e.g., Mehra and Prescott, 1985); and
 (3) the risk-free interest rate is   too low relative
to the observed aggregate rate of consumption  growth (Weil, 1989).
While there have been numerous attempts to explain these puzzles
by altering the preferences in the standard complete markets
model, there has  also been work that abandons the complete
markets assumption  and replaces it with
some version of either exogenously or endogenously incomplete
markets.   The Bewley models of chapters \use{selfinsure}  and
\use{incomplete}  are
examples of exogenously incomplete markets.   By ruling
out complete markets, this model structure  helps
with empirical  problems 1 and 3 above (e.g., see
Huggett, 1993), but not much with problem 2.     In chapter
\use{socialinsurance}, we study some models that can be thought of as
having endogenously incomplete markets.   They can also explain
puzzle 1 mentioned earlier in this paragraph;
  at this time it is not really known how far
they take us toward solving problem 2, though Alvarez and Jermann (1999)
report promise.


\nosechead{Micro foundations}

\noindent This book is  about micro foundations for macroeconomics.
Browning, Hansen, and Heckman (1999) identify two possible justifications
for putting microfoundations underneath macroeconomic models.
The first is  aesthetic and preempirical: models with micro foundations
are by construction coherent and explicit.   And because they
contain descriptions of agents' purposes, they  allow us to
analyze policy interventions using standard methods of welfare
economics.    Lucas (1987) gives a distinct second reason:
\auth{Lucas, Robert E., Jr.}%
       a model with micro foundations     broadens the
sources of empirical evidence that can be used to
assign numerical values to  the model's parameters.    Lucas
endorses  Kydland and Prescott's (1982) procedure of
borrowing parameter values from micro studies.   Browning, Hansen,
and Heckman (1999)  challenge
Lucas's recommended  empirical strategy.
Most seriously, they point out that  in many contexts the
specifications underlying the microeconomic studies  cited by
a calibrator
conflict with those of  the macroeconomic  model being
``calibrated.''    It is typically not obvious how
to transfer parameters from one data set and model specification
to another data set and model specification.
\auth{Heckman, James J.} \auth{Hansen, Lars P.}
\auth{Browning, Martin}


  Although we take seriously the doubts
about Lucas's justification for microeconomic foundations
that
Browning, Hansen and  Heckman raise, we remain strongly attached
to micro foundations. For us, it remains enough
to appeal to the first justification, namely, the coherence provided
by micro foundations and the virtues that come from
having the ability to ``see the agents'' in
an artificial economy.
  We see Browning,  Hansen, and  Heckman as raising
 many legitimate
questions about empirical strategies for implementing
macro models with micro foundations.
We don't think that the clock will soon be turned back to a time
when macroeconomics was done without
micro foundations.



\nosechead{Road map}

\noindent Chapter \use{overview} is either a preview or review or both. It is either a reader's guide to what is to come or a concise review of main themes
  that have been studied.  There is a case for
reading it quickly before diving into the other chapters, while not expecting fully to understand everything
that is written there. After many of the other chapters have been mastered, it could be useful to read it again.  

Chapter \use{timeseries} describes   two basic models of a time series:
a Markov chain and  a linear first-order difference equation.
In different ways, these models use the algebra of first-order
difference equations
to form tractable models of time series.  Each model
has its own notion of the state of a system.  These time
series models define essential   objects in terms of which
the choice problems of later chapters are formed and their
solutions are represented.

  Chapters \use{dynamicdp1}, \use{practical}, and \use{dplinear}
  introduce aspects of dynamic programming, including
numerical dynamic programming. Chapter \use{dynamicdp1} describes the
basic functional equation of dynamic programming, the Bellman
equation, and several of its properties. Chapter \use{practical}
describes some numerical ways for solving dynamic programs, based
on Markov chains.  Chapter \use{dplinear} describes linear
quadratic dynamic programming and some uses and extensions of it,
including how to use it to approximate solutions of problems that
are not linear quadratic. This chapter also tells how the Kalman
filter from chapter \use{timeseries} is
mathematically equivalent to the linear quadratic dynamic
programming problem from chapter \use{dplinear}.\NFootnote{The equivalence is through duality,
in the sense of mathematical programming.} Chapter \use{search1}
describes a classic two-action dynamic programming problem, the
McCall search model, as well as Jovanovic's extension of it, a
good application of  the Kalman filter.

  While single agents appear in   chapters  \use{dynamicdp1} through
\use{search1},
systems with multiple agents, whose environments  and choices must
be reconciled through markets,  appear for the first time
in chapters \use{recurpe} and \use{recurge}.
 Chapter \use{recurpe} uses linear quadratic dynamic
programming to introduce two important and related equilibrium concepts:
 rational expectations equilibrium and  Markov perfect
equilibrium.    Each of these equilibrium
concepts can be viewed as a fixed point in a space of beliefs
about what other agents intend to do;  and each    is
formulated using recursive methods.
Chapter \use{recurge} introduces two notions of competitive equilibrium
in dynamic stochastic pure exchange  economies, then applies
them to pricing various  consumption streams.

Chapter \use{ogmodels}  interprets  an overlapping generations model
as a version of the general competitive  model with
a peculiar preference pattern.        It then goes on to
use a sequential formulation of equilibria to display
how the overlapping generations model can be used to study
issues in monetary and fiscal economics, including Social Security.

Chapter \use{ricardian} compares an important aspect
of an overlapping generations model with
an infinitely lived agent  model  with a particular
kind of incomplete market structure.  This chapter is thus
our first  encounter with an incomplete markets model.
The chapter analyzes the Ricardian equivalence  theorem in two distinct
but isomorphic settings:  one a model with infinitely lived
agents who face borrowing constraints, another with overlapping generations
of two-period-lived agents with a bequest motive.    We describe
situations in which the timing of taxes does or does not
matter, and explain how binding borrowing
constraints in the infinite-lived model correspond
to nonoperational bequest motives in the overlapping generations
model.
%We also introduce the idea of ``natural'' and ``ad hoc'' debt
%limits, and why it matters which is imposed in the infinite
%horizon model.

Chapter \use{linappro} studies fiscal policy within a nonstochastic growth model with
distorting taxes.  This chapter studies  how foresight about future policies and
transient responses to past ones contribute to current outcomes. In particular, this chapter describes
`feedforward' and `feedback' components of mathematical formulas  for equilibrium outcomes.
Chapter \use{growth1} describes the recursive competitive equilibrium concept and applies it within the context of
the stochastic growth model.

Chapter \use{assetpricing1} studies asset pricing  and a host of practical
doctrines associated with asset pricing, including
Ricardian equivalence again and Modigliani-Miller theorems
for private and government finance. Chapter \use{assetpricing2} studies empirical strategies
for implementing asset pricing models. Building on work by Darrell Duffie, Lars Peter Hansen,  and their co-authors,  chapter  \use{assetpricing2} discusses  ways of characterizing asset pricing puzzles
associated with the preference specifications and market structures commonly used in other parts of macroeconomics.
It then describes alterations of those structures that hold promise for  resolving some of those puzzles.
\auth{Hansen, Lars P.}%
\auth{Duffie, Darrell}%
Chapter \use{growth} is about economic growth.  It describes
the basic growth model, and analyzes the key features
of the  specification of  the technology that allows
the model to exhibit balanced growth.

 Chapter \use{optax} studies competitive
equilibria distorted by taxes  and our first
mechanism design problems, namely,
ones that seek to find the optimal temporal pattern of
distorting taxes. In a nonstochastic economy, a
striking finding is that the optimal tax rate on capital
is zero in the long run.


Chapter \use{selfinsure} is about self-insurance.  We study
 a single agent whose limited menu
of assets gives him  an incentive
to self-insure by accumulating assets.     We study
a special case of what has sometimes been called
the ``savings problem,'' and analyze in detail the motive for
self-insurance and the surprising implications it
has  for the agent's ultimate consumption and asset holdings.
The type of agent studied in this chapter will be a component
of the incomplete markets models to be studied in chapter \use{incomplete}.

    Chapter \use{incomplete} studies incomplete
markets economies with heterogeneous agents and imperfect markets
for sharing risks.  The models of market incompleteness in this
chapter  come from simply ruling out markets in many assets,
without motivating  the absence of those asset markets   from the
physical structure of the economy. We  wait until chapter
\use{socialinsurance} to study   reasons that such markets may not
exist.

The next chapters describe   recursive
contracts.   Chapter
\use{stackel} describes what we
call ``dynamic programming squared'' and uses   linear quadratic dynamic programming
to explain it in a context in which key objects can be computed easily. A tell tale sign of a dynamic programming squared problem
is that there is a Bellman equation inside another Bellman equation.  Chapter \use{optaxrecur}
uses dynamic programming squared  to reformulate two optimal taxation models
from  chapter \use{optax} recursively.
 Chapter \use{socialinsurance} describes models in
the mechanism design tradition, work that starts to provide
a foundation for incomplete assets markets, and that recovers
specifications resembling  models
of chapter \use{incomplete}. Chapter \use{socialinsurance}
is about the optimal
provision of social insurance in the presence
of information  and enforcement problems.
 Relative to earlier chapters, chapter \use{socialinsurance}  escalates
the sophistication with which recursive methods are applied, by
utilizing promised values as state variables. Chapter
\use{socialinsurance2} extends the analysis to a general
equilibrium setting and draws out some implications for asset
prices, among other things. Chapter \use{uninsur1} uses recursive
contracts to design optimal unemployment insurance and
worker compensation schemes.

Chapters \use{credible} and \use{chang} apply some of the same ideas to  problems
in ``reputational macroeconomics,'' using promised values to
formulate a notion of credibility. We study how a  reputational
mechanism can make policies sustainable even when a government can't commit -- meaning choose a plan for all $t \geq 0$  once-and-for-all at time $0$ -- in the way
 assumed in the
 analysis of chapter \use{optax}. We use this reputational approach
in chapter \use{fiscalmonetary} to assess whether the Friedman rule is sustainable. Chapter
\use{wldtrade} describes a model of gradualism  in trade policy
that has  features in common with the first model of chapter
\use{socialinsurance}.

 Chapter \use{fiscalmonetary} switches
gears by adding money to a very simple competitive
equilibrium model, in a superficial way;  the excuse for
that superficial device      is that it permits us
to present  and unify ten  well-known monetary doctrines.
Chapter \use{townsend} presents a less superficial model of
money, the turnpike model of Townsend,
which is basically a special nonstochastic
version of one of the models of chapter \use{incomplete}.   The specialization
allows us to focus on a variety of monetary doctrines.

Chapter \use{search2}
 describes multiple agent models of search and matching. Except
for a section on money in a search model, we  focus  on applications to labor.
 To bring
out the economic forces at work in different frameworks,
we examine the general equilibrium effects of layoff taxes.  Chapter \use{mechanics_matching} investigates
some fundamental forces common to a variety of superficially different matching models.   Chapter \use{macrolaborII}
compares forces in an employment lotteries model with those operating in a time-averaging model
of aggregate labor supply.

  Two appendixes
 collect various technical results on
functional analysis and linear projections and hidden Markov models.


\nosechead{Alternative  uses of the book}

 \noindent We have used parts  of  this book to teach both first- and second-year
graduate courses in macroeconomics and monetary economics at the University
of Chicago, Stanford University, New York University, Princeton University, and the
Stockholm School of Economics.  Here are some  alternative plans
for courses:

\medskip
%\item{1.} A one-semester first-year course:  chapters 1-5, 7, 8, 9,
%and either chapter 10, 11, or 12.
\item{1.} A one-semester first-year course:  chapters
\use{timeseries}--\use{search1}, \use{recurge}, \use{ogmodels}, \use{ricardian},
 and either chapter \use{assetpricing1}, \use{growth}, or \use{optax}.

\medskip
%\item{2.}  A second-semester first-year course:  add chapters 6, 10, 11,
%12, parts of 13 and 14, and all of 15.
\item{2.}  A second-semester first-year course:  add chapters
\use{recurge}, \use{growth1}, \use{assetpricing1}, \use{assetpricing2}, \use{growth},
\use{optax},  parts of \use{selfinsure} and \use{incomplete}, and all of
\use{socialinsurance}.

\medskip
%\item{3.}  A first course in monetary economics:  chapters 8, 16,
%17, 18, and last section of 19.
 \item{3.}  A first course in
monetary economics:  chapters \use{ogmodels}, \use{credible}, \use{chang},
\use{wldtrade},
\use{fiscalmonetary}, \use{townsend}, and the last section of \use{search2}.
\medskip
\item{4.} A second-year macroeconomics course: select from
chapters \use{assetpricing1}--\use{macrolaborII}.
 %10-19.
%, 11, 12, 13, 14, 15, and 18.
\medskip
\item{5.} A self-contained course about recursive contracts:
chapters
\use{stackel}--%, \use{socialinsurance},
%\use{socialinsurance2},
%\use{uninsur1},
%\use{credible},
\use{wldtrade}. \medskip

As an example, Sargent used the following structure for
 a one-quarter first-year
course at the University of Chicago:   for the first and last
weeks of the quarter, students were asked  to read  the monograph
by Lucas (1987). Students were ``prohibited'' from reading the
monograph in the intervening weeks.   During    the middle eight
weeks of the quarter, students read material from   chapters
\use{search1} (about search theory); chapter \use{recurge} (about complete markets);
chapters \use{ogmodels}, \use{fiscalmonetary}, and
\use{townsend} (about models of money); and a little bit of
chapters
\use{socialinsurance}, \use{socialinsurance2}, and \use{uninsur1}
(on social insurance with incentive constraints). The substantive
theme of the course was the issues set out in a nontechnical way
by Lucas (1987).
  However, to understand
Lucas's arguments, it helps to know the tools and models studied
in the middle weeks of the course.     Those weeks also exposed
students to a range of alternative models that could
be used to measure Lucas's arguments against some of the
criticisms   made, for example,  by Manuelli and Sargent (1988).
\auth{Manuelli, Rodolfo}%

Another one-quarter course would  assign Lucas's  (1992) article
on efficiency and distribution in the first and last weeks. In the
intervening weeks of the course, assign chapters \use{selfinsure},
\use{incomplete}, and \use{socialinsurance}.
%13, 14, 15.


As another example, Ljungqvist used the following material in a
four-week segment on employment/unemployment in first-year
macroeconomics at the Stockholm School of Economics. Labor market
issues command a strong interest especially in Europe. Those
issues help motivate studying the   tools in chapters
\use{search1} and \use{search2} (about search and matching models), and parts
of \use{uninsur1} (on the optimal provision of unemployment
compensation). On one level, both chapters \use{search1} and
\use{search2} focus on labor markets as a central application of the
theories presented, but on another level, the skills and
understanding acquired in these chapters transcend the specific
topic of labor market dynamics. For example, the thorough practice
on formulating and solving dynamic programming problems in chapter
\use{search1} is generally useful  to any student of economics, and the
models of chapter \use{search2} are  an entry-pass to other
heterogeneous-agent models like those in chapter \use{incomplete}.
Further, an excellent way to motivate the study of recursive
contracts in chapter \use{uninsur1} is to  ask how unemployment
compensation should optimally  be provided in the presence of
incentive problems.

As a final example, Sargent used versions of the material in
\use{search1}, \use{linappro}, and \use{assetpricing2} to teach undergraduate
classes at Princeton and NYU.

\nosechead{Computer programs}

\noindent Various exercises and examples use Matlab programs.  These
programs are referred to in a special index at the end of the
book.  They can be downloaded from %$<$http://homepages.nyu.edu/pub/\raise-4pt\hbox{\~{}}ts43/source\hbox{\_{}}code/mitbook.zip$>$.
%$<$https://files.nyu.edu/ts43/public/books.html$>$.
$<$www.tomsargent.com/source\_code/mitbook.zip$>$.
Python and Julia programs for some of the models studied in this book are described at
$<$https://lectures.quantecon.org/$>$.
%\nosechead{Answers to exercises}
%
%\noindent We have created a web site with additional exercises
%and  answers to the exercises in the text.  It is at
%$<$http://www.stanford.edu/\raise-4pt\hbox{\~{}}sargent$>$.

\nosechead{Notation}

 \noindent We use the symbol~\hskip-.75em$\qed$ to denote the conclusion of a proof.
The editors of this book requested  that where possible,
 brackets and braces be used in place of multiple  parentheses
to denote composite functions.  Thus, the reader will
often encounter $f[u(c)]$   to express
the composite function $f \circ u$.

\nosechead{Brief history of the notion of the state}

  \noindent This book reflects progress economists
have made in refining the notion of state so that more and
more problems can be formulated recursively.
  The art in applying recursive methods is to find a convenient definition
of the  state.  It is often not obvious what the
state is, or even whether a finite-dimensional
 state {\it exists} (e.g., maybe the entire infinite
history of the system is needed to characterize  its current
position).
Extending the range of problems susceptible to
recursive methods has been one of the major
accomplishments of macroeconomic theory since 1970.
 In diverse contexts, this enterprise
 has been about  discovering a convenient
state  and constructing a first-order difference equation to
describe its motion. In models equivalent to  single-agent control
problems, state variables are either capital stocks or information
variables that help predict the future.\NFootnote{Any available
variables
 that {\it Granger cause}  variables impinging on
the optimizer's objective function or constraints enter the state
as information variables. See C.W.J. Granger (1969).}
\auth{Granger, C.W.J.}%
  In single-agent
 models of optimization
in the presence of measurement errors, the true state vector is
latent or  ``hidden'' from the optimizer and the economist, and
needs to be  estimated. Here {\it beliefs} come to  serve as the
patent state. For example, in a Gaussian setting, the mathematical
expectation and covariance matrix of the  latent state vector,
conditioned on the available history of observations, serves as
the state. In authoring  his celebrated filter, Kalman (1960)
showed how an estimator of the hidden state could be constructed
recursively by means of a difference equation that uses the
current observables to update the estimator of last period's
hidden state.\NFootnote{In competitive multiple-agent models  in
the presence of measurement errors, the dimension of the hidden
state threatens to explode
 because beliefs about beliefs about
$\ldots$ naturally
appear, a problem studied by Townsend
 (1983).  This threat has been  overcome through thoughtful
and economical definitions of the state.
For example, one way is to  give up on
seeking a purely
 ``autoregressive''  recursive structure and to include
 a moving
 average piece in the descriptor of beliefs.  See Sargent (1991).
Townsend's equilibria have the property that prices fully reveal
the private information of  diversely informed agents.} Muth
(1960); Lucas (1972), Kareken, Muench, and Wallace (1973);
Jovanovic (1979);  and Jovanovic and Nyarko (1996)  all used
versions   of the Kalman filter to study systems in which agents
make decisions with imperfect observations about the state.

\auth{Townsend, Robert M.} \auth{Wallace, Neil} \auth{Lucas,
Robert E., Jr.} \auth{Kareken, John} \auth{Muth, John F.}
\auth{Jovanovic, Boyan} \auth{Nyarko, Yaw}
 \auth{Hansen, Lars P.} \auth{Epple, Dennis} \auth{Roberds, William}%
For a while, it seemed that some very important problems in
macroeconomics could not be  formulated  recursively. Kydland and
Prescott (1977) argued that it would be difficult to apply
recursive methods to  macroeconomic policy design problems,
including two examples about taxation and a Phillips curve.
\auth{Kydland, Finn E.}\auth{Prescott, Edward C.}%
  As Kydland and Prescott
formulated them,  the problems were not recursive:  the fact that
the public's  forecasts of the government's future decisions
influence the public's current decisions made the government's
problem  simultaneous, not sequential.
But soon Kydland and Prescott (1980) and Hansen, Epple,
and Roberds (1985) proposed
a recursive formulation of such problems by expanding
the state of the economy to include a Lagrange multiplier
or {\it costate} variable associated with the government's
budget  constraint. The costate variable acts as
the marginal cost
 of keeping a   promise made earlier by the government.
Marcet and Marimon (1999) extended and formalized
a recursive version of such problems. % \auth{Marcet, Albert} \auth{Marimon, Ramon}

 A significant breakthrough in the application of
    recursive methods
was achieved by several researchers including Spear and Srivastava
(1987), Thomas and Worrall (1988), and Abreu, Pearce, and
Stacchetti (1990).
 \auth{Abreu, Dilip}%
\auth{Spear, Stephen E.}%
\auth{Thomas, Jonathan}%
 \auth{Worrall, Tim}%
 \auth{Pearce, David}%
\auth{Stacchetti, Ennio}%
 They discovered
a state variable for recursively formulating an
 infinitely repeated moral hazard problem. That  problem
requires the
principal  to  track  a history
of outcomes  and to use it  to construct statistics for drawing
inferences about the agent's actions.
Problems involving self-enforcement of contracts and a
government's
reputation share this feature.
A {\it continuation value}
promised by the principal to the agent can
 summarize the history.
Making the promised valued  a state variable allows
a recursive  solution in terms of a function
mapping the inherited promised value and random
variables  realized today into an action or
 allocation today and a promised value for
tomorrow.  The sequential nature of the solution allows us to
recover history-dependent strategies just as we use a stochastic
difference equation to find a ``moving average''
representation.\NFootnote{Related
 ideas are used by Shavell and Weiss (1979); Abreu, Pearce,
and Stacchetti (1986, 1990) in  repeated games; and Green  (1987)
and Phelan and Townsend (1991) in dynamic mechanism design. Andrew
Atkeson (1991) extended  these ideas to study  loans made by
borrowers who cannot tell whether they are making consumption
loans or investment loans.} \auth{Atkeson, Andrew} \auth{Townsend,
Robert M.} \auth{Phelan, Christopher} \auth{Shavell, Stephen}
\auth{Green, Edward J.}\auth{Abreu, Dilip} \auth{Pearce, David}
\auth{Stacchetti, Ennio} \auth{Weiss, Laurence}

  It is now standard to
use a continuation value  as a state variable
in models of credibility and dynamic incentives.     We shall
study several such models in this book, including ones for
optimal unemployment insurance and for designing loan contracts
that must overcome information and enforcement problems.

