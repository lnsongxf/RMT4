\noindent{\it Exercise \the\chapternum.5} \quad
\quad
  {\bf Ambiguity averse multiplier preferences} \index{multiplier preferences!ambiguity}%

  \medskip\noindent
Consider the recursion \Ep{Trecur2}, namely,
$$
 U_t = c_t - \beta \theta \log E_t \left[ \exp \left( {\frac{-U_{t+1}}{\theta}}\right)\right],
\eqno(1) $$
where $c_t = \log C_t, \beta \in (0,1)$, and $0 < \theta $.  Let $c_t$ follow
the stochastic process
$$ c_{t+1} = c_t + \mu + \sigma_c \varepsilon_{t+1} ,  \eqno(2) $$
where $\varepsilon_{t+1} \sim {\cal N}(0,1)$ is an i.i.d.\ random process.

\medskip
\noindent{\bf a.}  Guess a value function of the form
$U_t = k_0 + k_1 c_t$, where $k_0$ and $k_1$ are scalar constants.
In detail, derive formulas for $k_0, k_1$ that verify the recursion \Ep{Trecur2} (or (1) above).

\medskip
\noindent{\bf b.} Using the formulas for $k_0, k_1$ that you have derived, verify that when $c_t$ obeys (2), another way to express
recursion (1) is
$$ U_t = c_t + \beta E_t U_{t+1} - {\frac{\beta}{2 \theta }} {\rm var}_t (U_{t+1}) . \eqno(3) $$
 % $$ U_t = c_t + \beta E_t U_{t+1} + {\frac{\beta(1-\gamma)}{2}} {\rm var}_t (U_{t+1}) . \eqno(3) $$
Use representation (3) to offer an interpretation of why the consumer prefers plan A to plan B in figure \Fg{treeAB} and also why he prefers Plan C to Plan D in
figure \Fg{treeCD}.

\medskip
\noindent{\bf c.}  For recursion (1), verify in detail that
$$ \eqalign{ {\frac {\partial U_t}{\partial C_t}} & = {\frac{k_1}{C_t} } \cr
             {\frac {\partial U_t}{\partial C_{t+1}}} & = \beta ( {\frac{\exp(-\theta^{-1} U_{t+1})}{E_t \exp(-\theta^{-1} U_{t+1}) }} ){\frac{ k_1}{C_{t+1}}}. }
             $$

\medskip
\noindent{\bf d.} Explicitly verify  that the likelihood ratio defined as $g$ satisfies
$$ g(\varepsilon_{t+1}) \equiv \left({\frac{\exp \left( \left(
1-\beta \right) \left( 1-\gamma \right) U_{t+1}\right) }{E_{t}\left[
\exp \left( \left( 1-\beta \right) \left( 1-\gamma \right)
U_{t+1}\right) \right] }}\right)  = \exp(w  \varepsilon_{t+1}  - {\frac{1}{2}} w^2 )
 $$
where $w$ is given by
$$ w = - {\frac{\sigma_c}{\theta\left(1-\beta\right)}} .  $$
{\it Hint:} Once again, you might want to use the formula for the mean of a log normal random variable.

\medskip
\noindent{\it Exercise \the\chapternum.6} \quad
\quad
  {\bf Exponential affine sdf} \index{stochastic discount factor!exponential affine}%

  \medskip\noindent
  Consider an exponential affine model of a stochastic discount factor driven by the following vector autoregression for $z_t$:
$$ \EQNalign{ z_{t+1} & = \mu +   \phi z_t + C \varepsilon_{t+1} \cr
              r_t & = \delta_0 + \delta_1' z_t,  } $$
where $\phi$ is a stable $n \times n$ matrix, $C$ is an $n \times n$ matrix,  $\varepsilon_{t+1} \sim {\cal N}(0,I)$ is an i.i.d.\ $n\times 1$ random vector, and $z_t$ is an $n \times 1$ state vector.
The  logarithm of the stochastic discount factor is affine (i.e., linear plus a constant) in the state vector
$z_t$ from the vector autoregresssion:
$$ \EQNalign{ \Lambda_t & = \Lambda_0 + \Lambda_z z_t  \cr
              \log ( m_{t+1}) & = - r_t - {\frac{1}{2}} \Lambda_t' \Lambda_t - \Lambda_t' \varepsilon_{t+1} . }$$
Let the gross risky return $R_{j,t+1}$ on an asset be
$$ R_{j,t+1} = \exp(\nu_t(j) - {\frac{1}{2}} \alpha_t(j)' \alpha_t(j) + \alpha_t(j)' \varepsilon_{t+1} ), \eqno(1) $$
where $\nu_t(j) $ is a function of $z_t$ that makes $E_t (m_{t+1} R_{j,t+1}) = 1 $ be satisfied and
$ \alpha_t(j) = \alpha_0(j) + \alpha_z(j) z_t,  $
where $\alpha_0(j)$ is an $n \times 1$ vector and $\alpha_z(j)$ is an $n \times n$ matrix.

\medskip
\noindent{\bf a.} Show that (1)  implies
$$ E_t R_{j,t+1} = \exp(\nu_t(j)) .$$


\medskip
\noindent{\bf b.} Prove that $E_t (m_{t+1} R_{j,t+1}) = 1 $ implies that
$$ \nu_t(j) = r_t + \alpha_t(j)' \Lambda_t.  \eqno(2) $$

\medskip
\noindent{\bf c.} Interpret formula (2).

\medskip
\noindent{\bf d.}  Specify  stochastic dynamics for  aggregate consumption $C_t$ that would make the `ordinary' stochastic discount factor
 $m_{t+1} = \beta ({\frac{C_{t+1}}{C_t}})^{-\gamma} $ become (a special case of) an exponential affine model of a stochastic discount factor.


\medskip
\noindent{\it Exercise \the\chapternum.7} \quad
\quad
  {\bf Value function for CRRA}

\medskip
\noindent Consumption $C_t$ follows the stochastic process
$$ C_{t+1} = \exp(\mu + \sigma_c \varepsilon_{t+1}) C_t $$
where $\varepsilon_{t+1} \sim {\cal N}(0,1)$ is an i.i.d.\ scalar random process, $\mu > 0$,
and $C_0$ is an initial condition.
The consumer ranks  consumption streams  according to
$$ U_0 = E_0 \sum_{t=0}^\infty \beta^t ({\frac{C_t^{(1-\gamma)}}{1-\gamma}} ) ,\eqno(1) $$
where $\beta \in (0,1)$, $E_0$ denotes the mathematical expectation conditional on $C_0$, and
$\gamma \geq 0$.

\medskip
\noindent {\bf a.} Find a value function $U_0$ that satisfies (1), giving conditions
on $\mu, \sigma_c, \gamma, \beta$ that ensure that the right side of (1) exists.

 \medskip
\noindent {\bf b.}  Describe a recursion
$$ U_t = {\frac{C_t^{(1-\gamma)}}{1-\gamma}}  + \beta E_t U_{t+1} $$
whose solution at time $0$ satisfies (1).

%\vfil\eject
\medskip
\noindent{\it Exercise \the\chapternum.8} \quad
\quad
  {\bf Lucas tree economy}
  \medskip \noindent
A representative consumer in a Lucas tree economy has preferences over consumption streams
ordered by $$ E_0 \sum_{t=0}^\infty \beta^t \log C_t. \eqno(1) $$
There is one asset, a tree that yields dividends $y_t$ at time $t$, which are governed by a two-state Markov chain $P$ with
states $s_t \in \{0,1\}$ and
$$ y_t = \cases{ y_L, & if $s_t =  0 $; \cr
                 y_H , & if $s_t = 1 $,\cr}  $$
where $y_H > y_L > 0$.  The tree is the only source of goods in the economy.  There is a single representative consumer, so that
 $C_t = y_t$. The transition matrix $P = \bmatrix{ \pi_L & 1 - \pi_L \cr 1 - \pi_H & \pi_H } $, where $\pi_L \in (0,1)$ and $\pi_H \in (0,1)$.

 There is a market in trees. (There will be zero volume in equilibrium.) Let $p_t$ be the (ex-dividend) price of a claim to the fruit of the tree from time $t+1$ on.
 Ownership    of the tree at the beginning of $t$ entitles the owner of the tree to receive dividend $y_t$ at $t$ and then to sell the tree if he wants at price $p_{t+1}$ after collecting the
 dividend at time
 $t+1$.

\medskip
\noindent {\bf a.} Find a stochastic discount factor for this economy. Please tell how to compute the rate of return on one-period risk-free bonds in this economy.

\medskip \noindent
{\bf b.} Find an equilibrium pricing function mapping the state of the economy at $t$ into the price of the tree $p_t$.

\medskip \noindent
{\bf c.}  Describe the behavior of the  one period gross return on the tree.

\medskip\noindent
A zero-net worth  outside entrepreneur  comes into this economy and purchases all trees at time $\bar t > 0$. The outsider has no resources and finances his tree purchases by issuing (1)
infinite duration risk-free bonds promising to pay $\eta \in (0, y_L)$ each period $t \geq \bar t+1$, and (2) equity that pays share-holders $y_t - \eta$  in period $t \geq \bar t+1$.
The asset of the entrepreneur is the tree while his/her liabilities are the bonds and equities.

\medskip
\noindent {\bf d.} Please compute the equilibrium value of the risk-free bonds as a function of the promised coupon payment $\eta$ and the state of the economy.
\medskip
\noindent {\bf e.} Please compute the equilibrium value of equity as a function of the bond coupon payment $\eta$ and the state of the economy.
\medskip
\noindent{\bf f.} How do your answers in {\bf d} and {\bf e} compare with the value of the tree that you computed in part {\bf b}?

\vfil\eject

\medskip
\noindent{\it Exercise \the\chapternum.9} \quad
\quad
  {\bf Long-run risk, I} \index{long run risk}%
  \medskip

\noindent Consider the following pure exchange economy with one representative consumer and an exogenous
consumption process.
 Consumption $C_t = \exp(c_t)$ follows the stochastic process
$$\eqalign{ c_{t+1} & = \mu + z_t + c_t + \sigma_c \varepsilon_{t+1} \cr
            z_{t+1} & = \rho z_t + \sigma_z \varepsilon_{t+1} ,  } $$
where $\varepsilon_{t+1} \sim {\cal N}(0,1)$ is an i.i.d.\ scalar random process, $\mu > 0$,
and $c_0, z_0$ are  initial conditions, $\rho \in (0,1)$.  By setting $\rho$ close to but less than unity
and $\sigma_z$ to be small, $z_t$ becomes a slowly moving component of the conditional mean  of  $E_t (c_{t+1} - c_t)$.
The consumer ranks consumption streams according to
$$ U_0 = E_0 \sum_{t=0}^\infty \beta^t ({\frac{C_t^{(1-\gamma)}}{1-\gamma}} ) ,\eqno(1) $$
where $\beta \in (0,1)$, $E_0$ denotes the mathematical expectation conditional on $C_0$, and
$\gamma \geq 0$.

\medskip
\noindent {\bf a.} Find a formula for the consumer's  stochastic discount factor $m_{t+1}$.

\medskip \noindent {\bf b.} Compute $E_t (m_{t+1})$ and interpret it.

\medskip \noindent {\bf c.} Compute $E_t (m_{t+1} m_{t+2})$ and interpret it.

\medskip \noindent{\bf d.} Use your answers to parts {\bf b} and {\bf c} to tell how you would expect the term structure
of interest rates to behave over time in this economy.


\medskip
\noindent {\bf e.} Find a value function $U_0$ that satisfies (1), giving conditions
on $\mu, \sigma_c, \sigma_z,  \gamma, \beta$ that ensure that the right side of (1) exists.

\medskip

\noindent{\it Exercise \the\chapternum.10} \quad
\quad
  {\bf Long-run risk, II} \index{multiplier preferences!ambiguity}%
\index{long run risk}%

  \medskip\noindent

  \medskip\noindent
Consider again the recursion \Ep{Trecur2}, namely,
$$
 U_t = c_t - \beta \theta \log E_t \left[ \exp \left( {\frac{-U_{t+1}}{\theta}}\right)\right],
\eqno(1) $$
where $c_t = \log C_t, \beta \in (0,1)$, and $0 < \theta $.
Consumption  $C_t = \exp(c_t)$ follows the stochastic process
$$\eqalign{ c_{t+1} & = \mu + z_t + c_t + \sigma_c \varepsilon_{t+1} \cr
            z_{t+1} & = \rho z_t + \sigma_z \varepsilon_{t+1} ,  } $$
where $\varepsilon_{t+1} \sim {\cal N}(0,1)$ is an i.i.d.\ scalar random process, $\mu > 0$,
and $c_0, z_0$ are  initial conditions, $\rho \in (0,1)$.  By setting $\rho$ close to but less than unity
and $\sigma_z$ to be small, $z_t$ becomes a slowly moving component of the conditional mean  of  $E_t (c_{t+1} - c_t)$.



\medskip
\noindent{\bf a.}  Guess a value function of the form
$U_t = k_0 + k_1 c_t + k_2 z_t$, where $k_0$, $k_1$, and $k_2$ are scalar constants.
In detail, derive formulas for $k_0, k_1, k_2$ that verify the recursion \Ep{Trecur2} (or (1) above).

\medskip
\noindent{\bf b.} ({\it Optional extra credit}) \quad Derive a formula for the stochastic discount factor in this economy.

\medskip


\noindent{\it Exercise \the\chapternum.11} \quad
\quad
  {\bf Stochastic volatility} \index{stochastic volatility}%

  \medskip\noindent

\noindent Consider the following pure exchange economy with one representative consumer and an exogenous
consumption process.
 Consumption $C_t = \exp(c_t)$ follows the stochastic process
$$ c_{t+1}  = \mu + c_t + \sigma_c (s_t) \varepsilon_{t+1}, $$
where $\varepsilon_{t+1} \sim {\cal N}(0,1)$ is an i.i.d.\ scalar random process, $\mu > 0$,
and $c_0, s_0$ are  initial conditions, and $s_t$ is the time $t$ realization of a two state
Markov chain on $\{0,1\}$  with transition matrix  $P = \bmatrix{ \pi_0 & 1 - \pi_0 \cr 1 - \pi_1 & \pi_1 } $, where $\pi_0 \in (0,1)$ and $\pi_1 \in (0,1)$.
It is true that
$$ \sigma_c (s_t) = \cases{ \sigma_L, & if $s_t =  0 $; \cr
                  \sigma_H , & if $s_t = 1 $,\cr}  $$
where $0 < \sigma_L < \sigma_H$. At time $t$, the consumer observes $c_t, s_t$ at time $t$.
The consumer ranks consumption streams  according to
$$ U_0 = E_0 \sum_{t=0}^\infty \beta^t ({\frac{C_t^{(1-\gamma)}}{1-\gamma}} ) ,\eqno(1) $$
where $\beta \in (0,1)$, $E_0$ denotes the mathematical expectation conditional on $C_0, s_0$, and
$\gamma \geq 0$.
%
%\medskip
%\noindent {\bf a.} Find a formula for the consumer's  stochastic discount factor $m_{t,t+1}$.
%
%\medskip \noindent {\bf b.} Compute $E_t (m_{t,t+1})$ and interpret it.
%
%\medskip \noindent {\bf c.} Compute $E_t (m_{t,t+1} m_{t+1,t+2})$ and interpret it.
%
%
%\medskip \noindent{\bf d.} Use your answers to parts {\bf b} and {\bf c} to tell how you would expect the term structure
%of interest rates to behave over time in this economy.


\medskip
\noindent {\bf a.} Define the consumer's  stochastic discount factor $m_{t+1}$.

\medskip
\noindent {\bf b.} Find a formula for the consumer's  stochastic discount factor $m_{t+1}$.

\medskip \noindent {\bf c.} Compute $E_t (m_{t+1})$ and interpret it.

\medskip \noindent {\bf d.} Compute $E_t (m_{t+1} m_{t+2})$ and interpret it.


\medskip \noindent{\bf e.} Use your answers to parts {\bf c} and {\bf d} to tell how you would expect the term structure
of interest rates to behave over time in this economy.


\medskip \noindent{\bf f.} Describe how your approach to answering question {\bf c} would change if, instead of observing $c_t, s_t$ at time $t$, the consumer  observes
only current and past values $c_t, c_{t-1}, \ldots, c_0$ while having a prior distribution $s_0 \sim \tilde \pi_0(s_0)$.

\medskip
\noindent{\it Exercise \the\chapternum.12} \quad
\quad
  {\bf Unknown $\mu$}
  \medskip

\noindent Consider the following pure exchange economy with one representative consumer and an exogenous
consumption process.
 Consumption $C_t = \exp(c_t)$ follows the stochastic process
$$ c_{t+1}  = \mu + z_t + c_t + \sigma_c \varepsilon_{t+1} , $$
where $\varepsilon_{t+1} \sim {\cal N}(0,1)$ is an i.i.d.\ scalar random process, $\mu > 0$,
and $c_0$ is an   initial condition.  The consumer ranks consumption streams according to
$$ U_0 = E_0 \sum_{t=0}^\infty \beta^t  \log (C_t) ,\eqno(1) $$
where $\beta \in (0,1)$, $E_0$ denotes the mathematical expectation conditional on $C_0$, and
$\gamma \geq 0$.  The consumer does not know $\mu$ but at time $0$ believes that $\mu$ is described by
a prior probability density  $\mu \sim {\cal N}(\hat \mu_0, \sigma_\mu^2)$, where $\sigma_\mu > 0$.  At the beginning of
time $t+1$, the consumer has observed the history $\{c_{s+1} - c_s, s=1, \ldots , t\}$.

\medskip
\noindent {\bf a.} Find a formula for the consumer's  stochastic discount factor $m_{t+1}$ for $t \geq 0$.

\medskip \noindent {\bf b.} In this economy, how does the gross rate of return on a one-period risk free bond at time $t$ behave through time?
Can you tell whether it increases or decreases?

%\vfil\eject

\medskip
\noindent{\it Exercise \the\chapternum.13} \quad
\quad
  {\bf Long-run risk, III}
  \medskip

\noindent Consider the following pure exchange economy with one representative consumer and an exogenous
consumption process.
 Consumption $C_t = \exp(c_t)$ follows the stochastic process
$$\eqalign{ c_{t+1} & = \mu + z_t + c_t + \sigma_c \varepsilon_{t+1} \cr
            z_{t+1} & = \rho z_t + \sigma_z \varepsilon_{t+1} ,  } $$
where $\varepsilon_{t+1} \sim {\cal N}(0,1)$ is an i.i.d.\ scalar random process, $\mu > 0$,
and $c_0, z_0$ are  initial conditions, $\rho \in (0,1)$.  By setting $\rho$ close to but less than unity
and $\sigma_z$ to be small, $z_t$ becomes a slowly moving component of the conditional mean  of  $E_t (c_{t+1} - c_t)$.
The consumer ranks consumption streams according to
$$ U_0 = E_0 \sum_{t=0}^\infty \beta^t  \log (C_t) ,\eqno(1) $$
where $\beta \in (0,1)$, $E_0$ denotes the mathematical expectation conditional on $C_0$, and
$\gamma \geq 0$. % The consumer knows $\rho$ but does not know $\mu$.
%The consumer knows the values of $\rho $ and $\Biggl(\bmatrix{\hat \mu_0 \cr \hat z_0}, \Sigma_0\Biggr)$, but observes neither  $\mu$ nor $z_t$.
% At the beginning of period $t$, the consumer does observe $c_t - c_{t-1}$ and remembers
%the history $c_s - c_{s-1}, s = 1, \ldots, t$.
At time $0$ the consumer believes that $\mu$ is described by
a prior probability density  $\mu \sim {\cal N}(\hat \mu_0, \sigma_\mu^2)$, where $\sigma_\mu > 0$.
The consumer never observes $z_t$ and at the start of period $0$ believes
that $z_0$ is distributed independently of $\mu$ and that $z_0 \sim {\cal N}(\hat z_0, \sigma_{z_0}^2)$.    At the beginning of
time $t+1$, the consumer has observed the history $\{c_{s+1} - c_s, s=1, \ldots , t\}$.

\medskip
\noindent {\bf a.} Find a formula for the consumer's  stochastic discount factor $m_{t+1}$ for $t \geq 0$.

\medskip \noindent {\bf b.} Assume the values $(.995, .005,  0, .99, .005, .00005)$ for $(\beta, \hat \mu_0, \hat z_0, \rho, \sigma_c, \sigma_{z_0})$.
Please write a Matlab program to compute the gross rate of return on a one-period risk-free bond for $t = 0, \ldots, 10,000$.  Plot it.  In this economy, how does the gross rate of return on a one-period risk free bond at time $t$ behave through time?

\medskip


\noindent{\it Exercise \the\chapternum.14} \quad
\quad
  {\bf Affine term structure model}
  \medskip
\noindent Recall the affine term structure model of section \use{sec:affine_term_structure1}
 $$ p_t(n) = \exp\left( \bar A_n + \bar B_n z_t \right) .   $$
 Please verify that  $(\bar A_n,\bar B_n)$ can be computed recursively from
 $$ \eqalign{ \bar A_{n+1} & = \bar A_n + \bar B_n'(\mu - C \Lambda_0) + {\frac{1}{2}} \bar B_n' C C' \bar B_n - \delta_0  \cr
                \bar B_{n+1}' & = \bar B_n' (\phi - C \Lambda_z) - \delta_1' ,  }$$
 subject to the initial conditions $\bar A_1 = - \delta_0, \bar B_1 = -\delta_1$.
 {\it Hint:} Apply the formula for the mean of a log normal random variable.

 \medskip

%
\noindent{\it Exercise \the\chapternum.15} \quad
\quad
  {\bf Reverse engineering}

  \medskip
\noindent An econometrician has discovered  that the logarithm of consumption $c_t$ is well described by a stochastic process
$$ c_{t+1} - c_t = \mu + \sigma_c \varepsilon_{t+1} \leqno(0)  $$
where $\varepsilon_{t+1} \sim {\cal N}(0,1)$ is an iid scalar stochastic process.  The econometrician has also discovered
a stochastic discount factor of the
form
$$ \log m_{t+1} = - \delta - (c_{t+1} - c_t)  + w \varepsilon_{t+1} - {\frac{1}{2}} w^2 , \leqno(1) $$
where $\delta > 0, w = \sigma_c(1-\gamma), \gamma \geq 1$.  The stochastic discount factor
 works well in the sense that for a set of assets $j=1, \ldots, J$, GMM estimates of the Euler equations
 $$ E_t (R_{j,t+1} m_{t+1} ) = 1  \leqno(2)  $$
are satisfied to a good approximation.  In addition, the econometrician has found that the gross return $R_{j,t+1}$ on asset $j$ is
well described by
$$ \log (R_{j,t+1})  = \eta_{jt} + \alpha_j \varepsilon_{t+1} +  \sigma_j u_{j,t+1}  -  \left({\frac{1}{2}}\right)  ( \alpha_j^2 + \sigma_j^2),    \leqno(3) $$
where $u_{j,t+1} \sim {\cal N}(0,1)$, $ E u_{j,t+1} \varepsilon_{t+1} = 0$ for all $j$, and $\alpha_1 = 0$.

\medskip

\noindent{\bf a.}  Use  restriction (2) to get a formula for $\eta_{jt}$ as a function of the other parameters in equations (0), (1), (2), and (3).

\medskip
\noindent{\bf b.}  Recalling that $\alpha_1 = 0$, state a formula for $\eta_{1t}$ that is a special case of the formula that you derived in part {\bf a.}
Please interpret $\eta_{1t}$.


\medskip

\noindent{\bf c.}  From your answers to parts {\bf a} and {\bf b}, please derive a formula that relates $\eta_{jt}$ to $\eta_{1t}$.  Please interpret it
in terms of risk prices.

\medskip

\noindent{\bf d.} Reverse engineer an economic model in which $m_{t+1}$ described by (1) reflects the preferences of a representative consumer.

\medskip

\noindent{\bf e.} Describe the representative consumer's  preferences derived in part {\bf d} in the special case in which $\gamma = 1$.

\medskip

\noindent{\bf f.} Describe the representative consumer's  preferences derived in part {\bf d} in the general  case that $\gamma >1$.   Please say what aspects of preferences the parameter $\gamma$ governs (e.g., risk aversion or
intertemporal substitution or yet other things).


\medskip

\noindent{\it Exercise \the\chapternum.16} \quad  {\bf Incomplete Markets, I}



\medskip
\noindent   Consider the following incomplete-markets, pure exchange, two-person economy.
Consumer $i$, $i =1,2$, orders consumption streams $\{C_t^i\}_{t=0}^\infty$
by
$$ E_0 \sum_{t=0}^\infty \beta^t U(C_t^i)  \leqno (1) $$
where  $U(C) = {C^{1-\gamma}\over 1-\gamma}$, $\gamma > 0$,  and $\beta \in (0,1)$.
Consumer $i$ has a stochastic endowment of the consumption good described by
$$ \log Y_{t+1}^i = \log Y_t^i + \mu + \sigma \epsilon_{t+1}^i$$
where $\epsilon_{t+1}^i \sim {\cal N}(0,1)$ for $i=1,2$, and $Y_0^i$ is given for $i=1,2$.   The consumers trade
 a single asset: a risk-free bond whose gross rate of return between $t$ and $t+1$ is
 $R_t= \exp(r_t)$.  Each consumer faces a sequence of budget constraints
$$ C_t^i + R_t^{-1} b^i_{t+1} \leq Y_t^i + b^i _t, \quad t \geq 0 \leqno(2) $$
where $b^i_0 = 0 $ for $i =1,2$, and where $b^i_t$ is person $i$'s holdings of free bonds at the beginning of time $t$.
The gross interest rate $R_t$ is known at time $t$.
Each consumer faces $R_t$ as  a price taker and chooses a stochastic process for $\{C^i_t, b_{t+1}^i \}$ to maximize (1)
subject to (2) and the initial conditions $Y_0^i, b_0^i$.

\medskip

\noindent {\bf a.}  For consumer $i$, please compute a ``personal stochastic discount factor'' evaluated at a no-trade $C_t^i = Y_t^i$ allocation.
For $i \neq  j$, does person $i$'s personal stochastic discount factor equal person $j$'s?
%where  both these  personal discount factors are evaluated at
%a $C_t^i = Y_t^i$ allocation?
Please explain why or why not.

\medskip

\noindent {\bf b.} Where $\log \beta = -\rho$, verify that $C_t^i = Y^i_t, b^i_{t+1} = 0, r_t = \rho +\gamma \mu - .5 \sigma^2 \gamma^2$ are competitive
equilibrium objects for the incomplete markets economy with financial trades only in a  risk free bond.

\medskip

\noindent{\bf c.}  In what sense does this economy display multiple stochastic discount factors? Is there anything that is unique about the stochastic discount
factors?

\medskip
%\vfil\eject

\noindent{\it Exercise \the\chapternum.17} \quad  {\bf Incomplete Markets, II}

\medskip  
\noindent Consider a version of the economy described in exercise {\it 14.16} % \use{\the\chapternum.16}
but in which the endowment processes for the two consumers are now described
by
$$ \eqalign{\log Y_{t+1}^i  & = \log Y_t^i + \mu + x_t + \sigma \epsilon_{t+1}^i \cr
             x_{t+1} & = \lambda x_t + \sigma_x u_{t+1} } $$
where $|\lambda | < 1$, $\epsilon_{t+1} \sim {\cal N}(0,1)$,  and $u_{t+1} \sim {\cal N}(0,1)$.
Everything else about the economy is the same.

\medskip

\noindent{\bf a.}  Show that equilibrium objects are now $C_t^i = Y^i_t, b^i_{t+1} = 0$ and
 $$ r_t = \rho +\gamma (\mu + x_t) - .5 \sigma^2 \gamma^2 .$$

\medskip

\noindent{\bf b.}  Explain why the risk-free net interest rate varies over time while it did not in the problem 16 economy.


\medskip

\noindent{\it Exercise \the\chapternum.18} \quad  {\bf Incomplete Markets, III}

\medskip

\noindent Consider a version of the economy described in exercise {\it 14.16} in which the endowment processes for the two consumers are now described by
$$ \eqalign{\log Y_{t+1}^i  & = \log Y_t^i + \mu + \sigma_{t} \epsilon_{t+1}^i \cr
             \log \sigma_{t+1}^2 & = \log \sigma_{t}^2 + \sigma_\sigma u_{t+1} } $$
where  $u_{t+1} \sim {\cal N}(0,1)$ and $\epsilon_{t+1} \sim {\cal N}(0,1)$.

\medskip

\noindent {\bf a.} Show that there exists an equilibrium net risk-free interest rate process  $\{r_t\}$ such that $C_t^i = Y^i_t, b^i_{t+1} = 0$ are equilibrium objects.
Compute that $r_t$ process.

\medskip
\noindent{\bf b.} During what sorts of periods are risk-free interest rates low? When are they high?



\noindent{\it Exercise \the\chapternum.19} \quad \quad {\bf Relative entropy}
\medskip
\noindent Conditional on $x_t$, under the ``physical measure'', an $n$ dimensional state vector  distribution of $x_{t+1}$
is ${\cal N}(\mu + \phi x_t, CC')$ for all $t \geq 0$. Furthermore, $x_0 \sim {\cal N}(\mu_0, \Sigma_0)$.

\medskip
\noindent {\bf a.} Verify that $\{x_t\}_{t=0}^\infty$ has a vector autoregressive representation
$$ x_{t+1} = \mu + \phi x_t + C \varepsilon_{t+1} ,$$
where $\varepsilon_{t+1} \sim {\cal N}(0,I)$ for all $t\geq 0$.

\medskip

\noindent {\bf b.}  Let $\lambda_t = \lambda_0 + \lambda_1' x_t$. Verify that conditional on $x_t$,
$$ l(\varepsilon_{t+1}; \lambda_t) = \exp( - {\frac{\lambda_t' \lambda_t}{2}} - \lambda_t' \varepsilon_{t+1}) $$
is a log normal random variable with mean unity.

\medskip
\noindent {\bf c.}  Argue that $l_{t+1}$ can be regarded as a likelihood ratio, being careful to describe the probability distributions
in the numerator and denominator of the likelihood ratio.
\medskip

\noindent{\bf d.}  Let $\phi(x; \mu + \phi x_t, \Sigma_x)$ be a multivariate normal density with mean vector $\mu + \phi x_t$ and covariance matrix $\Sigma_x = C C'$.
Verify that multiplying this density by the likelihood ratio $l(\varepsilon_{t+1})$ produces a probability density for $x_{t+1}$ conditional
on $x_t$ that is normal with mean $\mu_x + \phi x_t - C \lambda_t$ and covariance $C C'$.

\medskip
\noindent {\bf e.} Where $l$ is a likelihood ratio, the {\it entropy\/} of a density $l \phi$ relative to a density $\phi$ is
defined as $ E l \log l $, where the expectation is taken under the distribution $\phi$.  But this multiplication by $l$ changes the
density to $l \phi$, so relative entropy also can be represented $\hat E \log (l) $ where the expectation $\hat E$ is taken with respect to
the density $\phi l$.  When $\log l_{t+1} = (-{\frac{\lambda_t' \lambda_t}{2}} - \lambda_t \varepsilon_{t+1}) $ and $\phi_t(\epsilon_{t+1}) \sim {\cal N}(0,I)$,
verify that under the  twisted density $ l_{t+1} \phi_{t+1}$,
$\varepsilon_{t+1} \sim {\cal N}( - \lambda_t , I)$.  Use these findings to verify that relative entropy equals
$ {\frac{\lambda_t' \lambda_t}{2}}$.

\medskip
\noindent{\it Note:} Relative entropy is a non-negative quantity that indicates the discrepancy of one distribution from another.


\medskip

\noindent{\it Exercise \the\chapternum.20} \quad {\bf Likelihood ratio process}
\medskip
\noindent Let $\{\varepsilon_{t+1} \}_{t=0}^\infty$ be a sequence of i.i.d.\ random vectors where $\varepsilon_{t+1} \sim {\cal N}(0,I)$.
Assume that a likelihood process $\{\xi_{t}\}_{t=0}^\infty$  satisfies the recursion
$$ {\frac{\xi_{t+1}}{\xi_t}} = \exp ( - \lambda' \varepsilon_{t+1} - {\frac{\lambda' \lambda}{2}} ), \quad t \geq 0  $$
where $\xi_0 = 1$.

\medskip
\noindent{\bf a.} Verify that
$$ \xi_t = \exp( - \lambda'( \varepsilon_t + \cdots + \varepsilon_1) - t {\frac{\lambda' \lambda}{2}} ) \xi_0 $$
so that $\log \xi_t \sim {\cal N}( -  t {\frac{\lambda' \lambda}{2}},  t \lambda' \lambda )$.
Verify that $E \xi_t = 1$ for all $t \geq 0$ under the assumption that $\{\varepsilon_{t+1} \}_{t=0}^\infty$ is
a sequence of i.i.d.\ random vectors where $\varepsilon_{t+1} \sim {\cal N}(0,I)$.

\medskip
\noindent{\bf b.}  Verify that $\{\xi_{t+1} \}_{t=0}^\infty$ is a martingale under the assumption that $\{\varepsilon_{t+1} \}_{t=0}^\infty$ is
a sequence of i.i.d.\ random vectors where $\varepsilon_{t+1} \sim {\cal N}(0,I)$.


\medskip

\noindent {\bf c.}  Verify that $\xi_t$ is a log normal random variable for $ t \geq 1$.  Please provide formulas for the mean, median, standard deviation, mode,
variance, and skewness of $\xi_t$.

\medskip
\noindent {\bf d.}  {\it Extra credit:}  Write a Python or Julia program to evaluate the cumulative distribution function for a log normal random variable.
Suppose that $\lambda = 1$.  Evaluate the cumulative distribution function (CDF) of $\xi_t$ for $t=1, 5, 100, 1000, 10000$.  Argue informally that as $t \rightarrow + \infty$, $\xi_t$ converges in distribution to $0$
despite the fact that $E \xi_t =1$ for all $t \geq 1$.  How can this be?  ({\it Hint:} Compute the skewness of the distribution of $\xi_t$ or plot the cdf of $\xi_t$ for various
values of the nonnegative integer $t$.)

\medskip


\noindent{\it Exercise \the\chapternum.21} \quad  \quad {\bf Distorted beliefs}

\medskip

\noindent
A representative agent with ``distorted beliefs'' prices assets.   The ``physical measure'' that governs the log of consumption growth is conistent with the law of
motion
$$ c_{t+1} - c_t = \mu + \sigma_c \varepsilon_{t+1} , \quad t \geq 0 \leqno(1) $$
where $\varepsilon_{t+1} \sim {\cal N}(0,1)$.  However, the representative agent believes that the disturbance $\varepsilon_{t+1} $
in  equation (1) actually has distribution ${\cal N}(w, 1)$  where $w$ is a  constant scalar. Call
the probability measure under this distorted belief ``the subjective measure'' and denote the mathematical expectations taken under
this measure $E^S (\cdot)$.
The representative agent uses a stochastic discount factor described by
$$ m_{t+1}^{*} = \exp(- \rho ) \exp( -(c_{t+1} - c_t) ),   $$
so he acts as if he has time separable preferences with CRRA preferences and coefficient of risk aversion equal to $1$.
Assets are priced so that the return $R_{j,t+1}$ on any asset $j = 1, \ldots, J$ obeys
$$  E_t^S m_{t+1}^* R_{j,t+1} = 1,  \leqno(3) $$
 where $ E_t^S $ is a conditional expectation with respect to  subjective  biased beliefs rather than the physical measure.





\medskip
\noindent {\bf a.} Please explain the role of the parameters $\rho$ and $w$ in helping specification (2) attain the Hansen-Jagannathan bounds.

\medskip

\noindent{\bf b.}  A macroeconomist from Minnesota always imposes rational expectations, and so, instead of fitting equation (2) to the data taking into account
the gap between the subjective and physical measure, fits the model
$$ E_t^P m_{t+1} R_{j,t+1}  = 1 , $$
where $E_t^P$ denotes a conditional expectation under the physical measure and $m_{t+1}$ is the stochastic discount factor fit by this rational expectations econometrician.
Please reverse engineer a stochastic discount factor $m_{t+1}$ that this econometrician would he uncover if he managed to fit  data desribed by equation (2) perfectly.

\medskip
\noindent{\bf c.}  Please interpret the stochastic discount factor $m_{t+1}$ discovered in part {\bf a} in terms of the preference structure employed by
Thomas Tallarini (2000).

\medskip

\noindent{\it Exercise \the\chapternum.22} \quad  \quad {\bf Costs of aggregate fluctuations}

\medskip

\noindent
Nature or ``the physical measure'' makes   the level of consumption $C_t = \exp (c_t)$ obey
$$C_{t+1} = \exp( \mu + \sigma_c \varepsilon_{t+1}) C_t, \leqno(1) $$
where $\varepsilon_{t+1}$ is ${\cal N}(0,1)$.


\medskip

\noindent{\bf a.}  Please compute  $E_{t}^P C_{t+j}$ and ${\rm std}_t^P C_{t+j}$ for
$j=1, 2, \ldots$,  where the superscript $P$ denotes the physical measure and ${\rm std}_t$ denotes
a conditional standard deviation.

\medskip
\noindent {\bf b.} A consumer of type $P$ evaluates consumption plans according to the utility recursion
$$ U_t^P = \log  C_t + \beta E_t^P U_{t+1}^P .  \leqno(2)  $$
Please find a formula for $U_t^P$ as a function of time $t$ state variables, which you are free to define.


\medskip
\noindent{\bf c.} A consumer of type $S$ believes that consumption growth is governed by equation
(1) where $\epsilon_{t+1} \sim {\cal N}(w, 1)$, where $w$ is a real valued scalar.
Let $E^S$ be a mathematical expectation under the (subjective) $S$ distribution.
A type $S$ consumer evaluates consumption plans according to the utility recursion
$$ U_t^S = \log  C_t + \beta E_t^S U_{t+1}^S . \leqno(3) $$
Please find a formula for $U_t^S$ as a function of time $t$ state variables, which you are free to define.

\medskip
\noindent{\bf d.} Please find a deterministic consumption process $\{C_t^{d,P}\}_{t=0}^\infty$
process with same expected growth rate as the process $\{C_t\}_{t=0}^\infty$ under the $P$ probability measure, i.e., a nonstochastic
process satisfying a recursion
$$ C_{t+1}^{d,P} = \exp( g^P) C_t^{d,P}  \leqno(4) $$
for all $t \geq 0$.  Tell how $g^P$ varies with $\mu$ and $\sigma_c$.


\medskip
\noindent{\bf e.} Please find a deterministic consumption process $\{C_t^{d,S} \}_{t=0}^\infty$
process with same expected growth rate as the process $\{C_t\}_{t=0}^\infty$ under the $S$ probability measure, i.e., a nonstochastic
process satisfying a recursion
$$ C_{t+1}^{d,S} = \exp( g^S) C_t^{d,S} \leqno(5) $$
for all $t \geq 0$.  Tell how $g^S$ varies with $\mu,\sigma_c$, and $w$.

\medskip
\noindent{\bf f.} A consumer of type $P$ is offered a choice between a consumption process (1) under the $P$ measure starting from a given  known and   $C_0$ and a deterministic
process (4) starting from a value  $\tilde C_0^P$ that gives the same value of $U_0$ satisfying recursion (2). Please compute $\tilde C_0^P$ as a function of $C_0$.


\medskip
\noindent{\bf g.} A consumer of type $S$ is offered a choice between a consumption process (1) under the $S$ measure for $\{\epsilon_{t+1}\}$ starting from a given  known and
$C_0$ and a deterministic
process (5) starting from a value  $\tilde C_0^S$ that gives the same value of $U_0$ satisfying recursion (2). Please compute $\tilde C_0^S$ as a function of $C_0$.



\medskip
\noindent{\bf h.} ({\it Extra credit}) By comparing $U_0^P$ to $U^S$, please tell how much initial consumption $C_0$ a consumer of type $S$ would be willing to pay
to become a consumer of type $P$.  Please interpret this strange experiment as one that compares a situation in which a consumer faces  both risk and ambiguity
with one in which he or she faces   risk only.



\medskip

\noindent{\it Exercise \the\chapternum.23} \quad  \quad {\bf Reverse engineering}

\medskip

\noindent Consider a  setting with complete markets in one-period ahead Arrow securities.  The state at $t$
is $X_t \in {\bf R}^+$. Let $x_t = \log X_t$.   Arrow securities are characterized by the pricing kernel
$$ Q(X_t, X_{t+1}) = {\frac{1}{\sigma_x \sqrt{ 2 \pi}}}
\exp \left( - \rho - \gamma (x_{t+1} - x_t ) + {\frac{(x_{t+1} - x_t - \mu)^2}{2 \sigma_x^2} } \right) .  \leqno(1) $$
Here $\rho > 0, \mu > 0, \sigma_x > 0$.
In this economy, the price at time $t$ of one unit of  consumption at date $t+1$  contingent on $ X_{t+1} \in {\cal A} $ is thus
$$ \int_A Q(X_t, X_{t+1})  d X_{t+1} . $$


\medskip

\noindent{\bf a.} Please reverse engineer an economy in which formula (1) is indeed the kernel for one-period Arrow securities.  Please
describe  preferences, endowments,  technologies, and probability densities  for this economy.

\medskip
\noindent{\bf b.} Please describe the sense in which this economy does or does not have a representative agent.

\medskip

\noindent {\bf c.} Please compute the yield on  one-period risk-free claims.

\medskip

\noindent{\bf d.}  For this economy, please describe a (or the) stochastic discount factor and also a (or the) ``physical measure''.  Is the stochastic discount factor that you
propose unique?

\medskip

\noindent {\bf e.}  Is this model of a stochastic discount factor capable of attaining Hansen-Jagannathan bounds with what you regard to be reasonable parameter values?
Please compute a formula for what Hansen and Jagannathan call ``the market price of risk''.




\medskip

\noindent{\it Exercise \the\chapternum.24} \quad  \quad {\bf Unpriced risk}

\medskip

\noindent Suppose that  the following exponential quadratic stochastic discount factor correctly prices all
returns:
$$ \EQNalign{ \lambda_t & = \lambda_0 + \lambda_z z_t \EQN sdflambda1 \cr
                \log ( m_{t+1}) & = - r_t - {\frac{1}{2}} \lambda_t' \lambda_t - \lambda_t' \varepsilon_{t+1} , }$$
where the risk vector  $\varepsilon_{t+1} \sim {\cal N}(0,I)$ is an i.i.d.\ vector that  appears in the first-order vector autoregression
$$ \EQNalign{ z_{t+1} & = \mu +   \phi z_t + C \varepsilon_{t+1}  \cr
              r_t & = \delta_0 + \delta_1' z_t  .} $$
 Let $u_{t+1} \sim {\cal N}(0,I)$ be  another  i.i.d.\ vector shock that is orthogonal to $\varepsilon_{s+1}$ for all $t$ and all $s$.
 Let  a gross return process $R_{j,t+1}$ be described by
$$ R_{j,t+1} = \exp(\nu_t(j) - {\frac{1}{2}} \alpha_t(j)' \alpha_t(j)  - {\frac{1}{2}} \eta_t(j)' \eta_t(j) + \alpha_t(j)' \varepsilon_{t+1} +
\eta_t(j)'
  u_{t+j}   ) \EQN Rjlaw $$
where
$$ \EQNalign{ \alpha_t(j) & = \alpha_0(j) + \alpha_z(j) z_t \cr
              \eta_t(j) & = \eta_0(j) + \alpha_z(j) z_t . } $$
Here $\alpha_0(j)$ and $\eta_0(j)$  are  $m \times 1$ vectors and $\alpha_z(j)$ and $\eta_z(j)$ are  $m \times m$ matrices.


\medskip
\noindent{\bf a.}  Please verify that $E_t R_{j,t+1} = \exp(\nu_t(j))$.

\medskip

\noindent {\bf b.} Please find a formula for
$\nu_t(j) $ as a function of $z_t$ that verifies $E (m_{t+1} R_{j,t+1})=1$.

\medskip
\noindent {\bf c.}  Please explain why $\nu_t(j)$ depends on $\alpha_t(j)$ the way it does.

\medskip
\noindent {\bf d.} Please explain why $\nu_t(j)$ depends on $\eta_t(j)$ the way it does.




