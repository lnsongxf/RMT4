
\input grafinp3
%\input grafinput8
\input psfig
%\eqnotracetrue

%\showchaptIDtrue
%\def\@chaptID{6.}

%\eqnotracetrue

%\hbox{}

\def\toone{{t+1}}
\def\ttwo{{t+2}}
\def\tthree{{t+3}}
\def\Tone{{T+1}}
\def\TTT{{T-1}}
\def\rtr{{\rm tr}}

\chapter{Recursive Competitive Equilibrium: I \label{recurpe}}
\footnum=0

\section{An equilibrium concept}

This chapter formulates competitive and oligopolistic
equilibria in some dynamic settings.
Up to now, we have studied  single-agent problems where
components of the state vector not under the control
of the agent were taken as given.  In this chapter, we
describe multiple-agent settings in which  components
of the state vector that one agent takes as exogenous are
determined by the decisions of other agents.   We study partial
equilibrium models of a kind applied in
microeconomics.\NFootnote{For example, see Rosen and Topel (1988)  and Rosen, Murphy, and
Scheinkman  (1994)}    We describe
two closely related equilibrium  concepts for such models:
a rational expectations or recursive competitive  equilibrium,
and a Markov perfect equilibrium.   The first
equilibrium concept jointly restricts a Bellman equation and
a transition law that is taken as given in that Bellman
equation.  The second equilibrium concept leads
to pairs (in the duopoly case) or sets (in the oligopoly case)
 of Bellman equations  and transition equations that are  to be
solved by simultaneous backward induction.

   Though the equilibrium concepts introduced in this chapter
 transcend  linear quadratic setups, we choose to
present them in the context of linear quadratic examples
because this renders the    Bellman equations
 tractable.

\auth{Rosen, Sherwin}
\auth{Murphy, Kevin}
\auth{Topel, Robert}
\auth{Scheinkman, Jose}
\auth{Lucas, Robert E., Jr.}
\auth{Prescott, Edward C.}
\auth{Mehra, Rajnish}

\section{Example: adjustment costs}

  This section describes a model of a competitive market with
producers who face adjustment costs.\NFootnote{The model is
a version of one analyzed by Lucas and Prescott (1971) and
Sargent (1987a).  The recursive competitive equilibrium concept
was used by Lucas and Prescott (1971) and described further
by Prescott and  Mehra (1980).} \index{Big $K$, little $k$}%
In the course of the exposition, we
 introduce and exploit  a version of the `big $K$, little $k$' trick that is widely used in macroeconomics and applied economic dynamics.\NFootnote{Also see section
 \use{sec:bigKlittlek} of chapter \use{growth1}.}
 The model consists of $n$ identical firms whose profit function
makes them want to forecast the aggregate output decisions
of other firms just like them in order to choose
their own output.    We assume that $n$ is a large number
so that the output of any single firm has a negligible effect on
aggregate output and, hence, firms are justified in treating their forecast
of aggregate output as unaffected by their own output decisions.
  Thus, one of $n$  competitive firms sells  output $y_t$
and chooses a production plan to maximize
$$ \sum_{t=0}^\infty \beta^t R_t  \EQN comp1 $$
where
$$ R_t = p_t y_t - .5 d (y_{t+1} - y_t )^2 \EQN comp2 $$
subject to $y_0$ being a given initial condition.  Here
$\beta \in (0,1)$ is a discount factor, and $d >0$ measures
a cost of adjusting the rate of output.  The firm is a
price taker.  The price $p_t$ lies on the inverse demand curve
$$ p_t = A_0 - A_1 Y_t \EQN comp3d $$
where $A_0 >0, A_1 > 0 $ and $Y_t$ is the market-wide level of
output, being the sum of output of $n$ identical firms.
The firm believes that market-wide output
follows the law of motion
$$ Y_{t+1} = H_0 + H_1 Y_t \equiv H(Y_t) , \EQN comp3 $$
where $Y_0$ is a known initial condition.  The
belief parameters $H_0, H_1$ are   equilibrium
objects, but for now we proceed on faith and
take them  as given.
The firm observes $Y_t$ and $y_t$ at time $t$ when it chooses
$y_{t+1}$.    The adjustment cost $d(y_{t+1}-y_t)^2$ gives
the firm the incentive to forecast the market price, but since the market price
is a function of market output $Y_t$ via the demand equation \Ep{comp3d}, this in turn motivates the firm to forecast future
values of $Y$.  To state the firm's optimization problem completely requires that
we specify laws of motion for all state variables, including ones like $Y$ that it cares
about but does not control.  For this reason, the perceived law of motion \Ep{comp3} for $Y$ is among the constraints
that the firm faces.

Substituting equation \Ep{comp3d} into equation \Ep{comp2} gives
$$ R_t = (A_0 - A_1 Y_t) y_t - .5 d (y_{t+1} - y_t )^2. $$
The firm's incentive to forecast the market price
translates into an incentive to forecast
the level of market output $Y$.
We can write the Bellman equation for the firm
as
$$ v(y,Y) = \max_{y'} \left\{ A_0 y - A_1 y Y - .5 d (y' - y)^2
      + \beta v(y', Y')\right\}  \EQN comp4$$
where the maximization is subject to the perceived law of motion $Y'=H(Y)$.
Here $'$ denotes next period's value of a variable.
The Euler equation for the firm's problem is
$$ - d(y' - y) + \beta v_y(y',Y') =0. \EQN comp5$$
Noting that for this problem the control is $y'$ and applying
the \idx{Benveniste-Scheinkman formula} from chapter \use{dynamicdp1} %\Ep{130}
 gives
$$ v_y(y,Y) = A_0 - A_1 Y + d(y' - y). $$
Substituting this equation into equation \Ep{comp5} gives
$$ -d(y_{t+1} - y_t) + \beta [A_0 - A_1 Y_{t+1} + d(y_{t+2} - y_{t+1} )] =0.
\EQN comp7  $$

In the process of
solving its Bellman equation,
the firm sets  an output   path  that satisfies equation
\Ep{comp7}, taking equation
 \Ep{comp3} as given,  subject to the initial conditions
$(y_0, Y_0)$ as well as an extra
terminal condition.  The  terminal condition is
$$ \lim_{t \rightarrow \infty } \beta^t y_t v_y(y_{t}, Y_t) = 0. \EQN comp8  $$
This is called the \index{transversality condition}%
transversality condition and acts as
 a first-order necessary  condition  ``at infinity.''
The firm's decision rule solves the difference
equation \Ep{comp7} subject to the given initial condition
$y_0$ and the terminal condition \Ep{comp8}.
Solving  Bellman equation \Ep{comp4} by backward induction automatically
incorporates both equations \Ep{comp7} and \Ep{comp8}.

  The  firm's optimal policy function is
$$ y_{t+1} = h(y_t, Y_t). \EQN comp9 $$
Then with $n$ identical firms, setting $Y_t = ny_t$ makes
 the actual law of motion for
output for the market
$$ Y_{t+1} = n h(Y_t/n, Y_t). \EQN comp9a $$
Thus, when firms believe that the law of motion for market-wide
 output is equation \Ep{comp3}, their optimizing behavior makes
the actual law of motion equation \Ep{comp9a}.

  For this model, we adopt the following
definition:

\specsec{Definition}:  A recursive competitive equilibrium\NFootnote{This
is also often called a rational expectations equilibrium.} of
the model with adjustment costs is a value function
$v(y, Y)$, an optimal policy function
$h(y, Y)$, and   a law of motion $H(Y)$ such that
\medskip
\noindent{\bf a.} Given $H,$ $v(y,Y)$ satisfies the firm's Bellman
equation and $h(y,Y)$ is the optimal policy function.
\medskip
\noindent{\bf b.}  The law of motion $H$ satisfies $H(Y)= nh(Y/n,Y)$.

\medskip
 A recursive competitive equilibrium
\index{recursive!competitive equilibrium}%
equates the actual and perceived laws of motion \Ep{comp3}
and \Ep{comp9a}.
 The firm's optimum problem induces a mapping ${\cal M}$
from a perceived law of motion for output $H$ to an actual
law of motion ${\cal M}(H)$.  The mapping is summarized in equation
\Ep{comp9a}.  The $H$ component of a  rational expectations equilibrium
is a fixed point of the operator ${\cal M}$.

This is a special case
of a recursive competitive equilibrium, to be defined
more generally in section  \use{sec:rce22}.
   How might we find an equilibrium?  The mapping ${\cal M}$ is not a contraction and  there is no guarantee that direct iterations
   on ${\cal M}$ will converge.\NFootnote{A literature that studies whether models populated  with agents who  learn can converge  to rational expectations
   equilibria features iterations on a modification of the mapping ${\cal M}$ that can be approximated as
   $\gamma {\cal M} + (1-\gamma)I$ where $I$ is the identity operator and $\gamma \in (0,1)$ is a relaxation parameter.
     See Marcet and Sargent (1989) and
   Evans and Honkapohja (2001) for statements and applications of this approach to establish conditions under which collections of adaptive agents who use least squares learning converge to a rational expectations equilibrium.  The Marcet-Sargent-Evans-Honkapohja approach provides foundations
   for a method that Krusell and Smith (1998) use to approximation a rational expectations equilibrium of an incomplete-markets economy.
   See chapter \use{incomplete}.\label{FT:MSEH}}
   In fact, in many contexts, including the present one, there exist admissible parameter values for which divergence of iterations on ${\cal M}$
   prevails.  The next subsection
shows another  method that works when the equilibrium solves
an associated planning problem.
For convenience, we'll assume from now on   that the number of firms $n$ is one, while
retaining the assumption of price-taking behavior.
\auth{Honkapohja, Seppo}%
\auth{Evans, George W.}%
\auth{Sargent, Thomas J.}%
\auth{Marcet, Albert}%

\subsection{A planning problem}

Our approach to computing an equilibrium is to seek to match the Euler equations
of the market problem with those for a planning problem that can
be posed as a single-agent dynamic programming problem.
The optimal quantities from  the planning problem are then the
recursive competitive equilibrium quantities, and the equilibrium
price is a  shadow price in the planning problem.

For convenience we set $n=1$.
To construct a planning problem, we first compute
the sum $S_t$ of  consumer and producer surplus at time $t$, defined as
$$ S_t = S(Y_t, Y_{t+1}) = \int_0^{Y_t} (A_0 - A_1 x) d \, x
      -.5 d (Y_{t+1} - Y_t)^2. \EQN comp10 $$
The first term is the area under the demand curve.
The planning problem is to choose a production
plan to maximize
$$   \sum_{t=0}^\infty \beta^t S(Y_t, Y_{t+1}) \EQN comp11 $$
subject to an initial condition $Y_0$.
The Bellman equation for the planning problem is
$$ V(Y) = \max_{Y'}\left\{A_0  Y - {A_1 \over 2} Y^2 - .5 d (Y' - Y)^2
     + \beta V(Y') \right\}. \EQN comp12 $$
The Euler equation is
$$ - d(Y' - Y) + \beta V'(Y') = 0. \EQN comp14 $$
Applying the Benveniste-Scheinkman formula %\Ep{130}
  gives
$$ V'(Y) = A_0 - A_1 Y + d(Y' - Y) . \EQN comp15 $$
Substituting this into equation \Ep{comp14} and rearranging gives
$$ \beta A_0 + d Y_t - [\beta A_1 + d(1+ \beta)]Y_{t+1} + d \beta Y_{t+2} =0 .
         \EQN comp16 $$

Return to equation
\Ep{comp7} and set $y_t = Y_t$ for all $t$. (Remember
that we have set $n=1$. When $n \neq 1$ we have to adjust
pieces of the argument for $n$.)
Notice that with $y_t=Y_t$, equations \Ep{comp16} and \Ep{comp7} are identical.
The Euler equation for the planning problem matches the second-order difference equation
that we derived by first finding the Euler equation of the representative firm and substituting into
it the expression $Y_t = n y_t$ that ``makes the representative firm representative''.  Thus, if
 it is appropriate to apply  the same terminal conditions for these two difference equations, which it is,
then we have verified that  a solution of the planning problem also is an equilibrium.
Setting $y_t = Y_t$ in
equation \Ep{comp7} amounts to dropping equation \Ep{comp3} and instead
solving for the coefficients $H_0, H_1$ that make
$y_t = Y_t$ true and that jointly solve equations
\Ep{comp3} and \Ep{comp7}.

It follows that for this example we can compute an equilibrium
by forming the optimal linear regulator problem corresponding
to the Bellman equation \Ep{comp12}.    The optimal policy
function for this problem is the  law of motion $Y'=H(Y)$ that a firm faces within a rational
expectations equilibrium.\NFootnote{Lucas and Prescott (1971) used the method of this section.  The method exploits
 the
connection between equilibrium and Pareto optimality expressed in
the fundamental theorems of welfare economics. See Mas-Colell,
Whinston, and Green (1995). \auth{Lucas, Robert E., Jr.}
\auth{Prescott, Edward C.} \auth{Mas-Colell, Andreu}
\auth{Whinston, Michael D.} \auth{Green, Jerry R.} } \auth{Mehra,
Rajnish}\index{recursive!competitive equilibrium}
\index{equilibrium!rational expectations}
\index{equilibrium!recursive competitive}
\section{Recursive competitive equilibrium}\label{sec:rce22}%
The equilibrium concept of the previous section
is widely used.  Following Prescott and Mehra (1980),
it is useful to define the  equilibrium concept more  generally
as a {\it recursive competitive equilibrium}.
  Let $x$ be a vector of state variables under the control
of a \idx{representative agent} and let $X$ be the vector of those
same variables chosen by ``the market.''  Let $Z$ be a vector
of other state variables chosen by ``nature,'' that is, determined
outside the model.
The representative agent's problem is
 characterized by the Bellman equation
$$ v(x, X, Z) = \max_u \{ R(x, X, Z, u) + \beta v(x', X', Z') \} \EQN repres1$$
where $'$ denotes next period's value, and where the
maximization is subject to the restrictions:
$$ \EQNalign{ x' & = g(x, X, Z, u) \EQN repres2 \cr
              X' & = G(X,Z)   \EQN repres3 \cr
              Z' & = \zeta(Z) . \EQN repres4 \cr}$$
Here $g$ describes the impact of the representative   agent's
controls $u$ on his state $x'$; $G$ and $\zeta$ describe his beliefs about
the evolution of the aggregate state.  The solution of the
representative agent's    problem is a decision rule
$$ u = h(x, X, Z).  \EQN solution $$

 To make the representative agent representative, we impose
$X = x$, but only ``after'' we have solved the agent's decision
problem.   Substituting equation \Ep{solution} and $X = x_t$ into equation
\Ep{repres2} gives the {\it actual} law of motion
$$ X' = G_A (X, Z), \EQN actual $$
where $G_A(X, Z) \equiv g[X, X, Z, h(X, X, Z)]$.
We are now ready to propose a definition:

\medskip

\specsec{Definition:}  A {\it recursive competitive equilibrium}
is a policy function $h$, an actual aggregate law of motion
$G_A$, and a perceived aggregate
law $G$ such that  (a)  given $G$, $h$ solves   the representative
agent's optimization problem; and (b)  $h$ implies
that $G_A = G$.


\medskip
This equilibrium concept is also sometimes called a {\it rational expectations
equilibrium}.  The equilibrium concept
makes $G$ an outcome.  The functions
giving the  representative
agent's expectations about the aggregate state variables
 contribute no free parameters
and are {\it outcomes} of the analysis.  There are no free parameters that characterize
expectations.\NFootnote{This is the sense in which rational expectations models make
expectations disappear.}
%In  exercise {\it \the\chapternum.1\/},  you are asked to implement this equilibrium
%concept.
\index{expectations!disappearance of}


\section{Equilibrium  human capital accumulation}

As  an example of a recursive competitive equilibrium, we formulate what we  regard as  a schooling model of the type used by Sherwin Rosen.
  \auth{Rosen, Sherwin}%
  A household chooses an amount of
labor to send to a school that takes four periods to produce an educated worker.  Time is a principal input into the schooling technology.
\index{Rosen schooling model}%
\subsection{Planning problem}

A planner chooses a contingency plan for new entrants $n_t$ to maximize
$$ E_0 \sum_{t=0}^\infty \beta^t \biggl\{ f_0 + (f_1 + \theta_t) N_t - {f_2 \over 2} N_t^2 - {d \over 2} n_t^2 \biggr\} $$
subject to the laws of motion
$$ \eqalign{ \theta_{t+1} & = \rho \theta_t + \sigma_\theta \epsilon_{t+1} \cr
             N_{t+1} & = \delta N_t + n_{t-3} ,} \EQN sherwin2000 $$
where $N_t$ is the stock of educated labor at time $t$, $n_t$ is the number of new entrants into school at time $t$,  $\delta \in (0,1)$ is one minus a depreciation rate,  $\theta_t$ is a technology shock, and $\epsilon_{t+1}$ is an i.i.d. random process distributed
as ${\cal N}(0,1)$. The planner confronts initial conditions $\theta_0, N_0, n_{-1}, n_{-2}, n_{-3}$. Notice how \Ep{sherwin2000} incorporates a
 four period time to build  stocks of labor.  The planner's problem can be formulated as a stochastic
discounted optimal linear regulator problem, i.e., a linear-quadratic dynamic programming problem of the type studied in chapter \use{dplinear}.  We ask the reader to verify that it  suffices to take $ X_t = \bmatrix{\theta_t \cr N_{t+3}}$ as the state for the planner's problem.
A solution of the planner's problem is then a law of motion $X_{t+1} = (A-BF)X_t + C \epsilon_{t+1}$ and a decision rule $n_t = -F X_t$.
\index{time to build!Rosen schooling model}%

For the purpose of defining a recursive competitive equilibrium, it is useful to note that it is also possible to define the state for the planner's problem
more profligately as
$\tilde X_t = \bmatrix{ \theta & N_t & n_{t-1} & n_{t-2} & n_{t-3} }'$ with associated decision rule $n_t = -\tilde F \tilde X_t$ and law of motion
$$\tilde X_{t+1} = (\tilde A - \tilde B \tilde F) \tilde X_t + \tilde C \epsilon_{t+1}. \EQN sherwin2009 $$
We can use this representation  to express  a shadow wage $ \tilde w_t = f_1 - f_2 N_t + \theta_t$ as $\tilde w_t = S_w \tilde X_t$.
\subsection{Decentralization}

A firm and a representative household are price takers in a recursive competitive equilibrium. The firm faces a competitive wage process
$\{w_t\}_{t=0}^\infty$ as a price taker and chooses a contingency plan for $\{N_t\}_{t=0}^\infty$ to maximize
$$ E_0 \sum_{t=0}^\infty \beta^t \biggl\{ f_0 + (f_1 + \theta_t) N_t - {f_2 \over 2} N_t^2 - w_t N_t \biggr\}. $$
The first-order condition for the firm's problem is
$$ w_t = f_1 - f_2 N_t + \theta_t ,  \EQN sherwin2001$$
which we can regard as an inverse demand function for the stock of labor.


A representative household chooses a contingency plan for $\{n_t, N_{t+4}\}_{t=0}^\infty$ to maximize
$$ E_0 \sum_{t=0}^\infty \beta^t \biggl\{ w_t N_t - {d \over 2} n_t^2 \biggr\}  \EQN sherwin2004 $$
subject to \Ep{sherwin2000} and initial conditions in the form of given values for $N_t$ for $t = 0, 1, 2, 3$.
To deduce  first order  conditions for this problem, it is helpful first to notice that \Ep{sherwin2000} implies
that for $j \geq 4$,
$$ N_{t+j} = \delta^{j-3} N_{t+1} + \delta^{j-4} n_t + \delta^{j-3} n_{t+1} + \ldots \delta n_{t+j-5} + n_{t+j-4}, \EQN sherwin2002 $$
so that
$$ {\partial \sum_{j=0}^\infty \beta^j w_{t+j} N_{t+j} \over \partial n_t } = \beta^4 \sum_{j=0}^\infty (\beta \delta)^j w_{t+j+4} .$$
It follows that the first-order conditions for maximizing \Ep{sherwin2004} subject to \Ep{sherwin2000} are
$$ n_t = d^{-1} E_t \beta^4 \sum_{j=0}^\infty (\beta \delta)^j w_{t+j+4}, \quad t \geq 0 \EQN sherwin2005 $$
We can regard \Ep{sherwin2005} as a supply curve for a flow of new entrants into the schooling technology. It expresses the supply of new entrants into
school $n_t$
as a linear function of the expected present value of wages.

A rational expectations equilibrium is a stochastic process $\{w_t, N_t, n_t\}$ such that (a) given the $w_t$ process,
$N_t, n_t$ solves the household's problem, and (b) given the $w_t$ process, $N_t$ solves the firms' problem.
Evidently, a rational expectations equilibrium can also be characterized as a $\{w_t, N_t, n_t\}$ process
that equates  demand for labor (equation   \Ep{sherwin2001}) to supply   of  labor (equations \Ep{sherwin2002} and \Ep{sherwin2005}).

To formulate the firm's and household's problems  within a recursive competitive equilibrium, we can guess that the shadow wage $\tilde w_t$ mentioned above equals the
competitive equilibrium wage.  We can then  confront the household with an exogenous
wage governed by the stochastic process for $w_t$ governed
by
the state space representation
$$ \eqalign{ \tilde X_{t+1} & = (\tilde A - \tilde B \tilde F) \tilde X_t + \tilde C \epsilon_{t+1} \cr
                    w_t & = S_w \tilde X_t .} $$

\section{Equilibrium occupational choice}

As  another  example of a recursive competitive equilibrium, we formulate a modification of a  Rosen schooling model designed to focus on occupational choice.\NFootnote{For applications see Siow (1984) and Ryoo and Rosen (2004).}  Like the model in the previous section, this one
focuses on the cost of acquiring human capital via a time-to-build technology.  Investment times  now differ across occupations.
\auth{Siow, A.}
\auth{Rosen, Sherwin}
\index{occupational choice}%

Output of a single good is produced via the following production function:
$$ Y_t = f_0 + f_1 \bmatrix{U_t \cr S_t } - \bmatrix{U_t \cr S_t }' f_2 \bmatrix{U_t \cr S_t }
\EQN sherwin1
$$
where $U_t$ is a stock of skilled labor and $S_t$ is a stock of unskilled labor, and $f_2$ is a positive semi-definite
matrix parameterizing whether skilled and unskilled labor are complements or substitutes in production.    Stocks of the
two types of labor evolve according to the laws of motion
$$\eqalign{ U_{t+1} & = \delta_U U_t + n_{Ut} \cr
            S_{t+1} & = \delta_S S_t + n_{St-2} } \EQN sherwin2 $$
where   flows into the two types of skills are restricted by
$$ n_{Ut } + n_{St} = n_t, \EQN sherwin3 $$
where $n_t$ is an exogenous  flow of new entrants into the labor market governed by
the  stochastic process
$$ n_{t+1} = \mu_n (1-\rho) + \rho n_t + \sigma_n \epsilon_{t+1}  \EQN sherwin4 $$
where $\epsilon_{t+1}$ is an i.i.d. scalar stochastic process with
time $t+1$ component distributed as ${\cal N}(0,1)$.
Equations \Ep{sherwin2}, \Ep{sherwin3}, \Ep{sherwin4} express a time-to-build or schooling  technology for
converting new entrants $n_t$ into increments in  stocks of unskilled labor (this takes one period of
 waiting) and of skilled labor (this takes three periods of  waiting).
Stocks of skilled and unskilled labors depreciate, say through death or retirement, at the rates
$(1-\delta_S), (1-\delta_U)$, respectively, where $\delta_S \in (0,1)$ and $\delta_U \in (0,1)$.
 In addition,
we assume that there is an output cost
of ${e\over 2} n_{st}^2$ associated with  allocating new workers (or `students') to the skilled worker pool.


\subsection{A planning problem}

Let's start with a planning problem, then  construct a competitive equilibrium.
Given initial conditions $(U_0, S_0, n_{S,-1}, n_{S,-2}, n_0)$, a planner chooses $n_{St}, n_{Ut}$ to maximize
$$ E_0 \sum_{t=0}^\infty \beta^t \biggl\{ f_0 + f_1 \bmatrix{U_t \cr S_t } - .5 \bmatrix{U_t \cr S_t }' f_2 \bmatrix{U_t \cr S_t }
    - {e \over 2} n_{St}^2  \biggr\} \EQN sherwin5
$$ subject to \Ep{sherwin2}, \Ep{sherwin3}, \Ep{sherwin4}.  This  is a stochastic discounted optimal linear
regulator problem.  Define the state
as $ X_t = \bmatrix{U_t & S_t & 1 & n_{S,t-1} & n_{S,t-2} & n_t }$ and the control as $n_{St}$.  An optimal
  decision rule has the form
  $n_{St} = - F X_t$ and the law of motion of the state under the optimal decision is
  $$ X_{t+1} = (A-BF) X_t + C \epsilon_{t+1}. \EQN sherwin5 $$

Define shadow wages
$$\bmatrix{\tilde w_{Ut} \cr \tilde w_{St}} = f_1 - f_2  \bmatrix{U_t \cr S_t } \equiv \bmatrix{S_U \cr S_S} X_t ,\EQN sherwin6 $$
where $S_U$ and $S_S$ are the appropriate selector vectors.
The expected present value of
entering school to become an unskilled worker is evidently
$$ E_t \beta \sum_{j=1}^\infty (\beta \delta_U)^{j-1} \tilde w_{U,t+j} = \beta  S_U (I - (A-BF)\beta \delta_U)^{-1} (A-BF)X_t $$
and the expected present value of entering school at $t$  to become
a skilled worker is
$$ E_t \beta^3\sum_{j=3}^\infty (\beta \delta_S)^{j-3} \tilde w_{S,t+j} = \beta^3 S_S (I - (A-BF)\beta \delta_S)^{-1}(A-BF)^3 X_t.$$


\subsection{Decentralization}

We can decentralize the planning problem by finding a recursive competitive equilibrium whose allocation matches that
associated with the planning problem.
A competitive firm hires
stocks of skilled and unskilled workers at competitive wages $w_{St}, w_{Ut}$     each period. Taking those wages as given,
it chooses $S_t, U_t$ to maximize
$$  E_0\sum_{t=0}^\infty \beta^t \biggl\{f_0 + f_1 \bmatrix{U_t \cr S_t } - .5  \bmatrix{U_t \cr S_t }' f_2 \bmatrix{U_t \cr S_t } - w_{Ut} U_t -
w_{St} S_t \biggr\}.   \EQN sherwin6 $$
Notice that the absence of intertemporal linkages in this problem makes it break into a sequence of static problems.  The firm doesn't
have to know the law of motion for wages.
The  firm  equates the marginal products of each type of labor to that type's wage.

  A representative family   faces wages $\{w_{St}, w_{Ut}\}$  as a price taker and chooses contingency
  plans for  $\{n_{St}, U_{t+1}, S_{t+1}\}_{t=0}^\infty $ to maximize
$$ E_0 \sum_{t=0}^\infty \beta^t \biggl\{  w_{Ut} U_t + w_{st} S_t - {e \over 2} n_{St}^2 \biggr\}  \EQN sherwin7 $$
subject to the perceived law of motion for $w_{Ut}, w_{St}$
$$ \eqalign{ \bmatrix{ w_{Ut} \cr w_{St} } & = U_w \tilde X_t \cr
              \tilde X_{t+1} & = \tilde A \tilde X_t + \tilde C \epsilon_{t+1} } \EQN sherwin8 $$
and \Ep{sherwin2} and \Ep{sherwin3}.
According to \Ep{sherwin7}, the family allocates $n_t$ between $n_{Ut}$ and $n_{St}$ to maximize
the expected present value of earnings from both types of labor, minus the present value of `adjustment costs'
${e \over 2} n_{St}^2$. The state vector confronting the representative family
is $\bmatrix{U_t & S_t & \tilde X_t }$ where $\tilde X_t$ has dimension comparable to $X_t$; $\tilde A X + C \varepsilon$ is a perceived
law of motion for $\tilde X$.  In a recursive competitive equilibrium, it will turn out that $\tilde A = A-BF$,
where $A-BF$ is the optimal law of motion obtained from  the planning problem.

\auth{Siow, A.}%
\auth{Rosen, Sherwin}%
In the spirit of   Siow (1984) and Sherwin Rosen, it is interesting to  focus on the  special case in which $e  = 0$.
Here the competitive equilibrium
features the outcome that
$$ \beta E_t \sum_{j=1}^\infty (\beta \delta_U)^j w_{U,t+j} = E_t \beta^3 \sum_{j=3}^\infty (\beta \delta_S)^{j-3} w_{S,t+j}. \EQN sherwin9 $$
This condition says that the family allocates new entrants to equate
the present values of earnings across occupations, a calculation that  takes into account that it takes longer to train for some
occupations than for others.
The laws of motion of competitive equilibrium quantities adjust to equalize the present values
of wages in the two occupations.


\section{Markov perfect equilibrium}

It is instructive to consider a dynamic model of duopoly.
A  market has two firms. Each firm recognizes that its
output decision will affect the aggregate output and therefore influence the
market price. Thus, we drop the assumption of price-taking behavior.\NFootnote{One
consequence of departing from the price-taking framework is that the market
outcome will no longer maximize welfare, measured as the sum of consumer and
producer surplus. See exercise {\it \the\chapternum.4\/} for the case of a monopoly.}
The  one-period return
function of firm $i$ is
$$ R_{it} = p_t y_{it} - .5 d (y_{i t+1} - y_{it})^2.  \EQN game1 $$
There is a demand curve
$$ p_t = A_0 - A_1 (y_{1t} +  y_{2t}) .  \EQN game2 $$
Substituting the demand curve into equation \Ep{game1} lets us express
the return as
$$ R_{it} = A_0 y_{it} - A_1 y_{it}^2 - A_1 y_{it}y_{-i,t}
     - .5 d (y_{it+1} - y_{it})^2 , \EQN game3 $$
where $y_{-i,t}$ denotes the output of the firm other than $i$.
Firm $i$  chooses a decision rule that sets
$y_{it+1}$ as a function of $(y_{it}, y_{-i,t})$  and that  maximizes
$$ \sum_{t=0}^\infty \beta^t R_{it} .$$
Temporarily assume that  the maximizing
decision rule is $y_{it+1}  = f_i(y_{it}, y_{-i,t})$.


Given the function $f_{-i}$, the Bellman equation
of firm $i$ is
$$ v_i(y_{it}, y_{-i,t}) = \max_{y_{it+1}} \left\{
    R_{it} + \beta v_i(y_{it+1}, y_{-i,t+1}) \right\},  \EQN game4 $$
%or
%$$ v_i(y_{it}, y_{-i,t}) = \max_{y_{it+1}} \left\{ (A_0 y_{it} - A_1 y_{it}^2 - A_1 y_{it}y_{-i,t}
%     - .5 d (y_{it+1} - y_{it})^2 ) + \beta v_i(y_{it+1}, y_{-i,t+1}) \right\},  $$
where the maximization is subject to the perceived
decision rule of the other firm
$$ y_{-i,t+1} = f_{-i}(y_{-i,t}, y_{it}). \EQN game5 $$
Note the cross-reference between the two problems for $i=1,2$.

We now advance  the following definition:
\medskip
\specsec{Definition}:
 A Markov perfect equilibrium is a  pair of  value
functions $v_i$ and a pair of policy functions
$f_i$ for $i=1,2$ such that
\medskip
\noindent{\bf a.} Given $f_{-i}$,$v_i$ satisfies
the Bellman equation  \Ep{game4}.
\medskip
\noindent{\bf b.}  The policy function $f_i$ attains the right side
of the Bellman equation \Ep{game4}.
\medskip

  The adjective Markov denotes that the equilibrium decision
rules depend on the current values of the state variables
$y_{it}$ only, not other parts of their histories.   Perfect means `complete', i.e., that
the equilibrium is constructed by backward induction and therefore
builds in optimizing behavior for each firm for all possible
future states, including many that will  not be realized when we iterate
 forward on the pair of equilibrium
strategies $f_i$.

\subsection{Computation}

  If it exists, a Markov perfect equilibrium can be computed
by iterating to convergence on the pair of Bellman equations
\Ep{game4}.  In particular, let $v_i^j,f_i^j$ be the value function
and policy function for firm $i$ at the $j$th iteration.  Then
imagine constructing the iterates
$$ v_i^{j+1}(y_{it}, y_{-i,t}) = \max_{y_{i,t+1}} \left\{
    R_{it} + \beta v_i^{j}(y_{it+1}, y_{-i,t+1}) \right\}, \EQN game4 $$
where the maximization is subject to
$$ y_{-i,t+1} = f^j_{-i}(y_{-i,t}, y_{it}). \EQN game5 $$

  In general, these iterations are difficult.\NFootnote{See Levhari and
Mirman (1980) for how a Markov perfect equilibrium can be computed
conveniently with logarithmic returns and Cobb-Douglas transition
laws. Levhari and Mirman
construct a model of fish and fishers.}
In the next section, we describe how
the calculations simplify for the case in which
the return function is quadratic and the transition laws are linear.
\index{Markov perfect equilibrium!fish and fishers}
\auth{Mirman, Leonard J.}
\auth{Levhari, David}\index{optimal linear regulator!dynamic game}
\index{Markov perfect equilibrium!linear}
\section{Linear Markov perfect equilibria}
In this section, we show how the optimal linear regulator
can be used to solve a model like that in the previous section.
That model should be considered to be
 an example of a dynamic game.  A dynamic  game consists
of these objects: (a) a list of players; (b) a list of dates and
actions available to each player at each date; and (c) payoffs
for each player expressed as  functions of the actions taken
by all players.

  The optimal linear regulator is a good tool for formulating  and solving
dynamic games.  The standard equilibrium
concept---subgame perfection---in these games requires
that each player's strategy be computed
by backward induction.   This leads to an interrelated pair
of Bellman equations. In linear quadratic dynamic games, these
``stacked Bellman equations'' become ``stacked Riccati equations''
with a tractable mathematical structure.
\index{Bellman equation!stacked}
\index{Riccati equation!stacked}
\index{linear quadratic!dynamic games}

We now consider the following two-player, linear quadratic {\it dynamic
game}.  An $(n \times 1)$ state vector $x_t$ evolves according to a
transition equation
$$x_{t+1} = A_t x_t + B_{1t} u_{1t} + B_{2t} u_{2t}  \EQN orig-0$$
where
 $ u_{jt}$ is a $(k_j \times 1)$ vector of controls of
player  $j$.  We start with a finite horizon formulation, where
$t_0$ is the initial date and $t_1$ is the terminal date
for the common horizon of the two players.
Player $1$ maximizes
$$ - \sum_{t=t_0}^{t_1 - 1}  \left( x_t^T R_1 x_t + u_{1t}^T Q_1 u_{1t} +
u_{2t}^T S_1 u_{2t}\right) \EQN orig-1 $$
where $R_1$ and $S_1$ are positive semidefinite and
 $Q_1$ is positive definite.
Player 2 maximizes
$$ - \sum_{t=t_0}^{t_1 - 1} \left( x_t^T R_2 x_t + u_{2t}^T Q_2 u_{2t} +
u_{1t}^T S_2 u_{1t} \right) \EQN orig-2 $$
where $R_2$ and $S_2$ are positive semidefinite and $Q_2$ is positive definite.
\index{Markov perfect equilibrium}

We formulate a Markov perfect equilibrium as follows.  Player
$j$ employs linear decision rules
$$u_{jt} = - F_{jt}  x_t, \ \ t = t_0, \ldots, t_1 - 1$$
where $F_{jt}$ is a $(k_j \times n)$ matrix.  Assume that  player $i$ knows
$\{F_{-i,t}; t = t_0, \ldots, t_1 - 1 \}$.  Then player  1's problem is to
maximize expression
 \Ep{orig-1} subject to the known law of motion \Ep{orig-0}
{\it and\/}
the known control law $u_{2t} = - F_{2t} x_t$ of player 2.
Symmetrically, player 2's problem is to maximize expression
\Ep{orig-2} subject
to equation \Ep{orig-0} and $u_{1t} = - F_{1t} x_t$. A Markov perfect
equilibrium is a pair of
sequences $\{F_{1t}, F_{2t};\, t = t_0, t_0 + 1 , \ldots,
t_1 - 1 \}$ such that $\{F_{1t}\}$ solves player 1's problem, given
$\{F_{2t}\}$, and $\{F_{2t}\}$ solves player 2's problem, given $\{F_{1t}\}$.
We have restricted each player's strategy to depend
only on $x_t$, and not on the {\it history} $h_t =\{(x_s, u_{1s}, u_{2s}),
s = t_0, \ldots, t\}$.
 This restriction on strategy spaces
 accounts for the adjective
``Markov'' in the phrase ``Markov perfect equilibrium.''

Player 1's problem is to maximize
$$ - \sum_{t=t_0}^{t_1 - 1}\Bigl\{ x_t^T (R_1 + F_{2t}^T S_1 F_{2t}) x_t
+ u_{1t}^T Q_1 u_{1t} \Bigr\} $$
subject to
$$x_{t+1} = (A_t- B_{2t} F_{2t}) x_t + B_{1t} u_{1t}. %% + \xi_{1t+1} .
$$
This is an \idx{optimal linear regulator} problem, and it can be solved
by working backward. Evidently, player 2's problem is also an
optimal linear regulator problem.


The solution of player 1's problem is given by
$$F_{1t} = ( B_{1t}^T P_{1t+1} B_{1t} + Q_1)^{-1}  B_{1t}^T P_{1t+1}
(A_t - B_{2t} F_{2t}) \EQN orig-3$$
$$t = t_0, t_0 + 1 , \ldots, t_1 - 1$$
where $P_{1t}$ is the solution of the following matrix Riccati difference
equation with terminal condition $P_{1t_{1}} = 0$:
{ \ninepoint
$$\eqalign{& \, P_{1t}
 = (A_t - B_{2t} F_{2t})^T P_{1t+1} (A_t - B_{2t} F_{2t}) +
 (R_1 +  F_{2t}^T S_1 F_{2t}) \cr
 - &(A_t   - B_{2t} F_{2t})^T P_{1t+1} B_{1t} (B_{1t}^T P_{1t+1} B_{1t} +
Q_1)^{-1} B_{1t}^T P_{1t+1} (A_t - B_{2t} F_{2t}).\cr} \EQN orig-4$$
} % endninepoint
The solution of player 2's problem is
$$F_{2t} = (B_{2t}^T P_{2t+1} B_{2t} + Q_2)^{-1} B_{2t}^T P_{2t+1} (A_t -
B_{1t} F_{1t}) \EQN orig-5$$
where $P_{2t}$ solves the following matrix Riccati difference
equation, with terminal condition $P_{2t_1} = 0$:
$$\eqalign {P_{2t} &= (A_t - B_{1t} F_{1t})^T P_{2t+1} (A_t - B_{1t} F_{1t}) +
(R_2 + F_{1t}^T S_2 F_{1t}) \cr
&- (A_t - B_{1t} F_{1t})^T P_{2t+1} B_{2t} \cr & (B_{2t}^T P_{2t+1} B_{2t} +
Q_2)^{-1} B_{2t}^T P_{2t+1} (A_t - B_{1t} F_{1t}).\cr} \EQN orig-6$$

 The equilibrium sequences $\{F_{1t}, F_{2t}; t =
t_0, t_0 + 1 , \ldots, t_1 - 1\}$ can be calculated from
the pair of coupled Riccati difference equations \Ep{orig-4} and \Ep{orig-6}.
In particular, we  use equations  \Ep{orig-3},
\Ep{orig-4}, \Ep{orig-5}, and \Ep{orig-6} to ``work backward'' from time
$t_1 - 1$.  Notice that given $P_{1t+1}$
and $P_{2t+1}$, equations \Ep{orig-3} and \Ep{orig-5} are a system of
$(k_2 \times n) + (k_1
\times n)$ {\it linear\/} equations in the $(k_2 \times n) + (k_1 \times n)$
unknowns in the matrices $F_{1t}$ and $F_{2t}$.

 Notice how $j$'s control law $F_{jt}$ is a function of $\{F_{is},
s \geq t, i \neq j \}$. Thus, agent $i$'s choice of $\{F_{it}; t = t_0, \ldots,
t_1 - 1\}$ influences agent $j$'s choice of control laws. However, in the
Markov perfect
equilibrium of this game, each agent is assumed
to ignore the influence that his choice exerts on the other agent's
choice.\NFootnote{In an equilibrium of a {\it Stackelberg\/} or {\it dominant
player\/} game, the timing of moves is so altered relative to the present game
that one of the agents called the {\it leader\/} takes
into account the influence that his choices exert on the other agent's
choices. See chapter \use{stackel}.}

  We often want to compute the solutions of such  games for
infinite horizons, in the hope that the decision rules
$F_{it}$ settle down to be time invariant as $t_1 \rightarrow +\infty$.
In practice, we usually  fix $t_1$ and compute the equilibrium
of an infinite horizon game by driving $t_0 \rightarrow - \infty$.   Judd
followed that procedure in the
following example.
\auth{Judd, Kenneth L.}

\subsection{An example}

This section describes the Markov perfect equilibrium of an
infinite horizon linear quadratic game proposed
by Kenneth Judd (1990).
\index{Markov perfect equilibrium!prices and inventory example}%
  The equilibrium is computed by iterating to convergence on the
pair of Riccati equations defined by the choice problems of
two firms.
  Each firm solves a linear quadratic optimization
problem, taking as given and known the sequence of linear decision rules
used by the other player.  The firms set prices and quantities of two
goods interrelated through their demand curves.
  There is no uncertainty.  Relevant variables are defined
as follows:
\smallskip
\item{} $I_{it}$ = inventories of firm $i$ at beginning of $t$.
\item{} $q_{it}$ = production of firm $i$ during period $t$.
\item{} $p_{it}$ = price charged by firm $i$ during period $t$.
\item{} $S_{it}$ = sales made by firm $i$ during period $t$.
\item{} $E_{it}$ = costs of production of firm $i$ during period $t$.
\item{}  $C_{it}$ = costs of carrying inventories for firm $i$ during $t$.
\medskip
\noindent The firms' cost functions are
\smallskip
\item{} $C_{it} = c_{i1} + c_{i2} I_{it} + .5 c_{i3} I_{it}^2$
\item{} $E_{it} = e_{i1} + e_{i2}q_{it} + .5 e_{i3} q_{it}^2$
\smallskip
\noindent where $e_{ij},c_{ij}$ are positive scalars.

\noindent Inventories obey the laws of motion
$$ I_{i,t+1} = (1 - \delta)  I_{it} + q_{it} - S_{it} $$
Demand is governed by the linear schedule
$$S_t = d p_{it} + B$$
where $S_t = \left[\matrix{S_{1t} & S_{2t}\cr}\right]'$,
 $d$ is a $(2\times 2)$ negative definite matrix, and
$B$ is a
vector of constants.  Firm $i$ maximizes the undiscounted sum
$$\lim_{T \to \infty}\ {1 \over T}\   \sum^T_{t=0}\   \left(
p_{it} S_{it} - E_{it} - C_{it} \right) $$
by choosing a decision rule for price and quantity of the form
$$u_{it} = -F_i  x_t$$
where $u_{it} =\left[ \matrix{p_{it} & q_{it}\cr}\right]'$,
 and the state is $x_t=\left[\matrix{I_{1t} & I_{2t}\cr}\right]$.

\mtlb{nnash.m}
\mtlb{nash.m}
In the web site for the book,
we supply a Matlab program {\tt nnash.m} that
computes a Markov perfect equilibrium of
the linear quadratic dynamic game
in which player $i$ maximizes
$$ - \sum_{t=0}^\infty \{ x_t' r_i x_t + 2 x_t' w_i u_{it} +u_{it}' q_i
  u_{it} + u_{jt}' s_i u_{jt} + 2 u_{jt}' m_i u_{it} \}$$
subject to the law of motion
$$x_{t+1} = a x_t + b_1 u_{1t}+b_2 u_{2t}$$
and a control law $u_{jt}= -f_j x_t$ for the other player; here
$a$ is $n \times n$; $b_1$ is $n \times k_1$; $b_2$ is $n \times k_2$;
$r_1$ is $n\times n$; $r_2$ is $ n \times n$;
$q_1$ is $k_1 \times k_1$; $q_2$ is $k_2 \times k_2$;
$s_1$ is $k_2 \times k_2$; $s_2$ is $k_1 \times k_1$;
$w_1$ is  $n \times k_1$;
$w_2$ is $n \times k_2$;
$m_1$ is $ k_2 \times k_1$; and  $m_2$ is $k_1 \times k_2$.
The equilibrium of Judd's model can be computed by
filling in the matrices appropriately.  A Matlab tutorial
{\tt judd.m} uses {\tt nnash.m} to compute the equilibrium.
\mtlb{judd.m}
\mtlb{judd.m}
\mtlb{nash.m}

\section{Concluding remarks}

  This chapter has introduced two
equilibrium concepts and illustrated how dynamic programming
algorithms are embedded in each.   For the linear models we have
used as illustrations,  the dynamic programs become optimal
linear regulators, making it tractable to compute equilibria
even for large state spaces.
We chose to define these equilibria concepts in
partial equilibrium settings
that are more natural for microeconomic applications than for
macroeconomic ones.  In the next chapter, we
use the recursive equilibrium concept to analyze
a general equilibrium in an endowment economy. That setting
serves as a natural starting point for addressing various
macroeconomic issues.

%describe a closely related equilibrium concept
%widely used in macroeconomics.

%\section{Exercises}
\showchaptIDfalse
\showsectIDfalse
\section{Exercises}
\showchaptIDtrue
\showsectIDtrue
\medskip
\medskip
\noindent
These problems aim to teach about (1) mapping problems into
recursive forms, (2) different equilibrium concepts,
and (3) using Matlab.  Computer programs
are available from the web site for the book.\NFootnote{The
web site is $<$www.tomsargent.com/source\_code/mitbook.zip$>$.} % $<$https://files.nyu.edu/ts43/public/books.html$>$.} %$<$http://homepages.nyu.edu/pub/\raise-4pt\hbox{\~{}}ts43/source\hbox{\_{}}code/mitbook.zip$>$.}

\medskip
\noindent {\it Exercise  \the\chapternum.1}\quad {\bf A competitive firm}
\medskip
\noindent A competitive  firm seeks to maximize
 $$\sum_{t=0}^\infty \beta^t R_t \eqno(1) $$
where $\beta \in (0,1)$, and time $t$ revenue $R_t$ is
$$ R_t = p_t y_t - .5 d (y_{t+1} - y_t)^2, \quad d > 0, \eqno(2) $$
where $p_t$ is the price of output, and $y_t$ is the time $t$ output
of the firm.  Here $ .5 d (y_{t+1}- y_t)^2$ measures the firm's
cost of adjusting its rate of   output.     The firm
starts with a given initial level of output $y_0$.  The price lies
on the market demand curve
$$ p_t = A_0 - A_1 Y_t , A_0, A_1 > 0 \eqno(3)$$
where $Y_t$ is the market level of output, which the firm
takes as exogenous, and which the firm believes follows the law of motion
$$ Y_{t+1} = H_0 + H_1 Y_{t}, \eqno(4) $$
with $Y_0$ as a fixed initial condition.

\medskip
\noindent{\bf a.}  Formulate the Bellman equation for the firm's
problem.
\medskip
\noindent{\bf b.} Formulate the firm's problem as a discounted
optimal linear regulator problem, being careful
to
describe all of the  objects needed.
What is the {\it state\/} for the firm's problem?

\medskip
\noindent{\bf c.}   Use the Matlab program {\tt olrp.m}
to solve the firm's problem for the following parameter values:
$A_0= 100, A_1=.05, \beta = .95, d=10, H_0 = 95.5$, and
$H_1 = .95$. Express
the solution of the firm's problem in the
form
$$ y_{t+1} = h_0 + h_1 y_t + h_2 Y_t,  \eqno(5) $$
giving values for the $h_j$'s.
\medskip
\noindent{\bf d.}  If there were $n$ identical competitive
firms all behaving according to  equation (5), what would  equation
(5)  imply for the {\it actual} law of motion
 (4) for the market supply $Y$?

\medskip
\noindent{\bf e.}   Formulate the  Euler equation for the firm's
problem.

\medskip
\noindent{\it Exercise \the\chapternum.2}  \quad {\bf Rational expectations}

\medskip
\noindent Now assume that  the firm in problem 1 is ``representative.''
We implement this idea  by setting $n=1$.
In equilibrium, we will require that $y_t = Y_t$, but we don't want
to impose this condition at the stage that the firm is optimizing (because
we want to retain  competitive behavior).   Define
a rational expectations equilibrium to be a pair of numbers $H_0, H_1$
such that if the representative firm solves   the problem ascribed
to it in problem 1, then the firm's optimal behavior given by equation
(5) implies that $y_t = Y_t \ \forall \ t \geq 0$.

\medskip
\noindent{\bf a.}   Use the program that you wrote for exercise {\it \the\chapternum.1\/}
 to determine which if any  of the following pairs  $(H_0, H_1)$
is a rational expectations equilibrium:
(i) (94.0888, .9211); (ii)  (93.22, .9433), and (iii)  (95.08187459215024,
 .95245906270392)?
%(Answer: set (iii) which implies $(h_0, h_1, h_2) = (95.0819, 1, -.0475)$.

\medskip
\noindent{\bf b.} Describe an iterative algorithm that uses the program that you wrote for exercise {\it \the\chapternum.1\/}
to compute a rational expectations equilibrium.
(You are not being asked actually to use the algorithm you
are suggesting.)

\medskip
\noindent{\it Exercise \the\chapternum.3} \quad{\bf Maximizing welfare}

\medskip
\noindent A planner seeks to maximize the welfare criterion
$$ \sum_{t=0}^\infty \beta^t S_t , \eqno(1) $$
where $S_t$ is ``consumer surplus plus producer surplus''
defined to be
 $$S_t = S(Y_t, Y_{t+1}) = \int_0^{Y_t} (A_0 - A_1 x) d \, x
 - .5 d (Y_{t+1} - Y_t)^2 .$$

\medskip
\noindent{\bf a.}  Formulate the planner's  Bellman equation.

\medskip
\noindent{\bf b.}  Formulate the planner's problem as an optimal
linear regulator, and, for the same parameter values in exercise \the\chapternum.1, solve it using
the Matlab program {\tt olrp.m}.  Represent the solution
in the form $Y_{t+1} = s_0 + s_1 Y_t$.
\medskip
\noindent{\bf c.}   Compare your answer  in  part b with
your answer to part a of exercise {\it \the\chapternum.2\/}.

\medskip
\noindent{\it Exercise  \the\chapternum.4}  \quad {\bf Monopoly}
\medskip
\noindent A monopolist faces the industry demand curve (3)
 and  chooses $ Y_t$ to maximize $ \sum_{t=0}^\infty
\beta^t R_t$ where
$R_t = p_t Y_t - .5 d(Y_{t+1} - Y_t)^2$ and where $Y_0$ is
given.

\medskip
\noindent{\bf a.}  Formulate the firm's
Bellman equation.

\medskip
\noindent{\bf b.}  For the parameter values listed in exercise {\it \the\chapternum.1\/},
 formulate and solve the firm's problem using {\tt olrp.m}.

\medskip
\noindent{\bf c.}  Compare your answer in part b with the answer
you obtained to part b of exercise {\it \the\chapternum.3\/}.

\medskip
\noindent{\it Exercise \the\chapternum.5} \quad {\bf Duopoly}
\medskip
\noindent   An industry consists of two firms that  jointly  face
the industry-wide inverse demand curve $ p_t = A_0 - A_1 Y_t  $, where now
$Y_t = y_{1t} + y_{2t}$.  Firm $i=1,2$ maximizes
$$ \sum_{t=0}^\infty \beta^t R_{it} \eqno(1) $$
where $R_{it} = p_t y_{it} - .5d(y_{i,t+1} - y_{it})^2$.


\medskip
\noindent{\bf a.}  Define a Markov perfect equilibrium for
this industry.
\medskip
\noindent{\bf b.}  Formulate the Bellman equation for each firm.
\medskip
\noindent{\bf c.}  Use the Matlab program {\tt nash.m} to compute
an equilibrium, assuming the parameter values  listed in
exercise {\it \the\chapternum.1\/}. \mtlb{nash.m}

\medskip \auth{Laibson, David I.} \auth{Phelps, Edmund S.}
\auth{Pollak, Robert A.} \auth{Gul, Faruk}
\auth{Pesendorfer, Wolfgang}
\index{self-control!Markov perfect equilibrium}
\index{Markov perfect equilibrium!self-control}
\index{Bellman equation!dynamic games}
\noindent
{\it Exercise \the\chapternum.6} \quad {\bf Self-control}

\medskip\noindent
This is a model of a human who  has time inconsistent preferences,
 of a type proposed by Phelps and Pollak (1968) and used by
Laibson (1994).\NFootnote{See Gul and Pesendorfer (2000) for
a single-agent recursive
representation of preferences  exhibiting temptation  and self-control.}
The human lives from $t=0, \ldots, T$.  Think of the human as
actually consisting of $T+1$  personalities, one for each period.
Each personality is a distinct agent (i.e., a distinct utility
function and constraint set).  Personality $T$ has preferences
ordered by $u(c_T)$ and personality $t < T$
has preferences that are ordered by
$$ u(c_t) + \delta \sum_{j=1}^{T-t} \beta^j u(c_{t+j}), $$ %\EQN laib-1  $$
where $u(\cdot)$ is a twice continuously differentiable,
increasing, and strictly concave function of consumption of
a single good; $\beta \in (0,1)$, and $\delta \in (0,1]$.  When
$\delta <1$, preferences of the sequence of personalities
are time inconsistent (that is, not recursive).
At each $t$, let there be a savings technology described by
$$ k_{t+1} + c_t \leq f(k_t), $$ %\EQN laib0 $$
where $f$ is a production function with $f' >0, f''\leq 0$.
\medskip
\noindent{\bf a.} Define a Markov perfect equilibrium for the
$T+1$ personalities.

\medskip
\noindent
{\bf b.} Argue that the Markov perfect equilibrium can be computed
by iterating on the following functional equations:
$$ \EQNalign{ V_{j+1}(k) & = \max_{c} \left\{u(c) + \beta
            \delta W_j(k') \right\}  \cr %\EQN laib1;a \cr
          W_{j+1}(k) & = u[c_{j+1}(k)] + \beta W_j[f(k)- c_{j+1}(k)] \cr}$$
          %\EQN laib2 \cr}$$
where $c_{j+1}(k)$ is the maximizer of the right side of
the first equation above  %\Ep{laib1;a}
for $j+1$, starting from $W_0(k) = u[f(k)]$.
Here $W_j(k)$ is the value of $u(c_{T-j}) + \beta u(c_{T-j+1}) +
\ldots + \beta^{T-j} u(c_T)$, taking the decision
rules $c_h(k)$ as given for $h=0, 1, \ldots, j$.

\medskip
\noindent
{\bf c.}  State the optimization problem of the  time $0$
person who is given the power to dictate the choices
of all subsequent persons. Write the Bellman equations for
this problem. The time $0$ person is said to have a commitment
technology for ``self-control'' in this problem.


\medskip
\noindent{\it Exercise \the\chapternum.7} \quad {\bf Equilibrium search}
\medskip
\noindent An economy consists of a continuum of {\it ex ante\/} identical
workers each of whom is either employed or unemployed.  A worker wants
to maximize  the expected value of $\sum_{t=0}^\infty \beta^t y_t$ where $\beta \in (0,1)$
and
%
%$$J_j(a,a') = \cases{ 1 &  if \ $g(a,s_j) = a'$ \cr
%                      0 &  otherwise .} $$
$$y_t = \cases{ w &  if \ employed \cr
                      c(U) & if \ unemployed .} $$
Each period,
an unemployed worker draws one and only one offer to work (until fired)
at a wage $w$ drawn from a c.d.f.\ $F(W) = {\rm Prob} (w \leq W)$
where $F(0)=0, F(B) = 1$ for $B >0$.  Successive draws from $F$ are
i.i.d.  If a worker accepts a job, he receives $w$ this period and enters
the beginning of next period as `employed'.  At the beginning of each
period, each such previously employed worker is exposed to a probability
of $\lambda \in (0,1)$ of being fired; with probability $1-\lambda$ he is not fired and again
receives the previously drawn $w$ as a wage.  If fired, the worker
becomes newly unemployed and has the same opportunity as all other unemployed workers, i.e.,
he draws an offer $w$ from c.d.f.\ $F$.  If an unemployed worker rejects that offer, he receives unemployment
compensation $c(U) = c\bigl[ {\frac{1}{1+ \exp(-6U)}}- .5 \bigr] $ and enters next period unemployed.
Here $U$ is the aggregate unemployment rate at the beginning of the period.  The unemployment
rate tomorrow $U^*$ is related to the unemployment rate $U$ today by the law of motion
$$ U^* = \lambda (1-U) + (1-\phi(U)) U ,$$
where $\phi(U)$ is the fraction of unemployed workers who accept a wage offer this period.
\medskip

\noindent{\bf a.} Write a Bellman equation for an unemployed worker.
\medskip

\noindent{\bf b.} Describe the form of an unemployed worker's optimal decision rule.

\medskip
\noindent{\bf c.} Describe how $\phi(U)$ is
implied by  a typical worker's optimal decision rule.

\medskip
\noindent{\bf d.}  Define a recursive competitive equilibrium for this environment.


