
\input grafinp3
%\input grafinput8
\input psfig
%\eqnotracetrue


%\showchaptIDtrue
%\def\@chaptID{7.}

%\eqnotracetrue

%\hbox{}

\def\toone{{t+1}}
\def\ttwo{{t+2}}
\def\tthree{{t+3}}
\def\Tone{{T+1}}
\def\TTT{{T-1}}
\def\rtr{{\rm tr}}

\chapter{Dynamic Stackelberg Problems\label{stackel}}
\footnum=0

\section{History dependence}

  Except for chapter \use{optax}, previous chapters described decision problems that are recursive in
what we can call ``natural'' state variables, meaning state variables that
describe stocks of capital, wealth, and information that helps forecast
future values of prices and quantities that impinge on  utilities
or profits.  In problems that are recursive in the natural state variables,
optimal decision rules are functions of the natural state variables.

\index{time consistency}
%  This chapter studies a class of
% problems that are not recursive in the natural state variables.
Kydland and Prescott (1977)  and Calvo (1978) gave
macroeconomic examples of decision problems
that %whose solutions  exhibited {\it time inconsistency\/}
are not  recursive in natural state variables.
At time $0$, a government
chooses  actions for all $t \geq 0$, knowing that it  confronts  a competitive market composed of many small private
agents
whose decisions are influenced by their
{\it forecasts\/} of the government's future actions. In particular, what private agents choose to do at date $t$ depends
 partly on what they expect the government  to do at dates $t+j, \ \forall t \geq 0$.   In a rational expectations equilibrium in a nonstochastic setting,
 the government's actions at time $t \geq 1$ equal private agents' earlier forecasts of
those actions.
Knowing that, the government uses its time $t \geq 1$ actions
to influence earlier actions by private agents.  The rational expectations equilibrium concept requires that the government  confirm
private sector forecasts. That
prevents the government's decision problem from being
recursive in  natural state variables and makes  the government's
decision rule at $t$ depend on the {\it history\/} of the natural
state variables from time $0$ to time $t$.

 \auth{Kydland, Finn E.} \auth{Prescott, Edward C.}%
It took  time for economists to learn how to formulate policy problems of this type
recursively.
Prescott  (1977) asserted that
 recursive optimal control theory (i.e., dynamic programming) did not apply
to problems with this structure.  This chapter and chapters \use{optaxrecur},
\use{socialinsurance}, and \use{credible} describe
how Prescott's initial pessimism about the inapplicability
of optimal control theory was overturned.\NFootnote{The important contribution by Kydland and Prescott (1980)
 dissipated
 Prescott's initial pessimism.}
An important finding is that if the natural state variables
are augmented with appropriate `forward looking' state variables, this class of problems
can be made recursive. This affords
computational advantages and yields substantial insights.
  This chapter displays these
within the tractable framework of linear-quadratic problems.

\section{The Stackelberg problem}

To exhibit the essential structure of
the decision problems that concerned Kydland and Prescott (1977) and Calvo (1979),
this chapter uses the optimal linear regulator problem of chapter \use{dplinear}
to solve a linear-quadratic version of
what is known as a dynamic Stackelberg problem.\NFootnote{In some settings
it is  called a Ramsey problem. See chapters \use{optax} and \use{optaxrecur}.}
For now we refer to the Stackelberg leader as a government and
the Stackelberg follower as a representative agent or
private sector.  In section \use{sec:monopfringe},   we'll give
an application with another  interpretation of these two types of agent.


Let $z_t$ be  an $n_z \times 1$  vector of natural state variables,
$x_t$ an $n_x \times 1$  vector of  endogenous forward-looking variables, and $u_t$ a vector of government choice variables.
Included in $x_t$ are prices and quantities that adjust
instantaneously to clear markets at time $t$.
The $z_t$ vector is inherited from the past.
The vector $x_t$ is  determined purely by {\it future} values of $z$ and $u$.
Nevertheless,   at $t \geq 1$, $x_t$ is  inherited from the past  because
values of $z$ and $u$ for all $ t \geq 0$ are set  by a Stackelberg plan devised at time $0$.
%   The model determinesunder a Stackelberg plan,
%  the ``jump variables'' $x_t$ at time
% $t$.


\specsec{Remark:}  For $t \geq 1$, $x_t$ will turn out to be both a {\it forward-looking\/} and a {\it backward-
looking\/} variable.  It is forward looking because it depends on forecasts of future actions of the Stackelberg leader.
It is backward looking because it is a promise about time $t$ outcomes that was  chosen earlier by the Stackelberg leader.

Let $ y_t = \left[\matrix{z_t \cr x_t \cr} \right]$. Let $u_t$ be a vector of variables chosen by
the government at $t$.
Define the
government's one-period loss function\NFootnote{The
 problem assumes that there
are no cross products between states and controls in the
return function.  There is a simple transformation that converts a problem
whose return function has cross products into an equivalent problem
that has no cross products. For example, see Hansen and Sargent (2008, chapter 4, pp. 72-73).}
$$ r(y, u)  =  y' R y  + u' Q u . \EQN target $$
Subject to an initial condition for $z_0$, but not for
$x_0$, a government at time $0$ wants to % chooses $(x_0, \{u_t, y_t\}_{t=0}^\infty)$  to
 maximize %  chooses $\{y_{t+1}, x_t, u_t \}_{t=0}^\infty$  to maximize
$$ -  \sum_{t=0}^\infty \beta^t r(y_t, u_t) \EQN new1 $$
in light of  the model
$$ \left[\matrix{I & 0 \cr
                 G_{21} & G_{22} \cr}\right]
    \left[ \matrix{   z_{t+1} \cr  x_{t+1} \cr} \right]
  = \left[ \matrix{ \hat A_{11}  &  \hat A_{12} \cr
                    \hat A_{21} & \hat A_{22}  \cr} \right]
\left[\matrix{ z_t \cr x_t \cr} \right]
    + \hat B u_t   . \EQN new2$$
We shall explain the meaning of the second block of equations in system \Ep{new2} soon.
We assume that the matrix on the left is invertible,
so that we can multiply both sides of the above equation by its
inverse to obtain
% \NFootnote{We
% have assumed that the matrix on the left of \Ep{new2} is invertible
% for ease of presentation.
% However, by appropriately using the
% invariant subspace methods described
% under step 2 below (see Appendix \the\chapternum\use{appBblkstack}),
% it is straightforward to adapt the computational
% method when this assumption is violated.}
$$ \left[ \matrix{   z_{t+1} \cr  x_{t+1} \cr} \right]
  = \left[ \matrix{ A_{11}  &   A_{12} \cr
                     A_{21} &  A_{22}  \cr} \right]
   \left[\matrix{ z_t \cr x_t \cr} \right]
    +  B u_t   \EQN new3$$
or
$$ y_{t+1} = A y_t + B u_t  . \EQN new30 $$
At time $0$, the government maximizes \Ep{new1} by choosing
$\{u_t, x_t, z_{t+1}\}_{t=0}^\infty$ subject to
\Ep{new3} and an initial condition for $z_0$.\NFootnote{Miller and Salmon (1982, 1985),
Hansen, Epple, and Roberds (1985), Pearlman, Currie, and Levine (1986),
Sargent (1987a, chapter XIII),
Pearlman (1992), and others have  studied
versions of this problem.}

\medskip

The  optimal decision rule  is history dependent, meaning that
$u_t$ depends not only on $z_t$ but also on lags of $z$.
 History
dependence has two sources: (a) the government's
commitment
 to a sequence of rules at time
$0$,\NFootnote{The government would
make different choices were it
to choose sequentially, that is,  were it to select its time $t$ action
at time $t$.} and
(b) the forward-looking behavior of the
private sector embedded in the second block of equations
\Ep{new3}.

The  second
block of equations of \Ep{new2}  or \Ep{new3}
typically includes first-order conditions
for private  agents' optimization problems (i.e., their Euler equations). These
summarize the forward-looking aspect of private agents' behavior.
We shall provide an example in section \use{sec:monopfringe} in which
the last $n_x$ equations
of \Ep{new3} or \Ep{new30} constitute {\it implementability constraints} that
 are formed by the Euler equations  of a competitive fringe or
private sector.
When combined with a stability condition to be imposed below, these
Euler equations
summarize the private sector's best responses to the
sequence of actions by the government. The government uses its understanding of these responses
to manipulate private sector actions.

To indicate features of the problem that make $x_t$ a vector of forward-looking
variables, write the second block of system \Ep{new2}
as
$$ x_t =  \phi_1 z_t + \phi_2 z_{t+1} +  \phi_3 u_t + \phi_0 x_{t+1} , \EQN eqn:xlawforward $$
where $\phi_0 = \hat A_{22}^{-1} G_{22}$.
The models we study in this chapter typically satisfy

\specsec{Condition A:}
The eigenvalues of $\phi_0$ % = \hat A_{22}^{-1} G_{22}$
are bounded in modulus by $1$.\NFootnote{It will suffice if the eigenvalues of $\phi_0$ are bounded in modulus
by $\beta^{-.5}$.} %${\frac{1}{\sqrt{\beta}}}$.

\medskip

\noindent Condition A makes equation  explosive if ``solved backward'' but stable if ``solved forward''.\NFootnote{See
appendix \use{appa1} of chapter \use{timeseries} and   chapter \use{dplinear}. Also, see Sargent (1987a, chapter IX).} So we solve
equation \Ep{eqn:xlawforward} forward to get
$$ x_t = \sum_{j=0}^\infty \phi_0^j \left[ \phi_1 z_{t+j} + \phi_2 z_{t+j+1} + \phi_3 u_{t+j} \right] . \EQN bell101 $$
In choosing $u_t$ for $t \geq 1$ at time $0$, the government  takes into account how future $z$'s and $u$'s affect earlier
$x$'s through equation \Ep{bell101}.

%
% Let $X^t$ denote the history of any variable $X$  from
% $0$ to $t$. Miller and Salmon (1982, 1985),
% Hansen, Epple, and Roberds (1985), Pearlman, Currie, and Levine (1986),
% Sargent (1987),
% Pearlman (1992), and others have all studied
% versions of the following problem:
% \medskip
% \noindent {\bf Problem S:}
% The  {\it
%  Stackelberg problem\/}
% is to  maximize \Ep{new1} by choosing an $x_0$ and
%  a sequence of decision rules, the time $t$ component of which
% maps the time $t$ history of the state
% $z^t$ into the time $t$ decision $u_t$ of the Stackelberg leader.
%    The Stackelberg leader commits to
% this sequence of decision  rules at time $0$.
% The maximization is subject to a given initial condition
% for $z_0$.
% %But $x_0$ is among the objects  to be chosen by the Stackelberg leader.
% %The jump variables $x_t$ are functions of both the natural state variables $z_t$ and
%the multipliers $\mu_{xt}$.

\specsec{Remark:}
We can regard $x_t$ in equation \Ep{eqn:xlawforward} or \Ep{bell101} as indexing the optimal behavior of the followers
in response to a Stackelberg plan $\{u_t\}_{t=0}^\infty$.  In chapters \use{optaxrecur} and  \use{chang}, we use counterparts to $x_t$ to index
competitive equilibria with distorting taxes.  In equation \Ep{eqn:xlawforward}, the effects of $\{u_{t+j}\}_{j=1}^\infty$ on $x_t$ are all
 intermediated through $x_{t+1}$.


\specsec{Remark:}
The   certainty equivalence principle
stated in chapter \use{dplinear} allows us
to  work with a nonstochastic model.
 We would attain the
same decision rule for the Stackelberg leader   if we were  to replace
$x_{t+1}$ with the forecast $E_t x_{t+1}$ and to add  a shock process
$C \epsilon_{t+1}$ to the right side of \Ep{new3}, where
$\epsilon_{t+1}$ is an i.i.d.\ random vector
 with mean   zero
and identity covariance matrix.


\section{Recursive formulation}\label{recurstack}%
For any vector $a_t$, define $\vec a_t = [a_t, a_{t+1}, \ldots]$.

\specsec{Definition:} Given $z_0$, the Stackelberg problem is to choose $\vec x_0, \vec z_1, \vec u_0$ that maximize
 criterion \Ep{new1} subject to \Ep{new3} for $t \geq 0$.  A Stackelberg plan is a $(\vec x_0, \vec z_1, \vec u_0)$ that solves the Stackelberg problem starting from a given $z_0$. To formulate a Stackelberg problem recursively,
 we formulate two Bellman equations in two sets of  state variables.

\subsection{Two Bellman Equations}
Define a feasible set of $(\vec y_1, \vec u_0)$ sequences
% $$
%    \Omega(y_0) = \left\{ (\vec y_1, \vec u_0) : - \sum_{t=0}^\infty \beta^t r(y_t, u_t) > -\infty \ {\rm and} \ y_{t+1} = A y_t + B u_t, \forall t \geq 0 \right\} $$
   $$
   \Omega(y_0) = \left\{ (\vec y_1, \vec u_0) : \ y_{t+1} = A y_t + B u_t, \forall t \geq 0 \right\} $$
In the definition of $\Omega(y_0)$, $y_0$ is taken as given.
We  express the Stackelberg problem in terms of  two subproblems:

\medskip
\specsec{subproblem 1}

$$   v(y_0) = \max_{(\vec y_1, \vec u_0) \in \Omega(y_0)} - \sum_{t=0}^\infty \beta^t r(y_t, u_t)
   \EQN stacksub1  $$

%\vfil\eject

\medskip
\specsec{subproblem 2}

$$
   w(z_0) = \max_{x_0} v(y_0) \EQN stacksub2 $$



\subsection{Subproblem 1}
The value function $v(y)$ in  subproblem 1 satisfies the Bellman equation

$$    v(y) = \max_{u, y^*} \left\{ - r(y,u) + \beta v(y^*) \right\} \EQN bell1
$$
where the maximization is subject to
$$     y^* = A y + B u,
  \EQN bell2 $$
where  $y^*$ denotes next period's value of $y$.
% The optimal value function $v(y) = - y' P y$ is attained by
% the linear decision rule
% $$ u = - F y $$
% where XXXX
% he linear regulator is
% $$ v(y_0) = -y_0' P y_0
% = {\rm max}_{\{  u_t, y_{t+1}\}_{t=0}^\infty} - \sum_{t=0}^\infty \beta^t
%   \left( y_t' R y_t +   u_t'   Q   u_t \right) \EQN olrp1a $$
% where the maximization is subject to a fixed initial condition for
% $y_0$ and the law of motion\NFootnote{In step 4, we acknowledge that the $x_0$ component
% is {\it not\/} given but is to be chosen by the Stackelberg leader.}
% $$ y_{t+1} = A y_t +   B   u_t . \EQN new30a  $$
 The problem takes the form of a linear quadratic dynamic programming problem, also called  an optimal linear regulator problem in
  chapter \use{dplinear}.  Substituting the (correct) guess  $v(y) = - y'P y$ into  Bellman equation \Ep{bell1}  gives
$$   - y' P y = {\rm max}_{  u, y^*} \left\{ -  y' R y -   u'Q     u - \beta y^{* \prime} P y^* \right\} , $$
which as in chapter \use{dplinear} gives rise to the algebraic matrix Riccati equation
$$    P = R + \beta A' P A - \beta^2 A' P   B (  Q  + \beta   B' P   B)^{-1}   B' P A
    \EQN bell3 $$
and the optimal decision rule  %    the formula for $F$ in the   decision rule
$$    u_t = - F y_t,
  \EQN bell5 $$
where
$$   F = \beta(   Q + \beta   B' P   B)^{-1}  B' P A . \EQN bell4 $$


% Thus, we can solve problem \Ep{new1}, \Ep{new30} by iterating
% to convergence on the difference equation counterpart to the algebraic Riccati equation \Ep{bell3}, or by using
% a faster computational method that emerges as a by-product in
% step 2.  This method is described in Appendix \the\chapternum\use{appBblkstack}.


\subsection{Subproblem 2}

The value function $v(y_0)$ satisfies $v(y) = - y_0' P y_0$ or
$$ v(y_0) = - z_0 ' P_{11} z_0 - 2 x_0' P_{21} z_0 - x_0' P_{22} x_0
     \EQN valuefny $$
where
$$P = \bmatrix{P_{11} & P_{12} \cr P_{21} & P_{22} }.$$
Choose $x_0$ by equating to zero the gradient of $v(y_0)$ with respect to  $x_0$:
$$ - 2 P_{21} z_0 - 2 P_{22} x_0 =0,   $$
which implies that
$$ x_0 = - P_{22}^{-1} P_{21} z_0. \EQN king6x0 $$
We have solved subproblem 2.



\specsec{Remark:}
From chapter \use{dplinear}, recall the formula $\mu_t = P y_t$ for the vector of
shadow prices $\mu_t = \bmatrix{ \mu_{zt} \cr \mu_{xt} }$ on the transition equations.
The shadow price $\mu_{xt}$ evidently equals
$$ \mu_{xt} = P_{21} z_t + P_{22} x_t.   \EQN eqnmux $$
So  \Ep{king6x0} is equivalent with
$$ \mu_{x0} = 0 . \EQN mu0condition  $$
The Lagrange multiplier $\mu_{xt}$ measures the cost to the government at $t \geq 0$
of confirming expectations about its time $t$ action that the followers had
at dates $s < t$.
Setting
$\mu_{x0}=0$ means that at time $0$ there are no prior expectations to confirm.  But
when $\mu_{xt} \neq 0 $ for $ t \geq 1$, it indicates that  it is costly to firm  the private sector's expectations about time $t \geq 1$ actions.
 The government takes these costs into account when it weighs the costs and benefits
of using its choice $ u_{t+j}, j >0$ to influence $x_t \ \forall t \geq 0$.


\subsection{Timing protocol}
Equations \Ep{bell5} and \Ep{king6x0} form a  recursive representation of a Stackelberg plan that
features the following timing protocol:
\smallskip

\item{1.} At times $t \geq 1$,  the government takes $(z_t, x_t)$ as given and chooses $ (u_t, z_{t+1}, x_{t+1}) $.

\medskip

\item{2.} At time $0$,  the government takes $z_0$ as given and chooses $(x_0, u_0, z_1, x_1)$.

\smallskip
\noindent In this timing protocol, the entirely  forward looking vector $x_t$ that obeys \Ep{eqn:xlawforward} is part of the  state vector confronting
the government at times $t \geq 1$ but not at time $t=0$.  It is  presented to the government
at times $t \geq 1$ as a promise to be kept.  The time $t$ government delivers $x_t$  by choosing $u_t, z_{t+1}, x_{t+1}$.\NFootnote{See
exercise \the\chapternum.2 for a  timing protocol that builds in time consistency.}

\subsection{Time  inconsistency}
The two subproblems in section \use{recurstack} express the time inconsistency of the optimal rule. In the recursive representation of
the Stackelberg program, different state variables confront the government at $t=0$, on the one hand, and dates $t \geq 1$, on the other.
At $t =0$, the government faces $z_0$ as a state vector and  chooses the forward looking
vector $x_0$ as well as the forward looking vector $x_1$ that will confront the
government at time $1$.  At dates $t \geq 1$,  the government confronts the state vector $x_t$  as  values promised at time $t-1$ that must
be confirmed at $t$.\NFootnote{Another manifestation of time-inconsistency is that $\mu_{xt}$ is zero at $t=0$ and different
from zero at $t \geq  1$.}
%The multiplier $\mu_{xt}$  accounts for how the optimal plan used $u_{t+j}, j \geq 0$ to influence the followers' choice of
%$x_s$  at dates $s <  t$ and assures  that the Stackelberg leader confirm those expectations.


Define $\vec a_1$ as the continuation of the sequence $\vec a_0$. Recall that a
 Stackelberg plan is a  $(\vec x_0, \vec z_1, \vec u_0)$ that solves the Stackelberg problem starting from a given $z_0$.

\specsec{Time inconsistency:}
  A concise way
to say that a Stackelberg plan is {\it time inconsistent\/} is to note that a continuation of a Stackelberg plan is not
a Stackelberg plan.\NFootnote{Why? Because $x_1$ does not solve subproblem 2 at $z_1$.}
\index{continuation!of a sequence}%
\index{continuation!of a Stackelberg plan}%



%
%\subsection{Digression on determinacy of equilibrium}
%Appendix \the\chapternum\use{appBblkstack} describes methods for  solving a system of difference
%equations of the form \Ep{new2} or \Ep{new3} with an arbitrary feedback rule that
%expresses the decision rule for
%$u_t$ as a function of current and  previous values of $y_t$ and perhaps previous values
%of itself.  The difference equation system has  a unique solution
%satisfying the stability condition $\sum_{t=0}^\infty \beta^t y_t \cdot  y_t$
%if the eigenvalues of the matrix \Ep{symplec2} split, with half being greater than
%unity and half being less than unity in modulus.  If more than half are less than
%unity in modulus, the equilibrium is said to be indeterminate \index{indeterminacy!of equilibrium}%
% in the sense that there are multiple equilibria starting from any initial condition.
%
%If we choose to represent the solution of a Stackelberg or Ramsey problem in the form
%\Ep{vonzer3}, we can substitute that representation for $u_t$ into
%\Ep{new3}, obtain a difference equation system in $y_t, u_t$, and ask whether
%the resulting system is determinate. To answer this question, we would use the method
%of Appendix \the\chapternum\use{appBblkstack}, form system \Ep{symplec2}, then check whether the generalized
%eigenvalues split as required.   Researchers have used this method to study the determinacy
%of equilibria under Stackelberg plans with representations like \Ep{vonzer3} and have discovered
%that sometimes an equilibrium can be indeterminate.\NFootnote{The existence of a Stackelberg plan is not at
%issue because we know how to construct one using the method in the text.}  See Evans
%and Honkapohja (2003) for a discussion of determinacy of equilibria under commitment
%in a class of equilibrium monetary models and
%how determinacy depends on how the decision rule of the Stackelberg leader is represented.  Evans and
%Honkapohja argue that casting a government decision rule in a way that leads to indeterminacy is a bad idea.
%\auth{Evans, George W.}
%\auth{Honkapohja, Seppo}
%\index{implementation!of Stackelberg plan}
\section{Large firm facing a competitive fringe}\label{sec:monopfringe}%
As an example, this section studies the equilibrium of an industry with
a large firm that acts as a Stackelberg leader with respect to a competitive
fringe.\NFootnote{Sometimes the large firm is called `the monopolist' even though there are
actually many firms in the industry.}%
  The industry produces a single nonstorable homogeneous good. One
large firm produces $Q_t$ and a representative firm in a competitive
fringe produces $q_t$.  The representative firm in the competitive
fringe acts as a price taker and chooses sequentially.  The large firm
commits to a policy at time $0$, taking into account its ability to
manipulate the price sequence, both directly through the effects of
its quantity choices on prices, and indirectly through the responses of
the competitive fringe to its forecasts of prices.\NFootnote{Hansen and Sargent
(2012) use this model as a laboratory to illustrate an equilibrium concept featuring
robustness in which at least one of the agents has doubts about the stochastic specification
of the demand shock process.}

  The costs of production are
${\cal C}_t = e Q_t + .5 g Q_t^2+ .5 c (Q_{t+1} - Q_{t})^2 $
for the large firm
and $ \sigma_t= d q_t + .5 h q_t^2 + .5 c (q_{t+1} - q_t)^2$
for the competitive firm,
where $d>0, e >0, c>0, g >0, h>0 $ are cost parameters.
There is a linear inverse demand curve
$$ p_t = A_0 - A_1 (Q_t + \overline q_t) + v_t, \EQN oli1 $$
where $A_0, A_1$ are both positive and  $v_t$ is a disturbance
to demand governed by
$$ v_{t+1}= \rho v_t + C_\epsilon \check \epsilon_{t+1} \EQN oli2 $$
and where $ | \rho | < 1$ and $\check \epsilon_{t+1}$ is an i.i.d.\
sequence of random variables with mean zero and variance $1$.
In \Ep{oli1}, $\overline q_t$ is equilibrium output  of the representative
competitive firm.  In equilibrium, $\overline q_t = q_t$, but we
must distinguish between $q_t$ and $\overline q_t$ in posing the optimum
problem of a competitive firm.

\subsection{The competitive fringe}

 The representative competitive firm regards $\{p_t\}_{t=0}^\infty$
as an exogenous  stochastic process and chooses
an output plan to
maximize
$$ E_0 \sum_{t=0}^\infty \beta^t \left\{
 p_t q_t - \sigma_t
% .5 c(q_{t+1} - q_t)^2 -.5h q_t^2  - d q_t
 \right\}, \quad \beta \in(0,1) \EQN oli3 $$
subject to $q_0$ given, where  %$c>0, d>0, h>0$ are cost parameters,
$E_t$ is the mathematical expectation based on time
$t$ information.
Let $i_t = q_{t+1} - q_t.$  We
regard $i_t$ as the representative firm's control at $t$.  The
first-order conditions
 for maximizing \Ep{oli3} are
$$  i_t =  E_t  \beta i_{t+1} -c^{-1} \beta h  q_{t+1}
  + c^{-1} \beta  E_t( p_{t+1} -d) \EQN oli4   $$
for $t \geq 0$.
We appeal to the  certainty equivalence principle stated on
page \use{certequiv} to justify working with
a non-stochastic version of \Ep{oli4} formed by dropping
the expectation operator and the random term $\check \epsilon_{t+1}$
from \Ep{oli2}.   We use
an insight  of Sargent (1979) and Townsend
(1983).\NFootnote{They used this method to compute a rational expectations
competitive equilibrium.  The  key step was
to eliminate price and output by substituting
  from the inverse demand curve and the production function into
the firm's first-order conditions to get a difference equation
in capital.}
  We shift  \Ep{oli1} forward one period, replace conditional
expectations with realized values,  use \Ep{oli1} to  substitute
for $p_{t+1}$ in  \Ep{oli4}, and set $q_t = \overline q_t$  and $i_t = \overline i_t$ for all
$t\geq 0$ to get
%{\ninepoint
$$ \overline i_t = \beta \overline  i_{t+1}  - c^{-1} \beta h \overline q_{t+1}
 + c^{-1} \beta (A_0-d) - c^{-1} \beta    A_1 \overline q_{t+1}
  -  c^{-1} \beta    A_1 Q_{t+1} + c^{-1} \beta    v_{t+1}. \EQN oli5 $$
%}%endninepoint
Given sufficiently stable sequences $\{Q_t, v_t\}$, we can solve \Ep{oli5}
and $\overline i_t = \overline q_{t+1} - \overline q_t$ to get a second-order difference equation
in $\bar q_t$, then use the method  for constructing a stable solution
of a second order linear difference equation described  in appendix A of chapter \use{timeseries}  to
express the competitive fringe's
output sequence  as a function of the (tail of the)
large firm's output sequence:
$$ \bar q_{t+1} = \lambda \bar q_t +k_0 + k_1 \sum_{j=0}^\infty (\beta \lambda)^j Q_{t+j+1} + k_2 \sum_{j=0}^\infty (\beta \lambda)^j v_{t+j+1} , $$
where $\lambda \in (0,1)$ and the $k_i$s  are constants that are functions of demand and cost parameters.
The dependence of $\bar q_{t+1}$ on future $Q_{t+j+1}$'s opens an avenue for the large firm to influence $\bar q_{t+1}$ by its choice
of  future $Q_{t+j+1}$'s.
  It is this feature that makes the  large firm's problem fail  to be
recursive in the natural state variables $\overline q, Q$. In effect, the large firm arrives
at time $t+j$ {\it not\/} in the position of being able to take past values of $\bar q_t$ as given because these have already
been influenced by the large firm's choice of $Q_{t+j}$.  Instead, the large firm
arrives at period $t >0$  facing the  constraint that it must
 confirm the expectations about its time $t$ decision
upon which the competitive fringe   based its decisions at dates
before $t$.

\subsection{The large firm's problem}

The large firm views the competitive firm's sequence of Euler equations
as  constraints on its own opportunities.
They are {\it implementability constraints\/} on the
large firm's choices.
 Including the implementability constraints \Ep{oli5},
we can represent
the constraints
in terms of the transition law facing the large firm:
%{\ninepoint
$$ \eqalign{ \left[\matrix{ 1 & 0 & 0 & 0 & 0 \cr
                  0 & 1 & 0 & 0 & 0 \cr
                  0 & 0 & 1 & 0 & 0 \cr
                  0 & 0 & 0 & 1 & 0 \cr
                  A_0 -d & 1 & - A_1 & - A_1 -h & c \cr }\right]
   \left[\matrix{ 1 \cr v_{t+1} \cr Q_{t+1} \cr \overline
 q_{t+1} \cr i_{t+1} \cr}
    \right]
  & = \left[ \matrix{ 1 & 0 & 0 & 0 & 0 \cr
             0 & \rho & 0 & 0 & 0 \cr
             0 & 0 & 1 & 0 & 0 \cr
             0 & 0 & 0 & 1 & 1 \cr
             0 & 0 & 0 & 0 & {c\over \beta} \cr} \right]
     \left[ \matrix{ 1 \cr v_t \cr Q_t \cr \overline
    q_t \cr i_t \cr} \right] \cr
& + \left[\matrix{ 0 \cr 0 \cr 1 \cr 0 \cr 0 \cr}\right] u_t
   , \cr}   \EQN oli6 $$
%}%endninepoint
where $u_t = Q_{t+1} - Q_t $ is the control of the large firm.
The last row portrays the implementability constraints \Ep{oli5}.
Represent \Ep{oli6} as
$$ y_{t+1} = A y_t + B u_t .  \EQN oli6a  $$

Although we have included  the competitive fringe's choice variable  $i_t$  as a component
of the ``state''  $y_t$ in the large firm's transition law  \Ep{oli6a},
$i_t$ is actually  a ``jump''
 variable. Nevertheless, the analysis  in earlier sections of this chapter
implies that the  solution of the large firm's
problem is encoded in the Riccati equation associated with
\Ep{oli6a} as the transition law.  Let's decode it.

To match our general setup, we partition $y_t$ as
$y_t' = \left[\matrix{z_t' &  x_t' \cr} \right]$ where
$z_t' = \left[\matrix{ 1 & v_t & Q_t & \overline q_t \cr}\right]$
and $x_t = i_t$.
 The large firm's problem is
$$
\max_{\{u_t, p_{t+1}, Q_{t+1}, \overline q_{t+1}, i_t\}}
 \sum_{t=0}^\infty \beta^t \left\{ p_t Q_t  - {\cal C}_t \right\} $$
subject to  the given initial condition
for $z_0$, equations \Ep{oli1} and \Ep{oli5} and $i_t = \overline q_{t+1} -
\overline q_t$,
 as well as the laws of motion
of the natural state variables $z$.     Notice that the large firm  in effect chooses the
price sequence, as well as the quantity sequence of the
competitive fringe, albeit subject to the restrictions imposed by
the behavior of consumers, as summarized by the demand curve
\Ep{oli1} and the implementability constraint \Ep{oli5} that
describes the best responses  of the competitive fringe.

By substituting \Ep{oli1} into  the above objective function,
the large firm's problem can be expressed as
$$
\max_{\{u_t\}}
 \sum_{t=0}^\infty \beta^t
    \left\{ (A_0 - A_1 (\overline q_t + Q_t) + v_t) Q_t - eQ_t - .5gQ_t^2 -
    .5 c u_t^2
 \right\} \EQN oli7  $$
subject to \Ep{oli6a}.
This can be written
$$
\max_{\{u_t\}}
 -  \sum_{t=0}^\infty \beta^t \left\{ y_t' R y_t +   u_t' Q u_t
   \right\} \EQN oli9 $$
subject to \Ep{oli6a}
where
$$  R =  - \left[\matrix{ 0 & 0 & {A_0-e \over 2} & 0 & 0 \cr
                       0 & 0 & {1 \over 2} & 0 & 0 \cr
                       {A_0-e \over 2} & {1 \over 2} & - A_1 -.5g
                   & -{A_1 \over 2} & 0 \cr
                   0 & 0 & -{A_1 \over 2} & 0 & 0 \cr
                  0 & 0 & 0 & 0 & 0 \cr} \right] $$
and $Q= {c \over 2}$.

%
% \subsection{Equilibrium representation}
%
%  We can use \Ep{king11} to  represent the
% solution of the large firm's problem  \Ep{oli9}  in the form:
% $$ \left[\matrix{z_{t+1} \cr \mu_{x,t+1}\cr}\right]
%    = \left[\matrix{m_{11} & m_{12} \cr
%                    m_{21} & m_{22}\cr}\right]
%      \left[\matrix{z_t \cr \mu_{x,t} \cr} \right]  \EQN oli11 $$
% or
% $$ \left[\matrix{z_{t+1} \cr \mu_{x,t+1}\cr}\right]
%    = m
%      \left[\matrix{z_t \cr \mu_{x,t} \cr} \right] . \EQN oli11 $$
%  The large firm is
% constrained to set $\mu_{x,0} \leq 0$, but will find it optimal to
% set it to zero.
% Recall that $z_t =\left[\matrix{ 1 & v_t & Q_t & \overline q_t \cr}\right]'$.
% Thus, \Ep{oli11}  includes the equilibrium law of motion for the quantity
% $\overline q_t$
% of the competitive fringe.  By construction,  $\overline q_t$ satisfies the Euler
% equation of the representative firm in the competitive fringe, as
% we elaborate in Appendix \the\chapternum\use{appCblkstack}.
%% TTTTTT
\subsection{Numerical example}
We computed the optimal Stackelberg plan
for parameter settings $A_0, A_1, \rho, C_\epsilon,\hfil\break
  c, d, e, g, h,  \beta $ = $100, 1, .8, .2, 1,  20, 20, .2, .2,
.95$.\NFootnote{These calculations were performed
by the Matlab program {\tt oligopoly5.m} or the Python program {\tt oligopoly.py}.}\mtlb{olipololy5.m}%
%, a modification of Stijn and Tom's earlier
%program with robustness.XXXXX}
 For these parameter values, a recursive representation of the Stackelberg plan is
$$u_t = (Q_{t+1} - Q_t) =\left[\matrix{-83.98 & -0.78 &  0.95 &  1.31  &  2.07  \cr}\right]
\left[ \matrix{z_t \cr x_t \cr}\right]  $$
for $ t \geq 0$ and
$$x_0 = \bmatrix{ 31.08 &   0.29  & -0.15  & -0.56     } z_0  .$$


 %Note in Figure  4.5 %\Fg{oli30}
%how starting from $0$ the implementation multiplier
%decreases toward its negative steady state value.  The negative
%value of the multiplier reflects the cost to the large firm
%of adhering to its plan.  The time inconsistency
%of the large firm's plan is reflected in the incentive
%the large firm would have to reset the multiplier to zero
%in any period and thereby reinitialize its plan (see Hansen,
%Epple, and Roberds (1985)).   Figure  4.5 %\Fg{oli30}
%and the other sample paths show that the large firm is acting
%to smooth total output $Q+q$, and that it does so by inducing
%a negative contemporaneous covariance between its own output
%and the price.
%%
%%%%%%%%%
%%\midinsert
%%$$ \grafone{oli10.eps,height=2.5in}
%%{{\bf Figure 4.2.}
%%Impulse response of $p, q, Q, \mu_x$ to innovation to demand
%%shock $\epsilon$.} $$
%%\endinsert
%%%%%%%%
%
%\midfigure{oli10f}
%\centerline{\epsfxsize=3truein\epsffile{oli10.eps}}
%\caption{Impulse response of $p, q, Q, \mu_x$ to innovation to demand shock
%$\epsilon$.}
%\infiglist{oli10f}
%\endfigure
%
%%%%%%%%
%%\midinsert
%%$$\grafone{oli60.eps,height=2.5in}{{\bf Figure 4.3.}
%%Impulse response of  $q+Q, w, v$ to $\epsilon$.} $$
%%\endinsert
%%%%%%%%%%%%
%
%\midfigure{oli60f}
%\centerline{\epsfxsize=3truein\epsffile{oli60.eps}}
%\caption{Impulse response of  $q+Q, w, v$ to $\epsilon$.}
%\infiglist{oli60f}
%\endfigure
%
%%%%%%%%%%%%
%%\midinsert
%%$$
%%\grafone{oli20.eps,height=2.5in}{{\bf Figure 4.4.} Sample path of $q+Q, q, Q$.}
%% $$
%%\endinsert
%%%%%%%%%%%%%%%
%
%\midfigure{oli20f}
%\centerline{\epsfxsize=3truein\epsffile{oli20.eps}}
%\caption{Sample path of $q+Q, q, Q$.}
%\infiglist{oli20f}
%\endfigure
%
%%%%%%%%%%%
%% \midinsert
%%$$\grafone{oli30.eps,height=3in}
%% {{\bf Figure 4.5.}  Sample path of $\mu_x, Q, p$.} $$
%% \endinsert
%%%%%%%%%%%%%%%%%%%%%%%%%%%%%
%
%\midfigure{oli30f}
%\centerline{\epsfxsize=3truein\epsffile{oli30.eps}}
%\caption{Sample path of $\mu_x, Q, p$.}
%\infiglist{oli30f}
%\endfigure
%
%%%%%%%%%%%%%%%%%%
%% \midinsert
%%$$
%%\grafone{oli50.eps,height=3in}{{\bf Figure 4.6.} Sample path of $v, Q, q, p$.}
%%$$
%% \endinsert
%%%%%%%%%%%%%%%%%%%
%
%\midfigure{oli50f}
%\centerline{\epsfxsize=3truein\epsffile{oli50.eps}}
%\caption{Sample path of $v, Q, q, p$.}
%\infiglist{oli50f}
%\endfigure

\section{Concluding remarks}
We shall
confront other problems in which optimal decision rules are history dependent in chapters \use{optaxrecur},  \use{socialinsurance},
\use{socialinsurance2},
and \use{credible} and shall see in various contexts how history
dependence can be represented recursively by
appropriately augmenting the natural state variables with
 forward-looking variables chosen by private agents.\NFootnote{For another application of the techniques
in this chapter and how they related to the method recommended by
Kydland and Prescott (1980), see Evans and Sargent (2013).}
In
chapters \use{socialinsurance}, \use{socialinsurance2}, and \use{credible}, we make dynamic
incentive and enforcement problems recursive  by augmenting the
state with continuation values of other decision
makers.\NFootnote{In chapter \use{socialinsurance}, we describe Marcet and Marimon's (1992, 1999) method of
constructing recursive contracts, which  is closely related to the method
that we have presented in this chapter.}
\auth{Marcet, Albert}%
 \auth{Marimon, Ramon}%
  \auth{Kydland, Finn E.} \auth{Prescott, Edward C.}%
  \auth{Evans, David} \auth{Sargent, Thomas J.}%

%
%\appendix{A}{History-dependent representation of Stackelberg plan}\label{appa1stack}%
%Substituting $u_t = - F y_t$ into the law of motion \Ep{bell2} gives the following motion
%for $y$:
%$$ \bmatrix{ z_{t+1} \cr x_{t+1} } = (A- BF) \bmatrix{ z_t \cr x_t }
%\EQN king11 $$
%or
%$$  \left[ \matrix{ z_{t+1} \cr x_{t+1} \cr} \right]
%   = \left[ \matrix{m_{11} & m_{12} \cr m_{21} & m_{22}\cr} \right]
%    \left[\matrix{ z_t \cr x_t \cr} \right] \EQN vonzer1 $$
%\auth{Von Zur Muehlen, Peter}%
% For present purposes, it is useful to eliminate $x_t$ as an argument of
% the decision rule $u_t = - F \bmatrix{z_t \cr x_t }$ by expressing it as a function of
% $[z_{t-1}, x_{t-1}]$ and then  expressing
%the decision rule for $u_t$ as a function
%of $z_t, z_{t-1},$ and $u_{t-1}$.
%This can be accomplished as
%follows.\NFootnote{Peter Von Zur Muehlen suggested
%this representation to us.} First write the feedback rule \Ep{bell5} for $u_t$
%$$u_t  = -F_1  z_{t} - F_2 x_t . \EQN vonzer2 $$
%Then where $F_2^{-1}$ denotes
%a generalized inverse of $F_2$,
% \Ep{vonzer2} implies $x_t = F_2^{-1}(u_t - F_1 z_t)$.
%Equate the right side of this expression to the right side
%of the second line of \Ep{vonzer1} lagged once and rearrange by using
%\Ep{vonzer2} lagged once to eliminate $\mu_{x,t-1}$
%to get
%% $$ u_t =  f_{12} m_{22} f_{12}^{-1} u_{t-1} + f_{11} z_t
%%    + f_{12}(m_{21} - m_{22} f_{12}^{-1} f_{11}) z_{t-1}
%%    \EQN vonzer3;a $$
%% or
%$$ u_t = \rho u_{t-1} + \alpha_0 z_t + \alpha_1 z_{t-1} \EQN vonzer3 $$
%for $t \geq 1$, where $\rho = F_2 m_{22} F_2^{-1} , \alpha_0 = -F_1 ,
%\alpha_1 =  -F_2(m_{21} - m_{22} F_2^{-1} F_1) $,\NFootnote{By making the instrument feed back on itself,
%the form of \Ep{vonzer3} potentially allows for
%``instrument-smoothing'' to emerge as an optimal rule under
%commitment. This insight partly motivated Woodford
%(2003) to use his model to interpret empirical evidence about
%interest rate smoothing in the United States.} while
%for $t =0$ the decision rule is
%$$ u_0 = -(F_1 + F_2 P_{22}^{-1} P_{21}) z_0.  \EQN vonzer4 $$
%The difference equation \Ep{vonzer3} can be solved backwards subject to the initial condition  \Ep{vonzer4} to deduce a sequence
%of history-dependent decision rules
%$$ u_t = \sigma_t(z^t),  \   t \geq 0 , $$
%where $z^t = \left[ z_t, z_{t-1}, \ldots, z_0 \right]$.
%% For $t=0$, the initialization $\mu_{x,0}=0$ implies
%% that
%% $$ u_0 = f_{11} z_0. \EQN vonzer3;c $$
%



%\section{Exercises}
\showchaptIDfalse
\showsectIDfalse
\section{Exercises}
\showchaptIDtrue
\showsectIDtrue
\medskip
% \noindent{\it Exercise \the\chapternum.1} \quad   There is no uncertainty.
% For $t \geq 0$, a  monetary authority sets the growth of the (log)
% of money according to
% $$ m_{t+1} = m_t + u_t \leqno(1)  $$
% subject to the initial condition $m_0>0$ given.  The demand for money
% is
%   $$  m_t - p_t = - \alpha (p_{t+1} - p_t), \alpha > 0,  \leqno(2)     $$
% where $p_t$ is the log of the price level.  Equation (2) can be
% interpreted as the Euler equation  of the holders of money.
%
% \medskip
% \noindent{\bf a.}  Briefly interpret how equation
% (2) makes the demand for real balances vary inversely with
% the expected rate of inflation.
% Temporarily (only for this part of the exercise) drop
% equation (1) and assume instead that $\{m_t\}$ is a given sequence
% satisfying $\sum_{t=0}^\infty m_t^2 < + \infty$.
% Please solve the difference equation (2) ``forward''
% to express $p_t$ as a function of current and future values of $m_s$.
% Note how future values of $m$ influence the current price level.
%
% \medskip
% At time $0$,  a  monetary authority chooses a possibly
% history-dependent strategy for setting $\{u_t\}_{t=0}^\infty$.  (The monetary
% authority commits to this strategy.)  The monetary authority orders
% sequences $\{m_t, p_t\}_{t=0}^\infty$ according to
% $$ - \sum_{t=0}^\infty .95^t \left[  (p_t - \overline p)^2 +
%     u_t^2 + .00001 m_t^2  \right]. \leqno(3) $$
% Assume that $m_0=10, \alpha=5, \bar p=1$.
% \medskip
% \noindent{\bf b.} Please briefly interpret  this problem
% as one where the monetary authority wants
% to stabilize the price level, subject
% to costs of adjusting the money supply and some implementability
% constraints.    (We include the term $.00001m_t^2$ for purely technical
% reasons that you need not discuss.)
%
% \noindent {\bf c.} Please write and run a Matlab program
% to find the optimal  sequence
% $\{u_t\}_{t=0}^\infty$.
% \medskip
% \noindent {\bf d.}  Display the optimal decision rule for $u_t$
% as a function of $u_{t-1},  m_t, m_{t-1}$.
% \medskip
% \noindent{\bf e.} Compute the optimal $\{m_t, p_t\}_t$
%  sequence for $t=0, \ldots,  10$.
%
% \medskip
% \noindent{\it Hint:} The optimal $\{m_t\}$ sequence must satisfy
% $ \sum_{t=0}^\infty (.95)^t m_t^2 < +\infty$.
% You are free to apply the Matlab program {\tt olrp.m\/}.
% % that is available
% %from the course web site or from Yongs Shin.



\noindent{\it Exercise \the\chapternum.1} \quad There is no uncertainty.
For $t \geq 0$, a  monetary authority sets the growth of the (log)
of money according to
$$ m_{t+1} = m_t + u_t \leqno(1)  $$
subject to the initial condition $m_0>0$ given.  The demand for money
is
  $$  m_t - p_t = - \alpha (p_{t+1} - p_t), \quad \alpha > 0,  \leqno(2)     $$
where $p_t$ is the log of the price level.  Equation (2) can be
interpreted as an Euler equation  of  holders of money.

\medskip
\noindent{\bf a.}  Briefly interpret how equation
(2) makes the demand for real balances vary inversely with
the expected rate of inflation.
Temporarily (only for this part of the exercise) drop
equation (1) and assume instead that $\{m_t\}$ is a given sequence
satisfying $\sum_{t=0}^\infty m_t^2 < + \infty$. Please verify that
equation (2) implies that $p_t = (1-\lambda ) m_t + \lambda p_{t+1}$,
where $\lambda = {\frac{\alpha}{1 + \alpha}} \in (0,1)$.
Please solve this difference equation  ``forward''
to express $p_t$ as a function of current and future values of $m_s$.
{\it Hint:} If necessary, please review appendix A of chapter \use{timeseries}.


\medskip
\noindent At time $0$,  a  monetary authority chooses a possibly
history-dependent strategy for setting $\{u_t\}_{t=0}^\infty$.  (The monetary
authority somehow commits to this strategy once and for all at time $0$.)  The monetary authority orders
sequences $\{m_t, p_t\}_{t=0}^\infty$ according to
$$ - \sum_{t=0}^\infty .95^t \left[  p_t^2 +
    u_t^2 + .00001 m_t^2  \right]. \leqno(3) $$
%$Assume that $m_0=10, \alpha=5$.
\medskip
\noindent{\bf b.} Please briefly interpret  this problem
as one where the monetary authority wants
to stabilize the price level, subject
to costs of adjusting the money supply rapidly and a set of  implementability
constraints.    (We include the term $.00001m_t^2$ for purely technical
reasons that you need not discuss.)

\medskip

\noindent {\bf c.} Please formulate a `dynamic programming squared' problem
to find the optimal  sequence
$\{u_t\}_{t=0}^\infty$. Please tell why it can be called a `dynamic programming squared' problem.

\medskip

\noindent{\bf d.} Define a plan and a continuation of a plan.


\medskip

\noindent {\bf e.} Describe a recursive representation of the optimal plan.


\medskip

\noindent{\bf f.} Tell whether you agree or disagree with the following statement.
``A continuation of an optimal plan is an optimal plan.'' Please describe the logic that causes you to agree or to disagree.

\medskip
\noindent{\bf g.} Please describe formulas to compute all elements of an optimal plan.  (You don't have to write
a Matlab or Python program to implement those formulas, but a Matlab or Python programmer should be able to write a program  based on your formulas.)



\medskip
\noindent{\it Exercise \the\chapternum.2}  \quad  {\bf Markov perfect policy makers}
\medskip
\noindent   Now let's redo the optimal policy problem in  exercise \the\chapternum.1  with timing protocols like those in
the Markov perfect equilibrium concept introduced in chapter 7.
There is a sequence of monetary policy authorities, each in office for only one period.  Let
$$ L(p_t, m_t, u_t) =  \left[  p_t^2 +
    u_t^2 + .00001 m_t^2  \right]. $$
The
policy authority in office at time $t$ chooses $u_t$ to maximize
% $$ - \sum_{j=0}^\infty .95^t \left[  p_{t+j}^2 +
%     u_{t+j}^2 + .00001 m_{t+j}^2  \right]. \leqno(3') $$
$$ - \sum_{j=0}^\infty .95^j L(p_{t+j}, m_{t+j}, u_{t+j}) \leqno(1) $$
subject to
$$ m_{t+1} = m_t + u_t ,$$
taking as given $m_t$. To make the time $t$ decision maker's  problem well posed, we must attribute views about $\{p_{t+j}, m_{t+j}\}_{j=1}^\infty$ to the time
$t$ decision maker. In the Markov perfect spirit, we assume that the time $t$ policy maker takes as given a policy rule $u_{t+j} = \overline g m_{t+j}$ that
it assumes will be chosen by all
successor monetary policy authorities $j \geq 1$.  We also assume that the  date $t$ policy authority believes that $p_{t+j} = \overline h m_{t+j}$ for all $j \geq 1$.
What about the public? If it were to believe that the law of motion of the money supply
is
$$ m_{t+1} = (1+ \overline g) m_t $$
for all $t \geq 0$, then to satisfy the difference equation $p_t = (1-\lambda) m_t + \lambda p_{t+1}$,  it would act to set the price level according to
$$ p_t = \overline h m_t,  \quad {\rm where} \ \overline  h = {\frac{(1-\lambda) }{1 - \lambda (1+\overline g)}}, \leqno(2) $$
and where we computed $\lambda \in (0,1)$ in the previous problem.

\medskip
\noindent{\bf a.} Please verify that equation (2) solves  $p_t = (1-\lambda) m_t + \lambda p_{t+1}$ .
\medskip

\noindent{\bf b.}  Consider the value function
$$ w(m_t) = - \sum_{j=0}^\infty \beta^j L (p_{t+j}, m_{t+j}, u_{t+j}) $$
where
$$ \eqalign{ p_{t+j} & = \overline h m_{t+j} \cr
            u_{t+j} & = \overline g m_{t+j} \cr
            m_{t+j+1} & = m_{t+j} + u_{t+j}  .} $$
Please interpret this value function in terms of the behavior that it assumes about (i) the sequence of monetary policy makers who choose
$\{u_{t+j}\}_{j=0}^\infty$, and (ii) the money holders who choose $\{p_{t+j}\}_{j=0}^\infty$.

\medskip
\noindent{\bf c.}  Now please consider the following functional equation:
$$ v(m_0) = \max_{u_0} \left\{ - L(p_0, m_0, u_0) + \beta w(m_1) \right\} \leqno(3) $$
where the maximization is subject to
$$ \eqalign{ m_1 & = m_0 + u_0  \cr
            p_0 & = \lambda \overline h [ m_0 + u_0] + (1 - \lambda) m_0 . } $$
Please interpret the equation $p_0  = \lambda \overline h [ m_0 + u_0] + (1 - \lambda) m_0 $ in terms of what it assumes about the
beliefs of the money holders who set the price level at time $0$.
Please interpret the functional equation (3) in terms of the beliefs of the time $0$ monetary authority who chooses $u_0$, in particular,
its beliefs about the decisions of successor monetary authorities.

\medskip
\noindent{\bf d.}  Let the optimizer of the right side of equation (3) be $u_0 = g m_0$.
Please define a Markov perfect equilibrium.   Tell who chooses what when.  Also tell what each decision maker assumes about
other pertinent decision makers.

\medskip
\noindent{\bf e.} Please describe a computer algorithm for computing a Markov perfect equilibrium, being careful first to describe all of the objects comprising
a Markov perfect equilibrium.

\noindent{\bf f.}  Is a continuation of a Markov perfect equilibrium a Markov perfect equilibrium?



\medskip

\noindent{\it Exercise \the\chapternum.3} \quad {\bf Duopoly}

\medskip
\noindent
There is  industry with two firms.
 The industry produces a single nonstorable homogeneous good. Firm $i = 1,2$ produces $Q_{it}$.
Costs of production for firm $i$ are
${\cal C}_{it} = e Q_{it} + .5 g Q_{it}^2+ .5 c (Q_{i,t+1} - Q_{it})^2 , $
where $ e >0,  g >0, c>0 $ are cost parameters.
There is a linear inverse demand curve
$$ p_t = A_0 - A_1 (Q_{1t} + Q_{2t} ) + v_t, \EQN oli1 $$
where $A_0, A_1$ are both positive and  $v_t$ is a disturbance
to demand governed by
$$ v_{t+1}= \rho v_t  $$
and where $ | \rho | < 1$. Assume that firm $1$ is a Stackelberg leader and that firm $2$ is a Stackelberg follower.

\medskip
\noindent{\bf a.}   Please formulate the decision problem of firm $2$ and derive  Euler equations that relate its current
decisions to  current and future decisions of firm 1.

\medskip
\noindent{\bf b.} Please formulate the decision problem of firm $1$ as Stackelberg leader.  Please tell how to solve it.

\medskip
\noindent{\bf c.}  Describe some calculations for answering the following question.  Starting from an initial state $Q_{1,0}, Q_{2,0}$ and initial situation in which
firm $1$ acts as Stackelberg leader and firm $2$ acts as follower, how much would firm $1$ be willing to pay to buy out firm $2$ and thereby acquire the ability to act as
the leader?


\medskip
\noindent{\bf d.}  Describe calculations that answer the following question.  Starting from an initial state $Q_{1,0}, Q_{2,0}$ and initial situation in which
firm $1$ acts as Stackelberg leader and firm $2$ acts as follower, how much would firm $2$ be willing to pay to buy  firm $1$ and thereby acquire the ability to act as
a monopolist?





%
%
% \medskip
% \noindent{\it Exercise \the\chapternum.3}  \quad A representative
% consumer has quadratic utility functional
% $$ \sum_{t=0}^\infty \beta^t \left\{ -.5 (b -c_t)^2 \right\} \leqno(1) $$
% where $\beta \in (0,1)$, $b = 30$,  and $c_t$ is time $t$ consumption.
% The consumer faces a sequence of budget constraints
% $$ c_t + a_{t+1} = (1+r)a_t + y_t - \tau_t \leqno(2) $$
% where $a_t $ is the household's holdings of an  asset at the beginning
% of $t$, $r >0$ is a constant net interest rate satisfying $\beta (1+r) <1$, $\tau_t$
% are lump sum taxes at time $t$,
%  and
% $y_t$ is the consumer's endowment at $t$.  The consumer's plan for
% $(c_t, a_{t+1})$  must obey the boundary condition
% $\sum_{t=0}^\infty \beta^t a_t^2 < + \infty$.
% Assume that $y_0, a_0$ are given
% initial conditions and that
% $y_t$ obeys
% $$ y_t = \rho y_{t-1}, \quad t \geq 1,  \leqno(3)$$
% where $|\rho| <1$.
% Assume that $a_0=0$, $y_0=3$, and $\rho=.9$.
%
%
% At time $0$, a    planner commits to a plan
% for taxes $\{\tau_t\}_{t=0}^\infty$.  The planner designs the plan
%    to maximize
% $$ \sum_{t=0}^\infty \beta^t
% \left\{ -.5 (c_t-b)^2 -   \tau_t^2\right\}  \leqno(4) $$
% over $\{c_t, \tau_t\}_{t=0}^\infty$ subject
%  to the implementability constraints
% (2) for $t\geq 0$ and
% $$\lambda_t =  \beta (1+r) \lambda_{t+1} \leqno(5) $$
% for $t\geq 0$, where $\lambda_t \equiv (b-c_t)$.
%
% \medskip
% \noindent{\bf a.}  Argue that (5) is the Euler equation for a consumer
% who maximizes (1) subject to (2), taking $\{\tau_t\}$ as a given sequence.
% \medskip
% \noindent{\bf b.}  Formulate the planner's problem as a Stackelberg problem.
% \medskip
% \noindent{\bf c.}  For $\beta=.95, b=30, \beta(1+r)=.95$,
%   formulate an artificial
% optimal linear regulator  problem and use it to solve the Stackelberg problem.
% \medskip
% \noindent{\bf d.} Give a recursive representation  of the
% Stackelberg plan for $\tau_t$.


\eqnotracefalse
