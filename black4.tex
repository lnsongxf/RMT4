
\input grafinp3
%\input grafinput8
\input psfig
%\eqnotracetrue
%\input epsf.tex
%\showchaptIDtrue
%\def\@chaptID{4.}

\hbox{}

\def\toone{{t+1}}
\def\ttwo{{t+2}}
\def\tthree{{t+3}}
\def\Tone{{T+1}}
\def\TTT{{T-1}}
\def\rtr{{\rm tr}}
\def\be{{\beta}}
\def\eps{{\epsilon}}
\footnum=0
\index{optimal linear regulator}

\chapter{Linear Quadratic Dynamic Programming\label{dplinear}}

\section{Introduction}

This chapter describes the class of dynamic programming
problems in which the return function  is quadratic and
the transition function is linear.  This specification
leads to the widely used optimal linear regulator problem,
for which the Bellman equation can
be solved quickly   using linear algebra.
  We consider the special case in which the
return function  and transition function  are both time
invariant, though the mathematics is almost identical when they are
permitted to be deterministic functions of time.
\index{dynamic programming!linear quadratic}%
\index{linear quadratic!dynamic programming}

After studying a recursive formulation and the associated Bellman equation,
in section \use{lagrangianformulation} we analyze a Lagrangian formulation that provides
useful insights about  how Lagrange multipliers on transition laws relate to
 gradients of  value functions.  These insights help us in chapter \use{stackel}
when we study how the methods of this chapter apply to problems in which  a Stackelberg leader chooses a sequence of actions
 to manipulate
future  decisions of a collection agents whose decisions depend on forecasts of the leader's decisions.
In that chapter, we shall get a sharp characterization of the time inconsistency of a Stackelberg plan.

  In section \use{sec:Kalmanch4} we tell how the Kalman  filtering problem from chapter \use{timeseries}
relates to the linear-quadratic dynamic programming problem.  Suitably reinterpreted,
formulas that solve the optimal linear regulator are
the Kalman filter.
\index{Kalman filter!and optimal linear regulator}%


\section{The optimal linear regulator problem}

 The undiscounted optimal linear regulator problem
is to maximize over choice of $\{u_t\}_{t=0}^\infty$ the criterion
$$ - \sum_{t=0}^\infty \{x'_t Rx_t+u'_tQu_t\},\EQN 141$$
subject to $x_{\toone}=Ax_t+Bu_t$, $x_0$ given.  Here $x_t$ is an
$(n\times 1)$ vector of state variables, $u_t$ is a $(k\times 1)$
vector of controls, $R$ is a positive semidefinite symmetric matrix,
$Q$ is a positive definite symmetric matrix, $A$ is an $(n\times n)$
matrix, and $B$ is an $(n\times k)$ matrix.

We guess that the
value function is quadratic, $V(x)= - x'Px$, where $P$ is a positive
semidefinite symmetric matrix.
Using the transition law to eliminate next period's state, the Bellman
equation becomes
$$ - x'Px=\max_u \{- x' Rx-u'Qu-(Ax+Bu)' P(Ax+Bu)\}.\EQN 142$$
The first-order necessary condition for the maximum problem on the
right side of equation \Ep{142} is\NFootnote{We use the following rules for
differentiating quadratic and bilinear matrix forms:
${\partial x' A x \over \partial x} = (A + A') x;
{\partial y' B z \over \partial y} = B z, {\partial
y' B z \over \partial z} = B' y$. See appendix \use{appendixLQA} of this chapter.}
$$(Q+B'PB)u=-B'PAx,\EQN 143$$
which implies the feedback rule for $u$:
$$u=-(Q+B'PB)^{-1} B'PAx\EQN 144$$
or
$u=-Fx,$
where $$F=(Q+B'PB)^{-1}B'PA. \EQN 145$$
  Substituting the optimizer \Ep{144} into
the right side of equation \Ep{142} and rearranging gives
$$P=R+A'PA-A'PB(Q+B'PB)^{-1} B'PA.\EQN 146$$
Equation \Ep{146} is called the {\it algebraic matrix Riccati} equation.
It expresses the matrix $P$ as an implicit function of the matrices
$R,Q,A,B$.  Solving this equation for $P$ requires
a computer whenever $P$ is larger than a $2 \times 2$ matrix.
\index{Riccati equation! algebraic}

In exercise {\it \the\chapternum.1\/}, you are asked to derive the Riccati
equation for the case where the return function is
modified to
$$ -(  x_t' R x_t + u_t' Q u_t  + 2 u_t' H x_t) .$$


\auth{Bertsekas, Dimitri}%
\auth{Sargent, Thomas J.}%
\subsection{Value function iteration}
  Under particular conditions to be discussed in section \use{sec:ch4stable} about  stability,
equation \Ep{146} has a unique positive
semidefinite solution that is approached in the limit as $j\to\infty$ by
iterations on the matrix Riccati difference equation\NFootnote{If
the eigenvalues of $A$ are bounded in modulus below unity, this result
obtains, but much weaker conditions suffice.  See
Bertsekas (1976, chap. 4) and Sargent (1980).}
$$P_{j+1}=R+A'P_j A-A'P_jB(Q+B'P_jB)^{-1} B'P_jA,\EQN 147;a$$
starting from $P_0=0$.
The policy function  associated with $P_{j}$ is
$$F_{j+1} =(Q+B'P_j B)^{-1}B'P_j A. \EQN 147;b$$
 Equation \Ep{147} is derived much like equation \Ep{146}
except that one starts from the iterative version of the Bellman
equation  rather than from the asymptotic version.

\subsection{Discounted linear regulator problem}

The discounted optimal linear
regulator problem is to maximize
$$- \sum_{t=0}^\infty \be^t \{x'_t Rx_t+u'_tQu_t\},\qquad 0<\be<1,\EQN 148$$
subject to $x_{\toone}=Ax_t+Bu_t,x_0$ given. This problem
leads to the following matrix Riccati difference
equation modified for discounting:
$$P_{j+1} =R+\be A'P_jA-\be^2A'P_jB(Q+\be B'P_jB)^{-1} B'P_jA.\EQN 149$$
The algebraic matrix Riccati equation is modified correspondingly.  The value
function for the infinite horizon problem is  $V(x_0)= - x'_0Px_0$, where
$P$ is the limiting value of $P_j$ resulting from iterations on equation
\Ep{149} starting
from $P_0=0$.  The optimal policy is $u_t=-Fx_t$, where $F=\be (Q+\be
B'PB)^{-1} B'PA$.

  The Matlab program {\tt olrp.m} solves the discounted
optimal linear regulator problem.  Matlab has a variety
of other programs that solve both discrete- and continuous-time
 versions of undiscounted
optimal linear regulator problems.
%The program {\tt policyi.m}
%solves the undiscounted optimal linear regulator problem using
%policy iteration, which we study next.
\mtlb{policyi.m}
\mtlb{olrp.m}
\subsection{Policy improvement algorithm}

The \idx{policy improvement algorithm}  can be applied to solve the
discounted optimal
linear regulator problem. We discussed aspects of this algorithm earlier in section \use{sec:Howard1}.
  Starting from an initial $F_0$ for
which the eigenvalues of $A - B F_0$ are less than
$1/\sqrt\beta$ in modulus, the algorithm iterates on the
two equations
$$\EQNalign{
   P_{j} &= R + F_j' Q F_j + \beta ( A - B F_j)' P_j (A - B F_j)
      \EQN policyi1 \cr
   F_{j+1} &= \beta (Q + \beta B' P_j B )^{-1} B' P_j A .
     \EQN policyi2  \cr  }$$
The first equation pins down the matrix for the quadratic form in the  value function associated with using a fixed rule $F_j$ forever.
The second equation gives the matrix for the  optimal first-period decision rule for a two-period problem with
second-period  value function $- x^{* \prime} P_j x^*$ where $x^*$ is the second-period state.
The first equation is an example of a {\it discrete Lyapunov\/}
or {\it Sylvester\/} equation,
 \index{Lyapunov equation}
which is to be solved for the matrix $P_j$ that determines the
value $ - x_t' P_j x_t$ that is associated with following policy $F_j$
forever.  The solution of this equation can be represented
in the form
$$ P_j = \sum_{k=0}^\infty \beta^k (A-BF_j)^{'k} (R + F_j' Q F_j)
                                (A - B F_j)^k.  $$
If the eigenvalues of the matrix $A - B F_j$ are bounded
in modulus by $1/\sqrt\beta$, then a solution of this equation exists.
There are several methods available for solving this equation.\NFootnote{The
Matlab programs {\tt dlyap.m} and {\tt doublej.m}
 solve discrete Lyapunov equations.
See Anderson, Hansen, McGrattan, and Sargent (1996).}
The Matlab program {\tt policyi.m}
solves the undiscounted optimal linear regulator problem using
policy iteration.
\mtlb{policyi.m}%
  This algorithm is typically much faster than the algorithm that
iterates on the matrix \idx{Riccati equation}.  Later we shall present
a third method for solving for $P$ that rests on the
link between $P$ and shadow prices for the state
vector.\auth{Hansen, Lars P.}  \auth{McGrattan, Ellen R.} \auth{Anderson, Evan W.}
\index{doubling algorithms}
\index{Lyapunov equation}
\mtlb{dlyap.m}
\mtlb{doublej.m}
\index{optimal linear regulator}

\section{The stochastic optimal linear regulator problem}

The stochastic discounted linear optimal regulator problem is to
choose a decision rule for $u_t$ to
maximize
$$ - E_0\sum_{t=0}^\infty \be^t \{x'_t Rx_t+u'_tQu_t\},\qquad
0<\be<1,\EQN 158$$
subject to $x_0$ given, and the law of motion
$$x_{\toone} =Ax_t+Bu_t+ C \eps_{\toone},\qquad t\ge 0,\EQN 159$$
where $\eps_{\toone}$ is an $(n\times 1)$ vector of random variables that is
independently and identically distributed according to the normal
distribution with mean vector zero and covariance matrix
$$E\eps_t\eps'_t= I .\EQN 160$$
(See Kwakernaak and Sivan, 1972, for an extensive study of the continuous-time
version of this problem; also see Chow, 1981.)  %The matrices $R,Q,A$, and $B$
%obey the assumption that we have described.
\auth{Kwakernaak, Huibert}  \auth{Sivan, Raphael}  \auth{Chow, Gregory}

The value function for this problem is
$$v(x)= - x'Px-d,\EQN 161$$
where $P$ is the unique positive semidefinite solution of the discounted
algebraic matrix Riccati equation corresponding to equation \Ep{149}.
As before, it is the limit
of iterations on equation
\Ep{149} starting from $P_0=0$. The scalar $d$ is given by
$$d=\be(1-\be)^{-1} {\rm trace} ( P C C') .  \EQN 162$$
  Furthermore, the optimal policy
continues to be given by $u_t=-Fx_t$, where
$$F=\be (Q+\be B'P'B)^{-1} B'PA.\EQN 163$$
A notable feature of this solution is: %that the feedback rule \Ep{163} is
%identical with the rule for the corresponding nonstochastic linear optimal
%regulator problem.  This outcome is the {\it certainty equivalence}
%principle.
\index{certainty equivalence!proof}
\tag{certequiv}{\the\pageno}%
\medskip
\noindent{\sc Certainty Equivalence Principle:} The  decision rule  \Ep{163}  that
solves the stochastic optimal linear regulator problem is
identical with the decision rule for the corresponding
nonstochastic linear optimal regulator problem.
\medskip
\noindent{\sc Proof:}
Substitute guess \Ep{161} into
the Bellman equation to obtain
$$ v(x) = \max_u \left\{- x' Rx-u' Qu-\be E [(Ax+Bu+ C \eps)'
 P(Ax+Bu+ C \eps)]-\be d \right\} ,$$
where $\eps$ is the realization of $\eps_{\toone}$ when $x_t=x$ and where
$E\eps|x=0$.
The
preceding equation implies
$$\eqalign{ v(x)=&\max_u\left\{ -  x' Rx-u' Qu-\be E\left\{ x'A'PAx+x'
A'PBu\right.\right.\cr
&+ x'A'P C \eps+ u'B'PAx+u'B'PBu+u'B'P C \eps\cr
&\left.\left. + \ \eps' C'
 PAx+\eps' C' PBu+\eps' C' P C \eps\right\} -\be d\right\}.\cr}$$
Evaluating the expectations inside the braces and using $E\eps|x=0$ gives
$$\eqalign{ v(x) =&\max_u - \left\{  x' Rx+u'Qu+ \beta x' A' P A x +
 \be  2x'A'PBu\right.\cr
&\left.+ \ \be u'B'PBu+\be E\eps'C' P C \eps\right\}-\be d  . \cr}$$
The first-order condition for $u$ is
$$(Q+\be B'PB)u=-\be B'PAx,$$
which implies equation \Ep{163}.
Using $E\eps'C' P C \eps= {\rm trace} (P C C)'$,
substituting equation
\Ep{163} into the preceding expression
for $v(x)$, and using equation \Ep{161} gives
$$P=R+\be A'PA-\be^2 A'PB(Q+\be B'PB)^{-1} B'PA,$$
and
$$d=\be(1-\be)^{-1} {\rm trace} ( P C C'). \quad \qed $$

\subsection{Discussion of certainty equivalence}

The  remarkable thing  is that, although through $d$ the
objective function \Ep{160} depends on $CC'$,
 the optimal decision rule $u_t=-Fx_t$ is independent of
$CC'$.  This is the message of  equation
\Ep{163} and the discounted algebraic Riccati
equation for $P$, which are identical with the formulas derived earlier under
certainty.  In other words, the optimal decision rule $u_t=h(x_t)$
is independent of the problem's
noise statistics.\NFootnote{Therefore, in linear quadratic
versions of the optimum savings problem, there are no
precautionary savings. Compare outcomes from  section \use{sec:LQmodel} of chapter \use{timeseries} and   chapters \use{selfinsure} and \use{incomplete}.} The
certainty
equivalence principle is a special property of the
optimal linear regulator problem and comes from the quadratic
objective function, the linear transition equation,
and the property $E (\eps_{t+1}  | x_t) =0$.
  Certainty
equivalence does not characterize stochastic control problems generally.
\index{certainty equivalence}

 % For the remainder of this chapter, we return to the nonstochastic
%optimal linear regulator, remembering the stochastic
%counterpart.

\section{Shadow prices in the linear regulator}

For several purposes,\NFootnote{In a planning problem in a linear quadratic economy,  the gradient of the value function
has information from which competitive equilibrium prices can be coaxed.
See Hansen and Sargent (2013).}
it is helpful to interpret the
gradient $ - 2 P x_t$ of the value function $- x_t' P x_t$ as
a shadow price or Lagrange multiplier.  Thus, associate
with the Bellman equation the Lagrangian
\index{shadow price}
\index{Lagrange multiplier}
$$\eqalign{ - x_t^\prime P x_t = V(x_t)
 &= \min_{\mu_{t+1}} \max_{u_t, x_{t+1}} -
\biggl\{ x^\prime_t R x_t + u^\prime_t Q u_t + x^\prime_{t+1}
   P x_{t+1}\cr
&+ 2 \mu^\prime_{t+1} [A x_t + B u_t - x_{t+1}]\biggr\},\cr}$$
where $2 \mu_{t+1}$ is a vector of Lagrange multipliers.
The first-order necessary conditions for an optimum with respect to
$u_t$ and $x_{t+1}$ are
$$\eqalign{2 Q u_t + 2B^\prime \mu_{t+1} &= 0\cr
2Px_{t+1} - 2\mu_{t+1} &= 0.\cr} \EQN{no label 1}$$
Using the transition law and rearranging gives the usual formula
for the optimal decision rule, namely,
 $u_t = - (Q + B^\prime P B)^{-1} B^\prime PA x_t$.
  Notice that by \Ep{no label 1}, the shadow
price vector satisfies $\mu_{t+1} = P x_{t+1}$.

In section \use{lagrangianformulation}, we shall  describe a computational strategy
that solves for $P$ by directly finding the optimal  multiplier process
$\{\mu_t\}$ and representing it as $\mu_t = P x_t$.  This
strategy exploits the {\it stability} properties of optimal
solutions of the linear regulator problem, which we now briefly
take up.  \index{stability properties}
%\vfil\eject
\subsection{Stability}\label{sec:ch4stable}%
After substituting the optimal control $u_t=-Fx_t$ into the
law of motion $x_{\toone}=Ax_t+Bu_t$, we obtain the optimal
``closed-loop system'' $x_{\toone}=(A-BF)x_t$.  This difference
equation governs the evolution of $x_t$ under the optimal
control.  The system is said to be {\it stable\/} if $\lim_{t\to\infty}
x_t=0$ starting from any initial $x_0\in R^n$.  Assume that the
\idx{eigenvalue}s of $(A-BF)$ are distinct, and use the eigenvalue
decomposition $(A-BF)=D\La D^{-1}$ where the columns of $D$
are the eigenvectors of $(A-BF)$ and $\La$ is a diagonal
matrix of eigenvalues of $(A-BF)$.   Write the ``closed-loop'' equation
as $x_{\toone}=D\La D^{-1} x_t$.  The solution of this difference
equation for $t>0$ is readily verified by repeated substitution
to be $x_t=D\La^t D^{-1} x_0$.  Evidently, the system is stable
for all $x_0\in R^n$ if and only if the eigenvalues  of $(A-BF)$
are all strictly less than unity in
absolute value.  When this condition is met, $(A-BF)$ is said
to be a ``stable matrix.''\NFootnote{It is possible to
amend the statements about stability in this section to
permit $A-BF$ to have a single unit eigenvalue associated with
a constant in the state vector.  See chapter \use{timeseries} for examples.}
\index{stable matrix}

A vast
literature is devoted to characterizing the conditions on $A,B,R$, and $Q$
that imply that $F$ is such that the optimal closed-loop system matrix $(A-BF)$ is stable.  These
conditions are surveyed by Anderson, Hansen,
McGrattan, and Sargent (1996) and can be briefly described
here for the undiscounted case $\be=1$.  Roughly speaking,
 the conditions on
$A,B,R$, and $Q$ are as follows:  First, $A$
and $B$ must be such that it is {\it possible\/} to pick a control law
$u_t=-Fx_t$ that drives $x_t$ to zero eventually, starting from any $x_0\in
R^n$ [``the pair $(A,B)$ must be stabilizable''].  Second, the matrix $R$ must
be such that it is {\it desirable\/} to drive $x_t$ to zero as
$t\to\infty$.

It would take us too far afield to go deeply into this body of
theory, but we can give a flavor of  the results by considering
the following special  assumptions and
their implications. Similar results can obtain under weaker conditions relevant for
economic problems.\NFootnote{See Kwakernaak and Sivan (1972) and
Anderson, Hansen, McGrattan, and Sargent (1996) for much weaker conditions.} \auth{Sargent,
Thomas J.}  \auth{Kwakernaak, Huibert}  \auth{Sivan, Raphael}
\auth{Anderson, Evan W.} \auth{McGrattan, Ellen R.} \auth{Hansen,
Lars P.} \specsec{Assumption A.1:}  The matrix $R$ is positive
definite.
\medskip
There immediately follows:

\medskip\noindent
{\sc Proposition 1:}  Under assumption A.1, if a solution to the
undiscounted regulator exists, it satisfies $\lim_{t\to\infty} x_t = 0$.

\medskip\noindent
{\sc Proof:}  If $x_t \not\to 0$, then $\sum^\infty_{t=0} x_t^\prime R x_t
\to \infty$.\qed

\medskip\noindent{\sc Assumption A.2:}  The matrix $R$ is positive semidefinite.

\medskip\noindent
 Under assumption A.2, $R$ is similar to a triangular matrix $R^\ast$:
$$R=T^\prime\ \pmatrix{R^\ast_{11} & 0\cr 0 & 0\cr}\ T$$
where $R^\ast_{11}$ is positive definite and $T$ is nonsingular.
Notice that
$x^\prime_t R x_t = x^\ast_{1t} R^\ast_{11} x^\ast_{1t}$
where $x^\ast_t = T x_t = \pmatrix{T_1\cr T_2\cr} x_t =
\pmatrix{x^\ast_{1t}\cr x^\ast_{2t}\cr}$.
Let
$x^\ast_{1t} \equiv T_1 x_t$.
These calculations support:

\medskip\noindent
{\sc Proposition 2:}  Suppose that a solution to the optimal linear
regulator exists under assumption A.2.
Then
$\lim_{t\to\infty} x^\ast_{1t}=  0.$
\medskip
The following definition is used in control theory:

\medskip
\index{stabilizable}%
\noindent{\sc Definition:}  The pair $(A,B)$ is said to be {\it stabilizable\/}
if there exists a matrix $F$ for which $(A-BF)$ is a stable matrix.
\medskip
The following indicates the flavor of  a variety of stability theorems from
control theory:\NFootnote{These
 conditions are discussed under the subjects of controllability,
stabilizability, reconstructability, and detectability in the literature on
linear optimal control.  (For continuous-time linear system, these concepts are
described by Kwakernaak and Sivan, 1972; for discrete-time systems, see
Sargent, 1980.)  These conditions subsume and generalize the transversality conditions
used in the discrete-time calculus of variations (see Sargent, 1987a).
  That is,
the case when $(A-BF)$ is stable corresponds to the situation in which it is
optimal to solve ``stable roots backward and unstable roots forward.'' See
Sargent (1987a, chap. 9). Hansen and Sargent (1981)
 describe the relationship
between Euler equation methods and dynamic programming for a class of linear
optimal control systems.  Also see Chow (1981).}$^,$\NFootnote{The conditions
under which $(A-BF)$ is stable are also the conditions under
which $x_t$ converges to a unique stationary distribution in the stochastic
version of the linear regulator problem.}
\auth{Hansen, Lars P.}  \auth{Chow, Gregory}

\medskip
\noindent{\sc Theorem:}  If $(A,B)$ is stabilizable and $R$ is positive definite,
then under the optimal  rule $F$, $(A-BF)$ is a stable matrix.
\medskip
In the next section, we assume that $A, B, Q, R$ satisfy
conditions  sufficient to invoke such a stability proposition, and
we use that assumption to justify a solution method that
solves the undiscounted linear regulator by searching among the
many solutions of the {\it Euler equations}
for a stable solution.
\index{Euler equation}
\index{duality}
\section{A Lagrangian formulation}\label{lagrangianformulation}%
This section describes a Lagrangian formulation of the
\idx{optimal linear regulator}.\NFootnote{Such formulations
are recommended by Chow (1997) and
Anderson, Hansen, McGrattan, and Sargent (1996).}
Besides being
useful computationally, this formulation carries
insights about  connections  between
stability and optimality and also
opens the way to constructing solutions of dynamic systems
 not coming directly from an  intertemporal optimization
 problem.\NFootnote{Blanchard and Kahn (1980); Whiteman (1983);
Hansen, Epple, and Roberds (1985); and
Anderson, Hansen, McGrattan and Sargent (1996) use and extend such methods.}
The formulation is also the basis for constructing fast algorithms for solving Riccati equations.
\auth{Blanchard, Olivier J.} \auth{Kahn, Charles}
\auth{Roberds, William}
\auth{Whiteman, Charles}
\auth{Epple, Dennis}
\auth{Hansen, Lars P.}

For the undiscounted optimal linear regulator problem, form the Lagrangian
$$\eqalign{{\cal L} &= - \sum^\infty_{t=0} \biggl\{ x^\prime_t R x_t +
u_t^\prime Q u_t\cr
&+ 2 \mu^\prime_{t+1} [A x_t + B u_t - x_{t+1}]\biggr\}.\cr} \EQN Lagr101 $$
First-order conditions for maximization with respect
to $\{u_t,x_{t+1}\}_{t=0}^\infty$ are
$$\eqalign{2 Q u_t &+ 2B^\prime \mu_{t+1} = 0\cr \mu_t &= R x_t + A^\prime \mu_{t+1}\ ,\ t\geq 1.\cr}\EQN c$$
Define $\mu_0$ to be the vector of shadow prices of $x_0$ and apply an envelope condition to
\Ep{Lagr101} to  deduce that
$$ \mu_0 = R x_0 + A' \mu_1 ,$$
which is a time $t=0 $ counterpart to the second equation of system  \Ep{Lagr101}.
Recall from the second equation of \Ep{no label 1} that $\mu_{t+1} = P x_{t+1}$, where
$P$ is the matrix that solves the algebraic Riccati equation.  Thus, $\mu_{t}$ is
 the gradient of the value function.
 The Lagrange multiplier vector $\mu_{t}$ is often called
the {\it costate} vector corresponding to the state vector $x_t$. \index{costate vector}%
Solve the first equation of \Ep{c} for $u_t$ in terms of $\mu_{t+1}$; substitute
into the law of motion $x_{t+1} = A x_t + B u_t$; arrange the resulting
equation and the second equation of \Ep{c} into the form
$$L\ \pmatrix{x_{t+1}\cr \mu_{t+1}\cr}\ = \ N\ \pmatrix{x_t\cr \mu_t\cr}\
,\ t \geq 0,$$
where
$$L = \ \pmatrix{I & BQ^{-1} B^\prime \cr 0 & A^\prime\cr}, N = \
\pmatrix{A & 0\cr -R & I\cr}.$$
When $L$ is of full rank (i.e., when $A$ is of full rank), we can write
this system as
$$\pmatrix{x_{t+1}\cr \mu_{t+1}\cr}\ = M\ \pmatrix{x_t\cr\mu_t\cr}\EQN a$$
where
$$M\equiv L^{-1} N = \pmatrix{A+B Q^{-1} B^\prime A^{\prime-1}R &
-B Q^{-1} B^\prime A^{\prime-1}\cr -A^{\prime -1} R & A^{\prime -1}\cr} .\EQN Mdef $$

We seek to solve the difference equation system \Ep{a} for a sequence $\{x_t\}_{t=0}^\infty$
that satisfies the initial condition for $x_0$ and a terminal condition
$\lim_{t \rightarrow +\infty} x_t =0$ that expresses our wish for a {\it stable} solution.
We inherit our wish for stability of the $\{x_t\}$ sequence from a desire to maximize
$-\sum_{t=0}^\infty \bigl[ x_t ' R x_t + u_t' Q u_t \bigr]$, which requires that $x_t' R x_t$ converge to
zero.

To proceed, we study  properties of the $(2n \times 2n)$ matrix $M$.  It is helpful to introduce
a $(2n \times 2n)$ matrix
$$ J= \pmatrix{0 & -I_n\cr I_n & 0\cr}.$$
The rank of $J$ is $2n$.

\medskip\noindent{\sc Definition:}  A matrix $M$ is called {\it symplectic} if
$$ MJM^\prime = J. \EQN e$$
\index{symplectic matrix}

%\beginleftbox  Tom, no equation d \endleftbox

\noindent
It can be verified directly that $M$ in equation
\Ep{Mdef} is symplectic.

 It
follows from equation
 \Ep{e} and from the fact   $J^{-1} = J^\prime = -J$ that for any symplectic
matrix $M$,
$$M^\prime = J^{-1} M^{-1} J.\EQN f$$
Equation \Ep{f} states that $M^\prime$ is related to the inverse of $M$
by a similarity transformation.  For square matrices, recall that  (a)
similar matrices share eigenvalues; (b) the eigenvalues of the inverse of
a matrix are the inverses of the eigenvalues of the matrix; and (c)
a matrix and its transpose have the same eigenvalues.
  It then follows from equation \Ep{f} that
the eigenvalues of $M$ occur in reciprocal pairs:  if $\lambda$ is an
eigenvalue of $M$, so is $\lambda^{-1}$.

\index{reciprocal pairs!of eigenvalues}

Write equation \Ep{a} as
$$y_{t+1} = M y_t\EQN h$$
where $y_t = \pmatrix{x_t\cr \mu_t\cr}$.  Consider the following
triangularization of $M$
$$V^{-1} M V= \pmatrix{W_{11} & W_{12} \cr 0 & W_{22}\cr}$$
where each block on the right side is $(n\times n)$, where $V$ is
nonsingular, and where $W_{22}$ has all its eigenvalues exceeding $1$ in modulus
and $W_{11}$ has all of its eigenvalues less than $1$ in modulus.  The
{\it Schur decomposition\/} and the {\it eigenvalue decomposition\/}
are two  such decompositions.\NFootnote{Evan Anderson's
Matlab program {\tt schurg.m} attains a convenient Schur decomposition
and is very useful for solving linear models  with distortions.
See McGrattan (1994) for  examples
of distorted economies whose equilibria can be computed using a Schur decomposition.}
\index{Schur decomposition}%
\index{eigenvalue!decomposition}%
\mtlb{schurg.m}%
Write equation \Ep{h} as
$$y_{t+1} = V W V^{-1} y_t. \EQN k$$
The solution of equation \Ep{k} for arbitrary initial condition $y_0$ is
evidently
$$y_{t} = V \left[\matrix{W^t_{11} & W_{12,t}\cr 0 & W^t_{22}\cr}\right]
\ V^{-1} y_0 \EQN l$$
where $W_{12,t} = W_{12}$ for $t=1$ and  for $t \geq 2$ obeys the recursion
$$W_{12, t} = W^{t-1}_{11} W_{12,t-1} + W_{12,t-1} W^{t-1}_{22}$$
%subject to the initial condition $W_{12,1}=W_{12}$
and where $W^t_{ii}$ is $W_{ii}$ raised to the $t$th  power.

Write equation \Ep{l} as
$$\pmatrix{y^\ast_{1t}\cr y^\ast_{2t}\cr}\ =\ \left[\matrix{W^t_{11} &
W_{12, t}\cr 0 & W^t_{22}\cr}\right]\quad \pmatrix{y^\ast_{10}\cr
y^\ast_{20}\cr}$$
where $y^\ast_t = V^{-1} y_t$, and in particular where
$$y^\ast_{2t} = V^{21} x_t + V^{22} \mu_t, \EQN m$$
and where $V^{ij}$ denotes the $(i,j)$ piece of
the partitioned $V^{-1}$ matrix.

Because $W_{22}$ is an unstable matrix, unless $y^\ast_{20} = 0$,
$y^\ast_t$ will diverge.
Let $V^{ij}$ denote the $(i,j)$ piece of
the partitioned $V^{-1}$ matrix.
 To attain stability, we must impose
$y^\ast_{20} =0$,
which from equation \Ep{m} implies
$$V^{21} x_0 + V^{22} \mu_0 = 0$$
or
$$\mu_0 = - (V^{22})^{-1} V^{21} x_0.$$ This equation replicates itself over
time in the sense that it implies
$$\mu_t = - (V^{22})^{-1} V^{21} x_t. \EQN invarsub1 $$
But notice that because $(V^{21}\ V^{22})$ is the second row block of
the inverse of $V$,
$$(V^{21} \ V^{22})\quad \pmatrix{V_{11}\cr V_{21}\cr} = 0$$
which implies
$$V^{21} V_{11} + V^{22} V_{21} = 0.$$
Therefore,
$$-(V^{22})^{-1} V^{21} = V_{21} V^{-1}_{11}.$$
So we can write
$$\mu_0 = V_{21} V_{11}^{-1} x_0 \EQN invarsub2 $$
and
$$\mu_t = V_{21} V^{-1}_{11} x_t.$$
%
However, we know from equations \Ep{nolabel1} that $\mu_t = P x_t$,
where $P$ occurs in the matrix that solves the \idx{Riccati equation}
\Ep{146}. Thus, the preceding argument establishes that
$$ P = V_{21} V_{11}^{-1}. \EQN easyricatti $$
This formula provides us with an alternative, and typically computationally very
efficient, way of computing the matrix $P$.

   This same method can be applied to compute the solution of
any system of the form \Ep{a} if a solution exists, even
if the eigenvalues of $M$ fail to occur in reciprocal pairs.  The method
will typically work so long as the eigenvalues of $M$ split   half
inside and half outside the unit circle.\NFootnote{See Whiteman (1983);
Blanchard and Kahn (1980); and Anderson, Hansen, McGrattan,
and Sargent (1996) for applications and developments of these methods.}
Systems in which  eigenvalues (properly adjusted for discounting) fail
to occur in reciprocal pairs arise when the system being solved
is an equilibrium of a model in which there are distortions that
prevent there being any optimum problem that the equilibrium
solves.  See Woodford (1999)  for an application of
such methods to solve for linear approximations
of equilibria of a monetary model with distortions.  See chapter \use{linappro} for some applications
to an economy with distorting taxes.
\auth{Whiteman, Charles}  \auth{Blanchard, Olivier J.}  \auth{Kahn, Charles}
\auth{Anderson, Evan W.}  \auth{Hansen, Lars P.}  \auth{McGrattan, Ellen R.}
\auth{Woodford, Michael}
%{\bf Mention inverse optimum problem.}

\section{The Kalman filter again}\label{sec:Kalmanch4}%
  Suitably reinterpreted, the same recursion \Ep{147}  that solves the optimal
linear regulator also determines the celebrated {\it Kalman filter} that we derived in section
\use{sec:Kalman100}
of chapter \use{timeseries}.
\index{Kalman filter}%
\index{filter!linear}%
Recall that the Kalman filter is a recursive algorithm
for computing  the mathematical expectation $E[ x_t \vert  y_{t-1}, \ldots,
y_0]$ of a hidden state  vector $x_t$, conditional on observing
a history $y_t, \ldots, y_0$ of a vector of noisy signals  on the
hidden state.   The Kalman filter can be used to formulate or
simplify a variety of signal-extraction and prediction
problems in economics.
\tag{KfilterLQ}{\the\pageno}%

We briefly remind the reader that the setting for the Kalman filter is the following linear
state-space  system.\NFootnote{We derived the Kalman filter as a
recursive  application of population  regression in chapter
\use{timeseries}, page \use{Kfilterchap1}.
\tag{Kfilterchap2}.} Given $x_0 \sim {\cal N}(\hat x_0,
\Sigma_0)$, let
$$\EQNalign{ x_{t+1} & = A x_t + C w_{t+1} \EQN kalman1;a \cr
             y_t &  = G x_t + v_t \EQN kalman1;b\cr}  $$
where $x_t$ is an $(n \times 1)$ state vector, $w_t$ is
an i.i.d.\ sequence Gaussian vector with  $E w_t w_t' = I$, and
$v_t$ is an i.i.d.\ Gaussian vector orthogonal to $w_s$ for
all $t,s$ with $E v_t v_t' = R$; and $A, C$, and $G$ are
matrices conformable to the vectors they multiply.  Assume
that the initial condition $x_0$ is  unobserved but
is known to have a Gaussian distribution with
mean $\hat x_0$ and covariance matrix $\Sigma_0$.   At time
$t$, the history of observations $y^t \equiv [y_t, \ldots, y_0]$
 is available to estimate the location of
$x_t$ and the location of $x_{t+1}$.  The \idx{Kalman filter}
is a recursive algorithm for computing
$\hat x_{t+1} = E[ x_{t+1} \vert y^t]$.
The algorithm is
$$ \eqalign{\hat x_{t+1} & = (A - K_t G) \hat x_t + K_t y_t \cr}
 \EQN kalman2 $$
where
$$\EQNalign{K_t & =  A \Sigma_t G'(G \Sigma_t G' + R )^{-1} \EQN kalman3;a\cr
            \Sigma_{t+1} & = A \Sigma_t A' + C C' -
              A \Sigma_t G' (G \Sigma_t G' + R)^{-1} G \Sigma_t A .
                  \EQN kalman3;b \cr}
                     $$
Here $\Sigma_t = E (x_t - \hat x_t)(x_t - \hat x_t)'$,  and
$K_t$ is called the {\it Kalman gain\/}.  \index{Kalman gain}%
  Sometimes the Kalman
filter is written in terms of the ``innovation representation''
$$ \EQNalign{  \hat x_{t+1} &= A \hat x_t + K_t a_t \EQN kalman4;a \cr
                  y_t & = G \hat  x_t + a_t \EQN kalman4;b \cr}   $$
where $a_t \equiv y_t - G \hat x_t \equiv y_t - E[y_t \vert y^{t-1}]$.
The random vector $a_t$ is called
the \index{innovation!in time series representation}
  {\it innovation} in $y_t$, being the part of
$y_t$ that cannot be forecast linearly from its own past.
  Subtracting  equation \Ep{kalman4;b} from
\Ep{kalman1;b} gives $a_t = G (x_t - \hat x_t) + v_t$;  multiplying
each side by its own transpose and taking expectations gives
  the following formula  for the innovation covariance matrix:
$$ E a_t a_t'  = G \Sigma_t G' + R . \EQN kalman5 $$



Equations
 \Ep{kalman3} display extensive similarities to equations
 \Ep{147}, the recursions for the optimal
linear regulator.  Indeed, the mathematical structures are identical when viewed properly.
Note that equation \Ep{kalman3;b} is
a Riccati equation.   With the judicious use
of matrix transposition and reversal of time, the two systems of equations
\Ep{kalman3} and \Ep{147}
can be made to match.\NFootnote{See Hansen and Sargent (ch.~4, 2008) for an account of how the LQ dynamic programming problem
  and the Kalman filter are connected through duality.  That chapter formulates the Kalman filtering problem in terms of a Lagrangian,
  then judiciously transforms the first-order conditions into an associated optimal linear regulator.} %In Appendix \use{lincontrol} on dual filtering
%and control (see Technical Appendixes), we compare
%versions of these equations and describe the concept of
%duality that links them.   Appendix \use{lincontrol} also contains
%a formal derivation of the Kalman filter.
% {\bf Lars XXXX: i propose to remove appendix B from the technical appendixes because we have covered the material enough in chapter 2 now.
See chapter \use{timeseries}, especially section \use{Kfilterapplications}, for some applications of the Kalman filter.\NFootnote{The
Matlab program {\tt kfilter.m}
computes the Kalman filter. Matlab has several
 programs that compute the Kalman filter for
discrete time  and continuous time models.}
\mtlb{kfilter.m}
\auth{Hansen, Lars P.}%
\auth{Sargent, Thomas J.}%
\index{Kalman filter!dual to linear regulator}




\section{Concluding remarks}

In exchange for their restrictions,
 the linear quadratic dynamic optimization problems of
this chapter acquire  tractability.     The Bellman
equation leads to Riccati difference  equations that
are so  easy to solve numerically that the curse of
dimensionality loses most of its force.  It is easy
to solve linear quadratic control or filtering with many state variables.
That it is difficult to solve  those problems otherwise is why
linear quadratic approximations are widely used.
% We describe
% those approximations in Appendix \use{appendixLQB} to this chapter.

  In chapter \use{recurpe}, we go beyond the single-agent optimization
problems of this chapter  to study
systems with multiple  agents who simultaneously solve
linear quadratic dynamic programming problems, with the decision rules of some agents influencing transition
   laws of variables appearing in other agents' decision problems.  We introduce two related equilibrium concepts
   to reconcile different agents' decisions.

\appendix{A}{Matrix formulas}\label{appendixLQA}
Let $(z,x,a)$ each be $n\times 1$ vectors, $A,C,D$, and $V$
each be $(n\times n)$ matrices, $B$ an $(m\times n)$ matrix, and $y$
an $(m\times 1)$ vector.  Then
$ {\partial a^\prime x\over \partial x} = a,
{\partial x^\prime A x\over \partial x} = (A + A^\prime) x,
{\partial^2 (x^\prime Ax)\over \partial x \partial x^\prime}=
(A+A^\prime),
{\partial x^\prime Ax\over \partial A} = x x^\prime,
{\partial y^\prime Bz\over \partial y} = Bz,
{\partial y^\prime Bz\over \partial z} = B^\prime y,
{\partial y^\prime Bz\over \partial B} = y z^\prime.$

The equation
$$A^\prime V A + C = V$$
to be solved for $V$ is called a {\it discrete Lyapunov equation}, and
its generalization
$$A^\prime V D + C = V$$
is called the discrete {\it Sylvester equation}.
The discrete Sylvester equation
\index{Sylvester equation!discrete Lyapunov equation}%
 has a unique solution
if and only if the eigenvalues $\{\lambda_i\}$
of $A$ and $\{\delta_j\}$ of $D$ satisfy the condition
$\lambda_i \delta_j \not=  1 \ \forall \ i, \ j.$

%
% \index{linear quadratic!approximations}
% \auth{King, Robert G.}
% \auth{Rebelo, Sergio}
% \auth{Plosser, Charles I.}
% \auth{Kydland, Finn E.}
% \auth{Prescott, Edward C.}
% \appendix{B}{Linear quadratic approximations}\label{appendixLQB}
% This appendix describes an important use of the optimal
% linear regulator: to approximate the solution of more complicated
% dynamic programs.\NFootnote{Kydland and Prescott (1982) used such
% a method, and so do many of their followers in the
% real business cycle literature. See King, Plosser, and Rebelo (1988) for
% related methods of real business cycle models.}
% Optimal linear regulator problems \index{optimal linear regulator}%
% are often used to approximate problems of the following form: maximize
% over $\{u_t\}^\infty_{t=0}$
% $$E_0 \sum^\infty_{t=0} \beta^t r(z_t) \EQN 1$$
%
% %\beginleftbox  Tom, this is where the w's begin.  were they supposed
% %to be omegas?  \endleftbox
%
% $$x_{t+1} = A x_t + B u_t + C w_{t+1} \EQN 2$$
% where $\{w_{t+1}\}$ is a vector of i.i.d.\ random
% disturbances with mean zero and finite variance,  and $r(z_t)$ is a
% concave and twice continuously differentiable function of
% $z_t \equiv \pmatrix{x_t\cr u_t\cr}$.  All nonlinearities in the
% original problem are absorbed into the composite function $r(z_t)$.
%
% \auth{Brock, William A.}  \auth{Mirman, Leonard J.}
% \subsection{An example: the stochastic growth model}
% Take a parametric version of Brock and Mirman's stochastic
% growth model, whose social planner chooses a policy
% for $\{c_t, a_{t+1}\}_{t=0}^\infty$ to maximize
% $$E_0 \sum^\infty_{t=0} \beta^t \ln c_t$$
% where
% $$\eqalign{c_t + i_t &= A a^\alpha_t \theta_t \cr
% a_{t+1} &= (1-\delta) a_t + i_t\cr
% \ln \theta_{t+1} &= \rho \ln \theta_t + w_{t+1}\cr}$$
% where $\{w_{t+1}\}$ is an i.i.d.\ stochastic process
% with mean zero and finite variance, $\theta_t$ is a
% technology shock, and $\tilde \theta_t \equiv \ln \theta_t$.
%   To get this problem
% into the form \Ep{1}--\Ep{2}, take $x_t = \pmatrix{a_t\cr \tilde
% \theta_t\cr}$, $u_t = i_t$,
% %\beginleftbox  Tom is this a displayed equation or text?  \endleftbox
% and $r (z_t) = \ln (A a^\alpha_t \exp \tilde \theta_t - i_t)$,
% and we write the laws of motion as
% $$\pmatrix{1\cr a_{t+1}\cr \tilde\theta_{t+1}\cr}\ = \ \pmatrix{1 & 0 & 0\cr
% 0 & (1-\delta) & 0\cr 0 & 0 & \rho\cr}\ \pmatrix{1\cr a_t\cr \tilde\theta_t\cr}
% \ + \ \pmatrix{0\cr 1\cr 0\cr}\ i_t\ + \ \pmatrix{0\cr 0\cr 1\cr}\ w_{t+1}$$
% where it is convenient to add the constant 1 as the first component
% of the state vector.
%
% \subsection{Kydland and Prescott's method}
%
% We want to replace $r(z_t)$ by a quadratic $z^\prime_t M z_t$.  We
% choose a point $\bar z$ and approximate with the first two terms of a
% Taylor series:\NFootnote{This setup is taken from McGrattan (1994) and
% Anderson, Hansen, McGrattan, and Sargent (1996).}
% $$\eqalign{\hat r(z) &= r(\bar z) + (z-\bar z)^\prime
% {\partial r\over \partial z}\cr
% &+ {1\over 2} (z-\bar z)^\prime {\partial^2 r\over
% \partial z\partial z^\prime} \ (z-\bar z).\cr} \EQN 3 $$
% If the state $x_t$ is $n\times 1$ and  the control $u_t$ is $k\times 1$,
% then the vector $z_t$ is $(n+k)\times 1$.  Let
% $e$ be the $(n+k)\times 1$ vector with 0's everywhere except for a 1 in
% the row corresponding to the location of the constant unity in the state
% vector, so that $1\equiv e^\prime z_t$ for all $t$.
%
% %\beginleftbox  couldn't read this ?? \endleftbox
%
% Repeatedly using $z^\prime e = e^\prime z = 1$, we can express equation
% \Ep{3}
% as
% $$\hat r (z) = z^\prime M z, $$
% where
% $$\eqalign{M =& e \left[r(\bar z) - \left({\partial r\over
% \partial z}\right)^\prime
% \bar z + {1\over 2} \bar z^\prime
%  {\partial^2 r\over \partial z\partial z^\prime}\ \bar z\right] e^\prime
%     \cr
% \noalign{\smallskip}
% &+{1\over 2} \left( {\partial r\over \partial z} e^\prime - e \bar z^\prime
% {\partial^2 r\over \partial z\partial z^\prime} - {\partial^2 r\over
% \partial z\partial z^\prime} \bar z e^\prime + e {\partial r \over
%   \partial z}^\prime \right)\cr
% \noalign{\smallskip}
% &+ {1\over 2} \left({\partial^2 r \over \partial z\partial z^\prime}\right)\cr}$$
% where the partial derivatives are evaluated at $\bar z$.
% Partition $M$, so that
% $$\eqalign{z^\prime M z &\equiv \pmatrix{x\cr u\cr}^\prime \
% \pmatrix{M_{11} & M_{12} \cr M_{21} & M_{22}\cr} \ \pmatrix{x\cr u\cr} \cr
% &= \pmatrix{x\cr u\cr}^\prime \pmatrix{R & W\cr W^\prime & Q\cr}\
%    \pmatrix{x\cr u\cr}. \cr}$$
%
%
%
% %$$\eqalign{M =& e \Bigl[r(\bar z) - \Bigl({\partial r\over
% %\partial z}\Bigr)^\prime
% %\bar z + {1\over 2} \bar z^\prime
%  %{\partial^2 r\over \partial z\partial z^\prime}\ \bar z\Bigr] e^\prime
% %    \cr
% %\noalign{\smallskip}
% %&+{1\over 2} \Bigl( {\partial r\over \partial z} e^\prime - e \bar z^\prime
% %{\partial^2 r\over \partial z\partial z^\prime} - {\partial^2 r\over
% %\partial z\partial z^\prime} \bar z e^\prime + e {\partial r \over
% %  \partial z}^\prime \Bigr)\cr
% %\noalign{\smallskip}
% %&+ {1\over 2} \Bigl({\partial^2 r \over \partial z\partial z^\prime}\Bigr)\cr}$$
% %where the partial derivatives are evaluated at $\bar z$.
% %Partition $M$, so that
% %$$\eqalign{z^\prime M z &\equiv \pmatrix{x\cr u\cr}^\prime \
% %\pmatrix{M_{11} & M_{12} \cr M_{21} & M_{22}\cr} \ \pmatrix{x\cr u\cr} \cr
% %&= \pmatrix{x\cr u\cr}^\prime \pmatrix{R & W\cr W^\prime & Q\cr}\
% %   \pmatrix{x\cr u\cr}. \cr}$$
% %
%
% \subsection{Determination of $\bar z$}
%
% Usually, the point $\bar z$ is chosen as the (optimal) stationary
% state of the {\it nonstochastic\/} version of the original nonlinear model:
% $$\eqalign{\sum^\infty_{t=0} \beta^t & r (z_t)\cr
% x_{t+1} &= A x_t + B u_t .\cr}$$
% This stationary point is obtained in these steps:
% \medskip
%
% \itemitem{1.}  Find the Euler equations.
% \itemitem{2.}  Substitute $z_{t+1} = z_t \equiv \bar z$ into the
% Euler equations and transition laws, and solve the resulting
% system of nonlinear equations for $\bar z$.  This purpose
%  can be accomplished,
% for example, by using the nonlinear equation solver {\tt fsolve.m}
% in Matlab.
%
% \mtlb{fsolve.m}
% \vfil\eject
%
% \subsection{Log linear approximation}
%
% For some problems, Christiano (1990) has advocated a \idx{quadratic approximation}
% in logarithms.  We illustrate his idea with the stochastic growth example.
% Define
% $$\tilde a_t = \log a_t \ ,\ \tilde \theta_t =\log \theta_t.$$
% Christiano's strategy is to take $\tilde a_t, \tilde\theta_t$ as the components
% of the state and write the law of motion as
% $$\eqalign{\pmatrix{1\cr \tilde a_{t+1}\cr \tilde\theta_{t+1}\cr}  &=\
% \pmatrix{1 & 0 & 0\cr 0 & 0 & 0\cr 0 & 0 &\rho\cr}\ \pmatrix{1\cr \tilde a_t\cr
% \tilde\theta_t\cr}   \cr
%  &+ \pmatrix{0 \cr 1\cr 0\cr}\ u_t \quad + \pmatrix{0\cr 0\cr 1\cr} w_{t+1}
% \cr}$$
% where the control $u_t$ is $\tilde a_{t+1}$. \auth{Christiano, Lawrence J.}%
%
% Express consumption as
% $$c_t = A(\exp \tilde a_t)^\alpha (\exp \tilde\theta_t) + (1-\delta)
% \exp \tilde a_t - \exp \tilde a_{t+1}.$$
% Substitute this expression into $\ln c_t \,  \equiv r (z_t)$, and
% proceed as before to obtain the second-order Taylor series approximation
% about $\bar z$.
%
% \subsection{Trend removal}
%
% It is conventional in the real business cycle literature to specify the
% law of motion for the technology shock $\theta_t$ by
% $$\tilde\theta_t = \log \left( {\theta_t\over \gamma^t}\right), \
% \gamma > 1$$
%
%
% $$\tilde\theta_{t+1} = \rho \tilde
%  \theta_t + w_{t+1}, \qquad \vert \rho \vert < 1.
% $$
% This inspires us to write the law of motion for capital as
% $$\gamma {a_{t+1}\over \gamma^{t+1}} = (1-\delta) {a_t\over \gamma^t} +
% {i_t\over \gamma^t}$$
% or
% $$\gamma \exp \tilde a_{t+1} = (1-\delta) \exp \tilde a_t +
% \exp(\tilde i_t)
% $$
% where
% $\tilde a_t \equiv \log \Bigl({a_t\over\gamma^t}\Bigr),
% \tilde i_t = \log \Bigl({i_t \over \gamma_t}\Bigr)$.
% By studying the Euler equations for a model with a growing
% technology shock $(\gamma > 1)$, we can show that there exists
% a steady state for $\tilde a_t$, but not for $a_t$.
%  Researchers often construct linear quadratic approximations
% around the nonstochastic steady state of $\tilde a$.


%\section{Exercises}

\showchaptIDfalse
\showsectIDfalse
\section{Exercises}
\showchaptIDtrue
\showsectIDtrue
\medskip
\eqnum=0
\noindent{\it Exercise \the\chapternum.1}  \quad Consider the
modified version of the optimal linear regulator problem where the
objective is to maximize
$$-  \sum_{t=0}^\infty \beta^t \left\{ x_t' R x_t + u_t' Q u_t
    + 2 u_t' H x_t \right\}$$
subject to the law of motion:
$$ x_{t+1} = A x_t + B u_t .$$
Here $x_t$ is an $n \times 1$ state vector, $u_t$ is a $k \times 1$
vector of controls, and $x_0$ is a given initial condition.   The matrices
$R, Q$ are positive definite and symmetric.
 The
maximization is with respect to sequences $\{u_t, x_t\}_{t=0}^\infty$.
\medskip
\noindent {\bf a.} Show  that the optimal policy has the form
$$ u_t = - (Q + \beta B'P B)^{-1} (\beta B' P A + H) x_t,$$
where $P$ solves the
algebraic matrix Riccati equation
$$ P = R + \beta A' P A - (\beta A' P B + H')
       (Q+\beta B'PB)^{-1} (\beta B' P A + H) . \eqno(1) $$ %\EQN ricccross $$
\medskip
\noindent{\bf b.}  Write a Matlab program to solve equation (1) % \Ep{ricccross}
by iterating on $P$ starting from $P$ being a matrix of zeros.

\medskip
\noindent{\it Exercise \the\chapternum.2} \quad  Verify that equations
 \Ep{policyi1} and \Ep{policyi2}
implement the policy improvement algorithm for the discounted
linear regulator problem.

\medskip
\noindent {\it Exercise \the\chapternum.3} \quad  A household chooses $\{c_t, a_{t+1}\}_{t=0}^\infty$ to maximize
$$ - \sum_{t=0}^\infty \beta^t \left\{(c_t - b)^2 + \gamma i_t^2
  \right\} $$
subject to
$$\EQNalign{ c_t + i_t & = r a_t + y_t \cr % \EQN tran1;a \cr
             a_{t+1} & = a_t + i_t \cr % \EQN tran1;b \cr
            y_{t+1} & = \rho_1 y_t + \rho_2 y_{t-1}. \cr} $$ %\EQN tran1;c \cr} $$
Here $c_t, i_t, a_t, y_t$ are the household's consumption,
investment, asset holdings, and exogenous labor income at $t$;
while $b > 0, \gamma > 0, r>0, \beta \in (0,1)$, and
$\rho_1, \rho_2$ are parameters, and $a_0, y_{0}, y_{-1}$ are initial conditions.
Assume that $\rho_1, \rho_2$ are such that $(1 -\rho_1 z - \rho_2 z^2) = 0$
implies $\vert z \vert > 1$.
\medskip
\noindent{\bf a.}  Map this problem into an optimal linear regulator
problem.
\medskip
\noindent {\bf b.}
For parameter values $[\beta, (1+r), b, \gamma, \rho_1,
\rho_2] = (.95, .95^{-1}, 30, 1, 1.2, -.3)$, compute the household's
optimal policy function using your Matlab program from exercise {\it \the\chapternum.1\/}.

\medskip
\noindent{\it Exercise \the\chapternum.4} \quad   Modify exercise {\it \the\chapternum.3\/} by assuming that the household
seeks to maximize
$$ - \sum_{t=0}^\infty \beta^t \left\{(s_t - b)^2 + \gamma i_t^2 \right\}$$
Here $s_t$  measures consumption services that are produced
by durables or habits according to
$$\EQNalign{s_t & = \lambda  h_{t} + \pi c_t  \cr % \EQN habit1;a \cr
            h_{t+1} & = \delta h_{t} + \theta c_t \cr} % \EQN habit1;b \cr}
$$
where $h_t$ is the stock of the durable good or habit,
$(\lambda, \pi, \delta, \theta)$ are  parameters, and
$h_{0}$ is an initial condition.

\medskip
\noindent{\bf a.}  Map this problem into a linear regulator problem.
\medskip
\noindent{\bf b.}  For   the same parameter values   as in
exercise {\it \the\chapternum.3\/} and
$(\lambda, \pi, \delta, \theta) = (1, .05, .95, 1)$, compute
the optimal policy for the household.

\medskip
\noindent{\bf c.}  For the same parameter values as in exercise {\it \the\chapternum.3\/}
and
$(\lambda, \pi, \delta, \theta) = (-1, 1, .95, 1)$, compute
the optimal policy.
\medskip
\noindent{\bf d.}\
 Interpret the parameter settings
in part b as capturing a  model   of durable consumption goods, and
the settings in part c as giving a model of habit persistence.

\medskip
\noindent{\it Exercise \the\chapternum.5}  \quad A household's labor income follows the
stochastic process
$$ y_{t+1} = \rho_1 y_t +  \rho_2 y_{t-1} + w_{t+1} + \gamma w_t, $$
where $w_{t+1}$ is a
 Gaussian martingale difference sequence
with unit variance.    Calculate
$$ E \sum_{j=0}^\infty \beta^j [y_{t+j} \vert  y^t, w^t],  \eqno(1) $$ %\EQN expect $$
where $y^t, w^t$ denotes the history of $y,w$
 up to $t$.
\medskip\noindent{\bf a.} Write a Matlab program to compute expression
(1). %\Ep{expect}.
\medskip
\noindent{\bf b.}  Use your program to evaluate expression  (1) % \Ep{expect}
for the parameter values
$( \beta, \rho_1, \rho_2, \gamma) = (.95, 1.2, -.4, .5)$.




\medskip\noindent
{\it Exercise  \the\chapternum.6} \quad {\bf Finding the state is an art}

\medskip\noindent
\noindent For $t \geq 0$, the endowment for a one-good economy  $d_t$ is governed by the second order stochastic difference equation
$$ d_{t+1} = \rho_0 + \rho_1 d_t + \rho_2 d_{t-1} + \sigma_d \epsilon_{t+1} $$
where $\epsilon_{t+1}$ is an i.i.d.\ process and $\epsilon_{t+1} \sim {\cal N}(0,1)$, $\rho_0, \rho_1$, and  $\rho_2$ are scalars, and $d_0, d_1$ are given initial conditions.
  A {\it stochastic discount factor}
is given by $s_t = \beta^t (b_0- b_1 d_t)$, where $b_0$ is a positive  scalar, $b_1 \geq 0$, and $\beta \in (0,1)$.   The value of the endowment at time $0$ is defined to be
$$ v_0 = E_0 \sum_{t=0}^\infty s_t d_t  \leqno(1) $$
and $E_0$ is the mathematical expectation operator conditioned on $d_0, d_{-1}$.

\medskip
\noindent{\bf a.}  Assume that $v_0$ in equation (1) is finite.  Carefully describe a recursive algorithm for computing
$v_0$.

\medskip
\noindent{\bf b.}   Describe conditions on $\beta, \rho_1, \rho_2$ that are sufficient to make  $v_0$  finite.

\medskip\noindent
{\it Exercise  \the\chapternum.7} \quad {\bf Dynamic Laffer curves}

\medskip\noindent
The demand for currency in a small country
is described by
$$ M_t / p_t = \gamma_1 - \gamma_2 p_{t+1}/p_t  , \leqno(1) $$
where $\gamma_1 > \gamma_2 > 0$, $M_t$ is the stock of currency
held by the public at the end of period $t$, and $p_t$ is the
price level at time $t$.  There is no randomness in the country,
so that there is perfect foresight.  Equation (1) is a Cagan-like
demand function for currency, expressing real balances as
an inverse function of the expected gross rate of inflation.

   Speaking of Cagan, the government is running
a permanent real  deficit of $g$ per period, measured in goods,
all of which it finances by currency creation.  The government's
budget constraint at $t$ is
$$ (M_t - M_{t-1})/p_t = g ,\leqno(2)  $$
where the left side is the real value of the new currency
printed at time $t$.  The economy starts at time $t=0$,
with the initial level of nominal currency stock $M_{-1}=100$ being
given.

   For this model, define an {\it equilibrium} as a  pair of
{\it positive}   sequences
$\{ p_t > 0, M_t > 0 \}_{t=0}^\infty$ that satisfy equations
(1) and (2) (portfolio balance and the government budget constraint,
respectively) for $t \geq 0$, and the initial condition assigned for
$M_{-1}$.
    \medskip

\noindent{\bf a.} Let $\gamma_1 =100, \gamma_2 = 50, g = .05$.
Write a computer program to compute  equilibria for this economy.
Describe your approach and display the program.

\smallskip

\noindent{\bf b.}  Argue that there exists a continuum of
equilibria. Find the {\it lowest} value of the initial price level
$p_0$ for which there exists an equilibrium.  ({\it Hint
1:}  Notice the positivity condition that is part of the
definition of equilibrium. {\it Hint  2:} Try using the
general approach to solving difference equations described in
section \use{lagrangianformulation}.)

\noindent{\bf c.}  Show that for all of these equilibria except
the one that is associated with the minimal $p_0$ that you
calculated in part b, the gross inflation rate and the gross money
creation rate both eventually converge to the {\it same} value.
Compute this value.

\smallskip
\noindent{\bf d.}  Show that there is a unique equilibrium with a
lower inflation rate than the one that you computed in part b.
Compute this inflation rate.
%  Find the initial value of the price level
%in the equilibrium that is associated with this inflation rate.

\smallskip
\noindent{\bf e.}  Increase the level of $g$ to $.075$.  Compare
the
 (eventual or asymptotic) inflation rate that you computed
in part b and the  inflation rate that you computed in part c.
Are your results consistent with the view that ``larger permanent deficits
cause larger inflation rates''?

\smallskip
\noindent{\bf f.}  Discuss your results from the standpoint of the
Laffer curve.

\smallskip
\noindent{\it Hint:} A Matlab program {\tt dlqrmon.m} performs the calculations.
It is available from the web site for the book.
\medskip

\noindent{\it Exercise \the\chapternum.8}  \quad
  A government faces an  exogenous stream of
government expenditures $\{g_t\}$ that it must finance. Total
government expenditures at $t$ consist of two components:
$$ g_t = g_{Tt} + g_{Pt} \leqno(1) $$
where $g_{Tt}$ is transitory expenditures and $g_{Pt}$ is
permanent expenditures.  At the beginning of period $t$, the
government observes the history up to $t$ of both $g_{Tt}$ and
$g_{Pt}$.  Further, it knows the stochastic laws of motion of
both, namely,
$$ \eqalign{ g_{Pt+1} &= g_{Pt} + c_1 \epsilon_{1,t+1} \cr
             g_{Tt+1} & = (1 -\rho) \mu_T + \rho g_{Tt} + c_2
             \epsilon_{2t+1} \cr } \leqno(2) $$
where $\epsilon_{t+1} = \left[\matrix{\epsilon_{1t+1} \cr
\epsilon_{2t+1} \cr}\right]$ is an i.i.d.\ Gaussian vector process
with mean zero and identity covariance matrix.  The government
finances its budget with a distorting taxes. If it collects $T_t$
total revenues at $t$, it bears a dead weight loss of $W(T_t)$
where $W(T) = w_1 T_t + .5 w_2 T_t^2 $, where $w_1 , w_2 > 0$. The
government's loss functional is
$$ E \sum_{t=0}^\infty \beta^t W(T_t), \quad \beta \in (0,1). \leqno (3) $$
The government can purchase or issue one-period risk-free loans at
a constant price $q$. Therefore, it faces a sequence of budget
constraints
$$ g_t + q b_{t+1} = T_t + b_t, \leqno(4)  $$
where $q^{-1}$ is the gross rate of return on one-period risk-free
government loans.  Assume that $b_0 = 0$.  The government also
faces the terminal value condition
$$ \lim_{t \rightarrow +\infty} \beta^t W'(T_t) b_{t+1} = 0 ,$$
which prevents it from running a Ponzi scheme. The government
wants to design a tax collection strategy expressing $T_t$ as a
function of the history of $g_{Tt}, g_{Pt}, b_t$ that {\it
minimizes\/} (3) subject to (1), (2), and (4).

\medskip
\noindent {\bf a.}  Formulate the government's problem as a
dynamic programming problem.  Please carefully define the state
and control for this problem.  Write the Bellman equation in as
much detail as you can.  Tell a computational strategy for solving
the Bellman equation. Tell the forms of the optimal value function
and the optimal decision rule.

\medskip \noindent
{\bf b.} \ Using objects that you computed in part {\bf a}, please state
the form of the law of motion for the joint process of $g_{Tt},
g_{Pt}, T_t, b_{t+1}$ under the optimal government policy.
\medskip
\noindent{\bf Some background:}  Assume now that the optimal tax
rule that you computed above has been in place for a very long
time.  A macroeconomist who is studying the  economy observes time
series on $g_t, T_t$, but {\it not\/} on  $b_t$ or the breakdown
of $g_t$ into its components $g_{Tt}, g_{Pt}$. The macroeconomist
has a very long time series for $[g_t, T_t]$ and proceeds to
compute a {\it vector autoregression\/} for this vector.
\medskip
\noindent{\bf c.}  Define a population vector autoregression for
the $[g_t, T_t]$ process. (Feel free to assume that lag lengths
are infinite if this simplifies your answer.)
\medskip
\noindent{\bf d.}  Please tell precisely how the vector
autoregression for $[g_t, T_t]$ depends on the parameters $[\rho,
\beta, \mu, q, w_1, w_2, c_1, c_2]$ that determine the joint
$[g_t, T_t]$ process according to the economic theory you used in
part a.
\medskip
\noindent{\bf e.} \  Now suppose that in addition to his
observations on $[T_t, g_t$], the economist gets an error-ridden
time series on government debt $b_t$:
$$ \tilde b_t = b_t + c_3 w_{3t+1} $$
where $w_{3t+1}$ is an i.i.d.\ scalar Gaussian process with mean
zero and unit variance that is orthogonal to $w_{is+1}$ for
$i=1,2$ for all $s$ and $t$. Please tell how the vector
autoregression for $[g_t, T_t, \tilde b_t]$ is related to the
parameters $[\rho, \beta, \mu, q, w_1, w_2, c_1, c_2, c_3]$. Is
there any way to  use the vector autoregression to make inferences
about those parameters?






\medskip
\noindent{\it Exercise \the\chapternum.9}  \quad
\medskip

\noindent   A planner chooses a contingency plan for
$\{c_t, k_{t+1}\}_{t=0}^\infty$ to maximize
$$ - .5 E_0 \sum_{t=0}^\infty \beta^t [(c_t - b_t)^2 + e i_t^2 ] $$
subject to the technology
$$ \eqalign{ c_t + i_t & = \gamma k_t + d_t \cr
             k_{t+1} & = (1-\delta) k_t + i_t, \cr } $$
the laws of motion for the exogenous shock processes
$$ \eqalign{ b_{t+1} & =  \mu_b (1- \rho_b) + \rho_b b_t + \sigma_b \epsilon_{b,t+1} \cr
             d_{t+1} & = \mu_d (1-\rho_d) + \rho_d d_t + \sigma_d \epsilon_{d,t+1}, \cr }$$
and given initial conditions $k_0, b_0, d_0$.
Here $k_t$ is physical capital, $c_t$ is  consumption, $b_t$ is a scalar stochastic  process for
bliss consumption, and $d_t$ is an exogenous endowment process,  $\beta \in (0,1)$, $e >0$, $\delta \in (0,1)$,  $\rho_b \in (0,1)$, $\rho_d \in (0,1)$, and the adjustment cost parameter
$e >0$.  Also, $\left[ \matrix{ \epsilon_{b,t+1} \cr \epsilon_{d,t+1}\cr}\right]$ is an i.i.d.\ process
that is distributed $ \sim {\cal N}(0, I)$. We assume that $\beta \gamma (1-\delta)=1$, a condition that Hall and Friedman imposed to form  permanent income models of consumption.  For convenience, group all parameters
into the vector
$$ \theta = \left[\matrix{ \beta &
                    \delta &
                    \gamma &
                    e &
                    \mu_b &
                    \mu_d &
                   \rho_b &
                   \rho_d  &
                   \sigma_b &
                   \sigma_d \cr} \right]. $$

\medskip
\noindent
{\bf Part I.}  Assume that the planner knows all parameters of the model.  At time $t$, the planner
observes the history of $d_s, b_s, k_s$ for $s \leq t$.

\medskip
\noindent {\bf a.}  Formulate the planning problem as a discounted dynamic programming problem.
\medskip
\noindent {\bf b.}  Use  the Bellman equation for the planning problem to describe the effects on the decision rule for $c_t$
and $k_{t+1}$ of an increase in $\sigma_b$.  Tell the effects of an increase in $\sigma_d$.
\medskip

\noindent {\bf c.}  Describe an algorithm to solve the Bellman equation.
\medskip

\noindent {\bf Part II.}
An econometrician observes a time series $\{c_t, i_t\}_{t=0}^T$ for the economy described in part I. (This economy is either a socialist economy with a benevolent planner or a competitive economy with complete
markets.)   The
econometrician does not observe $b_t, d_t, k_t$ for any $t$ but believes that
$$ \left[\matrix{ k_0 \cr
                  b_0 \cr
                  d_0 \cr}\right] \sim {\cal N}(\mu_0, \Sigma_0) .$$
The econometrician knows the value of $\beta$ but not  the remaining parameters
in $\theta$.
\medskip
\noindent {\bf a.} Describe as completely as you can how the econometrician can form maximum likelihood
estimates of the remaining parameters in $\theta$ given his sample $\{c_t, i_t\}_{t=0}^T$.  If possible,
find a recursive representation of the likelihood function.

\medskip
\noindent {\bf b.}  Suppose that the econometrician has a Bayesian prior distribution over the
unknown parameters in $\theta$.  Please describe an algorithm for constructing the Bayesian posterior
distribution for these parameters.




\medskip
\noindent{\it Exercise \the\chapternum.10}  \quad
\medskip
\noindent A consumer values consumption, asset streams $\{c_t, k_{t+1}\}_{t=0}^\infty$ according to
$$ -.5 E_0 \sum_{t=0}^\infty \beta^t (c_t - b)^2 \leqno(1) $$ where $\beta \in (0,1)$ and
$$\eqalign{ k_{t+1} & = R( k_t + y_t - c_t ) \cr
             y_{t+1} & = \mu_y(1-\rho_1 - \rho_2) + \rho_1 y_t + \rho_2 y_{t-1} + \sigma_y \epsilon_{t+1} \cr
             c_t & = \alpha y_t + (R-1) k_t, \ \ \alpha \in (0,1) \cr } $$
and $k_0, y_0, y_{-1}$ are given initial conditions, and $\epsilon_{t+1}$ is an i.i.d. shock with
$\epsilon_{t+1} \sim {\cal N}(0,1)$.
\medskip
\noindent {\bf a.} Tell how to compute the value of the objective function (1) under the prescribed
decision rule for $c_t$. In particular, write a Bellman equation and get as far as you can in solving it.
\medskip
\noindent {\bf b.}  Tell how to use the Howard policy improvement algorithm to get a better decision rule.




\medskip
\noindent{\it Exercise \the\chapternum.11} \quad {\bf Firm level adjustment costs}
\medskip
\noindent
A competitive firms sells  output $y_t$ at price $p_t$
and chooses a production plan to maximize
$$ \sum_{t=0}^\infty \beta^t R_t \leqno(1) $$
where
$$ R_t = p_t y_t - .5 d (y_{t+1} - y_t )^2 \leqno(2) $$
subject to $y_0$ being a given initial condition.  Here
$\beta \in (0,1)$ is a discount factor, and $d >0$ measures
a cost of adjusting the rate of output.  The firm is a
price taker.  The price $p_t$ lies on the demand curve
$$ p_t = A_0 - A_1 Y_t \leqno(3) $$
where $A_0 >0, A_1 > 0 $ and $Y_t$ is the market-wide level of
output, being the sum of output of $n$ identical firms.
The firm believes that market-wide output
follows the law of motion
$$ Y_{t+1} = H_0 + H_1 Y_t  \leqno(4) $$
where $Y_0$ is a known initial condition.
The firm observes $Y_t$ and $y_t$ at time $t$ when it chooses
$y_{t+1}$.

\medskip
\noindent{\bf a.}  Formulate a Bellman equation for the firm.
\medskip
\noindent{\bf b.}  For parameter values $\beta=.95, d=2, A_0 = 100, A_1 = 1, H_0 = 200, H_1 =.8$,
compute the firm's optimal value function and optimal decision rule.


\medskip
\noindent{\it Exercise \the\chapternum.12} \quad {\bf Firm level adjustment cost, II}
\medskip
\noindent
A competitive firms sells  output $y_t$ at price $p_t$
and chooses a production plan to maximize
$$ E_0  \sum_{t=0}^\infty \beta^t R_t \leqno(1) $$
where $E_0$ denotes a mathematical expectation conditional on time $0$ information,
$$ R_t = p_t y_t - .5 d (y_{t+1} - y_t )^2 \leqno(2) $$
subject to $y_0$ being a given initial condition.  Here
$\beta \in (0,1)$ is a discount factor, and $d >0$ measures
a cost of adjusting the rate of output.  The firm is a
price taker.  The price $p_t$ lies on the demand curve
$$ p_t = A_0 - A_1 Y_t + u_t  \leqno(3) $$
where $A_0 >0, A_1 > 0 $ and $Y_t$ is the market-wide level of
output, being the sum of output of $n$ identical firms. In (3), $u_t$ is a demand shock that follows the
first-order autoregressive process
$$ u_{t+1} = \rho u_t + \sigma_u \epsilon_{t+1} \leqno (4) $$
where $\epsilon_{t+1}$ is an i.i.d.\ scalar process with $\epsilon_{t+1} \sim {\cal N}(0,1)$ and $| \rho | < 1$.
The firm believes that market-wide output
follows the law of motion
$$ Y_{t+1} = H_0 + H_1 Y_t + H_2 u_t   \leqno(5) $$
where $Y_0$ is a known initial condition.
The firm observes $ P_t, Y_t$, and $y_t$ at time $t$ when it chooses
$y_{t+1}$.

\medskip
\noindent{\bf a.}  Formulate a Bellman equation for the firm.
\medskip
\noindent{\bf b.}  For parameter values $\beta=.95, d=2, A_0 = 100, A_1 = 1, H_0 = 200, H_1 =.8, H_2 =2, \rho=.9, \sigma_u =.05$,
compute the firm's optimal value function and optimal decision rule.


\medskip
\noindent{\it Exercise \the\chapternum.13} \quad {\bf Permanent income model}
\medskip
\noindent
A household chooses a process $\{c_t, a_{t+1}\}_{t=0}^\infty $ to maximize
$$ E_0 \sum_{t=0}^\infty \beta^t \{ -.5 (c_t -b)^2 - .5 \epsilon a_t^2  \}, \quad \beta \in (0,1)  $$
subject to
$$ \eqalign{ a_{t+1} + c_t & = R a_t + y_t \cr
            y_{t+1} & = (1-\rho_1 - \rho_2) + \rho_1  y_t + \rho_2 y_{t-1} + \sigma_y \epsilon_{t+1} }$$
where $c_t$ is consumption, $b >0$ is a bliss level of consumption, $a_t$ is financial assets at the beginning of $t$, $R = \beta^{-1}$ is the gross rate of return
on assets held from $t$ to $t+1$, and $\epsilon_{t+1}$ is an i.i.d.\ scalar process with $\epsilon_{t+}] \sim {\cal N}(0,1)$.
The household faces known initial conditions $a_0, y_0, y_{-1}$.

\medskip
\noindent{\bf a.}  Write a Bellman equation for the household's problem.
\medskip
\noindent{\bf b.}  Compute the household's value function and optimal decision rule for the following parameter values:
$b=1000, \beta = .95, R = \beta^{-1}, \rho_1 = 1.2, \rho_2 = -.4, \sigma_y = .05, \epsilon=.000001$.
\medskip
\noindent{\bf c.} Compute the eigenvalues of $A-BF$.

\medskip
\noindent{\bf d.} Compute the household's value function and optimal decision rule for the following parameter values:
$b=1000, \beta = .95, R = \beta^{-1}, \rho_1 = 1.2, \rho_2 = -.4, \sigma_y = .05, \epsilon=0$.
Compare what you obtain with your answers in part {\bf b}.


\medskip
\noindent{\it Exercise \the\chapternum.14} \quad {\bf Permanent income model again}
\medskip
\noindent
A household chooses a process $\{c_t, a_{t+1}\}_{t=0}^\infty $ to maximize
$$ E_0 \sum_{t=0}^\infty \beta^t \{ -.5 (c_t -b)^2 - .5 \epsilon a_t^2 \}, \quad \beta \in (0,1)  $$
subject to
$$ \eqalign{ a_{t+1} + c_t & = R a_t + y_t \cr
            y_{t+1} & = (1-\rho_1 - \rho_2) + \rho_1  y_t + \rho_2 y_{t-3} + \sigma_y \epsilon_{t+1} }$$
where $c_t$ is consumption, $b >0$ is a bliss level of consumption, $a_t$ is financial assets at the beginning of $t$, $R = \beta^{-1}$ is the gross rate of return
on assets held from $t$ to $t+1$, and $\epsilon_{t+1}$ is an i.i.d.\ scalar process with $\epsilon_{t+1} \sim {\cal N}(0,1)$.
The household faces known initial conditions $a_0, y_0, y_{-1}, y_{-2},  y_{-3}$.

\medskip
\noindent{\bf a.}  Write a Bellman equation for the household's problem.
\medskip
\noindent{\bf b.}  Compute the household's value function and optimal decision rule for the following parameter values:
$b=1000, \beta = .95, R = \beta^{-1}, \rho_1 = .55, \rho_2 = .3, \sigma_y = .05, \epsilon=.000001$.
\medskip
\noindent{\bf c.} Compute the eigenvalues of $A-BF$.

\medskip
\noindent{\bf d.} Compute the household's value function and optimal decision rule for the following parameter values:
$b=1000, \beta = .95, R = \beta^{-1}, \rho_1 = .55, \rho_2 = .3, \sigma_y = .05, \epsilon=0$.
Compare what you obtain with your answers in part {\bf b}.


