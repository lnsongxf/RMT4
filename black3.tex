
%\input grafinp3

%\input grafinput8
%\input psfig
%\eqnotracetrue


%\showchaptIDtrue
%\def\@chaptID{3.}

%\eqnotracetrue

%\hbox{}

\def\toone{{t+1}}
\def\ttwo{{t+2}}
\def\tthree{{t+3}}
\def\Tone{{T+1}}
\def\TTT{{T-1}}
\def\rtr{{\rm tr}}

 \index{curse of dimensionality}
\footnum=0
\chapter{Practical Dynamic Programming \label{practical}}
\section{The curse of dimensionality}
We often encounter problems where
it is impossible
to attain closed forms for iterating on the
Bellman equation.  Then we have to adopt  numerical
approximations.  This chapter  describes two
popular methods for obtaining numerical approximations.  The first
method replaces the original problem with another problem
that forces the state vector to live on a finite and
discrete grid of points, then applies discrete-state dynamic
programming to this problem.  The ``curse of dimensionality''
impels us to keep the number of points in the discrete state
space small.  The second approach uses polynomials to approximate
the value function.  Judd (1998) is   a comprehensive reference
about numerical analysis of dynamic economic models and contains
many insights about ways to compute  dynamic models.

 \index{discretization!of state space}
\section{Discrete-state dynamic programming}

We introduce the method of discretization of the state
space  in the context of a particular
discrete-state version of an \idx{optimal savings problem}.  An
infinitely lived household likes to consume one good that
it can acquire by spending labor income or accumulated savings.
The household has an endowment of labor at time $t$, $s_t$, that
evolves according to an $m$-state  Markov chain with transition
matrix $\cal P$ and state space $[\bar s_1, \bar s_2, \ldots, \bar s_m]$.
  If the  realization of the process at $t$ is
$\bar s_i$, then at time $t$ the household receives labor income
of amount $w \bar s_i$.  The wage $w$ is fixed over time.
We shall sometimes assume that $m$ is $2$, and that $s_t$
takes on value  0 in an unemployed state and 1 in an
employed state.  In this case, $w$ has the interpretation of
being the wage of employed workers.

  The household can choose to hold a single asset in
discrete amounts
$a_t \in {\cal A}$ where
${\cal A}$ is a grid
$[a_1 < a_2 < \cdots < a_n]$.
How the model builder chooses the end points of the
grid ${\cal A}$ is important, as we describe in detail
in chapter \use{incomplete} on incomplete market models.  The asset bears
a gross rate of return $r$ that is fixed over time.

The household's maximum problem, for given values of ($w,r$)
and given initial values ($a_0, s_0$), is to choose a policy for
 $\{a_{t+1}\}_{t=0}^\infty$ to maximize
$$ E \sum_{t=0}^\infty \beta^t u(c_t)  , \EQN ob1 $$
subject to
$$\eqalign{ c_t + a_{t+1} &= (r+1) a_t + w s_t  \cr
             c_t& \geq 0 \cr
             a_{t+1} &\in {\cal A}\cr }\EQN ob2$$
where
$\beta \in (0,1)$ is a discount factor and $r$ is fixed
rate of return on the assets.  We assume that
$\beta (1+r) < 1$.  Here $u(c)$ is a strictly
increasing, concave one-period utility function.  % Associated with
%this problem is the Bellman equation
%$$ v(a,s) = \max_{a' \in {\cal A}} \{u[(r+1)a +ws - a'] + \beta
%       E v(a',s') \vert s \} ,$$
%or for each $i \in [1, \ldots, m]$ and each $h \in [1, \ldots, n]$,
%$$ v(a_h, \bar s_i) = \max_{a' \in {\cal A}} \{u[(r+1)a_h+w \bar s_i - a']
% + \beta  \sum_{j=1}^m {\cal P}_{ij} v(a',\bar s_j)\} ,\EQN ob3$$
%where $a'$ is next period's value of asset holdings, and $s'$ is
%next period's value of the shock; here $v(a,s)$ is
%the optimal value of the objective function, starting
%from asset, employment state ($a,s$).  We seek a value function $v(a,s)$ that satisfies equation
%\Ep{ob3} and an associated policy function
%$a' = g (a,s)$ mapping this period's ($a,s$) pair into an optimal
%choice of assets to carry into next period.
 Associated with
this problem is the Bellman equation
$$ v(a,s) = \max_{a' \in {\cal A}} \biggl\{u[(r+1)a +ws - a'] + \beta
       E v(a',s') \vert s \biggr\} ,$$
       where $a$ is next period's value of asset holdings, and $s'$ is
next period's value of the shock; here $v(a,s)$ is
the optimal value of the objective function, starting
from asset, employment state ($a,s$).  We seek a value function $v(a,s)$ that satisfies equation
\Ep{ob3} and an associated policy function
$a' = g (a,s)$ mapping this period's ($a,s$) pair into an optimal
choice of assets to carry into next period.
Let assets live on the grid ${\cal A} = [a_1, a_2, \ldots, a_n]$. Then we can  express the Bellman equation as
$$ v(a_i, \bar s_j) = \max_{a_h \in {\cal A}} \biggl\{u[(r+1)a_i+w \bar s_j - a_h]
 + \beta  \sum_{l=1}^m {\cal P}_{j l} v(a_h,\bar s_l)\biggr\} ,\EQN ob3$$
for each $i \in [1, \ldots, n]$ and each $j \in [1, \ldots, m]$.
\auth{Hansen, Gary D.} \auth{Imrohoro\u glu, Selahattin}\auth{Wei,
Chao} \auth{Hall, George} \index{dynamic programming! discrete
state space}


\section{Bookkeeping}

For a discrete state space of small size, it is easy to solve the
Bellman equation numerically by manipulating matrices. Here is how
to write a computer program to iterate on the Bellman equation in
the context of the preceding model of asset
accumulation.\NFootnote{Matlab versions of the program have been
written by Gary Hansen, Selahattin \.Imrohoro\u glu, George Hall,
and Chao Wei.}
  Let there be $n$ states $[a_1, a_2, \ldots, a_n]$ for assets
and two states $[s_1,s_2]$ for employment status.  For $j=1,2$, define  $n \times 1$
vectors $v_j, j=1,2$, whose $i$th rows are determined
by $v_j(i) = v(a_i,s_j), i=1, \ldots, n$.  Let ${\bf 1}$ be the
$n \times 1$ vector consisting entirely of ones. For $j=1,2$, define two
$n \times n$ matrices $R_j$ whose $(i,h)$ elements are
$$ R_j(i,h) = u[(r + 1) a_i +w s_j - a_h],\quad i=1, \ldots, n, h= 1,
\ldots, n.$$
Define an operator $T([v_1,v_2])$ that maps a pair of $n \times 1$ vectors
$[v_1, v_2]$ into a pair of $n \times 1$ vectors $[tv_1, tv_2]$:\NFootnote{Programming
 languages like Python, Julia,  and Matlab execute maximum operations
over vectors very efficiently. For example,
for an $n \times m$ matrix $A$, the Matlab command {\tt [r,index]
=max(A)} returns
% $n \times 2$ matrix whose $i$th row
%is the vector $[\max_j A(i,j), j_{\max}]$, where $j_{\max}$ is
%the column where the maximizing element is located.
    the two $(1 \times m)$ row vectors {\tt r,index}, where
$r_j= \max_i A(i,j)$ and ${\rm index}_j$ is the row $i$
that attains $\max_i A(i,j)$ for column $j$ [i.e.,
${\rm index}_j = {\rm argmax}_i A(i,j)$].
This command  performs $m$ maximizations simultaneously.}
%$$\eqalign{ tv_j(i) & = \max_h \biggl\{ \bmatrix{R_j(i,1) & R_j(i,2) & \cdots & R_j(i,n) }\cr
%   &+ \beta {\cal P}_{j1} \bmatrix{ v_1(1) & v_1(2) & \cdots & v_1(n)}\cr
%   & +\beta {\cal P}_{j2} \bmatrix{ v_2(1) & v_2(2) & \cdots & v_2(n)} \biggr\} }$$
$$ tv_j(i) = \max_h \biggl\{ R_j(i,h) + \beta {\cal P}_{j1} v_1(h) + \beta {\cal P}_{j2} v_2(h) \biggr\}$$
for $j=1,2$,
   or
$$\eqalign{tv_1 &= \max\{ R_1 + \beta {\cal P}_{11} {\bf 1} v_1'
      + \beta {\cal P}_{12} {\bf 1} v_2' \} \cr
\noalign{\smallskip}
 tv_2 &= \max\{ R_2 + \beta {\cal P}_{21} {\bf 1} v_1'
      + \beta {\cal P}_{22} {\bf 1} v_2' \} .\cr} \EQN obB1$$
Here it is understood that the ``max'' operator
applied to an $(n \times m)$ matrix $M$ returns  an
$(n \times 1)$ vector whose $i$th element is the
maximum of the $i$th row of the matrix $M$.
These two equations can be written compactly as
$$ \left[\matrix{tv_1 \cr tv_2 \cr} \right]
   = \max \left\{ \left[\matrix{R_1 \cr R_2 \cr} \right]
       + \beta ({\cal P} \otimes {\bf 1}) \left[\matrix{v_1' \cr
           v_2' \cr } \right] \right\} ,\EQN obB2$$
where $\otimes$ is the Kronecker product.\NFootnote{If A is an m-by-n matrix and B is a p-by-q matrix, then the Kronecker product $A \otimes B$ is the mp-by-nq block matrix
$ A \otimes B = \bmatrix{ a_{11} B & \cdots & a_{1n}B \cr \vdots & \ddots & \vdots \cr a_{m1} B & \cdots & a_{mn} B }$.}

    The Bellman equation $ [v_1 v_2] = T([v_1, v_2])$
  can be solved by iterating to convergence on
$ [v_1, v_2]_{m+1} = T([v_1, v_2]_m).$

\index{policy improvement algorithm}

\section{Application of Howard improvement algorithm}

Often computation speed is important.  Exercise 3.1 showed
that the policy improvement algorithm can
be much faster than iterating on the Bellman equation.
It is also easy to implement the Howard improvement algorithm
in the present setting.
At time $t$, the system resides in one of $N$ predetermined positions,
denoted $x_i$ for $i=1,2,\ldots, N$.  There exists a predetermined set
$\cal M$ of $(N\times N)$ stochastic matrices $P$ that are the objects of
choice.  Here $P_{ij}={\rm Prob}\,[x_{t+1} = x_j\mid x_t=x_i]$, $i=1,\ldots,
N;\ j=1,\ldots, N$.

The matrices $P$ satisfy $P_{ij}\geq 0$, $\sum^N_{j=1}\, P_{ij} = 1$, and
additional restrictions dictated by the problem at hand that determine the
set $\cal M$.  The one-period return function is represented as $c_P$, a
vector of length $N$, and is a function of $P$.  The $i$th entry of $c_P$
denotes the one-period return when the state of the system is $x_i$ and the
transition matrix is $P$.  The Bellman equation is
$$v_P (x_i)=\max_{P\in{\cal M}}\, \{c_P(x_i) + \beta \sum^N_{j=1}\, P_{ij}\,
v_P(x_j)\}$$
or
$$v_P=\max_{P\in{\cal M}}\, \{c_P + \beta Pv_P\}\ .\EQN obC1$$
We can express this as
$$v_P = T v_P\ ,$$
where $T$ is the operator defined by the right side of \Ep{obC1}.  Following
Putterman and Brumelle (1979) and Putterman and Shin (1978), define the
operator
\auth{Putterman, Martin L.}  \auth{Brumelle, Shelby}  \auth{Shin, M.C.}
%
$$B=T-I,$$
so that
$$B v = \max_{P\in{\cal M}}\, \{c_P + \beta P v\}-v .$$
In terms of the operator $B$, the Bellman equation is
$$Bv=0 . \EQN obC2$$

The \idx{policy improvement algorithm} consists of iterations on the
following two steps.
\medskip
\item{1.}  For fixed $P_n$, solve
$$(I-\beta\, P_n)\, v_{P_n}= c_{P_n}\EQN obC3$$
for $v_{P_n}$.
\medskip
\item{2.}  Find $P_{n+1}$ such that
$$c_{P_{n+1}} + (\beta P_{n+1} - I)\, v_{P_n} = Bv_{P_n}\EQN obC4$$
\medskip
\noindent Step 1 is accomplished by setting
$$v_{P_n} = (I-\beta P_n)^{-1} c_{P_n} .\EQN obC5$$
\medskip\noindent
Step 2 amounts to finding a policy function
 (i.e., a stochastic matrix $P_{n+1}
\in {\cal M}$) that solves a two-period problem with $v_{P_n}$ as the terminal
value function.

Following Putterman and Brumelle, the \idx{policy improvement algorithm} can
be interpreted as a version of Newton's method for finding the zero of $Bv=v$.
Using equation
 \Ep{obC3} for $n+1$ to eliminate $c_{P_{n+1}}$ from equation \Ep{obC4} gives
$$(I-\beta P_{n+1})\, v_{P_{n+1}} + (\beta P_{n+1}-I)\, v_{P_n} = Bv_{P_n}$$
which implies
$$v_{P_{n+1}} = v_{P_n} + (I-\beta P_{n+1})^{-1} B v_{P_n} .\EQN obC6$$
From equation \Ep{obC4}, $(\beta P_{n+1} - I)$ can be regarded as the
gradient of
$B v_{P_n}$, which supports the interpretation of equation
 \Ep{obC6} as implementing
Newton's method.\NFootnote{Newton's method for finding the solution of
$G(z)=0$ is to iterate on
$z_{n+1} = z_n - G^\prime (z_n)^{-1}\, G(z_n).$}

% Previous version ==========================================
%  I altered indexes
%\section{Numerical implementation}
%  We shall illustrate Howard's \idx{policy improvement algorithm} by
%applying it to our savings example.  Consider
%a given feasible policy function $k'=f(k,s)$.
%For each $h$, define the $n \times n$ matrices $J_h$ by
%$$J_h(a,a') = \cases{ 1 &  if \ $g(a,s_h) = a'$ \cr
%                      0 &  otherwise .} $$
%Here $h = 1, 2, \ldots, m$ where $m$ is the number
%of possible values for $s_t$, and $J_h(a,a')$ is
%the element of $J_h$ with
%rows corresponding to initial assets $a$ and columns to
%terminal assets $a'$.  For a given policy function
%$a'=g(a,s)$ define the $n \times 1$ vectors $r_h$ with
%rows corresponding to
%$$ r_h(a) = u[(r+1)a + w s_h - g(a,s_h)] , \EQN obC7$$
%for $h = 1, \ldots ,m$.
%
%  Suppose the policy function $a'=g(a,s)$ is used forever.
%Let the value associated with using $g(a,s)$ forever
%be represented by the $m$ $(n \times 1)$ vectors
%$[v_1, \ldots, v_m]$, where $v_h(a_i)$ is the
%value starting from state $(a_i, s_h)$.  Suppose that
%$m=2$.  The vectors $[v_1,v_2]$ obey
%$$\left[\matrix{v_1 \cr v_2}\right]
%   = \left[\matrix{ r_1  \cr
%                     r_2 \cr}\right]
%    +\left[\matrix{ \beta {\cal P}_{11} J_1 & \beta {\cal P}_{12} J_1 \cr
%           \beta {\cal P}_{21} J_2 & \beta {\cal P}_{22} J_2 \cr}\right]
%           \left[\matrix{v_1 \cr v_2 \cr}\right]     .$$
%Then
%$$ \left[\matrix{v_1 \cr v_2 \cr }\right] =
%   \left[I - \beta \left(\matrix{{\cal P}_{11} J_1 & {\cal P}_{12} J_1 \cr
%          {\cal P}_{21} J_2 & {\cal P}_{22} J_2 \cr }\right)\right]^{-1}
%          \left[\matrix{r_1  \cr
%                        r_2 \cr }\right]  .\EQN obC8$$
%\medskip
%\index{policy improvement algorithm}
%\noindent Here is how to implement the Howard policy improvement algorithm.
%\medskip
%
%\item{} Step 1.  For an initial feasible policy function $g_j(k,j)$
%for $j=1$,
%form the $r_h$ matrices using equation
% \Ep{obC7}, then use equation \Ep{obC8} to evaluate the
%vectors of values  $[v_1^j, v_2^j]$ implied by using that
%policy forever.
%
%\medskip
%\item{} Step 2.  Use $[v_1^j, v_2^j]$ as the terminal value
%vectors in equation \Ep{obB2}, and perform one step on the Bellman equation
%to find a new policy function $g_{j+1}(k,s)$ for $j+1=2$.  Use
%this policy function, update $j$, and repeat step 1.
%
%\medskip
%\item{}Step 3.  Iterate to convergence on steps 1 and 2.
%\index{policy improvement algorithm! modified}
%
%\subsection{Modified policy iteration}
%   Researchers have had success using the following modification
%of policy iteration:  for $k \geq 2$, iterate $k$ times
%on Bellman's equation.  Take the resulting policy function
%and use equation \Ep{obC8} to produce a new candidate value function.  Then
%starting from this terminal value function, perform another
%$k$ iterations on the Bellman equation.  Continue in this
%fashion until the decision rule
%converges.


\section{Numerical implementation}
  We shall illustrate Howard's \idx{policy improvement algorithm} by
applying it to our savings example.  Consider
a  feasible policy function $a'=g(k,s)$.
For each $j$, define the $n \times n$ matrices $J_j$ by
$$J_j(a,a') = \cases{ 1 &  if \ $g(a,s_j) = a'$ \cr
                      0 &  otherwise .} $$
Here $j = 1, 2, \ldots, m$ where $m$ is the number
of possible values for $s_t$, and $J_j(a,a')$ is
the element of $J_j$ with
rows corresponding to initial assets $a$ and columns to
terminal assets $a'$.  For a given policy function
$a'=g(a,s)$ define the $n \times 1$ vectors $r_j$ with
rows corresponding to
$$ r_j(a) = u[(r+1)a + w s_j - g(a,s_j)] , \EQN obC7$$
for $j= 1, \ldots ,m$.

  Suppose the policy function $a'=g(a,s)$ is used forever.
Let the value associated with using $g(a,s)$ forever
be represented by the $m$ $(n \times 1)$ vectors
$[v_1, \ldots, v_m]$, where $v_j(a_i)$ is the
value starting from state $(a_i, s_j)$.  Suppose that
$m=2$.  The vectors $[v_1,v_2]$ obey
$$\left[\matrix{v_1 \cr v_2}\right]
   = \left[\matrix{ r_1  \cr
                     r_2 \cr}\right]
    +\left[\matrix{ \beta {\cal P}_{11} J_1 & \beta {\cal P}_{12} J_1 \cr
           \beta {\cal P}_{21} J_2 & \beta {\cal P}_{22} J_2 \cr}\right]
           \left[\matrix{v_1 \cr v_2 \cr}\right]     .$$
Then
$$ \left[\matrix{v_1 \cr v_2 \cr }\right] =
   \left[I - \beta \left(\matrix{{\cal P}_{11} J_1 & {\cal P}_{12} J_1 \cr
          {\cal P}_{21} J_2 & {\cal P}_{22} J_2 \cr }\right)\right]^{-1}
          \left[\matrix{r_1  \cr
                        r_2 \cr }\right]  .\EQN obC8$$
\medskip
\index{policy improvement algorithm}
\noindent Here is how to implement the Howard policy improvement algorithm.
\medskip

\item{} Step 1.  For an initial feasible policy function $g_\tau(a,j)$
for $\tau=1$,
form the $r_j$ matrices using equation
 \Ep{obC7}, then use equation \Ep{obC8} to evaluate the
vectors of values  $[v_1^\tau, v_2^\tau]$ implied by using that
policy forever.

\medskip
\item{} Step 2.  Use $[v_1^\tau, v_2^\tau]$ as the terminal value
vectors in equation \Ep{obB2}, and perform one step on the Bellman equation
to find a new policy function $g_{\tau+1}(a,s)$ for $\tau+1=2$.  Use
this policy function, increment $\tau$ by 1, and repeat step 1.

\medskip
\item{}Step 3.  Iterate to convergence on steps 1 and 2.
\index{policy improvement algorithm! modified}

\subsection{Modified policy iteration}
   Researchers have had success using the following modification
of policy iteration:  for $k \geq 2$, iterate $k$ times
on the Bellman equation.  Take the resulting policy function
and use equation \Ep{obC8} to produce a new candidate value function.  Then
starting from this terminal value function, perform another
$k$ iterations on the Bellman equation.  Continue in this
fashion until the decision rule
converges.


\section{Sample Bellman equations}
This section presents some examples.  The first two examples
involve no optimization, just  computing discounted expected utility.
Appendix \use{searchA} of chapter \use{search1} describes some related examples based
on search theory.

\subsection{Example 1: calculating expected utility}
 Suppose that the one-period utility function is the constant relative
risk  aversion form
$ u(c) = c^{1-\gamma}/(1-\gamma)$.   Suppose that
$c_{t+1} = \lambda_{t+1} c_t$ and that $\{\lambda_t\}$
is an $n$-state Markov process
 with transition matrix $P_{ij} = {\rm Prob} (\lambda_{t+1} =
\bar \lambda_j \vert \lambda_t = \bar \lambda_i)$.    Suppose
that we want to evaluate     discounted expected
utility
$$ V(c_0, \lambda_0) = E_0 \sum_{t=0}^\infty \beta^t u(c_t), \EQN util1 $$
where $\beta \in (0,1)$.
We can express this equation recursively:
$$ V(c_t, \lambda_t) = u(c_t) + \beta
 E_t  V(c_{t+1}, \lambda_{t+1}) \EQN util3$$
    We use a guess-and-verify
technique to solve equation \Ep{util3} for $V(c_t,\lambda_t)$.   Guess
that $V(c_t,\lambda_t) = u(c_t) w(\lambda_t)$ for some
function $w(\lambda_t)$. Substitute the  guess into
equation \Ep{util3}, divide both sides by $u(c_t)$, and rearrange to
get
$$ w(\lambda_t) = 1   + \beta E_t \left({c_{t+1} \over c_t}\right)^{1-\gamma}
                  w(\lambda_{t+1}) $$
or
$$ w_i = 1 + \beta \sum_j P_{ij} \left(\lambda_j\right)^{1-\gamma} w_j.
    \EQN util2 $$
Equation \Ep{util2} is a system of linear equations
in  $w_i, i=1, \ldots , n$ whose solution can be expressed
as
$$ w = \left[1-\beta P \ {\rm diag}(\lambda_1^{1-\gamma}, \ldots, \lambda_n^{1-\gamma})\right]^{-1}
  {\bf 1} $$
  where ${\bf 1}$ is an $n\times 1$ vector of ones.


\subsection{Example 2: risk-sensitive preferences}

   Suppose we modify the preferences of the previous example to
be of the recursive form
$$ V(c_t, \lambda_t) = u(c_t) +\beta {\cal R}_t V(c_{t+1}, \lambda_{t+1}),
 \EQN util4 $$
where $${\cal R}_t(V) = \left({2 \over \sigma}\right) \log E_t
\left[ \exp\left({\sigma V_{t+1} \over 2} \right) \right] \EQN
util6 $$ is an operator used by Jacobson (1973), Whittle (1990),
and Hansen and Sargent (1995) to induce a preference for
robustness to model misspecification.\NFootnote{Also see  Epstein
and Zin (1989) and Weil (1989) for a version of the ${\cal R}_t$
operator.}  Here $\sigma\leq 0$; when $\sigma < 0$, it represents
a concern for model misspecification, or an extra sensitivity to
risk.

  We leave it to the reader to propose a method for computing an
  approximation to a value function that solves the functional equation
  \Ep{util4}.  (Hint: the method used in example 1 will not apply
  directly because the homogeneity property exploited there fails
  to prevail now.)

 % \vfil\eject

     % Let's apply our guess-and-verify method again.
%If we make a guess of the same form as before, we now find
%$$ w(\lambda_t) = 1 + \beta\left({2\over \sigma}\right) \log E_t \left\{
%   \exp\left[{\sigma\over 2} \left({c_{t+1}\over c_t}\right)^{1-\gamma}
%    w(\lambda_{t+1}) \right]\right\} $$
%or
%$$ w_i = 1 + \beta {2 \over \sigma} \log \sum_j P_{ij}
%      \exp \left({\sigma \over 2} \lambda_j^{1-\gamma} w_j \right).
% \EQN util6$$
%Equation \Ep{util6} is a nonlinear system of equations in
%the $n \times 1$ vector  of $w$'s.   It can be solved by an iterative
%method: guess at an $n \times 1$ vector $w^0$,  use
%it on the right side of equation  \Ep{util6}  to compute a new guess
%$w^1_i, i=1, \ldots , n$, and iterate.



\subsection{Example 3:  costs of business cycles}
Robert E. Lucas, Jr.,  (1987) proposed that the cost of
business cycles be measured in terms of a proportional
upward shift  in the consumption process that would be required
to make  a representative consumer indifferent between  its
random consumption allocation and a nonrandom consumption
allocation with the same mean.     This measure of business
cycles is the  fraction $\Omega$ that satisfies
$$ E_0 \sum_{t=0}^\infty \beta^t u[(1+\Omega) c_t ]
         = \sum_{t=0}^\infty \beta^t u[E_0(c_t)]. \EQN costbus1  $$
Suppose that the utility function and the consumption process
are as in example 1.   Then for given $\Omega$,    the
calculations in example 1 can be used to calculate
the left side of  equation \Ep{costbus1}.   In particular,  the left
side just equals $u[(1+\Omega) c_0] w(\lambda)$, where
$w(\lambda)$  is calculated from equation \Ep{util2}.    To calculate
the right side, we have to evaluate
$$ E_0 c_t = c_0 \sum_{\lambda_t,  \ldots  ,\lambda_1}
     \lambda_t \lambda_{t-1} \cdots \lambda_1 \pi(\lambda_t
     |  \lambda_{t-1}) \pi(\lambda_{t-1} | \lambda_{t-2})
     \cdots \pi(\lambda_1 | \lambda_0), \EQN costbus2 $$
where the summation is  over all possible {\it paths\/} of growth
rates between $0$  and $t$. In the case of i.i.d.\ $\lambda_t$,
this expression simplifies to
 $$  E_0 c_t = c_0 (E \lambda)^t, \EQN  costbus3 $$
where $E \lambda_t$ is the unconditional mean of $\lambda$.
Under equation \Ep{costbus3}, the right side of  equation \Ep{costbus1}
is easy to evaluate.

  Given $\gamma, \pi$, a procedure for constructing the cost of cycles---more
precisely, the costs of deviations from mean trend---to the
representative consumer is first to compute the right side
of equation \Ep{costbus1}.
Then we solve the following equation for $\Omega$:
$$ u[(1+  \Omega)c_0] w(\lambda_0) =
          \sum_{t=0}^\infty \beta^t u[E_0(c_t)].  $$
%%%%


    Using a closely related but somewhat different stochastic
specification, Lucas (1987)   calculated $\Omega$.  He assumed
that   the endowment is a geometric trend with
growth rate $\mu$ plus an i.i.d.\ shock with
mean zero and variance $\sigma_z^2$.  Starting
from a base $\mu = \mu_0$, he  found
$\mu, \sigma_z$ pairs to which the household
is indifferent, assuming
various values of $\gamma$ that he judged to be within a reasonable
range.\NFootnote{See chapter \use{assetpricing2}
 for a discussion of reasonable values of
$\gamma$. See Table 1 of Manuelli and Sargent (1988) for a correction
to Lucas's calculations.}  Lucas found that  for reasonable
values of $\gamma$, it takes a very small
adjustment in the trend rate of growth
$\mu$ to compensate for even a substantial increase
in the ``cyclical noise'' $\sigma_z$, which meant to him
that the costs of business cycle fluctuations are small.

 Subsequent researchers have studied
how other preference specifications   would affect the calculated
costs.  Tallarini (1996, 2000) used a version of the preferences
described in example 2 and found larger costs of business cycles
when parameters are calibrated to match data on asset prices.
Hansen, Sargent, and Tallarini (1999) and Alvarez and Jermann
(1999) considered local measures of the cost of business cycles
and provided ways to link them to the \idx{equity premium} puzzle,
to be studied in chapter
\use{assetpricing2}.
\auth{Hansen, Lars P.}
 \auth{Tallarini, Thomas} \auth{Alvarez,
Fernando} \auth{Jermann, Urban}

\section{Polynomial approximations}
Judd (1998) describes a method for iterating on the Bellman
equation  using a polynomial to approximate the value function and
a numerical optimizer  to perform the optimization at each
iteration. We describe this method in the context of the Bellman
equation for a  particular problem that we shall encounter later.


\auth{Hopenhayn, Hugo A.}
\auth{Nicolini, Juan Pablo}
In chapter \use{socialinsurance}, we
shall study Hopenhayn and Nicolini's (1997)
model of optimal unemployment insurance. A planner wants to
provide incentives to an unemployed worker to search for a new job
while also partially  insuring the worker  against bad luck
in the search process.  The planner seeks to
deliver discounted expected utility $V$ to an unemployed worker
at minimum cost while providing proper incentives to search for work.
Hopenhayn and Nicolini   show that the minimum cost $C(V)$ satisfies
the  Bellman equation
$$ C(V)  = \min_{V^u} \left\{ c + \beta [1 - p(a)] C(V^u) \right\} \EQN hugo23 $$
where $c,a$ are given by
$$ c = u^{-1}\left[ \max\left(0,
     V+a -\beta \{p(a)V^e + [1-p(a)]V^u\} \right)  \right]. \EQN hugo21$$
and
$$ a = \max\left\{0, {\log[r \beta (V^e - V^u)] \over r } \right\}.
             \EQN hugo22 $$
Here $V$ is a discounted present value   that an insurer
has promised to an unemployed  worker, $V_u$ is a value for next
period that the insurer promises the worker if he
remains unemployed, $1-p(a)$ is the probability
of remaining unemployed if the worker exerts search effort $a$,
and $c$ is the worker's consumption level.   Hopenhayn and
Nicolini assume that
$p(a) = 1 - \exp(ra)$, $r >0$.
%   Equation
%\Ep{hugo23} is the Bellman equation associated with a cost
%minimization problem for the insurer.

\tag{compstra1}{\the\pageno}
\subsection{Recommended computational strategy}
To approximate the solution of the Bellman
equation \Ep{hugo23}, we apply a computational procedure   described
by Judd (1996, 1998).
The method uses a
 polynomial to approximate the $i$th iterate
$C_i(V)$ of  $C(V)$.
This polynomial is stored on the computer in terms
of  $n+1$ coefficients.  Then at each iteration, the Bellman equation
is to be solved at a  small number $m \geq n+1$
 values of
$V$.   This procedure gives values of the $i$th iterate of the value
function
$C_i(V)$ at those particular $V$'s.  Then we interpolate
(or ``connect
the dots'') to fill in the  continuous function $C_i(V)$.
Substituting this approximation $C_i(V)$ for $C(V)$ in equation \Ep{hugo23},
we pass  the minimum problem on
the right
side of  equation \Ep{hugo23} to a numerical minimizer.   Programming
languages like Matlab and Gauss  have easy-to-use algorithms for
minimizing continuous functions of several variables.
We solve one such numerical problem  minimization for each node value for
$V$. Doing so yields optimized value $C_{i+1}(V)$ at those node points.
We then interpolate to build up
$C_{i+1}(V)$.  We iterate on this scheme to convergence.
Before summarizing the algorithm,
we provide a brief description of Chebyshev polynomials.


\subsection{Chebyshev polynomials}
 Where $n$ is a nonnegative  integer and $x \in \bbR$,
 the $n$th \idx{Chebyshev polynomial}, is
$$ T_n(x) = \cos(n \cos^{-1} x). \EQN cheb1 $$
Given coefficients $c_j, j = 0, \ldots, n$,
 the $n$th-order Chebyshev polynomial approximator is
$$ C_n(x) =  c_0 + \sum_{j=1}^n c_j T_j(x). \EQN cheb2 $$

We are given a real-valued function $f$ of a single variable
 $x \in [-1, 1]$. For computational purposes,
we want to form an approximator to $f$ of the form \Ep{cheb2}.
Note that we can store  this approximator simply as
the $n+1$ coefficients  $c_j, j= 0, \ldots , n$.
To form the approximator, we evaluate $f(x)$ at
$n+1$ carefully chosen points, then use a least-squares formula
to form the $c_j$'s   in   equation \Ep{cheb2}.
Thus, to interpolate a function of a single variable   $x$ with domain
$x \in [-1, 1]$,  Judd (1996, 1998) recommends  evaluating
the function at the $m \geq n+1$ points $x_k, k= 1, \ldots, m$,
where
$$ x_k = \cos\left({2 k - 1 \over 2 m } \pi\right) , k=1, \ldots , m .
        \EQN cheb3 $$
Here $x_k$ is the zero of the $k$th Chebyshev polynomial
on $[-1, 1]$.
Given the $m \geq   n+1 $ values of     $f(x_k)$ for $k= 1, \ldots , m$,
choose the least-squares values of $c_j$
$$ c_j = { \sum_{k=1}^m f(x_k) T_j(x_k) \over
 \sum_{k=1}^m T_j(x_k)^2  }, \ j= 0, \ldots , n  \EQN cheb4 $$

\subsection{Algorithm: summary}\label{HNalgorithm}%
In summary, applied to the
Hopenhayn-Nicolini model, the numerical procedure consists of the following steps:
\medskip

\item{1.}  Choose upper and lower bounds for $V^u$, so
that $V$ and $V^u$ will be understood to
reside in the interval $[\underline V^u, \overline V^u]$.  In particular,
set $\overline V^u = V^e - {1 \over
\beta p'(0) }$, the bound required to
assure positive search effort,  computed in chapter \use{socialinsurance}.
Set $\underline V^u = V_{rm aut}$.


\item{2.}  Choose a degree $n$ for the approximator, a Chebyshev polynomial,
and a  number $m \geq n+1$ of nodes or grid points.

\item{3.}  Generate the $m$ zeros of the  Chebyshev polynomial
on the set $[1, -1]$, given by \Ep{cheb3}.

\item{4.} By a change of scale,  transform  the $z_i$'s to
corresponding points $V^u_\ell$ in $[\underline V^u, \overline V^u]$.

\item{5.} Choose initial values of the $n+1$ coefficients in
the Chebyshev polynomial, for example, $c_j =0, \ldots, n$.
 Use these coefficients to
define the function $C_i(V^u)$ for iteration number $i=0$.

\item{6.}   Compute the  function   $\tilde C_i(V)
\equiv c + \beta [1-p(a)] C_i(V^u)$, where
$c,a$ are determined as functions of $(V, V^u)$
 from equations \Ep{hugo21} and \Ep{hugo22}.
This computation builds in the functional forms and parameters
of $u(c)$ and $p(a)$, as well as $\beta$.


\item{7.}  For each point $V^u_\ell$, use a numerical
minimization program to find
$C_{i+1}(V^u_\ell) =  \min_{V^u} \tilde C_i(V_u)$.

\item{8.}   Using these $m$ values of $C_{j+1}(V^u_\ell)$, compute
new values of the coefficients in the Chebyshev polynomials
by using ``least squares'' [formula \Ep{cheb4}].
     Return to step 5 and iterate to convergence.

\mtlb{schumaker.m}
\subsection{Shape-preserving splines}
Judd (1998) points out that because they do not
preserve concavity,  using Chebyshev polynomials  to approximate
value functions can cause problems.
He recommends the Schumaker quadratic shape-preserving spline.  It ensures
\index{spline!shape preserving}%
that the objective in the maximization step  of iterating on
a Bellman equation  will be concave
and differentiable (Judd,  1998, p. 441).
Using Schumaker splines  avoids the type of internodal
oscillations associated with other polynomial approximation methods. The
exact interpolation procedure is described in Judd (1998, p. 233). A
relatively small number of  nodes usually is sufficient.
 Judd and
Solnick (1994) find that this approach outperforms  linear interpolation and
discrete-state approximation methods in a deterministic optimal growth
problem.\NFootnote{The Matlab program {\tt schumaker.m}
(written by Leonardo Rezende of the
University of Illinois) can be used to compute the spline.  Use the
Matlab command {\tt ppval} to evaluate the spline.}
\tag{compstra2}{\the\pageno}
\section{Concluding remarks}
This chapter has described two of three standard methods for
approximating solutions of
dynamic programs numerically: discretizing the  state
space and using polynomials to approximate the value function.
  The next chapter describes the third method: making
the problem have a quadratic return function and linear
transition law.   A  benefit of making the  restrictive
linear-quadratic assumptions is that they make solving a dynamic
program easy by exploiting the ease with which
stochastic linear difference equations can be manipulated.
