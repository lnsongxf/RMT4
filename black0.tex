%\eqnotracetrue
%\pageno=3
\footnum=0
\chapter{Overview\label{overview}}%


\section{Warning}

This chapter provides a nontechnical summary of some themes of
this book.
We debated whether to put this chapter first or last.  A way to use this
chapter is to read it twice, once before reading anything else in the book,
then again after having mastered the techniques presented  in the rest of the book.
That second time, this chapter will be easy and enjoyable reading,
and it will remind you of connections that transcend a variety of apparently disparate
topics.  But on first reading, this chapter will be difficult, partly because
the discussion is mainly literary and therefore incomplete.  Measure
what you have learned by comparing your understandings after those first
and second readings.  Or just skip this chapter and read it after
the others.

\section{A common ancestor}

Clues in our mitochondrial DNA tell  biologists that we  humans
share a common ancestor called Eve who lived 100,000 years ago.
All of macroeconomics too seems to have descended from a
common source, Irving Fisher's and Milton Friedman's
 consumption Euler
equation, the cornerstone of
 the permanent income theory of consumption.
Modern macroeconomics records the fruits and frustrations of a long
love-hate affair with the permanent income mechanism. As a way of
summarizing some important themes in our book, we briefly
chronicle some  of the high and low points of this long affair.


\section{The savings problem}
A consumer  wants to maximize
$$E_0 \sum_{t=0}^\infty \beta^t u(c_t) \EQN sintro2 $$
where $\beta \in (0,1)$, $u$ is a twice continuously
differentiable, increasing, strictly concave utility function, and
$E_0$ denotes a mathematical expectation conditioned on time $0$
information. The consumer faces a sequence of budget
constraints\NFootnote{We use a different notation in chapter
\use{selfinsure}: $A_t$ here conforms to $-b_t$ in chapter \use{selfinsure}.}
$$ A_{t+1} = R_{t+1} (A_t  + y_t -c_t) \EQN sintro3 $$
for $t\geq 0$,
where $A_{t+1} \geq \underline A$ is the consumer's holdings of an asset at the beginning
of period $t+1$, $\underline A$ is a lower bound on asset holdings,
 $y_t$ is a random endowment sequence, $c_t$ is consumption
of a single good, and $R_{t+1}$ is the gross rate of return on the asset
between $t$ and $t+1$.  In the general version of the problem,
both $R_{t+1}$ and $y_t$ can be random,  though  special cases of the
problem restrict $R_{t+1}$ further.
  A first-order necessary condition for this problem is
$$ \beta  E_t R_{t+1} {u'(c_{t+1}) \over u'(c_t)} \leq 1, \quad
=  \  {\rm if} \ A_{t+1} > \underline A. \EQN sintro1 $$
This Euler inequality recurs as either the cornerstone or the
straw man in many theories
contained in this book.

Different modeling choices
put \Ep{sintro1} to work in different ways. One
 can restrict $u, \beta$, the return process
 $R_{t+1}$, the lower bound on assets $\underline A$,
 the income process $y_t$, and   the consumption
process $c_t$ in various ways.
By making alternative choices about restrictions
to impose on subsets of these objects, macroeconomists have constructed
theories  about consumption, asset prices, and the distribution of
wealth.  Alternative versions of equation \Ep{sintro1} also underlie Chamley's (1986)
 and Judd's (1985b)
 striking
results about eventually not taxing capital.

\subsection{Linear quadratic permanent income theory}

  To obtain a version of the permanent income theory  of
Friedman (1955) and Hall (1978), set $R_{t+1} =R$, impose
$R=\beta^{-1}$, assume the quadratic utility function
$u(c_t)=-(c_t-\gamma)^2$, and allow consumption $c_t$ to
be negative. We also allow
%%that $u $ is quadratic so that $u'$ is linear.
%%Allow
$\{y_t\}$ to be an arbitrary  stationary
process, and dispense with the lower bound $\underline A$.
 The Euler inequality \Ep{sintro1} then implies that consumption is
a martingale:
$$ E_t c_{t+1} = c_t.  \EQN sintro4 $$
Subject to a boundary condition  that\NFootnote{The motivation for
using this boundary condition instead of a lower bound $\underline
A$ on asset holdings is that there is no ``natural'' lower bound on
asset holdings when consumption is permitted to be negative.
%%, as it is when $u$ is quadratic in $c$.
Chapters \use{recurge} and \use{incomplete}
 discuss  what are called ``natural  borrowing limits,''
the lowest possible appropriate values of $\underline A$ in the case
that $c$ is nonnegative.}
 $E_0 \sum_{t=0}^\infty \beta^t A_t^2 <\infty$,
equation \Ep{sintro4} and  the budget constraints \Ep{sintro3} can be solved
to yield
$$ c_t = \left[{r \over 1+r}\right]
\left[E_t \sum_{j=0}^\infty
 \left({1 \over 1+r}\right)^j y_{t+j} + A_t \right]
\EQN sintro5 $$
where $1+r = R$.  Equation \Ep{sintro5}
expresses consumption as a fixed marginal propensity to consume
${r \over 1+r}$ that is applied to the sum of human wealth -- namely
$\left[E_t \sum_{j=0}^\infty
 \left({1 \over 1+r}\right)^j y_{t+j}\right] $ -- and
 financial wealth, $A)t$.
This equation has the following notable features:
(1) consumption is smoothed on average across time: current
consumption depends only on the expected present value of nonfinancial
income;
(2)  feature (1) opens the way to Ricardian equivalence: redistributions
of lump-sum taxes over time that leave  the expected present value of
nonfinancial income unaltered do not affect consumption;
(3) there is certainty equivalence: increases in the conditional variances
of future incomes about  their forecast values do not affect consumption
(though they do diminish the consumer's utility);
(4) a by-product of certainty equivalence is that   the marginal propensities
to consume out of financial and nonfinancial wealth are equal.


\auth{Ligon, Ethan} \auth{Blundell, Richard}
\auth{Preston, Ian}%
This theory continues to be a workhorse in much
good applied  work (see Ligon (1998)
and  Blundell and
Preston (1999) for creative applications).
Chapter \use{dplinear} describes  conditions under which
certainty equivalence prevails, while chapters \use{timeseries} and \use{dplinear}
  also describe the structure of the cross-equation
restrictions   that the hypothesis of rational expectations imposes and that
 empirical studies
heavily exploit.







\subsection{Precautionary saving}\label{sec:intro_precautionary}%
A literature on ``the  savings problem'' or ``precautionary saving''
investigates the consequences of altering the assumption in the
linear quadratic permanent income theory that $u$
is quadratic, an assumption that makes   the marginal utility of consumption
become  negative for large enough $c$. Rather than assuming that
$u$ is quadratic, the literature on the savings problem
assumes that $u$ is increasing and strictly
concave.  This assumption keeps the marginal utility of
consumption above zero.   We retain other features of the
linear quadratic model ($\beta R=1$, $\{y_t\}$ is a stationary
process), but now impose a borrowing limit $A_t \geq \underline
a$.

With these assumptions, something amazing occurs:   Euler
inequality \Ep{sintro1} implies that the marginal  utility of
consumption is a {\it nonnegative\/}
supermartingale.\NFootnote{See chapter \use{selfinsure}. The situation
is simplest in the case that the $y_t$ process is i.i.d. so that
the value function can be expressed as a function of level $y_t +
A_t$ alone: $V(A+y)$.  Applying the Benveniste-Scheinkman formula
from chapter \use{dynamicdp1} shows
 that $V'(A+y)=u'(c)$, which implies  that when $\beta
R=1$, \Ep{sintro1} becomes $E_t V'(A_{t+1}+y_{t+1}) \leq
V'(A_t+y_t)$,
 which states that
the derivative of the value function is a nonnegative supermartingale.
%which shows that where $V(a)$ is the value function for the saving
%problem  with an i.i.d.\  endowment, condition \Ep{sintro1}
%implies that $V'(a)$ is a nonnegative martingale,
That in turn implies
that $A$ almost surely diverges  to $+\infty$.} That gives the
model the striking implication that $c_t \rightarrow_{as} +\infty$
and $A_t \rightarrow_{as} +\infty$, where $\rightarrow_{as}$ means
almost sure convergence. Consumption and wealth will fluctuate
randomly in response to income fluctuations, but so long as
randomness in income continues,
 they will drift upward over time without bound.
  If
randomness eventually expires in the tail of the income
 process, then both consumption
and income  converge.  But even
small perpetual random fluctuations
in income are enough to cause both consumption and assets to diverge
to $+\infty$.   This  response
of the optimal consumption
plan to randomness is required by the Euler equation \Ep{sintro1} and
is called precautionary savings.
By keeping the marginal utility of consumption positive,
precautionary savings models arrest the  certainty equivalence  that prevails
in the linear quadratic permanent income model.
Chapter \use{selfinsure} studies the savings problem in depth and struggles
to understand the workings of the powerful martingale convergence theorem.
The supermartingale convergence theorem also plays an important role in the model
insurance with private information in chapter \use{socialinsurance}.

\subsection{Complete markets, insurance, and the distribution of wealth}\label{sec:intro_complete}%
To build a model of the distribution of wealth, we consider a setting
with many consumers.
  To start, imagine a large number of {\it ex ante\/}
identical consumers with preferences \Ep{sintro2} who are allowed
to share their income risk by trading one-period contingent
claims.  For simplicity, assume that the saving possibility
represented by the budget constraint \Ep{sintro3} is no longer available\NFootnote{It
can be shown that even if it were available, people would not want
to use it.} but that it is replaced by access to an extensive set
of insurance markets. Assume that household $i$ has an income
process $y_t^i = g_i(s_t)$ where $s_t$ is a state vector governed
by a Markov process with transition density $\pi(s'|s)$, where $s$ and $s'$
are elements of a common state space ${\bf S}$.    (See
chapters \use{timeseries} and \use{recurge} for  material about Markov
chains and their uses in equilibrium models.) Each period every household can trade
one-period state-contingent claims to consumption next period.
Let $Q(s'|s)$ be the price of one unit of consumption next period
in state $s'$ when the state this period is $s$. When household
$i$ has the opportunity to trade such state-contingent securities,
its first-order conditions for maximizing \Ep{sintro2} are
$$ Q(s_{t+1}|s_t)  = \beta {u'(c^i_{t+1}(s_{t+1}))\over u'(c^i_t(s_t)) }
     \pi(s_{t+1}|s_t) .
  \EQN sintro15 $$
Notice that $\int_{s_{t+1}} Q(s_{t+1} | s_t) d s_{t+1}$
is the price of a risk-free claim on consumption one period ahead: it is
thus the reciprocal of the gross risk-free interest rate $R$. Therefore, if we sum
both sides of \Ep{sintro15} over $s_{t+1}$, we obtain our standard
consumption Euler condition \Ep{sintro1}  at
equality.\NFootnote{That the asset is risk-free becomes manifested
in $R_{t+1}$ being a function of $s_t$, so that it is known at
$t$.} Thus, the complete markets equation \Ep{sintro15} is
consistent with our complete markets Euler equation \Ep{sintro1}, but \Ep{sintro15}
imposes more. We will exploit
this fact extensively in chapter \use{optax}.


 In a widely studied special case, there is no aggregate risk, so
that $\int_{i}  y_t^i d \, i  = \int_i g_i(s_t) d \, i = {\rm constant}$.
In that case, it can be shown that the competitive equilibrium
state-contingent prices become
$$ Q(s_{t+1} | s_t) = \beta \pi(s_{t+1} | s_t) .  \EQN sintro16 $$
This, in turn, implies that the risk-free
gross rate of return $R$ is $\beta^{-1}$.\NFootnote{This follows because the
price of a risk-free claim to consumption tomorrow at date $t$ in state $s_t$ is  $\sum_{s_{t+1}} Q(s_{t+1} | s_t )
= \beta \sum_{s_{t+1}} \pi(s_{t+1} | s_t) = \beta.$}
If we substitute \Ep{sintro16} into \Ep{sintro15}, we discover that
$c^i_{t+1}(s_{t+1}) = c^i_t(s_t)$ for all $(s_{t+1}, s_t)$.
 Thus, the consumption
of consumer $i$ is constant across time and across states of nature
$s$, so that in equilibrium, all
idiosyncratic risk is insured away.
   Higher present-value-of-endowment consumers will have permanently higher
consumption than lower present-value-of-endowment consumers, so that there
is a nondegenerate cross-section distribution of wealth and consumption.
In this model, the cross-section distributions of wealth and consumption
replicate themselves over time, and furthermore each individual forever
occupies the same position in that distribution.

\auth{Quadrini, V.}\auth{R\'\i os-Rull, J.V.} \auth{Perri,Fabrizio}\auth{Krueger, Dirk}%
\auth{Davies, James B.}\auth{Shorrocks, Anthony F.}\auth{Rodriquez, S.B.}%
\auth{Diaz-Gim\' enez, J.}%
  A model that has the cross-section distribution  of wealth and consumption
being time invariant is not a bad approximation to the data.
But there is ample evidence that individual
households' positions {\it within\/} the distribution of wealth move over
 time.\NFootnote{See D\'\i az-Gim\' enez, Quadrini and  R\'\i os-Rull (1997);
Krueger and Perri (2004, 2006); Rodriguez, D\'\i az-Gim\' enez,  Quadrini
and  R\'\i os-Rull (2002); and Davies and  Shorrocks (2000).} Several models
described in this book alter consumers' trading opportunities in
ways designed
 to frustrate risk sharing enough to cause individuals' position
in the  distribution of wealth to change with luck and enterprise.
One class that emphasizes luck is the set of incomplete markets
models started by  Truman Bewley. It eliminates the household's
access to almost all markets and returns it to the environment of
the precautionary savings model.

\subsection{Bewley models}

At first glance,
the precautionary savings model with $\beta R=1$ seems like
a bad starting point
for building a theory
that aspires to explain a situation in which cross-section distributions
of consumption and wealth are constant over time even as individuals
experience random fluctuations within that distribution.
  A panel of households described by the precautionary savings model with
$\beta R =1$ would  have cross-section distributions  of wealth
and consumption that march upward and never settle down. What
have come to be called Bewley models are constructed by lowering
the interest rate $R$ to allow those cross-section distributions
to settle down.\NFootnote{It is worth thinking about the sources of the following differences.  In  the complete markets model sketched in subsection
\use{sec:intro_complete},  an {\it equilibrium\/} risk-free gross interest rate $R$ satisfies
$R \beta=1$ and  each consumer completely smooths consumption across both states and time, so that the
distribution of consumption trivially converges. The precautionary savings model of section \use{sec:intro_precautionary}
{\it assumes\/} that $R \beta =1$ and derives the outcome that each consumer's consumption and financial wealth both diverge toward
$+ \infty$. Why can $\beta R =1$ be compatible with non-exploding individual consumption and wealth levels in the complete markets model
of subsection \use{sec:intro_complete}, but not in the precautionary savings model of subsection \use{sec:intro_precautionary}?}
 Bewley models are arranged so that the cross
section distributions of consumption, wealth, and income are
constant over time and so that  the asymptotic stationary
distributions of consumption, wealth, and income for an individual
consumer across time equal the corresponding cross-section
distributions across people.
 A Bewley model can thus
be thought of as starting with a continuum of consumers operating according to the
precautionary savings model with $\beta R=1$ and its diverging
individual asset process. We then lower the interest rate enough
to make assets converge to a distribution whose cross-section
average clears  a market for a risk-free asset.   Different
versions of Bewley  models are distinguished by what the risk-free
asset is. In some versions it is a consumption loan from one
consumer to another; in others it is fiat money; in others it can
be either consumption loans or fiat money; and in yet others it is
claims on physical capital.  Chapter \use{incomplete} studies
these alternative interpretations of the risk-free asset.




\topfigure{chap0pic1}
\centerline{\epsfxsize=3.5true in\epsffile{Bewley_chap1.eps}}
\caption{Mean of time series average of household consumption
as function of risk-free gross interest rate $R$.}
\infiglist{chap01}
\endfigure

As a function of a constant gross interest rate $R$,
Figure \Fg{chap0pic1}
plots the time series average of asset holdings for an
individual consumer.  At $R=\beta^{-1}$, the time series mean
of the individual's assets diverges, so that $E a(R)$ is infinite.
 For $R < \beta^{-1}$, the mean exists.  We require that a continuum
of {\it ex ante\/} identical but {\it ex post\/}
 different consumers share the same
time series average $E a(R)$ and also that the distribution of
$a$ over time for a given agent equals the distribution of $A_{t+1}$ at a point in time across
agents.  If the
asset in question is a pure consumption loan, we require
as an equilibrium condition that $E a(R) =0$, so that
borrowing equals lending. If the asset is fiat money, then
we require that $E a(R) = {M \over p}$, where $M$ is a fixed stock of fiat
money and $p$ is the price level.

Thus, a Bewley model lowers the interest rate $R$
enough to offset the precautionary savings force that with $\beta R=1$ propels
assets upward
in the savings problem.
  Precautionary saving remains an important force in Bewley
models: an increase in the volatility of income generally pushes
the $E a(R)$ curve to the right, driving the equilibrium $R$ downward.






\subsection{History dependence in standard consumption models}

Individuals' positions in the wealth distribution
are frozen in  the complete markets model,    but not
in the Bewley model, reflecting the absence or presence, respectively,
 of {\it history dependence\/} in
 equilibrium allocation rules for consumption.
   The preceding version of the complete markets model
erases history dependence, while the savings problem model and the
Bewley model do not.


History dependence is present in these models in an easy to handle
recursive way because the household's asset level completely
encodes the history of endowment realizations that it has
experienced.  We want a way of representing history dependence
more generally in contexts where a stock of assets does not
suffice to summarize history. History dependence can be
troublesome because without a convenient low-dimensional state
variable to encode history, it  requires that there be a separate
decision rule for each date that expresses the time $t$ decision
as a function of the history at time $t$, an object with a number
of arguments that grows exponentially with $t$. As analysts, we have a strong
incentive to find a low-dimensional state variable. Fortunately, economists have
made tremendous strides in handling history dependence with
recursive methods that summarize a history with
a single number and that permit compact time-invariant expressions for
decision rules. We shall discuss history dependence later in this
chapter and will encounter many such examples in chapters
\use{stackel} through \use{wldtrade}.
% \use{credible},  \use{socialinsurance}, and
%\use{socialinsurance2}.



\subsection{Growth theory}
  Equation \Ep{sintro1} is also a key ingredient of growth theory
(see chapters \use{linappro} and \use{growth}). In the  one-sector
 growth
model, a representative  household solves a version of  the savings problem
in which
the single asset is interpreted as a claim on the return from a
physical capital stock $K$ that enters a  constant returns to
scale production function $F(K,L)$,  where $L$ is labor input.
When returns to capital are tax free, the theory equates the
gross rate of return $R_{t+1}$ to the gross marginal product of
capital net of depreciation, namely,   $F_{k,t+1} +(1-\delta)$, where
$F_k(k,t+1)$ is the marginal product of capital
 and $\delta$ is a
depreciation  rate. Suppose that we add leisure to the utility
function, so that we replace $u(c)$ with the more
general one-period utility function
$U(c,\ell)$, where $\ell$ is the household's leisure.
Then the appropriate version of the consumption Euler
condition  \Ep{sintro1} at equality becomes
$$ U_c(t) = \beta U_c(t+1) [F_k(t+1) + (1-\delta) ]. \EQN sintro40 $$
The constant returns to scale property implies that
$F_k(K,N) = f'(k)$, where $k=K/N$ and $F(K,N)=N f(K/N)$.
If there exists a steady state in which $k$ and $c$ are constant over time,
then equation \Ep{sintro40} implies that
it must satisfy
$$   \rho + \delta = f'(k)  \EQN sintro41 $$
where $\beta^{-1} \equiv (1+\rho)$. The value of $k$ that solves this equation
is called the ``augmented Golden rule'' steady-state level of the capital-labor
ratio. This celebrated equation shows how technology (in the form of
$f$ and $\delta$) and time preference (in the form of $\beta$) are the
determinants of the steady-state level of capital when income from
capital is not taxed.  However,  if income from capital is
taxed at the flat rate marginal
rate $\tau_{k,t+1}$,
then the Euler equation \Ep{sintro40} becomes modified
$$ U_c(t) = \beta U_c(t+1) [F_k(t+1) (1-\tau_{k,t+1})
 + (1-\delta) ].  \EQN sintro42 $$
If the flat rate tax on capital is constant and if a steady-state
$k$ exists, it must satisfy
$$ \rho + \delta = (1-\tau_k) f'(k) . \EQN sintro45 $$
This equation shows how taxing capital diminishes the steady-state
capital labor ratio.   See chapter \use{linappro} for an extensive
analysis of the one-sector growth model when the government levies
time-varying flat rate taxes on  consumption, capital, and labor,
as well as offering an investment tax credit.


\subsection{Limiting results from dynamic optimal taxation}
Equations \Ep{sintro41} and \Ep{sintro45} are central to the
dynamic theory of optimal taxes. Chamley (1986) and Judd (1985b)
\auth{Judd, Kenneth L.}\auth{Chamley, Christophe}%
forced the
government to finance an exogenous stream of government purchases,
gave it the capacity to levy time-varying flat rate taxes on labor
and capital at different rates, formulated an optimal taxation
problem (a so-called Ramsey problem), and studied the possible
limiting behavior of the optimal taxes. Two Euler  equations play
a decisive role in determining the limiting tax rate on capital in
a nonstochastic economy:  the household's Euler equation
\Ep{sintro42}, and  a similar consumption Euler equation for the
Ramsey planner that takes the form
$$ W_c(t) = \beta W_c(t+1) [F_k(t+1) + (1-\delta) ]\EQN sintro43 $$
where
$$ W(c_t, \ell_t) = U(c_t,\ell_t) + \Phi[U_c(t) c_t - U_\ell(t) (1-\ell_t) ]
  \EQN sintro44 $$
and where $\Phi$ is a Lagrange multiplier on the government's
intertemporal budget constraint. As Jones, Manuelli, and Rossi
(1997) emphasize,\auth{Jones, Larry E.}\auth{Manuelli, Rodolfo}
\auth{Rossi, Peter E.}%
 if the function $W(c,\ell)$ is simply viewed
as a peculiar utility function, then what is called the primal
version of the Ramsey problem can    be viewed as an ordinary
optimal growth problem with period utility function $W$ instead of
$U$.\NFootnote{\label{fnt:conflict1}Notice that so long as $\Phi   >0$ (which occurs
whenever taxes are necessary), the objective in the primal version
of the Ramsey problem disagrees with the preferences of the
household over $(c,\ell)$ allocations. This conflict is the source
of a time-inconsistency problem in the Ramsey problem with
capital.}


In a Ramsey allocation, taxes must be such that {\it both\/}
\Ep{sintro40} and \Ep{sintro43} always hold, among other equations.
Judd and Chamley note the following implication of the two
Euler equations \Ep{sintro40}  and \Ep{sintro43}.
If the government expenditure sequence converges and if
a steady state exists in which $c_t, \ell_t, k_t, \tau_{kt}$ all converge,
then it must be true that \Ep{sintro41} holds {\it in addition\/}
 to  \Ep{sintro45}.
But both of these conditions can prevail only
if $\tau_k =0$.   Thus, the steady-state properties of two versions of our consumption
Euler  equation \Ep{sintro1} underlie Chamley and Judd's remarkable result
that asymptotically it is optimal not to tax capital.

  In stochastic versions of dynamic optimal taxation problems, we shall glean
additional insights  from
\Ep{sintro1} as embedded  in the asset-pricing
equations \Ep{sintro7a} and \Ep{sintro8a}.
  In optimal taxation problems,
the government has the ability to manipulate asset prices through its
influence on the equilibrium consumption  allocation that
contributes to  the stochastic discount factor  $m_{t+1,t}$ defined in equation \Ep{sintro7a} below.
The Ramsey government seeks a way wisely
to use its power to
revalue its existing debt by altering state-history
prices. To appreciate what the Ramsey government is doing, it helps to know
the theory of asset pricing.


\auth{Lucas, Robert E., Jr.} \auth{Breeden, Douglas T.}

\subsection{Asset pricing}
 The dynamic asset  pricing theory of Breeden (1979)  and Lucas (1978)
also starts with \Ep{sintro1}, but alters what is fixed and what is
free.  The Breeden-Lucas
 theory is silent about the endowment process $\{y_{t}\}$ and sweeps
it into the background. It
fixes a function $u$ and a discount factor $\beta$,
and takes a consumption process $\{c_t\}$ as given.  In particular,
assume that $c_t = g(X_t)$, where $X_t$ is a Markov process with
transition c.d.f.\ $F(X'|X)$.  Given these inputs, the theory is assigned the task of
restricting the rate
of return on an asset, defined by Lucas as a
claim on the consumption endowment:
$$ R_{t+1} = {p_{t+1}    + c_{t+1} \over p_t}$$
where $p_t$ is the price of the asset.  The Euler inequality
\Ep{sintro1} becomes
$$ E_t \beta {u'(c_{t+1}) \over u'(c_t)}\left({p_{t+1} + c_{t+1}
     \over p_t}\right)  =1. \EQN sintro6 $$
This equation can be solved for a pricing function
$p_t =p(X_t)$. In particular, if we substitute $p(X_t)$ into
\Ep{sintro6}, we get Lucas's functional equation for
$p(X)$.

\subsection{Multiple assets}

If the consumer has access to several assets,
 a version of \Ep{sintro1} holds for each
asset:
$$ E_t \beta {u'(c_{t+1}) \over u'(c_t)} R_{j,t+1} =1  \EQN sintro7 $$
where $R_{j,t+1}$ is the gross rate of return on asset $j$.
Given a utility function $u$, a discount factor $\beta$,
and the hypothesis of rational expectations (which allows the researcher
to use empirical projections as counterparts of the  theoretical projections
$E_t$),
equations \Ep{sintro7} put extensive restrictions across the moments
of a vector time series for $[c_t, R_{1,t+1}, \ldots, R_{J,t+1}]$.
 A key finding of the literature (e.g., Hansen and Singleton, 1983)
 is that for $u$'s with plausible
curvature,\NFootnote{Chapter \use{assetpricing2} describes Pratt's (1964)
mental
experiment for deducing plausible  curvature.}
consumption is too smooth for $\{c_t, R_{j,t+1}\}$ to satisfy
equation \Ep{sintro7}, where $c_t$ is measured as aggregate
consumption.

Lars Hansen and others have elegantly organized this evidence as follows.
Define the stochastic discount factor $$m_{t+1,t}  = \beta {u'(c_{t+1} )
\over u'(c_t)} \EQN sintro7a $$
 and write \Ep{sintro7} as
  $$ E_t m_{t+1,t} R_{j,t+1}  = 1 . \EQN sintro8 $$
Represent the gross rate of return as
$$ R_{j,t+1} = {o_{t+1} \over q_t } $$
where $o_{t+1}$ is a one-period payout on the asset and $q_t$ is
the price of the asset at time $t$.  Then \Ep{sintro8} can be expressed
as
$$ q_t = E_t m_{t+1,t} o_{t+1}. \EQN sintro8a $$
The structure of \Ep{sintro8a} justifies calling $m_{t+1,t}$
a stochastic discount factor: to determine the price of an
asset, multiply the random payoff for each state by the discount factor
for that state, then add over states by taking a conditional expectation.
Applying the definition of a conditional covariance and a Cauchy-Schwartz
inequality to this equation implies
$$ {q_t \over E_t m_{t+1,t}} \geq E_t o_{t+1} -
  {\sigma_t(m_{t+1,t}) \over  E_t m_{t+1,t}} \sigma_t(o_{t+1}) \EQN sintro9 $$
where $\sigma_t(y_{t+1})$ denotes  the conditional standard deviation of
$y_{t+1}$.
Setting $o_{t+1}=1$ in \Ep{sintro8a} shows that $E_t m_{t+1,t}$ must be
the time $t$
price of a risk-free one-period security.   Inequality
\Ep{sintro9} bounds the ratio of the price of a risky security $q_t$
to the price of a risk-free security $E_t m_{t+1,1}$ by the right side,
which equals the expected
payout on that risky asset {\it minus\/} its conditional
standard deviation $\sigma_t(o_{t+1})$ {\it times\/}
 a ``market price of risk''
 ${\sigma_t(m_{t+1,t}) /  E_t
  m_{t+1,t}}$.
   %An efficient portfolio attains this bound.
By using data only on payouts $o_{t+1}$ and prices $q_t$,
inequality \Ep{sintro9} has been used to estimate the market price
of risk without restricting how  $m_{t+1,t}$  relates
to consumption.
If we take these atheoretical  estimates of
 ${\sigma_t(m_{t+1,t}) /  E_t
  m_{t+1,t}}$
 and compare them with the theoretical
values of
 ${\sigma_t(m_{t+1,t}) /  E_t
  m_{t+1,t}}$ that we get
with a
plausible curvature for $u$, and
by imposing $\hat m_{t+1,t} = \beta {u'(c_{t+1})\over u'(c_t)}$ for aggregate
consumption,
 we find that the theoretical
$\hat m$ has far too little volatility to account for the
atheoretical estimates of the conditional coefficient of variation
of $m_{t+1,t}$.    As we discuss extensively in chapter \use{assetpricing2},
this outcome reflects the fact that aggregate consumption is too
smooth to account for atheoretical estimates of the market price of risk.


  There have been two broad types of response to the empirical challenge.
The first retains \Ep{sintro8} but abandons \Ep{sintro7a} and instead
 adopts
a statistical  model for $m_{t+1,t}$. Even without the link that
equation \Ep{sintro7a} provides to consumption, equation
\Ep{sintro8} imposes restrictions across  asset returns and
$m_{t+1,t}$ that can be used to identify the $m_{t+1,t}$ process.
Equation \Ep{sintro8} contains no-arbitrage conditions   that
restrict the joint behavior of returns. This has been a fruitful
approach in the affine  term structure literature (see Backus and
Zin  (1993), Piazzesi (2000), and Ang and Piazzesi
(2003)).\NFootnote{Affine term structure models generalize earlier
models that implemented rational expectations versions of the
expectations theory of the term structure of interest rates. See
Campbell and Shiller (1991), Hansen and Sargent (1991), and
Sargent (1979).} \auth{Campbell, John Y.} \auth{Piazzesi, Monika}
\auth{Ang, Andrew} \auth{Hansen, Lars P.} \auth{Zin, Stanley E.}
\auth{Backus, David K.}

  Another approach has been to disaggregate and
to  write the household-$i$ version of \Ep{sintro1}:
$$ \beta  E_t R_{t+1} {u'(c_{i,t+1}) \over u'(c_{it})} \leq 1, \quad
= \  \ {\rm if} \  A_{i,t+1} > \underline A_i. \EQN sintro10 $$ If at
time $t$, a subset of households are on the corner, \Ep{sintro10}
will hold with equality only for another  subset  of households.
Households in the second set price assets.\NFootnote{David
Runkle (1991) and Gregory Mankiw and Steven Zeldes (1991)
  checked \Ep{sintro10}
for subsets of agents.} \auth{Mankiw, Gregory} \auth{Zeldes,
Stephen P.}\auth{Runkle, David}

Chapter \use{socialinsurance2} describes a model of Harald Zhang
(1997) and Alvarez and Jermann (2000, 2001).  The model introduces
participation (collateral) constraints and shocks in a way that
makes a changing subset of agents $i$  satisfy \Ep{sintro10}.
 Zhang and Alvarez and Jermann formulate
these models by adding participation constraints to the recursive
formulation of the consumption problem based on \Ep{sintro22}.
Next we briefly describe the structure of these models and their
attitude toward our theme equation,  the consumption Euler
equation \Ep{sintro1}.  The idea of Zhang and Alvarez and Jermann
was to  meet the empirical asset pricing  challenges by disrupting
\Ep{sintro1}.   As we shall see, that requires  eliminating some
of the assets that some of the households can trade. These
advanced models exploit a convenient method for representing and
manipulating history dependence. \auth{Zhang, Harold}
\auth{Alvarez, Fernando} \auth{Jermann, Urban}



\section{Recursive methods}
 The pervasiveness  of the consumption Euler inequality will be a
significant substantive theme of this book. We now turn to a
methodological theme, the imperialism of  the recursive method
called dynamic programming.

The notion that underlies dynamic programming
is a finite-dimensional object called the
{\it state\/} that,
from the point of view of current and future payoffs,
 completely summarizes the current situation
of a decision maker.
If an optimum problem has a low-dimensional state vector,
immense simplifications follow.  A recurring theme of modern macroeconomics
and of this book is that
finding an appropriate state vector is an art.

To illustrate the idea of the state in  a simple setting,
  return to the savings problem and assume that the
consumer's endowment process is a time-invariant function
of a state $s_t$ that follows a Markov process with time-invariant one-period
transition density $\pi(s'|s)$ and initial  density $\pi_0(s)$,
 so that $y_t = y(s_t)$.
To begin, recall the description
\Ep{sintro5}
of consumption
that prevails
 in the special linear quadratic version of the savings problem. Under our
present assumption that $y_t$ is a time-invariant function of the Markov
state, \Ep{sintro5} and the household's budget constraint imply
the following representation of the household's decision rule:
$$\EQNalign{ c_t & =   f(A_t, s_t) \EQN sintro21;a \cr
             A_{t+1} & =  g( A_t, s_t) . \EQN sintro21;b \cr} $$
Equation \Ep{sintro21;a} represents consumption  as a
time-invariant function of a state vector $(A_t, s_t)$.  The Markov
component $s_t$ appears in
\Ep{sintro21;a} because it contains all of the information
that is useful in forecasting future endowments (for the linear quadratic
model, \Ep{sintro5} reveals the household's incentive to forecast future
incomes),  and the asset level $A_t$ summarizes the individual's current financial
wealth. The $s$ component is assumed to be
exogenous to the   household's decisions
and has a stochastic motion governed by $\pi(s'|s)$.
But the future path of $A$ is chosen by the household and is described
by \Ep{sintro21;b}. The system formed by \Ep{sintro21} and the Markov
transition density $\pi(s'|s)$ is said to be {\it recursive\/}
 because
it expresses a current decision $c_t$ as a function  of the state and tells how
to update  the state.
By iterating \Ep{sintro21;b},
notice that $A_{t+1}$ can be expressed as a function of
the history  $[s_t, s_{t-1}, \ldots,
s_0]$
and $A_0$. The endogenous state variable financial wealth thus
encodes all payoff-relevant aspects of the history of the
exogenous component of the state $s_t$.

  Define the   value function $V(A_0,s_0)$ as the optimum value
of the savings problem starting from initial state $(A_0, s_0)$.
The value function $V$ satisfies the following functional
equation, known as a Bellman equation:
$$ V(A,s) = \max_{c,A'}\left\{ u(c) + \beta E[ V(A',s') |s] \right\}
  \EQN sintro50 $$
where the maximization is subject to
$A' = R(A+y-c)$ and $y = y(s)$.
Associated with a  solution $V(A,s)$ of the Bellman equation is
the pair of policy functions
$$\EQNalign{ c & = f(A,s) \EQN sintro51;a \cr
   A' & = g(A,s) \EQN sintro51;b \cr} $$
from \Ep{sintro21}.
The {\it ex ante\/} value  (i.e., the value of
\Ep{sintro2} before $s_0$ is drawn)
of the savings problem
is then
$$ v(A_0) = \sum_{s} V(A_0,s) \pi_0(s) . \EQN sintro52 $$
We shall make ample use of the {\it ex ante\/} value function.





\subsection{Dynamic programming and the Lucas Critique}
Dynamic programming is now recognized as
a powerful method for studying private
agents' decisions and also the decisions
of a government that wants to design an optimal
policy in the face of constraints imposed on it by private agents'
best responses to that government policy.  But it has taken
a long time for the power of dynamic programming to be realized
for government policy design problems.

Dynamic programming had been applied since the late 1950s
to  design government decision rules
to control an economy whose transition laws included
rules that described the decisions of private agents.
 In  1976 Robert E. Lucas, Jr.,\ published his now famous critique of
dynamic-programming-based econometric
policy evaluation procedures.
 The heart of Lucas's critique was
the implication for government policy evaluation of
a basic property that pertains to any optimal decision rule
for private agents with a form
 \Ep{sintro51} that attains a Bellman equation like \Ep{sintro50}.
The property is that the optimal decision rules $(f,g)$ depend on
the transition density $\pi(s'|s)$ for the exogenous component of
the state $s$. As a consequence, any widely understood government
policy that alters the law of motion for a state variable like
$s$ that appears in private agents' decision rules should alter
those private decision rules.    (In the applications that Lucas
had in mind, the $s$ in private agents' decision problems included
variables useful for predicting tax rates, the money supply, and
the aggregate price level.)  Therefore, Lucas asserted that
econometric policy
evaluation procedures that assumed that private agents' decision
rules are fixed in the face of alterations in government policy
are flawed.\NFootnote{They were flawed because they assumed ``no
response'' when they should have assumed ``best response'' of private
agents' decision rules to government decision rules.}
 Most econometric policy evaluation procedures at the time
were vulnerable to Lucas's criticism.
 To construct valid policy evaluation procedures,
 Lucas advocated building new models that
would attribute rational expectations to decision makers.\NFootnote{That
is, he wanted private decision rules to solve dynamic programming
problems with the correct transition density $\pi$ for $s$.}
Lucas's discussant Robert Gordon predicted that after that ambitious
task had been accomplished, we could then use dynamic programming
to compute optimal policies, i.e., to solve Ramsey problems.


\subsection{Dynamic programming  challenged}
But Edward C.
 Prescott's 1977 paper
entitled ``Should Control  Theory Be Used for Economic Stabilization?''
asserted that Gordon was too optimistic.
Prescott
claimed that in his 1977 JPE paper with
Kydland he had proved that
it was ``logically impossible'' to use dynamic programming to  find
optimal government policies in settings where  private traders
face genuinely dynamic problems.  Prescott
said that  dynamic programming was inapplicable to government policy
design problems because the structure of 
best responses of {\it current private decisions\/} to {\it future\/}
government policies prevents the government policy design problem from
being recursive (a
manifestation of the time inconsistency of optimal government plans).
The optimal government plan would therefore require a  government
commitment technology, and the government policy
must take the form of a sequence of history-dependent
decision rules that could not be expressed as a function
of natural state variables.

\subsection{Imperialistic response of dynamic   programming}

Much of the subsequent history of macroeconomics
belies
Prescott's claim of ``logical impossibility.''
  More and more problems that smart
people like Prescott in 1977 thought could not be attacked
with dynamic programming {\it can\/} now   be solved with
dynamic programming.
Prescott didn't put it this way in 1977, but today we would:
in 1977 we lacked a way to handle history dependence
 within  a dynamic programming framework.
Finding a recursive way to
handle history dependence  is a major achievement of the past 35 years
and an important methodological theme of this book that opens
the way to a variety of important applications.

We shall encounter important traces of
the fascinating history of this topic in various chapters.  Important
contributors to the task of overcoming Prescott's challenge seemed to work in isolation
from one another, being unaware of the complementary approaches
being followed elsewhere. Important contributors included
 Shavell and Weiss (1979);
Kydland and Prescott (1980); Miller and Salmon (1985); Pearlman;
Currie and Levine (1985); Pearlman (1992), and Hansen, Epple, and
Roberds (1985). These researchers achieved truly independent
discoveries of the same important idea. \auth{Salmon, Mark}
\auth{Miller, Marcus} \auth{Pearlman, J. G.}\auth{Levine, P.L.}
\auth{Currie, D.A.}\auth{Weiss, Laurence}\auth{Shavell, Stephen}
\auth{Kydland, Finn E.}\auth{Prescott, Edward C.}\auth{Hansen,
Lars P.} \auth{Roberds, William}

As we discuss in detail in chapter \use{stackel},
one important approach   amounted to putting a government  costate vector
on the costate  equations of the private decision makers,
then proceeding as usual to use optimal control for the government's
problem.
(A costate equation
is a version of an Euler equation.)
 Solved forward, the costate equation
depicts the dependence of private  decisions on forecasts
of future government policies that Prescott was worried about.
The key idea in this approach  was to formulate
the government's problem by taking the costate
equations of the private sector as additional {\it constraints\/}
on the government's problem. These  amount to promise-keeping
constraints (they are cast in terms of derivatives of value functions,
 not value functions themselves, because costate vectors are
gradients of value functions). After adding the  costate equations
of the  private sector (the ``followers'') to the transition
law of the government (the  ``leader''), one could then solve
the government's problem by using dynamic programming as usual.
One simply writes down a Bellman equation for the  government
planner taking the private sector costate variables as
pseudo-state variables. Then it is   almost business as
usual (Gordon was correct!). We say ``almost'' because     after the
Bellman equation is solved, there is one more step: to pick the
initial value of the private sector's costate.  To maximize the government's
criterion, this initial
condition should be set to  zero because initially there are no
promises to keep. The government's optimal decision   is a
function of the natural state variable {\it and\/} the costate
variables.
   The date $t$
 costate variables encode history and record the ``cost'' to the
government at $t$ of confirming the   private sector's  prior expectations
about the government's time $t$ decisions, expectations
that were embedded in the private sector's  decisions before $t$.
The solution is time inconsistent (the government would  always like to
reinitialize the time $t$ multiplier to zero and thereby discard
past promises, but that is ruled out by the assumption that the
government is committed to follow the optimal plan).
See chapter \use{stackel} for  many technical details, computer programs,
 and an application.

\subsection{History dependence and ``dynamic programming squared''}
 Rather than pursue the ``costate on the costate'' approach further,
we now turn  to a closely related approach that we illustrate in a
dynamic contract design problem. While superficially
different from the government policy design problem, the contract problem
  has
many features in common with it.   What is again needed is a recursive way
to encode history dependence.   Rather than use costate variables,
we move up a derivative and work with promised values. This leads
to value functions appearing inside value functions or
``dynamic programming squared.''

Define the  history $s^t$ of the Markov state
by $s^t = [s_t, s_{t-1}, \ldots, s_0]$ and let $\pi_t(s^t)$ be the
density over histories induced by $\pi, \pi_0$.
Define a consumption allocation  rule
as a sequence of functions, the time component of which maps
$s^t$ into a choice of time $t$ consumption, $c_t = \sigma_t(s^t)$, for
$t \geq 0$.
Let $c= \{\sigma_t(s^t)\}_{t=0}^\infty$.
Define the ({\it ex ante\/}) value associated with  an allocation
rule as
$$v(c) = \sum_{t=0}^\infty \sum_{s^t} \beta^t u(\sigma_t(s^t)) \pi_t(s^t).
\EQN sintro60 $$
For each possible realization of the period zero state $\overline s_0$,
there is a {\it continuation history\/} $s^t|_{\overline s_0}$.
The observation that a continuation history is itself a complete history
is our first hint that a recursive formulation is
possible.\NFootnote{See chapters \use{recurge} and \use {credible}
for discussions of the recursive structure of histories.}
For each possible realization of the first period $s_0$,
 a consumption allocation rule implies a one-period  {\it continuation
consumption rule\/} $c|_{\overline s_0}$. A continuation consumption
rule is itself a consumption rule that maps histories into
time series of consumption.
 The one-period continuation history
treats the time $t+1$ component of the original history  evaluated
at $\overline s_0$ as the time $t$
component of the continuation history.
 The period $t$  consumption of the
one-period continuation consumption allocation conforms to the
time $t+1$ component of original consumption allocation evaluated
at $\overline s_0$.
The time and state separability of \Ep{sintro60} then allow us to
represent $v(c)$ recursively
as
$$ v(c) = \sum_{s_0} [ u(c_0(s_0) )+ \beta v(c|_{s_0})]\pi_0(s_0) ,
  \EQN sintro61 $$
where $v(c|_{s_0})$ is the value of the continuation allocation.
We call $v(c|_{s_0})$ the continuation value.
In a special case
that successive components of $s_t$ are i.i.d.\ and have a discrete
distribution,
we can write \Ep{sintro61} as
$$ v = \sum_s [u(c_s) + \beta w_s ] \Pi_s, \EQN sintro22 $$
where $\Pi_s = {\rm Prob}(y_t = \overline y_s)$ and
$[\overline y_1 < \overline y_2  \cdots < \overline y_S]$
is a grid on which the endowment resides,
 $c_s$ is
consumption in state $s$ given $v$, and $w_s$ is the continuation
value in state $s$ given $v$.
Here we use $v$ in \Ep{sintro22} to denote what was $v(c)$ in \Ep{sintro61}
and $w_s$ to denote what was $v(c|_s)$ in \Ep{sintro61}.
%The qualification `given $v$' is very important because in this
%notation $v$ encodes all
%history dependence.

So far this has all been for an arbitrary consumption plan.
Evidently, the {\it ex ante\/}
value $v$ attained by an {\it optimal\/} consumption program must
satisfy
$$ v = \max_{\{{c_s, w_s}\}_{s=1}^S} \sum_s [u(c_s) + \beta w_s ] \Pi_s
\EQN sintro23 $$ where the maximization is subject to constraints
that summarize the individual's opportunities to trade current
state-contingent consumption $c_s$ against future state-contingent
continuation values $w_s$. In these problems, the value of $v$ is
an outcome that depends, in the savings problem for example, on
the household's initial level of assets. In fact, for the savings
problem with i.i.d.\ endowment shocks, the outcome is that $v$ is
a monotone function of $A$. This monotonicity allows the following
remarkable representation. After solving for the optimal plan, use
the monotone transformation to let $v$ replace $A$ as a state
variable and represent the optimal decision rule in the form
$$\EQNalign{ c_s & =   f(v,s) \EQN sintro24;a \cr
             w_s & =  g(v,s) . \EQN sintro24;b \cr} $$
%which are of the form \Ep{sintro21} with the state $s$ indexing the endowment
%$y$, the initial promised $v$ taking the role of $A$, and
%$w_s$ being the continuation value, i.e., next period's value of $A$.

The promised value $v$ (a forward-looking variable if there ever was one)
 is also the variable that
functions as an index of history  in \Ep{sintro24}.  Equation
\Ep{sintro24;b}  reminds us that   $v$ is a  ``backward looking'' variable
that registers the cumulative impact of past   states $s_t$.
The definition of $v$ as a promised value,  for example in
\Ep{sintro23}, tells us that $v$ is also a forward-looking variable
that encodes expectations (promises)  about future consumption.



\subsection{Dynamic principal-agent problems}
The right side of  \Ep{sintro23} tells the terms on which
the  household is willing
to trade
current utility for continuation utility.
Models that confront enforcement and information problems use the
trade-off identified by \Ep{sintro23} to design intertemporal
consumption plans that optimally balance risk sharing  and intertemporal
consumption smoothing against the need to offer correct incentives.
Next we turn to such models.



We remove the household from the market and hand it over to a
planner or principal who offers the household a contract that the
planner designs to deliver an {\it ex ante\/} promised value $v$
subject to enforcement or information constraints.\NFootnote{Here
we are sticking close to two models of Thomas and Worrall (1988,
1990).} Now $v$ becomes a state variable that occurs in the {\it
planner's\/} value function. We assume that the only way the
household can transfer his endowment over time is to deal with the
planner.  The saving or borrowing technology \Ep{sintro3} is no
longer available to the agent, though it might be to the planner.
We continue to consider the i.i.d.\  case mentioned above. Let
$P(v)$ be the {\it ex ante\/} optimal value of the planner's
problem. The presence of a value function (for the agents) as an
argument of the value function of the principal causes us
sometimes to  speak of ``dynamic programming squared.''
\index{dynamic programming!squared}%
 The planner
``earns'' $y_t - c_t$ from the agent at time $t$ by commandeering
the agent's endowment but returning consumption $c_t$. The value
function  $P(v)$ for a planner who must deliver promised value
$v$  satisfies
$$ P(v) = \max_{\{c_s, w_s\}_{s=1}^S} [\overline y_s - c_s + \beta P(w_s) ] \Pi_s,
\EQN sintro70 $$
where the maximization is subject to the promise-keeping
constraint \Ep{sintro22} and some other constraints that depend on
details of the problem, as we indicate shortly.  The other
constraints are context-specific
incentive-compatibility constraints
%%manifestations of \Ep{sintro23}
and describe the best response
of the agent to the arrangement offered by the principal.
Condition \Ep{sintro22} is a {\it promise-keeping\/} constraint.
The planner is constrained to provide a vector of $\{c_s, w_s  \}_{s=1}^S$
that delivers the value $v$.

We briefly describe two
types of contract design problems and the constraints
that confront the planner because of the opportunities
that the environment grants the agent.



%\subsection{Sacrificing smooth consumption for incentives}



To model  the problem
of enforcement without an information problem, assume that while the
planner can observe $y_t$ each period, the household always has the
option of consuming its endowment $y_t$ and receiving an {\it ex ante\/}
 continuation
value $v_{\rm aut}$ with which to enter the next period, where  $v_{\rm aut}$
is the {\it ex ante\/} value the consumer receives by always consuming
his endowment.  The consumer's freedom  to walk away induces the planner
to structure the insurance contract so that   it is never in the
household's interest to defect from the contract (the contract must be ``self-enforcing'').
A self-enforcing contract
requires that the following participation constraints be satisfied:
$$ u(c_s) + \beta w_s \geq u(\overline y_s) + \beta v_{\rm aut}
  \quad \forall s.  \EQN sintro30 $$
A self-enforcing  contract provides imperfect insurance when
occasionally  some of these participation constraints are binding.
When they are binding, the planner sacrifices consumption smoothing
in the interest of
providing incentives for the contract to be self-enforcing.

An alternative specification eliminates
the enforcement problem by assuming
that once the household enters the contract, it does not have the option
 to walk away.
A planner wants to supply insurance to the household in the most efficient
way, but now the planner cannot observe the household's endowment.
The planner must trust the household to report its endowment.  It is assumed
that the household will truthfully report its endowment only if
it wants to.
This leads the planner
to add to the promise-keeping constraint \Ep{sintro22} the
 following truth-telling constraints:
%%$$  u(c_s) + \beta w_s \geq u(c_\tau) + \beta w_\tau
%% \quad \forall (s,\tau). \EQN sintro31 $$
$$  u(c_s) + \beta w_s \geq u(\overline y_s - \overline y_\tau + c_\tau) + \beta w_\tau
 \quad \forall (s,\tau), \EQN sintro31 $$
where constraint \Ep{sintro31} pertains to a situation when the household's
true endowment is $\overline y_s$ but the household considers
to falsely report that the endowment instead is $\overline y_\tau$.
The left  and right  sides of \Ep{sintro31} are the utility
of telling the truth and lying, respectively. If the household
(falsely) reports $\overline y_\tau$, the planner awards the household
a net transfer $c_\tau-\overline y_\tau$ and a continuation value
$w_\tau$.
If \Ep{sintro31} holds for all $\tau$, the household will always choose to report
the true state $s$.

As we shall see in chapters \use{socialinsurance} and \use{socialinsurance2},
the planner elicits truthful reporting by manipulating
how continuation values vary with the reported state.   Households
that report a low income today might receive a transfer today, but
they suffer an adverse  consequence by getting a diminished continuation
value starting tomorrow.  The planner structures this
menu of choices so that only
low-endowment households, those that badly want a transfer today, are willing
to accept the diminished continuation value that  is the consequence
of reporting that low income today.


At this point, a supermartingale convergence theorem raises its ugly head
again.  But this time     it propels consumption and continuation utility
{\it downward\/}.  The super martingale result leads to what
some people have termed the ``immiseration'' property of models in which
dynamic contracts are used to deliver  incentives to reveal
information.

To enhance our appreciation for  the immiseration result, %  -- or put differently, to disrupt
%the planner's ability to solicit information by manipulating future values --
we now touch on another aspect of
macroeconomic's love-hate affair  with the Euler inequality \Ep{sintro1}.
    In both of the  incentive models just described, one with an enforcement
problem, the other with an information problem, it is
 important that the household
not have access to a good risk-free investment technology like
that represented in the constraint \Ep{sintro3} that makes
\Ep{sintro1} the appropriate first-order condition in the savings
problem. Indeed, especially in the model with limited information,
the  planner makes ample use of his ability to reallocate
consumption intertemporally in ways that can violate \Ep{sintro3}
in order to elicit accurate information from the household. In
chapter \use{socialinsurance}, we shall follow Cole and
Kocherlakota (2001) by allowing the household   to save (but not
to {\it dissave\/}) a risk-free  asset that bears fixed gross
interest rate $R =\beta^{-1}$.   The Euler inequality  comes back
into play  and alters the character of the insurance arrangement
so  that outcomes resemble ones that occur in a Bewley model,
provided that the debt limit in the Bewley model is chosen
appropriately. \auth{Cole, Harold L.}\auth{Kocherlakota, Narayana R.}

\vskip-.2cm
\subsection{More applications}
We shall study many more  applications of dynamic programming
and dynamic programming squared, including models of search in labor
markets,  reputation and credible
public policy, gradualism in trade policy, unemployment insurance,
and monetary economies.  It is time to get to work
seriously studying the mathematical and economic tools
that we need to   approach these exciting topics.  Let us begin.

\eqnotracefalse
