\input grafinp3
%\input grafinput8
\input psfig

%\showchaptIDtrue
%\def\@chaptID{19.}
%\input gayejnl.txt
%\input gayedef.txt
\def\lege{\raise.3ex\hbox{$>$\kern-.75em\lower1ex\hbox{$<$}}}

%\hbox{}
\footnum=0
\chapter{Matching Models Mechanics\label{mechanics_matching}}


\auth{Lucas, Robert E., Jr.} \auth{Prescott, Edward C.}
\auth{Ljungqvist, Lars} \auth{Sargent, Thomas J.}
\auth{McCall, John} \auth{Shimer, Robert} \auth{Jovanovic, Boyan}
\auth{Pissarides, Christopher} \auth{Petrongolo, Barbara}

\section{Introduction}
%One of the workhorses in macro labor is the framework of
%matching models, as presented in section \use{sec:matchingmodel}
%of chapter \use{search2}.
We reserve  the term search models to denote ones in the spirit of McCall (1970), like the search-island
model of Lucas and Prescott (1974) described in
section \use{sec:LucasPrescott}. What are now widely called matching models  have   matching
functions that are designed to represent congestion externalities concisely.\NFootnote{We encountered these earlier
 in section \use{sec:matchingmodel}. In chapter \use{search1}, %%we applied  the
the word `matching' described  Jovanovic's (1979a)  analysis
of a process in which workers and firms  gradually  learn about match quality.
In macro labor, the term `matching models' has
come instead to mean models that postulate matching functions.}$\!\!^,\!$\NFootnote{Petrongolo and
Pissarides (2001) call
the matching function a black box because it describes outcomes of labor market frictions without
explicitly modeling them.}
This chapter explores some of the mechanics of
matching models, especially those governing the responses of labor market outcomes to productivity shocks.



 To get big responses of unemployment to movements
in productivity, matching models require a high elasticity of market
tightness %-- the ratio of vacancies to unemployment --
with respect to
productivity.  Shimer (2005) pointed out  that
for common calibrations of what was then a standard matching model, the elasticity
of market tightness is too low to explain business cycle fluctuations.
To increase that elasticity, researchers  reconfigured
matching models in various ways:  by elevating the utility
of leisure, by making wages sticky, by assuming alternating-offer wage
bargaining, by introducing costly acquisition of credit, or by
assuming fixed matching costs.  Ljungqvist and Sargent (2017) showed that beneath this apparent
diversity there resides an essential unity: all of these redesigned  matching models increase responses of
unemployment to movements in productivity by diminishing what Ljungqvist and Sargent called the
{\it fundamental surplus} fraction, a name they gave  to an upper bound
on the fraction of a job's output that the {\it invisible hand}
can allocate to vacancy creation. Business cycle and
welfare state dynamics of an entire class of reconfigured matching
models  operate through this common channel.


Across a variety of matching models, the fundamental surplus fraction is the
single intermediate channel through which economic forces
generating a high elasticity of market tightness with
respect to productivity must operate.  Differences in the  fundamental surplus
explain  why unemployment responds sensitively to
movements in productivity in some matching models but not in
others. The role of the fundamental surplus in generating that response sensitivity
transcends diverse matching models having
very different outcomes along other dimensions that include the elasticity of
wages with respect to productivity %%the size of match surpluses,
and
whether or not outside values affect bargaining outcomes.




For any model with a matching function,  to arrive at the fundamental surplus take the output of a job, then deduct the
sum of the value of leisure,
the annuitized values of layoff costs and training costs and a worker's
ability to exploit a firm's cost of delay under
alternating-offer wage bargaining, and  any other items that must
be set aside. The fundamental surplus is an
upper bound on what the invisible hand could allocate
to vacancy creation. If that fundamental surplus constitutes
a small fraction of a job's output, it means that a given change
in productivity translates into a  much larger percentage
change in the fundamental surplus. Because such large movements in the
amount of resources that could potentially be used for vacancy
creation cannot be offset by the invisible hand,
significant variations in market tightness ensue,  causing large
movements in unemployment.

In contrast to search models, matching models with inputs
of unemployed workers and vacancies in matching functions are
typically plagued by externalities. What types of workers -- perhaps
differentiated by education, skill, age -- and what types of jobs
-- perhaps differentiated by required skills and strengths  --
does the analyst make sit within the same matching functions?
Broadly speaking, matching analyses can be divided into those
that focus on  congestion externalities, and those that
seek to eliminate such  externalities, in order to facilitate analytical
tractability.
% along several important dimensions.
For example, we  describe a way
of proliferating matching functions and  assigning workers to them  that can be interpreted as  expressing
 `directed search' and that succeeds in arresting congestion
externalities and improving analytic tractability along some dimensions.

%A key dimension along which matching models differs
%is how they envision   a key force captured with a matching function:
%congestion externalities.   What types of workers -- perhaps
%differentiated by education, skill, age --
%and what types of jobs -- perhaps differentiated by skill and strength
%requirements --
%does the analyst make sit within the same matching functions.
%By proliferating  matching functions
%and thereby arresting congestion externalities along some dimensions,
%we can approach notions of 'directed search' in tractable ways.




\auth{Ch\'eron, Arnaud} \auth{Hairault, Jean-Olivier} \auth{Langot, Francois}
\auth{Menzio, Guido} \auth{Telyukova, Irina A.} \auth{Visschers, Ludo}

To illustrate the two types of matching analyses that either
emphasize or eliminate externalities, we turn to aging as one
key source of heterogeneity. Ch\'eron,
Hairault and Langot (2013), and Menzio, Telyukova and Visschers (2016)
study overlapping generations models in which unemployed
workers either enter a single matching function or are assigned
to type-specific matching functions. In a version of the
model of Ch\'eron, Hairault and Langot, we show that it is optimal  to subsidize the continuing employment
of old workers and to tax that of young workers  in order properly
to rearrange  the age composition of  unemployed workers sitting inside
 a single matching function. The
age-specific matching functions of
Menzio, Telyukova and Visschers make
those externalities vanish and unleash market forces that make job finding rates decrease with
 age.
Equilibrium computation  turns out to be `block recursive' because agents' value and policy functions depend on
realizations of  exogenous shocks but not on the distribution
of agents across employment and unemployment states. This
makes it easy to compute out-of-steady-state dynamics
as well as equilibria with aggregate shocks.


\section{Fundamental surplus}
\index{fundamental surplus}%
With exogenous separation,  a comparative steady state analysis decomposes  the elasticity of market tightness with respect to productivity into
two multiplicative factors,  both of which are bounded from below by unity.
In a matching model  of variety $j$, let $\eta^j_{\theta,y}$ be the elasticity of
market tightness $\theta$ with respect to productivity $y$:
$$
 \eta^j_{\theta,y}
\;\equiv\; {d\, \theta \over d\, y} \; {y \over \theta}
\;=\; \Upsilon^j \; \frac{y}{y-x^j} .    \EQN MLI_generic
$$
The first factor $\Upsilon^j$  has  an   upper bound
coming  from a consensus  about values of the elasticity
of matching with respect to unemployment. The second factor
$y/(y-x^j)$ is the inverse of what we  define to be   the
`fundamental surplus fraction'. The fundamental surplus $y-x^j$ equals a quantity that  deducts from productivity $y$ a
value $x^j$ that the `invisible hand' cannot allocate to
vacancy creation, a quantity whose economic interpretation differs across models.  Unlike $\Upsilon^j$, the fraction $y/(y-x^j)$ has no
widely agreed upon upper bound.
To get a high elasticity of market
tightness requires that $y/(y-x^j)$ must  be large, i.e., that  what we call the
fundamental surplus fraction must be small.\NFootnote{We call $y-x$ the fundamental surplus and
$\frac{y-x}{y}$ the fundamental surplus fraction.}
 Across reconfigured matching models, many details
differ, but
what ultimately matters is the fundamental surplus.


\subsection{Sensitivity of unemployment to market tightness}

To set the stage for studying how small changes in productivity
can have large effects on unemployment, we start by computing
the elasticity of unemployment with respect to market tightness.
The derivative of steady-state unemployment in equation
\Ep{unemp} with respect to market tightness is
$$
{{d}\, u \over {d} \, \theta } \;=\;
- { s \left[q(\theta) \,+\, \theta \,q'(\theta) \right] \over
  [s \,+\, \theta \,q(\theta)]^2 }
\;=\; -
\left[ 1 \,+\, { \theta \,q'(\theta) \over q(\theta)}  \right]
{ u\, q(\theta) \over   s \,+\, \theta \,q(\theta) }
\;=\;  - (1-\alpha)\,
{ u \,q(\theta) \over  s \,+\, \theta \,q(\theta) }
             \,,
$$
where the second equality uses equation \Ep{unemp} and factors
$q(\theta)$ from the expression in square brackets of the numerator,
and the third equality is obtained after invoking
the constant elasticity of matching with respect to unemployment,
$\alpha=-q'(\theta)\, \theta / q(\theta)$.
So the elasticity of
unemployment with respect to market tightness is
$$
\eta_{u,\theta}
%\;&\equiv&\; {d\, u \over d\, \theta} \; {\theta \over u}
\;=\; -(1-\alpha)\,
{ \theta \,q(\theta) \over  s \,+\, \theta \,q(\theta) }
\;=\; -(1-\alpha)\,
\left(1\,-\, { s \over  s \,+\, \theta \,q(\theta) } \right)
\;=\;  - (1-\alpha)\,(1-u)\,,                  %%%%%%%%\EQN ML1_unempelast
$$
where the second equality is obtained after adding and subtracting
$s$ to the numerator, and the last third equality invokes
expression \Ep{unemp}.


Thus, to shed light on what contributes to significant volatility
in unemployment, we seek forces that can make market tightness
$\theta$ highly elastic with respect to productivity.


%\subsection{Decomposition of the elasticity of market tightness}
\subsection{Nash bargaining model}
           \label{sec:ML1_Nash_elasticity}
%Our first focus is on the standard matching model with Nash
%bargaining, for which equilibrium expression \Ep{equili}
%%for market tightness
%can be rewritten as
In the standard version of the Nash bargaining model, the equilibrium
expression \Ep{equili} for market tightness
can be rewritten as
$$
{1-\phi \over c} (y-z) \;=\; {r+s \over q(\theta)}\,+\,\phi\,\theta\,.
                                                     \EQN ML1_theta
$$
Implicit differentiation  yields
$$\EQNalign{
{d\, \theta \over d\, y} \;&=\; - {
{\displaystyle  1-\phi \over \displaystyle c} \over
- \left( {\displaystyle - q'(\theta)\,(r+s) \over \displaystyle q(\theta)^2}
\,+\, \phi \right)  }
\;=\; - { \left( {\displaystyle r+s \over \displaystyle q(\theta)}
\,+\, \phi \, \theta \right) \; {\displaystyle 1 \over \displaystyle y-z} \over
- \left( {\displaystyle \alpha (r+s) \over \displaystyle \theta q(\theta)}
\,+\, \phi \right)  }                                   \cr
\noalign{\vskip.2cm}
\;&=\;  {\hfill (r+s)\,+\,\phi\,\theta\,q(\theta)  \over
\alpha (r+s) \,+\, \phi\,\theta\,q(\theta) } \; \; {\theta \over y-z}
\;\equiv\; \Upsilon^{\rm Nash} \; {\theta \over y-z}
\,,                                                 \EQN ML1_thetadiff \cr}
$$
where the second equality is obtained after using
equation \Ep{ML1_theta}
to rearrange the numerator, while in the denominator, we invoke
the constant elasticity of matching with respect to unemployment;
the third equality
follows from multiplying and dividing by $\theta\,q(\theta)$. The
elasticity of market tightness with respect to productivity
is then given by
$$
\eta_{\theta,y}
%%\;\equiv\; {d\, \theta \over d\, y} \; {y \over \theta}
\;=\;
{\hfill (r+s)\,+\,\phi\,\theta\,q(\theta)  \over
\alpha (r+s) \,+\, \phi\,\theta\,q(\theta) } \; \; {y \over y-z}
\;\equiv\;
\Upsilon^{\rm Nash} \; {y \over y-z} \,.    \EQN ML1_thetaelast
$$
This multiplicative decomposition of the elasticity of market
tightness is central to our analysis. Similar decompositions
prevail in all of the reconfigured matching models to be
described below and those in Ljungqvist and Sargent (2017).
The first factor $\Upsilon^{\rm Nash}$
in expression \Ep{ML1_thetaelast},
 has counterparts in other setups.   A consensus
about reasonable parameter values   %%calibrations
bounds its contribution
to the elasticity of market tightness. Hence, the
magnitude of the elasticity of market tightness depends mostly on
the second factor in expression \Ep{ML1_thetaelast}, i.e.,
the inverse of what %in section \ref{sec:intro}
we  define to be the fundamental surplus
fraction.

In
the standard matching model with Nash bargaining, the fundamental surplus
is simply what remains  after deducting  the worker's value
of leisure from productivity; $x=z$ in expression \Ep{MLI_generic}.
To induce  them  to work, workers have to receive at least  the
value of  leisure, so the invisible hand cannot allocate  that value to
vacancy creation.


\subsection{Shimer's  critique}
\auth{Shimer, Robert}%
%As noted above, Shimer's (2005) critique is that for common
%calibrations of the standard matching model,
%the elasticity of market tightness
%%with respect to productivity
%is too low to explain business cycle fluctuations.
Shimer  (2005) observed that  the average job finding rate
$\theta\, q(\theta)$ is large relative to the observed value of
the sum   of the net interest rate and the separation rate $(r+s)$. When
 combined with reasonable parameter values for a worker's
bargaining power $\phi$ and the elasticity of matching with
respect to unemployment $\alpha$, this  implies that the first factor
$\Upsilon^{\rm Nash}$ in expression \Ep{ML1_thetaelast},
is close to
its lower bound of unity. More generally,
the first
factor in \Ep{ML1_thetaelast} is bounded from above by $1/\alpha$.
Because reasonable values of the elasticity $\alpha$ imply an upper bound on
the first factor,  the second factor
$y/(y-z)$ in expression \Ep{ML1_thetaelast} becomes
critical for  generating movements in market tightness.
For values of leisure within a commonly assumed range
well below productivity, the second factor is not large
enough to generate the high volatility of market
tightness associated with observed business cycles. This is  Shimer's critique.


Shimer (2005, pp.\ 39-40)  documented that comparisons of steady states described by   expression
\Ep{ML1_thetaelast}
provide a good approximation to average outcomes from
simulations of an economy subject to aggregate productivity shocks.
% Anticipating that  good  approximations will also  prevail in other
% matching frameworks,
Inspired by his finding, we will
derive  steady states under some alternative specifications. These
will shed light on properties of stochastic simulations
to be reported in section \use{sec:FS_simulations}.     %later sections.





\subsection{Relationship to worker's outside value}
%The match surplus is the capitalized surplus accruing to a firm
%and a worker in a current match.  It is the
%difference between the present value of the match and the sum of
%the worker's outside value and the  firm's outside value.
By rearranging equation \Ep{U_eq} and imposing the first
Nash-bargaining outcome of equations \Ep{split}, $E-U=\phi S$,
the worker's outside value can be expressed as
$$
U \;=\; \frac{z}{1-\beta} \,+\, \frac{\beta}{1-\beta}
\theta q(\theta)\,\phi S
\;=\; \frac{z}{1-\beta}\,+\,\Psi^{\rm m.surplus}_u
                             \,+\,\Psi^{\rm extra}_u\,,
                                                    \EQN U_composition
$$
where the second equality decomposes $U$ into three nonnegative parts: (i) the
capitalized value of choosing leisure in all future periods,
$z(1-\beta)^{-1}$; (ii)  the sum % $\Psi^{\rm m.surplus}_u$
of the discounted values of the worker's share of match
surpluses in his or her as yet unformed future
matches\NFootnote{Let $\Psi^{\rm m.surplus}_n$ be the
analogous capital value of an employed worker's share of all
match surpluses over  lifetime,
including  current
employment. The capital values $\Psi^{\rm m.surplus}_u$
and $\Psi^{\rm m.surplus}_n$ solve the Bellman equations
$$\EQNalign{
\Psi^{\rm m.surplus}_u\;&=\; 0\,+\, \beta \Bigl\{\theta q(\theta)
\Psi^{\rm m.surplus}_n \,+\, \left[1-\theta q(\theta)\right]
\Psi^{\rm m.surplus}_u \Bigr\}\,,
                                                    \cr
\Psi^{\rm m.surplus}_n\;&=\; \psi\,+\, \beta \Bigl\{(1-s)
\Psi^{\rm m.surplus}_n \,+\, s \Psi^{\rm m.surplus}_u \Bigr\}\,,
                                                     \cr}
$$
where $\psi$ is an annuity that, when paid for
the duration of a match, has the same expected present value as
a worker's share of the match surplus, $E-U=\phi S$:
$$
\sum_{t=0}^{\infty} \beta^t (1-s)^t \psi \;=\; \phi S
\qquad \Longrightarrow
\qquad \psi\,=\; (r+s) \beta \phi S  \,.
$$}
$$
\Psi^{\rm m.surplus}_u \;=\;
\frac{r+s}{r} \, \frac{\theta\,q(\theta)}{r+s\,+\,\theta\,q(\theta)}
\, \phi S\,;
                                                 \EQN U_part2
$$
 and, key to our new perspective, (iii)
the  parts %$\Psi^{\rm extra}_u$
of fundamental surpluses
from future employment matches that are not allocated to match
surpluses
$$
\Psi^{\rm extra}_u \;=\; \frac{\theta q(\theta)}{r+s}
\, \Psi^{\rm m.surplus}_u \,,
                                                  \EQN U_part3
$$
which can be deduced from equation \Ep{U_composition}
after replacing $\Psi^{\rm m.surplus}_u$ with
expression \Ep{U_part2}.

We can use decomposition \Ep{U_composition} of a worker's outside
value $U$
%in equation (\ref{U_composition})
to shed light on the activities  of the
`invisible hand' that make the elasticity of market
tightness with respect to productivity be low for
common calibrations of  matching models.
Those parameter settings entail a value
of leisure $z$ well below productivity  and a
significant share $\phi$ of match surpluses being awarded to workers,
which together with  a high job finding
probability $\theta q(\theta)$ imply that the sum
$\Psi^{\rm m.surplus}_u +\Psi^{\rm extra}_u$ in equation
\Ep{U_composition} forms a substantial part of a
worker's outside value. Furthermore, $\Psi^{\rm extra}_u$
is the much larger term in that sum, which follows from
expression \Ep{U_part3} and the assumption that $\theta q(\theta)$
is large relative to $r+s$. That big term
$\Psi^{\rm extra}_u$ makes it easy for the
invisible hand to realign a worker's outside value in a
way that leaves the match surplus almost unchanged when
productivity changes. Offsetting changes
in $\Psi^{\rm extra}_u$ can absorb the impact of productivity
shocks so that resources devoted to vacancy creation can remain
almost unchanged, which in turn explains why unemployment
does not respond sensitively to productivity.

\auth{Hagedorn, Marcus} \auth{Manovskii, Iourii}
But in Hagedorn and Manovskii's (2008)  calibration with a high value of
leisure,
the fundamental-surplus components of a worker's outside
value are so small that there is little room for the invisible hand to  realign things as we have described,
making the equilibrium amount
of resources allocated to vacancy creation respond
sensitively to variations in productivity.
That  results in a high
elasticity of market tightness with respect to productivity. Put
differently, since the fundamental surplus is a part of
productivity, it follows that a given change
in productivity translates into  a greater percentage change in
the fundamental surplus by a factor of $y/(y-z)$, i.e., the
inverse of the fundamental surplus fraction. Thus, the small
fundamental surplus fraction in  calibrations
like Hagedorn and Manovskii's having high values of leisure imply large percentage
changes in the fundamental surplus. Such large changes in the
amount of resources that could potentially be used for
vacancy creation cannot be offset by the invisible hand
and hence variations in productivity lead to
large variations in vacancy creation, resulting in a high
elasticity of market tightness with respect to
productivity.\NFootnote{It is instructive to consider a
single perturbation, $\phi=0$, to common calibrations of the
standard matching model, for which a worker's outside
value in expression \Ep{U_composition} solely equals
the capitalized value of leisure and the worker receives no
part of fundamental surpluses,
$\Psi^{\rm m.surplus}_u +\Psi^{\rm extra}_u=0$.
What explains  that the elasticity of market tightness with respect to
productivity remains low for such perturbed parameter settings in which
large fundamental surpluses  end up  affecting only  firms' profits that
in  equilibrium are all used for vacancy creation? The answer
lies precisely in the outcome that firms' profits would then be
truly large; therefore, even though variations in productivity then
affect firms' profits directly, the percentage wise impact of
productivity shocks on such huge profits is negligible,  so market
tightness and unemployment  hardly changes. This shows that
decomposition \Ep{U_composition} of a worker's outside value can
only go so far to shed light on the sensitivity of market tightness to
changes in productivity, because what ultimately matters is
evidently the size of the fundamental
surplus fraction in expression \Ep{ML1_thetaelast}.}
%For further
%discussion of profits and fundamental surpluses, see
%section \ref{sec:profits}.}

%Having described  how the fundamental surplus
%supplements the concept of a worker's outside value, we now tell

\subsection{Relationship to match surplus}
How does the fundamental surplus relate to the match surplus?
The fundamental
surplus is an upper bound on resources that the invisible hand
can allocate to vacancy creation.  Its magnitude as a
fraction of output is the prime determinant of the
elasticity of market tightness with respect to productivity.\NFootnote{We express the fundamental
surplus as a flow value while the match surplus is typically a
capitalized value.}
%The smaller the
%fundamental surplus fraction the larger is the elasticity of
%market tightness.
In contrast, although it
%too measures resources that might be
is directly connected to resources that are
devoted to vacancy creation,  match surplus that is small
relative to output
has no direct bearing on the elasticity of market tightness.
Recall that in the standard matching model, the zero-profit
condition for vacancy creation implies that the expected present
value of a firm's share of match surpluses  equals
the average cost of filling a vacancy. Since common
calibrations award firms a significant share of
match surpluses,  and since vacancy cost expenditures are calibrated
to be relatively small,
%relative to the output of a filled job,
it follows that
equilibrium match surpluses must form  small parts of output
across various matching models, regardless of the elasticity
of market tightness in any particular model.

%%From an accounting perspective,
%A fundamental surplus yields both  a match surpluses and firms' profits.
%% emerge from fundamental surpluses.
Fundamental surpluses yield match surpluses, which in turn include
firms' profits.
A small fundamental surplus
fraction necessarily implies small match surpluses and small firms' profits.
But small match surpluses and small firms' profits don't necessarily  imply
small fundamental surpluses. Therefore,   the size of the fundamental surplus fraction
is the only reliable indicator
of  the magnitude
of the elasticity of market tightness with respect to productivity, a situation  conveyed by expression \Ep{ML1_thetaelast}.






%Dynamics that are intermediated through the fundamental
%surplus occur in other popular
%setups, including those with sticky wages, alternative
%bargaining protocols and costly acquisition of credit.
%For example, it  matters little
%if the source of  a diminished fundamental surplus fraction
%is Hagedorn and Manovskii's (\citeyear{HagedornManovskii})
%high value of leisure for workers, Hall's (\citeyear{Hall2005})
%sticky wage, Hall and Milgrom's (\citeyear{HallMilgrom})
%cost of delay for firms that participate in  alternating-offer bargaining,
%or Wasmer and Weil's (\citeyear{WasmerWeil}) upfront cost for firms
%to secure credit.
%A small fundamental surplus fraction
%causes variations in productivity to have large effects on
%resources devoted to vacancy creation either because  workers
% insist on being compensated for their  losses of leisure,
%or because  firms  have to pay the sticky wage, or because workers
%  strategically exploit the firm's cost
%of delay under an alternating-offer bargaining protocol,
%or because firms must bear the cost of acquiring credit.


\subsection{Fixed matching cost}
\auth{Pissarides, Christopher}%
Pissarides (2009) contributed what for us is  another good laboratory in which to study the pervasive role
of the fundamental surplus  when he argued that fixed
matching costs increase the elasticity of market
tightness with respect to productivity. So in addition to a vacancy posting cost $c$ per period,
we now assume that a firm incurs a fixed cost $H$ when matching
with a worker. Our job is  to verify that the addition of these
 costs diminishes the fundamental surplus fraction.


Under the assumption that a fixed matching cost $H$ is
incurred after the firm and the worker have bargained over the
consummation of a match (e.g., a training cost before work
commences),\NFootnote{For the alternative assumption that
the firm incurs the fixed matching cost before
bargaining with the worker, as well as for analyses of layoff
costs upon separation, see Ljungqvist and Sargent (2017).}
the match surplus $S$ becomes
$$
S \;=\; \left\{\sum_{t=0}^\infty \beta^t (1-s)^t
        \left[ y - (1-\beta) U \right] \right\} \,-\, H\,
  \;=\; \frac{y-(1-\beta)U-(1-\beta(1-s))H}{1-\beta(1-s)}\,.
                                               \EQN fixed_a_S
$$
By Nash bargaining, the firm receives $S_f$ and the worker $S_w$:
% of
%that match surplus, as given by
$$
S_f = (1-\phi)S \hskip1cm \hbox{\rm and} \hskip1cm S_w = \phi S .
                                               \EQN fixed_a_splitS
$$
A worker's value as unemployed is
$$
U \;=\; z \,+\, \beta \left[\theta q(\theta) S_w + U\right],
$$
which by using \Ep{fixed_a_splitS} can be rearranged to
$$
U \;=\; \frac{z \,+\, \beta \theta q(\theta)
         \frac{\displaystyle \phi}{\displaystyle 1-\phi} S_f}
         {1-\beta}\,.                             \EQN fixed_a_U
$$
Equations \Ep{fixed_a_S}, \Ep{fixed_a_splitS} and
\Ep{fixed_a_U} imply that  a firm's match surplus satisfies
$$
S_f \;=\; (1-\phi)\,
          \frac{y-z-\beta(r+s)H}
               {\beta(r+s)+ \beta\theta q(\theta) \phi}\,,
                                                  \EQN fixed_a_Sf
$$
where we have used $\beta=(1+r)^{-1}$ and $1-\beta(1-s)=\beta(r+s)$.

A firm's match surplus must also satisfy the zero profit
condition for vacancy creation:
$$
c\;=\; \beta q(\theta) S_f  \hskip.75cm \Longrightarrow \hskip.75cm
S_f \;=\; \frac{c}{\beta q(\theta)}\,.     \EQN fixed_a_zeroprofit
$$
Expressions \Ep{fixed_a_Sf} and \Ep{fixed_a_zeroprofit}
for a firm's match surplus imply that  the equilibrium
 $\theta$ satisfies
$$
\frac{1-\phi}{c} [y-z -\beta (r+s) H] \;=\;
\frac{r+s}{q(\theta)}\,+\,\phi\,\theta\,.
                                                \EQN fixed_a_theta
$$
Paralleling the steps of implicit differentiation in section
\use{sec:ML1_Nash_elasticity},
we arrive at the elasticity of
market tightness with respect to productivity for the model with a fixed matching cost:
     %%(\ref{fixed_a}).
$$
\eta_{\theta,y}
%% \;\equiv\; {\partial \theta \over \partial y} \; {y \over \theta}
\;=\; \Upsilon^{\rm Nash} \;
        {y \over y-z-\beta (r+s) H} \,.           \EQN fixed_a
$$
The only difference between the elasticity of market tightness with
a fixed matching cost \Ep{fixed_a} and the
earlier expression \Ep{ML1_thetaelast} without such a cost
is the additional term  $\beta (r+s) H$ that is deducted from the fundamental surplus.
So long as the firm continues to operate, this is an annuity payment $a$ having the same expected present
value as the fixed matching cost:
$$
\sum_{t=0}^{\infty} \beta^t (1-s)^t a \;=\; H
\qquad \Longrightarrow
\qquad a\,=\; [1-\beta(1-s)]H \,=\, \beta (r+s) H\,.
$$
%where the flow of annuity payments on the left  side of the first
%equation starts in the first
%period of operating and ceases when the job is destroyed.
The ``invisible hand'' cannot allocate  those resources
to vacancy creation, so it is appropriate to subtract this annuity value
when computing the fundamental surplus.


We have thus  reaffirmed Pissarides's (2009) insight that  the addition of a
fixed matching cost increases the elasticity
of market tightness and shown how the effect works through the fundamental surplus.  In addition,  our analysis thus adds the insight that the
  quantitative effect coming from  that fixed cost is
inversely related to the  size of the fundamental
surplus fraction.




\subsection{Sticky wages}
\auth{Hall, Robert E.} \index{sticky wage}%
The standard assumption of Nash
bargaining in matching models is  one way to determine a
wage, but not the only one.  Matching frictions create  a
range of wages that a firm and worker both prefer
 to breaking  a match.  Hall noted that  a constant wage  can
be consistent with no private inefficiencies in
contractual arrangements within a matching model.  That motivated
Hall (2005) to assume sticky wages, in the form of  a constant wage in his main
analysis, as a way of responding to the Shimer critique.   Hall
posited a `wage norm'  $\hat w$
inside the Nash bargaining set that must be paid to workers.
Here we show that an appropriately defined fundamental surplus
fraction determines how does such a constant wage affects the elasticity of market
tightness with respect to productivity.

Given a constant wage $w=\hat w$, an equilibrium is
characterized by the zero-profit condition for vacancy creation
in expression \Ep{wage1} of the standard matching model
$$
\hat w\;=\; y \,-\, {r+s \over q(\theta)} c \,.      \EQN ML1_fixwage
$$
There exists an equilibrium for any constant wage
$\hat w \in [z, y-(r+s)c]$. The lower bound
is a worker's utility of leisure and the upper bound
is determined by the zero-profit condition for vacancy
creation evaluated at the  point where the  probability of a firm filling a vacancy
is at its maximum value of $q(\theta)=1$.
%Rearrange expression \Ep{ML1_fixwage} to get
%$$
%q(\theta)\, (y -\hat w) \;=\; (r+s)\, c
%$$
%and
%implicitly  differentiate to get
%\$$
%{d\, \theta \over d\, y} \;=\; - {
%q(\theta) \over q'(\theta) (y -\hat w) }
%\;=\; {\theta \over \alpha (y -\hat w) }\,,    \EQN ML1_thetadiff2
%$$
%where the second equality follows from invoking
%the constant elasticity of matching with respect to unemployment,
%$\alpha=-q'(\theta)\, \theta / q(\theta)$.
After implicitly differentiating \Ep{ML1_fixwage},
we can compute the elasticity of market tightness as
$$
\eta_{\theta,y}
%% \;\equiv\; {\partial \theta \over \partial y} \; {y \over \theta}
\;=\; { 1 \over \alpha } \;\; {y \over y-\hat w}
\;\equiv\; \Upsilon^{\rm sticky}\; {y \over y-\hat w}\,.
                                                \EQN ML1_thetaelast2
$$
This equation resembles the earlier one for $\eta_{\theta,y}$  in
\Ep{ML1_thetaelast}. Not surprisingly, if the constant wage
 equals the value of leisure, $\hat w = z$, then the
elasticity \Ep{ML1_thetaelast2} is equal to that earlier
elasticity of market tightness in the standard
matching model with Nash bargaining when the worker has a zero
bargaining weight, $\phi=0$. With such lopsided
bargaining power, the equilibrium wage would indeed be the constant
value $z$ of leisure.

This outcome reminds us that the first factor in
expression \Ep{ML1_thetaelast} can play only a limited role in magnifying
the elasticity $\eta_{\theta,y}$ because it is bounded from above by the inverse of
the elasticity of matching with respect to unemployment, $\alpha$.   In
\Ep{ML1_thetaelast2},  the upper bound is attained.
So again it is the second factor, the inverse of the fundamental
surplus fraction, that tells whether the elasticity
of market tightness is high or low. The pertinent definition of
the fundamental surplus is now the difference between productivity and
the stipulated constant wage.

In Hall's (2005) model, all
of the fundamental surplus goes to vacancy creation (as also occurs  in the
standard matching model with Nash
bargaining when  the worker's
bargaining weight is zero). A given percentage
change in productivity is multiplied  by a factor  \hbox{$y/(y-\hat w)$}
to become a larger percentage
change in the fundamental surplus.
Because all of the fundamental surplus now goes to vacancy creation, there
is a correspondingly magnified impact on unemployment. Numerical simulations of economies with aggregate
productivity shocks in section \use{sec:FS_Hall_simul} reaffirm this interpretation.



\subsection{Alternating-offer wage bargaining}\label{sec:HallMilgrom}%
Hall and Milgrom (2008) proposed yet another response to the Shimer critique.
Instead of Nash bargaining, a firm and a worker take
turns making  wage offers. The threat is not to break
up and receive  outside values, but instead to continue to bargain
because that choice has a strictly higher payoff than accepting
the outside option. After each unsuccessful bargaining round, the
firm incurs a cost of delay $\gamma > 0$ while the worker enjoys
the value of leisure $z$. There is also a probability $\delta$ that
the job opportunity is exogenously destroyed between bargaining rounds, sending  the
worker to the unemployment pool.
\auth{Hall, Robert E.} \auth{Milgrom, Paul R.}%
\index{alternating-offer bargaining}%

It is optimal for both bargaining  parties  to make  barely
acceptable offers. The firm always offers $w^f$ and
the worker always offers $w^w$. Consequently, in an equilibrium, the
first wage offer is accepted. Hall and Milgrom assume that firms make the first
wage offer.

 Hall and Milgrom (2008, p.~1673)
chose to emphasize that ``the limited influence of unemployment
[the outside value of workers] on the wage results in large
fluctuations in unemployment under plausible movements in
[productivity].'' It is more accurate to emphasize
that  the key
force is actually  that an appropriately defined fundamental surplus fraction has to be
calibrated to be small. Without a small fundamental surplus
fraction, it matters little that the outside value has been
prevented  from influencing  bargaining. To  illustrate this, we  compute the
elasticity of market tightness with respect to productivity and look under the hood.

After a wage agreement, a firm's value of a filled job, $J$, and
the value of an employed worker, $E$,
are still  given
by expressions \Ep{J_eq} and \Ep{E_eq} in the standard
matching model. These  can be rearranged to become
$$\EQNalign{
E \;&=\; {w \,+\, \beta \,s\,U \over 1-\beta(1-s)} \,,
                                               \EQN ML1_employ \cr
\noalign{\vskip.2cm} \cr
J \;&=\; {y - w \over 1-\beta(1-s)} \,,        \EQN {ML1_job}  \cr}
$$
where we have imposed a zero-profit condition  $V=0$ on vacancy
creation in the second expression. Thus, using
expression \Ep{ML1_employ}, the indifference condition for
a worker who has just received a wage offer $w^f$ from the firm and is choosing
whether  to decline  the offer and wait until the next period
to make a counteroffer $w^w$ is
$$
{w^f \,+\, \beta \,s\,U \over 1-\beta(1-s)}
 \;=\; z\,+\, \beta \left[(1-\delta)
       {w^w \,+\, \beta \,s\,U \over 1-\beta(1-s)}
        \,+\, \delta\, U\right].                     \EQN ML1_employW
$$
Using expression \Ep{ML1_job}, the analogous condition for
a firm contemplating a counteroffer from the worker is
$$
{y - w^w \over 1-\beta(1-s)}  \;=\; -\gamma \,+\,
   \beta (1-\delta)\, {y - w^f \over 1-\beta(1-s)} \,.  \EQN ML1_jobW
$$


After collecting and simplifying the terms that involve the
worker's outside value $U$, expression \Ep{ML1_employW} becomes
$$
{w^f \over 1-\beta(1-s)}
 \;=\; z\,+\, \beta (1-\delta)\,{w^w \over 1-\beta(1-s)} \,+\,
 \beta \,{1-\beta \over 1-\beta(1-s) } \,(\delta-s)\,U.
                                               \EQN ML1_employWnew
$$
As emphasized by Hall and Milgrom, the worker's outside value $U$
has a small influence on bargaining: when  $\delta=s$, the
outside value disappears from expression \Ep{ML1_employWnew}. That
is, with  bargaining that  ends   either with an
agreement or with destruction of the job, the outside value will
matter only if  job destruction probabilities differ
before and after reaching an agreement. To strengthen
Hall and Milgrom's (2008) observation that under their bargaining protocol
the outside value has at most a
small influence, we proceed
under the assumption that $\delta=s$, which makes the two indifference
conditions \Ep{ML1_employWnew} and \Ep{ML1_jobW} become
$$\EQNalign{
w^f \;&=\; (1-\tilde \beta)\,z\,+\, \tilde \beta\, w^w
                                              \EQN ML1_employW2 \cr
\noalign{\vskip.2cm}  \cr
y - w^w \;&=\; -(1-\tilde \beta)\,\gamma \,+\,
        \tilde \beta \, (y - w^f)    \,,      \EQN ML1_jobW2   \cr}
$$
where $\tilde \beta \equiv \beta (1-s)$.
Solve for $w^w$ from \Ep{ML1_jobW2} and substitute into
\Ep{ML1_employW2} to get
%%\begin{equation}
%%w^f \;=\; (1-\tilde \beta)\,z \,+\,
%%      \tilde \beta\left[(1-\tilde \beta)(y+\gamma)
%%                      \,+\,\tilde \beta\,w^f\right], \label{ML1_wageF}
%%\end{equation}
%%which can be rearranged to read
$$
w^f \;=\; {(1-\tilde \beta)\left[ z\,+\,\tilde \beta(y+\gamma)\right] \over
           1-\tilde \beta^2 }
\;=\; { z\,+\,\tilde \beta(y+\gamma) \over 1+\tilde \beta }\,.
                                                   \EQN ML1_wageF2
$$
This is the wage that a firm would immediately offer a worker  when first matched;  the offer would
be accepted.\NFootnote{When firms
make the first wage offer, a necessary
condition for an equilibrium is that $w^f$ in expression
\Ep{ML1_wageF2} is less than productivity $y$, i.e., the parameters
must satisfy $z + \tilde \beta \gamma < y$.}
In an equilibrium, this
wage must also be consistent with the no-profit condition for vacancy
creation. Substitution of $w=w^f$ from expression
\Ep{ML1_wageF2} into the no-profit condition \Ep{wage1}
of the standard matching model results in the following
expression for equilibrium market tightness:
$$
{ z\,+\,\tilde \beta(y+\gamma) \over 1+\tilde \beta }  \;=\;
y \,-\, {r + s \over q(\theta)} \, c .       \EQN ML1_altbarg
$$
After implicit differentiation,
we can compute the elasticity of market tightness as
$$
%%\eta_{\theta,y}\Big|_{\rm first\, wage\, offer\, by\, firm}
\eta_{\theta,y}
%% \;\equiv\; {d\, \theta \over d\, y} \; {y \over \theta}
\;=\; {1 \over \alpha} \; \;
      {y \over y-z- \tilde \beta \,\gamma}
\;=\; \Upsilon^{\rm sticky}\;
         {y \over y-z- \tilde \beta \,\gamma}\,,  \EQN ML1_thetaelast3
$$
where the fundamental surplus is the productivity
that remains after making deductions for the value of leisure $z$
and a firm's discounted cost of delay $\tilde \beta \gamma$.
The latter item
captures the worker's prospective gains from his
 ability to exploit the
cost that delay imposes on the firm.  What remains of productivity is the
fundamental surplus that could potentially be allocated by the
`invisible hand'  to  vacancy creation in an equilibrium.

To summarize, the alternative bargaining
protocol of Hall and Milgrom (2008) does suppress the influence of the
worker's outside value during bargaining. But
this outcome would be irrelevant if  Hall and Milgrom had  not
calibrated a small fundamental surplus fraction, as numerically
illustrated in their own parameterized model in section
\use{sec:FS_HallMil_simul}.







\section{Business cycle simulations}\label{sec:FS_simulations}%
\auth{Hall, Robert E.}%
To illustrate that a small fundamental surplus fraction is essential for
generating ample unemployment volatility over the business cycle in
matching models, we use  Hall's (2005) specification
with discrete time and a  random productivity process.
The monthly discount factor $\beta$ corresponds to a 5-percent annual
rate and the value of leisure is $z=0.40$. The elasticity of
matching with respect to unemployment is $\alpha=0.235$, and the
exogenous monthly separation rate is $s=0.034$. Aggregate productivity
takes on five  values $y_s$ uniformly spaced around a mean of
one on the interval $[0.9935, 1.00565]$, and is governed by
a monthly transition probability matrix $\Pi$ with probabilities
that are zero except as follows: $\pi_{1,2}=\pi_{4,5}=2(1-\rho)$,
$\pi_{2,3}=\pi_{3,4}=3(1-\rho)$, with the upper triangle of the
transition matrix symmetrical to the lower triangle and the
diagonal elements equal to one minus the sums of the nondiagonal
elements. The resulting serial correlation of $y$ is $\rho$, which
is parameterized to be $\rho=0.9899$.
To facilitate the
sensitivity analysis, following Ljungqvist and Sargent (2017), we alter
Hall's model period from one month to one day.




\subsection{Hall's sticky wage}\label{sec:FS_Hall_simul}%
Following  Hall (2005),  we posit a
fixed wage $\hat w = 0.9657$, which  equals the flexible wage that
would prevail at the median productivity level under standard Nash
bargaining (with equal bargaining weights, $\phi=0.5$).
Figure \Fg{figMLIHall1} reproduces Hall's figures 2 and 4
for those two models. The solid line and the upper dotted line
depict unemployment rates at different productivities for the
sticky-wage model and the standard Nash-bargaining model,
respectively.\NFootnote{Unemployment is a state variable
that is not just a function of the current productivity, as are
all of the other variables, but depends on the history of the
economy. But high persistence of productivity and the
high job-finding rates make the unemployment rate that is  observed at
a given productivity level be well approximated by expression
\Ep{unemp}
evaluated at the market tightness $\theta$ prevailing at that
productivity (see Hall (2005, p.~59)).}
Unemployment is almost invariant to productivity under Nash bargaining
but responds sensitively under the sticky wage. These outcomes are
explained by differences in job-finding rates, as shown by the
dashed line and the lower dotted line for the
sticky-wage model and the standard Nash-bargaining model,
respectively, expressed at our daily frequency.\NFootnote{Our
daily job-finding rates are roughly $1/30$ of the monthly
rates in Hall (2005, figures 2 and 4),
confirming our conversion from a monthly to a daily frequency.}
Under the sticky wage, high productivities cause firms to post many
vacancies, making it easy for unemployed workers to find
jobs, while the opposite is true when productivity is low.

\midfigure{figMLIHall1}
\centerline{\epsfxsize=3truein\epsffile{MLI_Hall1.eps}}
\caption{Sticky-wage model. Unemployment rates and daily job-finding
rates at different productivities (given a fixed wage $\hat w = 0.9657$),
where the dotted lines with almost no slopes are counterparts from
a standard Nash-bargaining model.}
\infiglist{figMLIHall1}
\endfigure

%Starting from this verification of our conversion of Hall's
%(2005) model into a daily frequency,
We conduct a sensitivity analysis of the choice of the fixed wage.
The solid line in Figure \Fg{figMLIHall2} shows how
the average unemployment rate varies with the fixed wage $\hat w$.
A small set  of wages spans outcomes ranging from
%nearly zero to very high average unemployment.
             %%REMARK: We truncated the graph so the lowest
             %%unemployment rate is two percent.
 very low to  very high average unemployment rates.
Small variations in a fixed wage close to
productivity generate large changes in the fundamental surplus
fraction, $(y-\hat w)/y$.  Free entry of firms makes that map
directly into the amount of resources devoted to vacancy creation.
The dashed line in Figure \Fg{figMLIHall2}
delineates  implications
for the volatility of unemployment. The standard deviation of
unemployment is nearly zero at the left end of the graph where the
job-finding probability is almost one for all productivity levels.
Unemployment volatility then increases for higher constant wages
until, outside of the graph at the right end, vacancy creation
becomes so unprofitable that average unemployment converges to its
maximum of 100 percent, causing there to be no more fluctuations.



\midfigure{figMLIHall2}
\centerline{\epsfxsize=3truein\epsffile{MLI_Hall2.eps}}
\caption{Sticky-wage model. Average unemployment rate and standard
deviation of unemployment for
different postulated values of the fixed wage.}
\infiglist{figMLIHall2}
\endfigure


At Hall's fixed wage $\hat w = 0.9657$, Figure \Fg{figMLIHall2}
shows a standard deviation of unemployment equal to
1.80 percentage points, which is close to the target of 1.54 to which
Hall (2005) calibrated his model.


\subsection{Hagedorn and Manovskii's high value of leisure}
           \label{sec:FS_HagMan_simul}%
\auth{Hagedorn, Marcus} \auth{Manovskii, Iourii}%
 It turns out that  by elevating the value of
leisure,
the standard Nash-bargaining model can attain the
same volatility of unemployment as does the sticky wage model of the previous subsection.
To illustrate this, we use  Hall's (2005) parameterized environment
but
now simply assume standard Nash wage bargaining in order to study
 Hagedorn and Manovskii's (2008) analysis of the consequences of
  positing a high value $z=0.960$ of leisure  and a low
bargaining power of workers  $\phi=0.0135$.
%%who focus on a high value
%%of leisure, $z=0.955$, and a low bargaining power of workers,
%%$\phi=0.052$. At such a parameterization,
These parameter values imply a
high standard deviation of 1.4 percentage points for unemployment.
Figure \Fg{figMLINash1}, which depicts outcomes for different
constellations of $z\in [0.4, .99]$ and $\phi\in[0.001, 0.5]$, sheds light
 on the sensitivity of outcomes to  the choice of parameters.
To construct the figure, for each pair $(z,\phi)$, we adjusted the efficiency parameter $A$
of the matching function to make the average unemployment rate stay at
5.5 percent. Because it implies a
a small fundamental surplus fraction, a high value of leisure is essential for obtaining large variations in market tightness and a high volatility of unemployment.

%As detailed in section \ref{sec:HagedornManovskii}
%(and elaborated upon in online appendix \ref{app:wageelasticity}),


To match the elasticity of wages with respect to productivity,
Hagedorn and Manovskii (2008) require a low bargaining power for workers.
%If we use the Hall (2005) parameter values and Hagedorn and Manovskii's
%$(z,\phi)=(0.960, 0.0135)$ in the otherwise same  parametrized
%environment of Hall (2005),
Given the above parameterization with $(z,\phi)=(0.960, 0.0135)$,
we obtain a  wage elasticity
of 0.44, which  is approximately the value that
Hagedorn and Manovskii had targeted. To conduct a sensitivity analysis to variations in $z$ and $\phi$,
Figure \Fg{figMLINash2}
employs   the same computational approach underlying Figure \Fg{figMLINash1}.
  The figure confirms
that  a low $\phi$ is required to obtain a low wage
elasticity.\NFootnote{Note that the
axes in Figure \Fg{figMLINash2} are rotated relative to
Figure \Fg{figMLINash1}, for easy viewing of the relationship.}


\auth{Rogerson, Richard} \auth{Shimer, Robert}
Taken together, Figures \Fg{figMLINash1} and \Fg{figMLINash2}
seem to settle a difference of opinions in favor of  Hagedorn and Manovskii (2008, p.~1696), who argued
that ``the volatility of labor market tightness is almost independent
of [$\phi$] and is determined only by the level of $z$.''
Rogerson and Shimer (2011, p.~660) apparently disagreed when they instead  emphasized
that wages are rigid under the calibration of ``Hagedorn and
Manovskii (2008), although it is worth noting that the authors
do not interpret their paper as one with wage rigidities. They
calibrate ... a small value for the workers' bargaining power
[$\phi$]. This significantly amplifies productivity shocks ...''
But Figures \Fg{figMLINash1} and \Fg{figMLINash2} indicate that  the low wage elasticity of
Hagedorn and Manovskii (2008)
is incidental to and neither necessary nor sufficient to obtain
a high volatility of unemployment. We suggest that instead of stressing the importance of a rigid wage, as Rogerson and Shimer
did, what should be concluded is the general principle that
the fundamental surplus fraction must be small in order to amplify business cycle responses to productivity changes.

\midfigure{figMLINash1}
\centerline{\epsfxsize=3truein\epsffile{MLI_Nash1.eps}}
\caption{Nash-bargaining model. Standard deviation of unemployment in
percentage points for different
constellations of the value of leisure $z$, and the bargaining
power of workers $\phi$.}
\infiglist{figMLINash1}
\endfigure

\midfigure{figMLINash2}
\centerline{\epsfxsize=3truein\epsffile{MLI_Nash2.eps}}
\caption{Nash-bargaining model. Wage elasticity with respect to
productivity for different
constellations of the value of leisure $z$, and the bargaining
power of workers $\phi$. (Note that the axes are rotated relative
to Figure \Fg{figMLINash1}.)}
\infiglist{figMLINash2}
\endfigure


\subsection{Hall and Milgrom's alternating-offer bargaining}\label{sec:FS_HallMil_simul}%
\auth{Hall, Robert E.}\auth{Milgrom, Paul R.}%
 Hall and Milgrom's (2008)
model of alternating-offer wage bargaining is another way to
increase unemployment volatility. Except for
the wage formation process, their environment is  Hall's (2005). But Hall and Milgrom   parameterize it  differently.
One difference between  Hall and Milgrom's parameterization and  Hall's (2005) plays  an especially  important role in setting  the
 fundamental surplus: Hall and Milgrom' raised the value of leisure  to $z=0.71$ from Hall's value of $z= 0.40$.
 Section \use{sec:HallMilgrom} taught us that
the values of leisure and of  the firm's cost of delay in
bargaining $\gamma$ are likely to be critical determinants of
 the elasticity of market tightness with respect
to productivity and hence of the volatility of unemployment.

But that is not what
Hall and Milgrom (2008) chose to emphasize. Instead, they stressed how much the
outside value of unemployment is suppressed in alternating-offer
wage bargaining since disagreement no longer leads to
unemployment but instead to another round of bargaining. So from Hall and Milgrom's perspective
a key parameter
is the exogenous rate $\delta$ at which parties break up
between bargaining rounds. Figure \Fg{figMLIHallMilgrom} shows
how different constellations of $(\gamma, \delta)$ affect the
standard deviation of unemployment.  %% in percentage points.
For each pair $(\gamma,\delta)$, we adjust the efficiency
parameter $A$ of the matching function to make the average
unemployment rate stay at 5.5 percent. Because
Hall and Milgrom (2008) assumed that productivity shocks are not
the sole source of unemployment fluctuations, leading them to lower their target  standard deviation
of unemployment to 0.68 percentage points -- a target
attained with their parameterization
$(\gamma, \delta)=(0.27, 0.0055)$
%%$\gamma= 0.27$ and $\delta=0.0055$
and reproduced in Figure \Fg{figMLIHallMilgrom}.

Figure \Fg{figMLIHallMilgrom} supports our earlier finding
that the cost of delay $\gamma$ together with the value of
leisure $z$ are the keys to  generating higher volatility
of unemployment. Without a cost of delay sufficiently high
to reduce the fundamental surplus fraction, the exogenous separation rate between
bargaining rounds  matters little.\NFootnote{To be specific, our formula
\Ep{ML1_thetaelast3} for the steady-state comparative statics
is an approximation of the elasticity of market tightness at
the rear end of Figure \Fg{figMLIHallMilgrom} where
the exogenous rate $\delta$ at which parties break up
between bargaining rounds is equal to Hall and Milgrom's
(2008) assumed job destruction rate of $0.0014$ per day.}

\midfigure{figMLIHallMilgrom}
\centerline{\epsfxsize=3truein\epsffile{MLI_HallMilgrom.eps}}
\caption{Alternating-offer bargaining model. Standard deviation of unemployment in percentage
points for different constellations of firms' cost of delay
$\gamma$ in bargaining  and the exogenous separation rate
$\delta$ while bargaining.}
\infiglist{figMLIHallMilgrom}
\endfigure

Although Hall and Milgrom (2008, p.~1670) notice that their ``sum
of $z$ and $\gamma$ is $\ldots$ not very different from the value
of $z$ by itself in $\ldots$ Hagedorn and Manovskii's
calibration'' (as studied in our section \use{sec:FS_HagMan_simul}),
they demphasized this  similarity and instead emphasized
  differences in mechanisms across Hagedorn and Manovskii's model and theirs.
Focusing on the  fundamental surplus tells us that it is
their similarity that should
be stressed.   Hall and Milgrom's and Hagedorn and Manovskii's  models are united in requiring a small fundamental
surplus fraction  to generate high unemployment volatility
over the business cycle.


\subsection{Matching and bargaining protocols in a  DSGE model}
Christiano, Eichenbaum and Trabandt (2016) compare consequences
of assuming
alternative-offer bargaining (AOB) and Nash bargaining
in a dynamic stochastic general equilibrium (DSGE)
model with a matching function. They find that,  if they adjust structural parameters across the two models  to  fit the  data,
 models parameters estimated under   the two alternative assumptions are  able to account for
the data equally well. That  includes comparable performance in   generating observed  unemployment volatility.
%They find that  the two
%assumptions about bargaining allow the
%model perform equally well in accounting for patterns in
%the data, including generating unemployment volatility.
The solid line in Figure \Fg{figMLI_CLT}
depicts responses of unemployment to a neutral technology
shock that are virtually identical across the two models.
But  beneath those nearly identical responses there resides  a substantial difference in estimates of a key parameter under
the two assumptions, namely,
the replacement rate from unemployment insurance, a parameter that
corresponds to our value of leisure $z$. They estimate a value of  0.37 under the AOB
model versus 0.88 with the Nash bargaining model.
\auth{Christiano, Lawrence J.}\auth{Eichenbaum, Martin}%
\auth{Trabandt, Mathias} \index{DSGE model}%

%Christiano, Eichenbaum and Trabandt (2016) compare the implications
%of assuming alternative-offer bargaining (AOB) and Nash bargaining
%over a wage in  a DSGE framework with a matching function.
%The solid line in Figure \Fg{figMLI_CLT}
%depicts the  response of unemployment to a neutral technology
%shock, which is virtually identical across the two models.
%Though, a key difference in estimated parameter values is
%the replacement rate from unemployment insurance (that
%corresponds to our value of leisure $z$): 0.37 in the AOB
%model versus 0.88 in the Nash bargaining model.



\midfigure{figMLI_CLT}
\centerline{\epsfxsize=3truein\epsffile{AllComparisonFigure_BW.eps}}
\caption{Impulse response of unemployment to a neutral
technology shock in the DSGE analyses. The solid lines refer
to estimates of   AOB and Nash bargaining models,
respectively. The
dashed lines refer to perturbed models where parameter
values for the replacement ratio and,  in the AOB model,
for a firm's cost to make a counteroffer are cut in half. The
two solid (dashed) lines are almost indistinguishable, except for the Nash bargaining model being slightly below
the AOB model.}
\infiglist{figMLI_CLT}
\endfigure
%
%
%
%\midfigure{figMLI_CLT}
%\centerline{\epsfxsize=3truein\epsffile{MLI_HallMilgrom.eps}}
%\caption{Impulse response of unemployment to a neutral
%technology shock in the DSGE analyses. The solid line refers
%to both estimated models of AOB and Nash bargaining,
%respectively. The
%dashed line refers to the perturbed models where parameter
%values are cut in half for the replacement ratio, as well as
%for a firm's cost to make a counteroffer in the AOB model.}
%\infiglist{figMLI_CLT}
%\endfigure


 Christiano et al.\ (2016, pp.~1551-1552) remark that
   their high estimate of the value of leisure in the
Nash bargaining model ``$\ldots$  is reminiscent of Hagedorn
and Manovskii's (2008) argument that a high replacement ratio has
the potential to boost the volatility of unemployment''.\NFootnote{See  section \use{sec:FS_HagMan_simul} above.}
To elaborate, Christiano et al. demonstrate  that if they restrict the replacement
rate in the Nash bargaining model to be the same as that of the
AOB model and then recalculate the impulse response functions, then there occurs
a dramatic deterioration in the performance of the Nash
bargaining model. Thus,  the dashed line in Figure \Fg{figMLI_CLT}
show how  unemployment becomes much less responsive to
a neutral technology shock under that perturbation in the replacement rate.

Christiano et al.\ (2016, p.~1547) proceed to interpret  their low estimate of the value of leisure in the
AOB model as meaning that ``the replacement ratio does not play a critical role
in the AOB model's ability to account for the data.''
Their account conceals that the fundamental surplus  is really at work once
again. Christiano et al.
 generously conducted for us
%what can be regarded as a reverse of their perturbation of the AOB model,
a perturbation of the AOB model that can be regarded as the
analogue to their perturbation of the Nash bargaining model;
namely, a cutting in half
of both the replacement rate 0.37 and a firm's cost of delay
in bargaining, where the latter in their model is a firm's
cost of  making a counteroffer  calibrated to 0.6 of a firm's
daily revenue per worker.\NFootnote{Christiano et al.\ (2016)
assume that it takes one day for a wage offer to be extended, with a firm and a worker alternating in
making an offer.}
As  sections
\use{sec:HallMilgrom} and \use{sec:FS_HallMil_simul} lead us to expect, this
perturbation of the AOB model also brings a dramatic
deterioration in performance, one as bad as that of the perturbed Nash
bargaining model: e.g., the dashed line depicting a
dampened impulse response of unemployment to a neutral
technology shock in Figure  \Fg{figMLI_CLT}
is virtually identical across the two perturbed models.
We conclude from this exercise that contrary to what Christiano et al. say,  the replacement ratio is  critical
in the AOB model too, and that  what is needed to make the fundamental surplus fraction small
in that model
is a combination of very high
values of the replacement rate and a firm's cost of delay in
bargaining.








\auth{Ch\'eron, Arnaud} \auth{Hairault, Jean-Olivier} \auth{Langot, Francois}
\auth{Menzio, Guido} \auth{Telyukova, Irina A.} \auth{Visschers, Ludo}%
%%\section{Matching model with overlapping generations}
\section{Overlapping generations in one matching function}\label{sec:M_OLG1}%
  To emphasize the important role of  congestion externalities, it
  is useful to study a matching model in which workers are heterogeneous along one or more
  dimensions, for example, age.
%Heterogeneous agents have become an important ingredient of many macroeconomic
%models.
 Ch\'eron,
Hairault, and Langot (2013) and Menzio, Telyukova, and Visschers (2016)
study overlapping generations  models under alternative arrangements in which unemployed
workers either enter a single matching function or are assigned to type-specific
matching functions. In this section,
we adopt a framework of Ch\'eron, Hairault
and Langot. They  assume a single matching function and an exogenous retirement
age $T+1$. Each period, a retiring generation is replaced by a new generation of the same
size, normalized to unity.  All newborn workers enter the labor market being
unemployed.


At the beginning of each period, a new
productivity is drawn  from a differentiable
cumulative distribution function $G(\epsilon)$
for each newly filled job and also  for each ongoing job.
The continuous random variable $\epsilon$ has the support
$[\underline \epsilon,1]$.
%At the beginning of each period, a new
%productivity is drawn  from a
%cumulative distribution function $G(\epsilon)$ with
%$\epsilon\in[0,1]$ for each newly filled job and also for each ongoing job.
After observing the job
productivity, the firm decides whether  to operate the job. If a job
is not operated, the match between the firm and worker is broken, and the
worker returns to the pool of unemployed. In an equilibrium, jobs with  productivities below
age-specific reservation productivities $R_i$ for  workers of
age $i$ are terminated. This setting gives rise to
intergenerational labor market externalities   like  those that
arose with the  heterogeneous jobs model of  section \use{Sec_matching_heterojobs}.

We assume that the worst productivity realization $\underline \epsilon$
is so low that it triggers job separations for any age in the
decentralized equilibrium as well as in the social planner solution.
Hence, all reservation productivities are at interior solutions.
If that requires a negative parameter value $\underline \epsilon$, it
can be thought of as a 'labor hoarding cost' for a (temporarily)
non-productive job, which can only be avoided by breaking the match
and not operating the job.



\subsection{A steady state}

In a steady state, there is a list of  time-invariant unemployment rates
$\{u_i\}_{i=1}^{T}$, where the index $i$ denotes the
age of workers. Since newborn workers enter as
unemployed,  $u_1=1$. Given equilibrium market tightness
$\theta$ and reservation productivities $\{R_i\}_{i=2}^{T}$,
unemployment rates across ages evolve as
$$
u_i \,=\, u_{i-1} \Bigl[1 - \theta q(\theta) \Bigl(1- G(R_i) \Bigr) \Bigr]
         + (1-u_{i-1}) \,G(R_i)\,,                             \EQN M_OLG_u
$$
for $i=2, \ldots, T$. Note that the unemployed of age $i-1$ can
be matched to jobs in the subsequent period and hence, the reservation
productivity $R_i$ determines which of those jobs are operated. Total
unemployment is $u = \sum_{i=1}^T u_i$, with an economy-wide
unemployment rate of $u/T$.

For a job with productivity $\epsilon$ that is matched and acceptable
to a worker of age $i$ (i.e., $\epsilon \geq R_i$), a firm's value, $J_i(\epsilon)$, and an employed worker's value, $E_i(\epsilon)$, are
$$\EQNalign{
J_i(\epsilon) \;&=\; \epsilon\,-\, w_i(\epsilon)\,+\,
    \beta \left[ \int_{R_{i+1}}^1 J_{i+1}(\epsilon ') dG(\epsilon ')
     \,+\, G(R_{i+1}) V \right]
                                                           \EQN M_OLG_Ji \cr
V\;&=\; -c \,+\, \beta q(\theta) \sum_{i=2}^{T}
     \left[{u_{i-1} \over u} \left(\int_{R_i}^{1} J_i(\epsilon) dG(\epsilon)
           + G(R_i) V\right)\right] \cr
   &  \hskip1.5cm + \beta (1-q(\theta))V   \EQN M_OLG_V \cr
E_i(\epsilon) \;&=\; w_i(\epsilon) \,+\,
     \beta \left[ \int_{R_{i+1}}^1 E_{i+1}(\epsilon ') dG(\epsilon ')
     \,+\, G(R_{i+1}) U_{i+1} \right]
                                                           \EQN M_OLG_Ei \cr
U_i \;&=\; z\,+\, \beta \theta q(\theta) \left[
\int_{R_{i+1}}^1 E_{i+1}(\epsilon ') dG(\epsilon ')
     \,+\, G(R_{i+1}) U_{i+1} \right]   \cr
     & \hskip1.5cm + \beta (1-\theta q(\theta)) U_{i+1}  \cr
     \;&=\; z\,+\, \beta U_{i+1}
      \,+\, \beta \theta q(\theta)
\int_{R_{i+1}}^1 \bigl[ E_{i+1}(\epsilon ')-U_{i+1} \bigr] dG(\epsilon '),
                                                           \EQN M_OLG_Ui \cr}
$$
where the value  $V$ of a vacancy reflects a firm's probabilities
of being matched with workers of different ages. A free entry
condition requires that a vacancy earn zero expected profits, $V=0$,
and so equation \Ep{M_OLG_V} can be rewritten as
$$
q(\theta)\,=\, { c \over
       \beta \sum_{i=2}^{T} {u_{i-1} \over u}
\int_{R_i}^{1} J_i(\epsilon) dG(\epsilon)} \,.          \EQN M_OLG_noprofit
$$
The expression for the value of an unemployed worker of age $i$
in the second equality of equation \Ep{M_OLG_Ui}  shows
that a successful match earns the worker a surplus of employment over
the value of remaining unemployed, $E_{i+1}(\epsilon ') -U_{i+1}$.

For an acceptable firm-worker match with a worker of age $i$ and
job productivity $\epsilon \geq R_i$, the match surplus is
$$
S_i(\epsilon) \,=\, J_i(\epsilon) \,+\, E_i(\epsilon)\,-\, U_i \geq 0.
                                                           \EQN M_OLG_Si
$$
Nash bargaining implies  that  the surplus is
divided between a worker and a firm according to
$$
 E_i(\epsilon)\,-\, U_i \,=\, \phi S_i(\epsilon) \,=\,
{\phi \over 1-\phi} J_i(\epsilon) \,.                      \EQN M_OLG_Nash
$$
After substituting expressions \Ep{M_OLG_Ji}, \Ep{M_OLG_Ei} and
\Ep{M_OLG_Ui} in equation \Ep{M_OLG_Si}, the surplus of an acceptable
match becomes
$$\EQNalign{
S_i(\epsilon) \;&=\; \epsilon - z\,+\,
    \beta  \int_{R_{i+1}}^1 \left[ J_{i+1}(\epsilon ')
                      +  E_{i+1}(\epsilon ') \right] dG(\epsilon ')
                     \,-\, \beta \left[1-G(R_{i+1})\right] U_{i+1}  \cr
    &\hskip1.4cm -\; \beta \theta q(\theta)
                     \int_{R_{i+1}}^1 \left[ E_{i+1}(\epsilon ')
                      -  U_{i+1} \right] dG(\epsilon ')             \cr
    &=\; \epsilon - z\,+\,
    \beta \left[1-\theta q(\theta) \phi\right]
          \int_{R_{i+1}}^1 S_{i+1}(\epsilon ') \, dG(\epsilon ')\,,
                                                       \EQN M_OLG_Si_1 \cr}$$
where the last equality invokes \Ep{M_OLG_Si} for match
surpluses in period $t+1$, and associated Nash bargaining outcomes
\Ep{M_OLG_Nash},
$E_{i+1}(\epsilon ')-U_{i+1}=\phi S_{i+1}(\epsilon ')$.

We conclude that the surplus function satisfies
$$
S_i(\epsilon) \,=\, \max\left\{\epsilon - z\,+\,
    \beta \left[1-\theta q(\theta) \phi\right]
    \int_{\underline \epsilon}^1 S_{i+1}(\epsilon ') \, dG(\epsilon ')\,,
\; 0 \right\}.                                        \EQN M_OLG_Si_2
$$
The value $R_i$ of the productivity
$\epsilon$ at which the first argument behind the max operator
is zero is
$$
R_i \;=\; z \,-\,  \beta \bigl[1 - \theta q(\theta) \,\phi \bigr]
\int_{\underline \epsilon}^1 S_{i+1}(\epsilon ') dG(\epsilon ')\,.
                                                     \EQN M_OLG_cutoff
$$
Since the surplus function is zero for a worker who has retired from the labor market,
$S_{T+1}(\epsilon ') =0$, so we see from equation \Ep{M_OLG_cutoff} that
 reservation productivity in the last period before retirement is
$R_T = z$, i.e., acceptable jobs are those having productivities above
the value of leisure for an unemployed worker. Furthermore, before the worker's  last period in the labor market, the reservation
productivity is strictly less than $z$ because by staying in the match,
a firm-worker pair is assured of a new productivity draw next period,
and without having to incur any vacancy posting costs.\NFootnote{When
a worker has all the bargaining power,
$\phi=1$, we see from equation \Ep{M_OLG_cutoff} that the reservation
productivity at age $i<T$ would be less than $z$ if and only if
the job finding rate of an unemployed worker is less than one,
$\theta q(\theta)<1$, which reflects the worker's value of staying
on the job in order to be assured of a new productivity
draw next period. But of course, if $\phi=1$, there would not exist
an equilibrium with job creation since firms then could not
recover their vacancy costs. The other extreme of a firm having
all the bargaining power, $\phi =0$, is consistent with job
creation in an equilibrium. With such lopsided bargaining power, the
equilibrium wage would be the value of leisure, $w_i(\epsilon)=z$ (see
equation \Ep{M_OLG_wage} below); and at reservation productivity $R_i$ in
equation \Ep{M_OLG_cutoff}, the firm would be just indifferent to
operating the job and incurring the loss $w_i(R_i)- R_i= z-R_i$ in
return for the expected present value of the total match surplus next
period, $\beta \int_0^1 S_{i+1}(\epsilon ') dG(\epsilon ')$.}



%%%%%%%%%%%%%%%%%%%%%%%%
%The surplus is zero for $\epsilon < R_i$ when a job is terminated, but
%positive for $\epsilon \geq R_i$ when the surplus satisfies
%$$
%S_i(\epsilon) \,+\, U_i \,=\, \epsilon \,+\,
%\beta \left[ \int_0^1 S_{i+1}(\epsilon ') dG(\epsilon ') \,+\, U_{i+1}
%\right],$$
%and hence the surplus functions satisfy
%$$
%S_i(\epsilon) \,=\, \max\left\{\epsilon \,+\,
%\beta \int_0^1 S_{i+1}(\epsilon ') dG(\epsilon ') \,-\,
%(U_i - \beta U_{i+1})\,,  \; 0 \right\}.            \EQN M_OLG_Si_2
%$$
%
%The value $\bar \epsilon_i$ of the productivity
%$\epsilon$ at which the first argument behind the max operator in
%equation \Ep{M_OLG_Si_2} is zero is
%$$\EQNalign{
%\bar \epsilon_i \;&=\; - \beta \int_0^1 S_{i+1}(\epsilon ') dG(\epsilon ')
%         \,+\,U_i\,-\,\beta U_{i+1} \cr
%\;&=\; z \,-\,  \beta \bigl[1 - \theta q(\theta) \,\phi \bigr]
%\int_{R_{i+1}}^1 S_{i+1}(\epsilon ') dG(\epsilon '),   \EQN M_OLG_cutoff }
%$$
%where we have used equation \Ep{M_OLG_Ui} to eliminate $U_i$
%and then used  the Nash bargaining outcome in equation \Ep{M_OLG_Nash}.
%Since $S_{i+1}(\epsilon ')=0$ for $\epsilon ' < R_{i+1}$, we have
%appropriately  integrated   over the region
% at which values are not zero. If the cutoff value
%$\bar \epsilon_i$ in equation \Ep{M_OLG_cutoff} is positive, it equals the
%optimal reservation productivity, i.e.,
%$$
%R_i = \max \{\bar \epsilon_i,\, 0\}.
%$$
%Since the surplus function is zero for a worker who has retired from the
%labor market,
%$S_{T+1}(\epsilon ') =0$, so we see from equation \Ep{M_OLG_cutoff} that
% reservation productivity in the last period before retirement is
%$R_T = z$, i.e., acceptable jobs are those having productivities above
%the value of leisure for an unemployed worker. Furthermore, before the
%worker's  last period in the labor market, the reservation
%productivity is strictly less than $z$ because by staying on the job,
%a worker is assured of a new productivity draw next period, while
%if the worker were to be  unemployed, he or she  would be uncertain
%about being matched with a firm  next period.

\subsection{Reservation productivity is increasing in age}

By substituting \Ep{M_OLG_cutoff} in \Ep{M_OLG_Si_2}, the match
surplus can be expressed as
$$
S_i(\epsilon) \,=\, \max\left\{\epsilon - R_i \,, \; 0 \right\}.
                                                        \EQN M_OLG_Si_3
$$
We will now confirm that the reservation productivity is strictly
increasing in age, $R_i < R_{i+1}$, and hence, the match surplus in
\Ep{M_OLG_Si_3} decreases in age, $S_i(\epsilon) \geq S_{i+1}(\epsilon)$.
Use expression \Ep{M_OLG_Si_3} and apply
integration by parts to the integral over the future surplus in
equation \Ep{M_OLG_cutoff}:
$$\EQNalign{
\int_{R_{i+1}}^1 & S_{i+1}(\epsilon ') \, dG(\epsilon ')
\;=\; \int_{R_{i+1}}^1 \left[\epsilon ' - R_{i+1} \right] dG(\epsilon ') \cr
\;&=\; [1 - R_{i+1}] \, G(1) \,-\, [R_{i+1} - R_{i+1}] \, G(R_{i+1})
- \int_{R_{i+1}}^1 G(\epsilon ') d \epsilon' \cr
\;&=\; 1 - R_{i+1} - \int_{R_{i+1}}^1  G(\epsilon ') d \epsilon'
\,=\, \int_{R_{i+1}}^1  [1 - G(\epsilon ')] d \epsilon' \,.
\EQN M_OLG_intsurplus \cr}
$$
Since $S_{i+1}(\epsilon ')=0$ for $\epsilon ' < R_{i+1}$, we have
appropriately integrated over the region at which values are not zero.
Substituting expression \Ep{M_OLG_intsurplus} into equation
\Ep{M_OLG_cutoff} shows that   reservation productivities are
determined recursively by
$$\EQNalign{
R_i \,&=\,  z \,-\,  \beta \bigl[1 - \theta q(\theta) \,\phi \bigr]
\int_{R_{i+1}}^1 [1 - G(\epsilon ')] d \epsilon'  \EQN M_OLG_cutoff_2;a \cr
\noalign{\hbox{for $i=2,\ldots,T-1$, and}}
R_T \,&=\,  z \,.                                    \EQN M_OLG_cutoff_2;b \cr}
$$
The momentum of the reduction in the reservation productivity when
going from age $T$ to $T-1$ continues throughout the recursions,
as the integral in \Ep{M_OLG_cutoff_2;a} is computed for
a successively widened range when working backward; that is, the
negative of the enlarged term with the integral on the right side of
\Ep{M_OLG_cutoff_2;a} causes the reservation productivity for the
next younger age on the left side of \Ep{M_OLG_cutoff_2;a} to become
smaller, which in the subsequent recursion further widens the range of
integration and hence, further reduces the right side of
\Ep{M_OLG_cutoff_2;a}, and so on. Thus, the reservation
productivity is strictly increasing in age.

The described feedback loop is an equilibrium outcome as follows.
On the one hand, when the reservation productivity is increasing in
age, the range of acceptable productivity draws shrinks over the life
cycle and hence, the expected match surplus unambiguously falls with
age. On the other hand, since the expected match surplus for older
workers is lower, labor hoarding on their jobs becomes less valuable
and hence, their reservation productivities are higher than those of
younger workers. At the very end, for workers of the highest age $T$
in the labor market, who will be retired next period, labor hoarding
cannot be rational so the reservation productivity
becomes equal to the value of leisure.


%%%%%%%%%%%%%%%%%%%%%%%%%%%%%%%%
%Since the surplus function is weakly decreasing in age,
%$S_i(\epsilon) \geq S_{i+1}(\epsilon)$, we can conclude
%from equation \Ep{M_OLG_cutoff} that the reservation productivity is
%weakly increasing in age. To elaborate, suppose that the
%reservation productivity is at an interior solution for age $i$,
%$R_i=\bar \epsilon_i \in(0,1)$, and therefore is also at interior
%solutions for older ages, $R_j\in(0,1)$ for $j\geq i$. Apply
%integration by parts to the integral over the future surplus in
%equation \Ep{M_OLG_cutoff}:
%$$\EQNalign{
%\int_{R_{i+1}}^1 & S_{i+1}(\epsilon ') \, dG(\epsilon ')  \cr
%\;&=\; S_{i+1}(1) \, G(1) \,-\, S_{i+1}(R_{i+1}) \, G(R_{i+1})
%- \int_{R_{i+1}}^1 S_{i+1}'(\epsilon ') \,G(\epsilon ') d \epsilon' \cr
%\;&=\; S_{i+1}(1)
%- \int_{R_{i+1}}^1  G(\epsilon ') d \epsilon'
%\,=\, \int_{R_{i+1}}^1  [1 - G(\epsilon ')] d \epsilon' ,
%\EQN M_OLG_intsurplus \cr}
%$$
%where the second equality invokes $G(1)=1$, $S_{i+1}(R_{i+1})=0$, and
%$S_{i+1}'(\epsilon ')=1$. The third equality uses $S_{i+1}'(\epsilon ')=1$
% and an  interior solution to the reservation productivity
%$R_{i+1}$ to deduce  that $S_{i+1}(1) = \int_{R_{i+1}}^1 1 \,d\epsilon '$.
%Substituting expression \Ep{M_OLG_intsurplus} into equation
%\Ep{M_OLG_cutoff} shows that   reservation productivities are
%determined recursively by
%$$\EQNalign{
%R_j \,&=\,  z \,-\,  \beta \bigl[1 - \theta q(\theta) \,\phi \bigr]
%\int_{R_{j+1}}^1 [1 - G(\epsilon ')] d \epsilon'  \EQN M_OLG_cutoff_2;a \cr
%\noalign{\hbox{for $j=i,\ldots,T-1$, and}}
%R_T \,&=\,  z \,.                                    \EQN M_OLG_cutoff_2;b %\cr}
%$$
%%%%%%%%%%%%%%%%%%%%%%%%%%%%%%%%%%%%%%%%%%%%%


\subsection{Wage rate is decreasing in age}\label{Sec_M_OLG_wage}%
The outcome that the reservation productivity increases with age means that
employed older workers have a higher average productivity than younger
ones. However, as measured by wage rates conditional on job productivity, we shall
now show that
 older workers earn less than younger workers. Recall
from wage equation \Ep{wage2} in the standard matching model that the
equilibrium wage is a function of both a job's productivity and
a worker's value of unemployment. The latter decreases
with a worker's  age in our model.

After substituting expression \Ep{M_OLG_noprofit} for $q(\theta)$ in
equation \Ep{M_OLG_Ui} and utilizing Nash bargaining outcome \Ep{M_OLG_Nash},
the value of an unemployed worker of age $i$ becomes
$$
U_i \,=\, z\,+\, \beta U_{i+1}
      \,+\, {\phi \over 1-\phi} \, \theta \, c \, \kappa_{i+1}, \EQN M_OLG_Ui_2
$$
where
$$
\kappa_{i+1} \equiv { \int_{R_{i+1}}^1 S_{i+1}(\epsilon ') dG(\epsilon ')
                  \over
           \sum_{j=2}^{T} {u_{j-1} \over u}
               \int_{R_j}^{1} S_j(\epsilon) dG(\epsilon)} .   \EQN M_OLG_kappa
$$
Since the surplus function is decreasing in age,
it follows that $\kappa_i > \kappa_{i+1}$ is also decreasing in age.  In addition, starting with
the terminal value $U_{T+1}=0$ and then working backwards,  the value
of unemployment in equation \Ep{M_OLG_Ui_2} can be shown  to
be decreasing in age.


To compute the equilibrium wage for a worker of age $i$, we start with the Nash bargaining outcome
in expression \Ep{M_OLG_Nash}
$$
 E_i(\epsilon)\,-\, U_i \,=\,
\phi \bigl[J_i(\epsilon) \,+\, E_i(\epsilon)\,-\, U_i \bigr] \EQN M_OLG_Nash_2
$$
and  rearrange it  to  get
$$\EQNalign{
(1-\phi) U_i \;&=\; E_i(\epsilon)
          \,-\, \phi \bigl[J_i(\epsilon) \,+\, E_i(\epsilon) \bigr]  \cr
\;&=\; w_i(\epsilon) - \phi \epsilon  + (1-\phi) \beta G(R_{i+1}) U_{i+1} \cr
& \hskip.3cm + \beta \int_{R_{i+1}}^1
\Bigl\{ E_{i+1}(\epsilon ') - \phi \bigl[J_{i+1}(\epsilon ')
                    + E_{i+1}(\epsilon ')\bigr] \Bigr\}    d G(\epsilon') \cr
\;&=\; w_i(\epsilon) - \phi \epsilon  + (1-\phi) \beta G(R_{i+1}) U_{i+1}
+ (1-\phi) \beta [1-G(R_{i+1})] U_{i+1}                                      \cr
& \hskip.3cm + \beta \int_{R_{i+1}}^1
\Bigl\{ E_{i+1}(\epsilon ') - U_{i+1} - \phi \bigl[J_{i+1}(\epsilon ')
        + E_{i+1}(\epsilon ') - U_{i+1}\bigr] \Bigr\}       d G(\epsilon') \cr
\;&=\; w_i(\epsilon) - \phi \epsilon  + (1-\phi) \beta U_{i+1} , \cr}
$$
where the second equality is obtained by eliminating
$J_i(\epsilon)$ and $E_i(\epsilon)$ by using equations
\Ep{M_OLG_Ji} and \Ep{M_OLG_Ei}, and the third equality follows
from adding and subtracting $(1-\phi)\beta[1-G(R_{i+1})]U_{i+1}$.
The integral on the right side of the third equality  is zero according to Nash bargaining
outcome \Ep{M_OLG_Nash_2};  after further simplification, we arrive
at the last fourth equality. From the outermost left and right
 sides of the above succession of equalities the equilibrium wage of a worker of age $i$
satisfies
$$
w_i(\epsilon) \,=\, \phi \epsilon + (1-\phi)\left[U_i - \beta U_{i+1}\right].
$$
After eliminating  $U_i$ by using equation \Ep{M_OLG_Ui_2},
we arrive at
$$
w_i(\epsilon) \,=\, z \,+\, \phi [ \epsilon - z + \theta\, c\, \kappa_{i+1}].
                                                             \EQN M_OLG_wage
$$

Since $\kappa_{T+1}=0$, it follows that the wage in the last period
before retirement is $w_T=z + \phi (\epsilon - z)$, i.e., the worker
receives the outside payoff to an unemployed worker, $z$, plus a worker's
Nash bargaining share of the surplus from a one-period match,
$\phi (\epsilon - z)$. Wages before  that last period ($i<T$) are
higher by virtue  of the higher outside value of younger workers
as captured by the term $\phi \, \theta \, c \, \kappa_{i+1}$
in expression \Ep{M_OLG_wage}, which is
decreasing in age.

\subsection{Welfare analysis}

We study a planning
 problem with equal Pareto weights on subsequent generations
and individual agents who do not discount the future, $\beta=1$. An
optimal allocation  maximizes steady-state
output net of vacancy costs plus the value of leisure enjoyed by
the unemployed. Given an unemployment rate $u_{i-1}$ at
the end of age $i-1$, the output of workers of age $i$ is
the product of the fraction of age $i$ workers employed,
$\bigl[u_{i-1} \theta q(\theta) +1-u_{i-1}\bigr]
\bigl[1-G(R_i)\bigr]$,
and their average productivity,
$\bigl[1-G(R_i)\bigr]^{-1}\int_{R_i}^1 \epsilon \,d G(\epsilon)$.
Omitting  the constant $z u_1$ from the objective function, the planner's optimization problem becomes
$$\EQNalign{
& \max_{\theta, \{R_i, u_i\}_{i=2}^T} \;\; \sum_{i=2}^T
\left\{ \bigl[u_{i-1} \theta q(\theta) +1-u_{i-1}\bigr]
        \int_{R_i}^1 \epsilon \,d G(\epsilon) + z u_i\right\}
 -c \theta \sum_{j=1}^T u_j                      \cr
\noalign{\vskip.2cm}
%%%\noalign{\hbox{subject to }}
&\hbox{subject to }\ \
 u_{i} \;=\; u_{i-1} \Bigl\{1-\theta q(\theta) \left[1-G(R_i) \right]\Bigr\}
     + (1-u_{i-1}) G(R_i) \cr
&\hskip8cm \hskip.3cm  \hbox{\rm for     } i=2,\ldots , T   \cr
&\hbox{given \ \ } u_1=1\,.                                 \cr}
%&\hskip5cm \hskip.3cm  \hbox{\rm for     } i=2,\ldots , T, \hbox{given \ \ } u_1=1 \,,  \cr}
%&\hbox{given \ \ } u_1=1\,.                                \cr}
$$


%We confine our analysis to settings of  primitives that give rise to interior
%solutions for reservation productivities for workers of all ages.
First-order necessary conditions at interior solutions are
$$\EQNalign{
\theta:\hskip.5cm &\sum_{i=2}^T
   u_{i-1} \bigl[q(\theta) + \theta q'(\theta) \bigr]
        \int_{R_i}^1 \epsilon \,d G(\epsilon) -c\sum_{j=1}^T u_j  \cr
\noalign{\vskip.2cm}
&\hskip 1cm   \,-\, \sum_{i=2}^T\lambda_i \bigl[q(\theta)+\theta q'(\theta) \bigr]
               \left[1-G(R_i) \right] u_{i-1} \;=\; 0  \cr
\noalign{\vskip.2cm}
R_i:\hskip.5cm &\bigl[u_{i-1} \theta q(\theta) + 1-u_{i-1}\bigr]
                                                \bigl(-R_i \, g(R_i)\bigr) \cr
\noalign{\vskip.2cm}
&\hskip 1cm   \,+\, \lambda_i \Bigl[ u_{i-1} \theta q(\theta) g(R_i)
       + (1-u_{i-1}) g(R_i)\Bigr]  \;=\; 0             \cr
\noalign{\vskip.2cm}
u_i:\hskip.5cm &z - c \theta -\lambda_i + \bigl[\theta q(\theta) -1\bigr]
               \int_{R_{i+1}}^1 \epsilon \,dG(\epsilon)                    \cr
\noalign{\vskip.2cm}
&\hskip 1cm   \,+\, \lambda_{i+1} \Bigl\{1-
     \theta q(\theta) \left[1-G(R_{i+1}) \right] -G(R_{i+1})\Bigr\}  \;=\; 0\,,
                                             \hskip1cm   \cr}
$$
where the last two terms on the left side of the last equation should
be understood to be zero for the last period before retirement, $u_T$, i.e.,
the first-order condition  becomes
$z - c \theta -\lambda_T=0$. After rearranging and simplifying, let us rewrite the
first-order conditions as
$$\EQNalign{
\theta: \hskip.5cm & c \, u \,=\, q(\theta) [1-\alpha]
    \sum_{i=2}^T u_{i-1} \left\{ \int_{R_i}^1 \epsilon \,d G(\epsilon)
   \,-\, \lambda_i \left[1-G(R_i) \right] \right\}
                                   \hskip1cm        \EQN M_OLG_foc_theta \cr
\noalign{\vskip.2cm}
R_i:\hskip.5cm &\lambda_i \,=\, R_i               \EQN M_OLG_foc_Ri  \cr
\noalign{\vskip.2cm}
u_i:\hskip.5cm &\lambda_i \,=\; z - c \,\theta
    - \bigl[1 - \theta q(\theta)\bigr] \cr
&\hskip2cm  \cdot\left\{ \int_{R_{i+1}}^1 \epsilon \,d G(\epsilon)
   \,-\, \lambda_{i+1} \left[1-G(R_{i+1}) \right] \right\},
                                                      \EQN M_OLG_foc_ui  \cr}
$$
where $u$ is total unemployment, $u=\sum_{j=1}^T u_j$, and $\alpha$ is the
elasticity of matching with respect to unemployment,
$\alpha = - q'(\theta) \,\theta / q(\theta)$, as described in equation
\Ep{matchingfunction}. By substituting \Ep{M_OLG_foc_Ri} into
\Ep{M_OLG_foc_theta} and \Ep{M_OLG_foc_ui}, and by applying integration
by parts,\NFootnote{Integration by parts yields
$$ \int_{R_i}^1 \epsilon\, dG(\epsilon)
    = G(1) - R_i G(R_i) - \int_{R_i}^1 G(\epsilon)\, d\epsilon
    = R_i \bigl[1-G(R_i)\bigr]
       + \int_{R_i}^1 \bigl[ 1 - G(\epsilon)\bigr] d\epsilon\,,
$$
where the last equality is obtained by adding and subtracting $R_i$.}
the following equations characterize the  optimum of the Pareto problem:
$$\EQNalign{
q(\theta) \;&=\; {c \over [1-\alpha]
    \sum_{i=2}^T {u_{i-1} \over u}
  \int_{R_i}^1 \bigl[1-G(\epsilon)\bigr] \,d \epsilon}  \EQN M_OLG_opt1 \cr
\noalign{\vskip.2cm}
R_i \,&=\; z - c \,\theta
    - \bigl[1 - \theta q(\theta)\bigr]
          \int_{R_{i+1}}^1 \bigl[1-G(\epsilon)\bigr] \,d \epsilon
                                                             \EQN M_OLG_opt2;a \cr
\noalign{\hbox{ for $i=2,\ldots, T-1$, and}}
R_T \,&=\; z - c \,\theta \,.                                \EQN M_OLG_opt2;b \cr}
$$
Given a market tightness $\theta$, it follows immediately from
equations \Ep{M_OLG_opt2} that the socially optimal reservation
productivity is increasing in age, just as it is in the decentralized or market economy
analyzed above.
However, note that the reservation productivity \Ep{M_OLG_opt2;b} in the
last period before retirement is lower than the corresponding
reservation productivity \Ep{M_OLG_cutoff_2;b} in the market economy, which
seems to suggest that an optimal labor market policy calls for
 employment  subsidies for older workers that lower  the
reservation productivities attached to hiring them. We will confirm this conjecture
and also show that  employment of younger workers should
 be taxed.

To prepare to study  an optimal labor market policy, it is useful to
substitute expression \Ep{M_OLG_opt1} for $q(\theta)$ in
equation \Ep{M_OLG_opt2;a}
$$
R_i +  \int_{R_{i+1}}^1 \bigl[1-G(\epsilon)\bigr] \,d \epsilon
\,=\; z - c \,\theta + {\theta \,c \over 1-\alpha} \hat \kappa_{i+1},
                                                        \EQN M_OLG_optR_1
$$
where
$$
\hat \kappa_{i+1} \equiv
{ \int_{R_{i+1}}^1 \bigl[1-G(\epsilon)\bigr] d \epsilon \over
               \sum_{j=2}^{T} {u_{j-1} \over u}
               \int_{R_j}^{1} \bigl[1-G(\epsilon)\bigr] d \epsilon }
                                                        \EQN M_OLG_kappaHAT
$$
for $i=2,\ldots, T-1$.
Add and subtract $\theta\, c\, \hat \kappa_{i+1}$ to the
right  side of equation \Ep{M_OLG_optR_1} to get
$$
R_i +  \int_{R_{i+1}}^1 \bigl[1-G(\epsilon)\bigr] \,d \epsilon
\,=\; z - c \,\theta \bigl[1- \hat \kappa_{i+1} \bigr]
+ {\alpha \over 1-\alpha} \theta \,c \,\hat \kappa_{i+1}.
                                                        \EQN M_OLG_optR_2
$$


\subsection{The optimal policy}
%When a social optimum entails interior solutions for reservation
%productivities of all ages,%
The optimal allocation can be supported
by  age-specific subsidies $\delta_i$ to employment  (taxes if
negative)\NFootnote{We assume that
any deficit or surplus from the proposed scheme of employment subsidies
and taxes are offset with lump-sum transfers imposed on all agents.}
so long as workers' bargaining strength $\phi$
satisifies the Hosios condition, $\phi=\alpha$. We will assume that
the Hosios condition holds along with our assumption from
the previous section that $\beta=1$.

Introducing subsidies to employment alters  equation
\Ep{M_OLG_Si_2} for the match surplus to
%$$
%S_i(\epsilon) \,=\, \max\left\{\epsilon \,+\, \delta_i \,+\,
%\beta \int_{\underline \epsilon}^1 S_{i+1}(\epsilon ') dG(\epsilon ')
%\,-\, (U_i - \beta U_{i+1})\,,
%\; 0 \right\}.                             \EQN M_OLG_S_subsidy
%$$
$$
S_i(\epsilon) \,=\, \max\left\{\epsilon \,+\, \delta_i \,-\,z\,+\,
    \beta \left[1-\theta q(\theta) \phi\right]
    \int_{\underline \epsilon}^1 S_{i+1}(\epsilon ') \, dG(\epsilon ')\,,
\; 0 \right\}.                                  \EQN M_OLG_S_subsidy
$$
In addition, equation \Ep{M_OLG_Ji} for a firm's value of a
filled job is modified to become
$$
J_i(\epsilon) \,=\, \epsilon \,+\, \delta_i \,-\, w_i(\epsilon)\,+\,
    \beta \int_{R_{i+1}}^1 J_{i+1}(\epsilon ') dG(\epsilon ').
                                              \EQN M_OLG_J_subsidy
$$
Following  the same steps used to derive  wages in
section \use{Sec_M_OLG_wage}, we arrive at
$$
w_i(\epsilon) \,=\, z \,+\, \phi [ \epsilon + \delta_i
                                   - z + \theta\, c\, \kappa_{i+1}].
                                          \EQN M_OLG_wage_subsidy
$$

At the reservation productivity $R_i$, we know that $J_i(R_i)=0$
so from equation \Ep{M_OLG_J_subsidy}, we have
$$\EQNalign{
0 \,&=\, R_i \,+\, \delta_i \,-\, w_i(R_i)\,+\,
    \beta (1-\phi) \int_{R_{i+1}}^1 S_{i+1}(\epsilon ') dG(\epsilon ') \cr
   \,&=\, R_i \,+\, \delta_i \,-\, w_i(R_i)\,+\,
    \beta (1-\phi) \int_{R_{i+1}}^1 \bigl[1-G(\epsilon ')\bigr] d \epsilon ',
                                 \hskip1cm   \EQN M_OLG_cutoff_subsidy \cr}
$$
where the first equality invokes the Nash bargaining outcome \Ep{M_OLG_Nash},
$J_{i+1}(\epsilon ')= (1-\phi) S_{i+1}(\epsilon ')$,
and the second equality uses expression \Ep{M_OLG_intsurplus}.
Similar invocations of relationships
\Ep{M_OLG_Nash} and \Ep{M_OLG_intsurplus} in the no-profit
condition \Ep{M_OLG_noprofit} and in equation \Ep{M_OLG_kappa}
for $\kappa_{i+1}$ establish that
$$
q(\theta)\,=\, { c \over
       \beta (1-\phi) \sum_{i=2}^{T} {u_{i-1} \over u}
\int_{R_i}^{1} \bigl[1-G(\epsilon)\bigr]d \epsilon}
                                             \EQN M_OLG_noprofit_subsidy
$$
$$
\kappa_{i+1} = \hat \kappa_{i+1},              \EQN M_OLG_kappa_same
$$
respectively, where $\hat \kappa_{i+1}$ is given by equation
\Ep{M_OLG_kappaHAT}.

After substituting expression \Ep{M_OLG_wage_subsidy}
for $w_i(R_i)$ in equation \Ep{M_OLG_cutoff_subsidy}, and
using equation \Ep{M_OLG_kappa_same}, we find that an
equilibrium is characterized by
$$
R_i +  \beta \int_{R_{i+1}}^1 \bigl[1-G(\epsilon)\bigr] \,d \epsilon
\,=\; z - \delta_i
+ {\phi \over 1-\phi} \theta \,c \,\hat \kappa_{i+1},
                                               \EQN M_OLG_R_subsidy
$$
By comparing expressions \Ep{M_OLG_optR_2} and \Ep{M_OLG_R_subsidy},
and recalling our
assumptions that $\phi=\alpha$ and $\beta=1$, it follows that an
age-specific employment subsidy of
$\delta_i= c \,\theta \bigl[1- \hat \kappa_{i+1} \bigr]$
 would attain the socially
optimal reservation productivity whenever the market tightness
is the same. By inspecting equations \Ep{M_OLG_opt1} and
\Ep{M_OLG_noprofit_subsidy},
we can also confirm (via a circular or fixed-point argument) that market
tightness $\theta$ is indeed the same whenever the reservation
productivites are the same.

Thus, we have shown that employment in the last period before
retirement, should be subsidized by $\delta_T=c\,\theta$.
Subsidies $\delta_i= c \,\theta \bigl[1- \hat \kappa_{i+1} \bigr]$ to employment at earlier ages
taper off with time  to retirement, since $\hat \kappa_{i+1}$
is decreasing in age; at a sufficiently young age,
the subsidy becomes negative and turns into a tax on employment of
young workers (when $\hat \kappa_{i+1} > 1$). Note that, except
for one caveat, $\hat \kappa_{i+1}$ as defined in
\Ep{M_OLG_kappaHAT} is the expected next-period surplus for an employed
worker of age $i$ relative to a weighted average across employed workers
of all ages, where the weights are age-specific unemployment,
$u_i$, as a fraction of total unemployment, $u$. The caveat is
that these weights sum to less than one because unemployment of
the youngest generation, $u_1=1$, is included in $u$ while there are no
employed workers in that generation. However, this caveat just serves to  emphasize
that there is a critical cutoff age $i$ at which  $\hat \kappa_{j+1} >1$
for all $j\leq i$, since  the expected next-period surplus of such a
young employed worker, which tends to be greater than an economy-wide  weighted
average,  is compared to something less than a weighted
average of expected next-period surpluses of all employed workers.

The justification  for the subsidy $\delta_T=c\,\theta$ to employed workers
in the last period before retirement is that if one of them joins the
ranks of the unemployed, the economy incurs a vacancy cost per unemployed
equal to $c \, \theta$ with no potential gain in terms of future matches.
So long as this cost exceeds a worker's value of leisure when unemployed
net of the output in the present job, $c\,\theta \geq z - \epsilon$, it
is socially optimal for the worker to  remain employed; the subsidy accomplishes this
 by lowering the reservation productivity to
$R_T= z-c \, \theta$.
Similarly, employed workers further  from retirement  are also subsidized, but by less, in order to
ameliorate  congestion in the matching function. Interestingly, the argument is reversed for sufficiently young
workers whose employment should instead be taxed, because otherwise they would
 fail to internalize the positive externality that they
exert in the matching function.



\auth{Menzio, Guido} \auth{Telyukova, Irina A.} \auth{Visschers, Ludo}%
\index{directed search}%
\section{Directed search: age-specific matching functions}
\label{sec:OLG2}%
Following Menzio, Telyukova and Visschers (2016), we now
assume age-specific matching functions. Within a particular  submarket and
 matching function, firms post vacancies for a
particular age of  unemployed workers;  only workers of
that age  are allowed to sit in that  matching
function. Such a directed search setting leads to a
block recursive structure in which agents' value and policy
functions and measures of market tightness are independent of
the distribution of workers across states of employment and
unemployment. Two important features are
(1)  computation of equilibria simplifies;
(2)  the congestion externalities of section \use{sec:M_OLG1}
vanish because there is no longer a  mixture  of heterogeneous
workers sitting inside a matching function.

To facilitate a transparent presentation, we shut down differences in
productivity, each employed worker produces $y$;
and let age be the only source
of heterogeneity. Matches break up exogenously
with probability $s$.



\subsection{Value functions and market tightness}

A key difference from the  section \use{sec:M_OLG1} setting is that there
are now age-specific measures of market tightness $\theta_i$ and
values of vacancy creation $V_i$. Corresponding to  value
functions \Ep{M_OLG_Ji}--\Ep{M_OLG_Ui},
we have
$$\EQNalign{
J_i \;&=\; y \,-\, w_i \,+\, \beta (1-s) J_{i+1}  \EQN M_OLG2_Ji \cr
V_i \;&=\; -c \,+\, \beta q(\theta_i) J_{i+1}     \EQN M_OLG2_Vi \cr
E_i \;&=\; w_i \,+\, \beta
    \left[ (1-s) E_{i+1} \,+\, s U_{i+1} \right]  \EQN M_OLG2_Ei \cr
U_i \;&=\; z\,+\, \beta \theta_i q(\theta_i) E_{i+1}
     \,+\, \beta \left[1 - \theta_i q(\theta_i) \right] U_{i+1}   \cr
    \;&=\; z\,+\, \beta U_{i+1} \,+\, \beta \theta_i q(\theta_i)
               \left[ E_{i+1} - U_{i+1} \right],   \EQN M_OLG2_Ui \cr}
$$
where we have already imposed the zero-profit condition in
vacancy creation on the right sides of these
equations. After also imposing $V_i=0$ on the left
side of \Ep{M_OLG2_Vi}, a zero-profit condition  becomes
%the break-even condition for the
%creation vacancies for workers of age $i$ becomes
$$
q(\theta_i)\,=\, { c \over \beta J_{i+1} }  \,.   \EQN M_OLG2_noprofit
$$

The age-specific match surplus is given by
$$
S_i \;=\; J_i \,+\, E_i \,-\, U_i                 \EQN M_OLG2_surplus
$$
and as before, Nash bargaining determines how this surplus
is shared by a worker and a firm according to
$$
E_i\,-\,U_i \;=\; \phi S_i
             \;=\; \frac{\phi}{1-\phi} J_i\,.      \EQN M_OLG2_Nash
$$
After substituting expressions \Ep{M_OLG2_noprofit} and
\Ep{M_OLG2_Nash} into \Ep{M_OLG2_Ui}, the value of an
unemployed worker of age $i$ becomes
$$
U_i \;=\; z+ \beta U_{i+1} + \beta \theta_i
        { c \over \beta J_{i+1} } \, \frac{\phi}{1-\phi} J_{i+1}
    \;=\; z + \beta U_{i+1} + {\phi \over 1-\phi}\, c \,\theta_i\,.
                                                  \EQN M_OLG2_UiII
$$
Next, substitutions of expressions \Ep{M_OLG2_Ji},
\Ep{M_OLG2_Ei} and \Ep{M_OLG2_UiII} into \Ep{M_OLG2_surplus}
yield an equation for the match surplus in terms of market
tightness;
$$\EQNalign{
S_i \;&=\; J_i\,+\,E_i\,-\,U_i \;=\; y - w_i + \beta (1-s) J_{i+1}   \cr
  &\hskip.5cm +w_i+ \beta \left[ (1-s) E_{i+1} \,+\, s U_{i+1} \right]
   -\left[z + \beta U_{i+1} +{\phi \over 1-\phi}\, c \,\theta_i\right] \cr
  &=\; y - z - {\phi \over 1-\phi}\, c \,\theta_i + \beta (1-s) S_{i+1}\cr
    \;&=\; \sum_{j=0}^{T-i} \beta^j (1-s)^j
          \left[ y - z -\frac{\phi}{1-\phi}\, c \,\theta_{i+j} \right]\,,
                                              \EQN M_OLG2_surplusII \cr}
$$
where the last equality emerges after continued
substitutions of subsequent match surpluses at higher ages,
$S_{i+j}$, with the terminal value $S_{T+1}=0$.

To arrive at a characterization  of equilibrium market
tightness in terms of primitives, we start by deriving two
expressions for wages that must  be satisfied in
an equilibrium. A first  expression  is
based on the no-profit condition for vacancy creation.
Specifically, solve forward for the value of a filled
job in equation \Ep{M_OLG2_Ji}
$$\EQNalign{
J_{i+1} \;&=\; y - w_{i+1} + \beta (1-s) J_{i+2}
        \;=\; \sum_{j=0}^{T-i-1} \beta^j (1-s)^j
          \left[ y - w_{i+1+j} \right]                              \cr
        \;&=\; \frac{1-\beta^{T-i} (1-s)^{T-i}}{1-\beta (1-s)} \,y
           \,-\, \sum_{j=0}^{T-i-1} \beta^j (1-s)^j w_{i+1+j} \,.
                                                 \EQN M_OLG2_J_recurs \cr}
$$
%$$
%\frac{1-\beta^{T-i} (1-s)^{T-i}}{1-\beta (1-s)} \,y
%           \,-\, \sum_{j=0}^{T-i-1} \beta^j (1-s)^j w_{i+1+j}
%\;=\; \beta^{-1} \frac{c}{q(\theta_i)}\,,
%$$
By substituting this expression into \Ep{M_OLG2_noprofit},
multiplying through by \hbox{$[1-\beta(1-s)]$}, and rearranging,
we obtain
$$
\left[1-\beta(1-s)\right] \sum_{j=0}^{T-i-1} \beta^j (1-s)^j w_{i+1+j}
  \;=\; \left[1-\beta^{T-i} (1-s)^{T-i}\right]y
        \,-\, \frac{r+s}{q(\theta_i)}\, c \,,      \EQN M_OLG2_wage1
$$
where the last term uses $\beta^{-1}[1-\beta(1-s)]=[1-r-(1-s)]=r+s$.
A second expression requires that  wage payments are
consistent with Nash bargaining. Specifically, use
equations \Ep{M_OLG2_Ei} and \Ep{M_OLG2_UiII} and
solve forward for a worker's part of the match surplus:
$$\EQNalign{
E_{i+1}-U_{i+1}\;&=\; w_{i+1}
    + \beta \left[ (1-s) E_{i+2} \,+\, s U_{i+2} \right]
-\left[z + \beta U_{i+2} +{\phi \over 1-\phi}\, c \,\theta_{i+1}\right] \cr
     &=\; w_{i+1} - z - {\phi \over 1-\phi}\, c \,\theta_{i+1}
    + \beta (1-s) \left[ E_{i+2} \,-\, U_{i+2} \right]                 \cr
   \;&=\; \sum_{j=0}^{T-i-1} \beta^j (1-s)^j w_{i+1+j}
     -\frac{1-\beta^{T-i} (1-s)^{T-i}}{1-\beta (1-s)} \,z   \cr
&\hskip.5cm  -\; \frac{\phi}{1-\phi} \,c
                \sum_{j=0}^{T-i-1} \beta^j (1-s)^j \theta_{i+1+j}\,.
                                                 \EQN M_OLG2_EU_recurs \cr}
$$
By substituting \Ep{M_OLG2_J_recurs} and \Ep{M_OLG2_EU_recurs}
into \Ep{M_OLG2_Nash}, $(1-\phi)[E_{i+1}-U_{i+1}]=\phi J_{i+1}$,
and rearranging, we arrive at
$$\EQNalign{
\left[1-\beta(1-s)\right] &\sum_{j=0}^{T-i-1} \beta^j (1-s)^j w_{i+1+j}
  \;=\; \left[1-\beta^{T-i} (1-s)^{T-i}\right]
             \left[\phi\,y +(1-\phi)\, z\right]            \cr
  &+\, \left[1-\beta(1-s)\right] \phi\, \,c
       \sum_{j=0}^{T-i-1} \beta^j (1-s)^j \theta_{i+1+j}\,.
                                                    \EQN M_OLG2_wage2 \cr}
$$
Set the right sides of \Ep{M_OLG2_wage1} and \Ep{M_OLG2_wage2}
equal to each other and rearrange to get
$$\EQNalign{
 & \left[1-\beta(1-s)\right] \frac{\phi}{1-\phi} \,c
      \sum_{j=0}^{T-i-1} \beta^j (1-s)^j \theta_{i+1+j}             \cr
  & \;=\; \left[1-\beta^{T-i} (1-s)^{T-i}\right]
             \left[y - z\right]
    \,-\,  \frac{r+s}{(1-\phi) q(\theta_i)}\, c .
                                                    \EQN M_OLG2_theta \cr}
$$
This expression determines  age-specific market tightnesses
and  is a counterpart of expression \Ep{equili} in the standard
matching model with infinitely-lived workers.\NFootnote{By imposing
an infinite lifespan, $T=\infty$, and constant market tightness,
$\theta_i =\theta$, expression \Ep{M_OLG2_theta} reduces to  \Ep{equili}. Likewise, the same manipulations
of expressions \Ep{M_OLG2_wage1} and \Ep{M_OLG2_wage2}, including
the imposition of constant wages, $w_i = w$, yield
 expressions \Ep{wage1} and \Ep{wage2}
in the standard matching model with infinitely-lived workers.}

Equilibrium market tightnesses can be computed
recursively. Starting at age $T$, firms choose not to
create any vacancies, $\theta_T=0$, since workers of age $T$
are known to retire next period and hence, there would be no
return to  investing  in such vacancies. Solving backward,
measures of market tightness can then be computed recursively
from expression \Ep{M_OLG2_theta}.


\subsection{Job finding rate is decreasing in age}

To determine the dynamics of age-specific market tightness,
we compute  differences in match surpluses between two
adjacent ages $i$ and $i+1$, for $i= 1, 2, \dots, T-1$,
using expression \Ep{M_OLG2_surplusII}:
$$\EQNalign{
S_i-S_{i+1} \;&=\; \sum_{j=0}^{T-i} \beta^j (1-s)^j \left[y - z\right]
                -\sum_{j=0}^{T-i-1} \beta^j (1-s)^j \left[y - z\right]\cr
&\hskip.5cm  -\; \frac{\phi}{1-\phi} \,c
               \sum_{j=0}^{T-i} \beta^j (1-s)^j \theta_{i+j}
             +\; \frac{\phi}{1-\phi} \,c
                \sum_{j=0}^{T-i-1} \beta^j (1-s)^j \theta_{i+1+j} \cr
            \;&=\; \beta^{T-i} (1-s)^{T-i} \left[y - z\right]
             \; -\; \frac{\phi}{1-\phi} \,c \,\theta_i             \cr
&\hskip.5cm  +\; \frac{\phi}{1-\phi} \,c
                 \sum_{j=0}^{T-i-1} \beta^j (1-s)^j
    \left[1-\beta(1-s)\right]\theta_{i+1+j}                        \cr
           \;&=\; \beta^{T-i} (1-s)^{T-i} \left[y - z\right]
             \; -\; \frac{\phi}{1-\phi} \,c \,\theta_i             \cr
&\hskip.5cm  +\; \left[1-\beta^{T-i} (1-s)^{T-i}\right] \left[y-z\right]
             \;-\; \frac{r+s}{(1-\phi)q(\theta_i)}\,c              \cr
           \;&=\; y-z \;-\;\frac{r+s\,+\,\phi\,\theta_i\,q(\theta_i)}
                    {(1-\phi)q(\theta_i)}\,c\,, \EQN M_OLG2_surdiff \cr}
$$
where we replace the term with the summation sign in the second
equality  by the right  side of equation
\Ep{M_OLG2_theta}.

Under the assumption of long-lived workers, i.e., a large $T$,
young workers' experiences should resemble  those of
infinitely-lived workers. This can be confirmed from expression
\Ep{M_OLG2_surdiff}, evaluated at some young ages $i$ and $i+1$,
for which match surpluses would have to be practically the same.
Specifically, a near-zero value of the last equality
in \Ep{M_OLG2_surdiff} shows that the equilibrium value of
$\theta_i$ would be practically identical to that of market
tightness in the standard matching model with infinitely-lived
workers, as given by equation \Ep{equili}. The incentives of those firms and young workers
to engage in job creation are practically identical with those in a
standard matching model with infinitely-lived workers because the
matches into which they enter will almost certainly break before workers retire,
 and at most of those future separations, the
worker will  still be much younger than
retirement age $T+1$. Therefore, in the setting of this section, young workers' experiences are
similar to those of infinitely-lived workers.

Because workers have  finite lives,  equilibrium values of
$\theta_i$  have to be less than market tightness
in the standard matching model with infinitely-lived workers.
Subject to the risk of workers retiring before an exogenous
job destruction shock, the `invisible hand'
compensates firms that create vacancies with a higher probability
of filling vacancies, i.e., a lower equilibrium value of market
tightness. By the last equality in \Ep{M_OLG2_surdiff}, it follows
that $S_i - S_{i+1}$ is strictly positive, so  match
surpluses are decreasing in age, $S_i > S_{i+1}$. By
substituting \Ep{M_OLG2_Nash} into \Ep{M_OLG2_noprofit},
$q(\theta_i)\,=\, c / (\beta (1-\phi) S_{i+1})$, we can
confirm that market tightness also declines
in age and hence, a worker's job finding rate  decreases with
age.

The declining job finding rate becomes especially pronounced
towards the end of a worker's labor market career, more so with  a low
exogenous separation rate $s$, i.e., when jobs are expected to
last long in the absence of retirement. As an illustration,
Figure \Fg{DS_job_finding_prob} reports a numerical example
where the job finding index is a worker's job finding probability
relative to that of the youngest worker (with the highest job
finding probability).\NFootnote{As in common
parameterizations of matching models, the elasticity $\alpha$ of a
Cobb-Douglas matching function with respect to unemployment
and a worker's bargaining power $\phi$ are set equal to each other and near
 the middle of the unit interval, $\alpha=\phi=0.5$, and
the replacement ratio in unemployment is around half of
worker productivity, $z/y=0.6$.
The annual discount rate is 4 percent, and a worker's labor
market career lasts 45 years. (The model period is set  to
be one day.) For different economies indexed  by their exogenous job destruction rates $s$,
 the vacancy
cost $c$ and a multiplicative efficiency parameter $A$ in the
matching function are chosen to yield an unemployment
rate of 5 percent in a corresponding matching model with
infinitely-lived workers. (When there is no calibration target
for vacancies, fixing either $c$ or $A$ amounts to a  normalization, with
the other parameter then being used to target an unemployment rate.)}
The axis labeled mean job duration, calculated as $1/s$,
identifies different economies defined by their exogenous job
destruction rate $s$. For each such economy, the age-specific
job finding index is shown for older workers defined by their
times to  retirement.


\midfigure{DS_job_finding_prob}
\centerline{\epsfxsize=3truein\epsffile{DS_job_finding_prob.eps}}
\caption{Job finding index of older workers, defined by their
time left until retirement, in different economies, defined by
their mean job duration. The index is a worker's job finding
probability relative to that of the youngest worker in respective
economy.}
\infiglist{DS_job_finding_problay_p5f}
\endfigure



\subsection{Block recursive equilibrium computation}
\index{block recursive equilibrium}%
\auth{Menzio, Guido} \auth{Shi, Shouyong}%
The value and policy functions and measures
of market tightness described in the previous subsections are independent
of the distributions of workers across age and employment and
unemployment states. It is analytically convenient to be able
to derive those quantities before  computing
distributions of workers in a steady state or along transition paths  or  in response to aggregate shocks.
For an example of
business cycle analysis in a matching model with directed
search, see Menzio and Shi (2011) who bring out the benefits  of such block recursive structures.
They show how agents' value and policy functions
%%, and measures of market tightness,
depend on the aggregate state of the economy
through realization of aggregate shocks only, and not
through endogenous distributions of workers across
employment and unemployment states.

In  our present framework, it is  easy to
compute a steady state. Under the assumption of a stationary
population in which the number of new labor market
entrants of age 1 equals the number of retiring  workers of
age $T+1$, the age-specific unemployment rates in a steady
state are computed as follows. At age 1, all new entrants
are unemployed since it takes at least one period to be
matched with a vacancy, $u_1=1$. The unemployment rates for
subsequent ages $i=2, 3, \ldots, T$ are then sequentially
computed as
$$
u_i \,=\, \Bigl[1 - \theta_{i-1} q(\theta_{i-1}) \Bigr] u_{i-1}
         + s (1-u_{i-1}) \,.                      \EQN M_OLG2_u
$$

Besides having enough matching functions to assure that each type of
worker-job pair is matched within its
own matching function, a block recursive equilibrium requires
that matching and production technologies both exhibit constant returns to scale and that financial funds
are available in perfectly elastic supply. The latter condition is actually
an equilibrium outcome under a common assumption in the
matching literature that preferences are linear in consumption.
Linear preferences make workers not only risk neutral but
also indifferent to the timing of consumption at a gross
interest equal to the inverse of their subjective discount
factor, $\beta^{-1}$, so that they are  willing to finance
new vacancy creation and to acquire ownership of existing
firms, valued by their `match capital' in filled jobs.
Moreover, under the assumption of linear preferences, those
asset holdings can be omitted from a worker's Bellman
equations because they do not affect equilibrium outcomes
in labor markets.\NFootnote{Many expositions of matching models don't  spell out a
general equilibrium  but instead
simply list workers and firms as separate actors,
taking  workers to be  hand-to-mouth consumers of
wage income or unemployment benefits, and firms as
profit maximizers while being silent about how they are  financed.
But as suggested above, an equilibrium of a
matching model with risk-neutral preferences can be interpreted
as a general equilibrium  analogous to ones in  growth models of earlier chapters:  workers own  firms that are merely intermediaries who  hire
factors of production and operate technologies to maximize
shareholder value.}
%In principle, except for the friction in labor
%markets, the functioning of firms and their financing
%are no different than that in the standard growth model
%in earlier chapters. {\bf Lars XXXXX: I think this footnote should be
%clarified or expanded. I find the last sentence
%needs some elaboration if it is to be linked to thoughts earlier in the
%footnote.}}

Other assumptions, such as a small
open economy and an internationally given interest rate,
could support a block recursive equilibrium,
%% and its accompanying analytical tractability,
but even when block recursivity fails to prevail,
the assumption of directed search simplifies equilibrium computation by eliminating congestion
externalities that would otherwise arise if heterogeneous
workers and jobs were to reside inside the same matching function.


%One alternative to linear preferences would be the assumption
%of a small open economy that with an internationally given interest
%rate together with complete markets. Contingent claims would
%then be used finance both investments in vacancy creation and
%households' demand for smooth consumption streams in the face
%of idiosyncratic employment risk. The analytical tractability
%would be preserved in such a block recursive {\it partial}
%equilibrium.



\subsection{Welfare analysis}
Because directed search eliminates congestion externalities,
we can expect a
decentralized equilibrium to  be Pareto optimal, conditional
on  Hosios efficiency conditions being satisfied. Moreover,
 the  block recursive structure in the present
framework means that the welfare maximum of each generation can be
studied in isolation from other generations. The
reason is that a block recursive structure
eliminates  feedbacks from aggregates to outcomes that can be
deduced from individual decision problems.
Recall that value and policy functions and measures of market
tightness can be derived
before computing distributions of workers across age and employment and
unemployment states.
%{\bf XXXXX distributions across??? please fill in}.
Our discussion in the previous subsection alerts us to
one caveat attached to this consequence of block recursiveness in our  present
overlapping generations
framework: each new generation enters the economy as
unemployed and  without the resources needed to create the vacancies
that will allow them eventually to become employed. But this caveat  constitutes no
obstacle to connecting the welfare analysis of
a single generation to that of  the overall economy, since
all agents are  indifferent to intertemporal trades that occur
at a gross rate of return $\beta^{-1}$. That is, any earlier born
worker with resources available in the present period and who will
be alive next period, is willing to postpone consumption to
fund vacancy creation in the present period in exchange
for repayment with interest next period.

The planner's optimization problem for a single  generation  $i$ becomes
$$\EQNalign{
& \max_{\theta_1, \{\theta_i, u_i\}_{i=2}^T} \;\;
        z u_1 \,-\, \theta_1 u_1 \,c                           \cr
&\hskip1.8cm  + \, \sum_{i=2}^T \beta^{i-1}
\Bigl\{ y \bigl[(1-s)(1-u_{i-1})
        + \theta_{i-1} q(\theta_{i-1}) u_{i-1}\bigr]
        + z u_i - \theta_i u_i c  \Bigr\}                      \cr
\noalign{\vskip.2cm}
&\hbox{subject to }\ \
 u_{i} \;=\;  \bigl[1-\theta_{i-1} q(\theta_{i-1})\bigr] u_{i-1}
     + s (1-u_{i-1})\,,
\hskip.5cm \hbox{\rm for     } i=2,\ldots , T ,  \cr
&\hbox{given \ \ }\hskip.8cm u_1=1\,.                                 \cr}
$$
The optimal choice of market tightness for the last age is
$\theta_T=0$ since there is  a cost term for posting such
vacancies but no future return because workers will  retire
at age $T+1$. The first-order conditions for the remaining
choice variables are
$$\EQNalign{
\theta_i:\hskip.5cm &-\beta^{i-1} u_i c
    + \beta^i y \bigl[q(\theta_i) + \theta_i q'(\theta_i) \bigr] u_i \cr
      &\hskip2.1cm   - \beta^i \lambda_{i+1}
                \bigl[q(\theta_i) + \theta_i q'(\theta_i) \bigr] u_i
 \;=\; 0 \,,
\hskip.5cm \hbox{\rm for     } i=1,\ldots , T-1   \cr
u_i:\hskip.5cm &\beta^{i-1} [z - \theta_i c ] - \beta^{i-1} \lambda_i
    + \beta^i y \bigl[-(1-s) + \theta_i q(\theta_i) \bigr]        \cr
     &\hskip2.5cm  - \beta^i \lambda_{i+1}
                \bigl[1 - \theta_i q(\theta_i) - s \bigr]  \;=\; 0 \,,
\hskip.5cm \hbox{\rm for     } i=2,\ldots , T-1   \cr
u_T:\hskip.5cm &\beta^{T-1} [z - \theta_T c ] - \beta^{T-1} \lambda_T
                                                         \;=\; 0   \,.}
$$
After rearranging and simplifying, the first-order conditions
can be written
$$\EQNalign{
\theta_i: \hskip.5cm & c \,=\, \beta q(\theta_i) [1-\alpha]
    (y - \lambda_{i+1})\,,
     \hskip.5cm \hbox{\rm for     } i=1,\ldots , T-1
                                   \hskip2cm
       \EQN M_OLG2_foc_thetai \cr
\noalign{\vskip.2cm}
u_i:\hskip.5cm &\lambda_i \,=\; z - \theta_i \,c
    - \beta \bigl[1 - \theta_i q(\theta_i) - s \bigr]
                                      (y - \lambda_{i+1})\,, \cr
&\hskip5cm  \hbox{\rm for     } i=2,\ldots , T-1   \hskip2cm
                                         \EQN M_OLG2_foc_ui  \cr
u_T:\hskip.5cm &\lambda_T \,=\; z \,,    \EQN M_OLG2_foc_uT  \cr}
$$
where $\alpha$ is the elasticity of matching with respect to
unemployment, $\alpha = - q'(\theta) \,\theta / q(\theta)$, and
the last expression invokes $\theta_T=0$. These equations
enable us to solve backward for measures of optimal market
tightness. Starting with terminal shadow value $\lambda_T=z$,
equation \Ep{M_OLG2_foc_thetai} for $i=T-1$ determines optimal
market tightness $\theta_{T-1}$. Next, given $\lambda_T$ and
$\theta_{T-1}$, equation \Ep{M_OLG2_foc_ui} for $i=T-1$
determines $\lambda_{T-1}$, and in that manner, we can continue
solving backward until we eventually  recover all measures of
optimal market tightness.

By applying this algorithm, the optimal market tightness
\hbox{$\theta_i$, for $i=1,\ldots,$} $T-1$, can be expressed recursively as
$$\EQNalign{
 & \left[1-\beta^{T-i} (1-s)^{T-i}\right]
             \left[y - z\right]                                   \cr
  & \;=\;  \left( \frac{r+s}{(1-\alpha) q(\theta_i)}\,+\,
           \left[1-\beta(1-s)\right] \frac{\alpha}{1-\alpha}
   \sum_{j=0}^{T-i-1} \beta^j (1-s)^j \theta_{i+1+j} \right) c .
                                    \hskip1.5cm   \EQN M_OLG2_PO \cr}
$$
Comparing this with equilibrium expression \Ep{M_OLG2_theta} confirms
our conjecture that the equilibrium allocation
is Pareto optimal so long as a worker's bargaining
power is equal to the elasticity of matching with respect
to unemployment, $\phi=\alpha$.



\section{Concluding remarks}
Ljungqvist and Sargent (2017) showed that in a variety of
matching models, productivity
changes affect both business cycle and welfare state dynamics through a single
intermediate channel called the fundamental surplus.
Thus, in studying welfare state dynamics,  Mortensen and Pissarides (1999b)
and Ljungqvist and Sargent (2007)  attribute the outbreak of
European unemployment after the late 1970s to changes in the
economic environment in conjunction with the  generous
unemployment benefits offered by  European government, i.e., a higher $z$ in the formulation of matching models in this chapter.
 In
a matching model with directed search by workers with permanently
different productivities, Mortensen and Pissarides (1999b)  model `skill-biased' technology shocks in terms of a mean preserving
spread of the distribution of productivities.
There is a convex inverse relationship between the unemployment rate
and worker productivity across submarkets, so
moving workers to a lower range of productivities causes a larger
increase in unemployment than a  decrease that would  caused by moving
workers to higher productivities. Because the relationship
becomes more convex for a higher  value of $z$,
 unemployment increases more in high $z$ Europe than in low $z$ America. In a matching model with
skill accumulation and unemployment benefits that are paid as a
fixed replacement rate of a worker's past earnings, Ljungqvist and
Sargent (2007) study how European unemployment erupts in
`turbulent' times, modeled as an increased risk of skill loss at   layoff events,
both under random search in a single matching function but more so
under directed search.
They conclude that ``the cost of posting vacancies is the
lynchpin, or to use a less kind metaphor, the tail that wags the
dog, of matching models.'' Then how is it that vacancy costs that are
commonly calibrated to be small relative to aggregate output turn out to  wag
the dog in some matching models  but not in others? The answer is that it all
depends  on whether  the fundamental surplus
fraction is small.  Here it helps to remember that the fundamental surplus fraction serves as  an upper bound on the fraction of a job's output
that the invisible hand can allocate to vacancy creation.
\index{unemployment!European}%
\auth{Lucas, Robert E., Jr.} \auth{Prescott, Edward C.}%
\auth{Ljungqvist, Lars} \auth{Sargent, Thomas J.}%
\auth{Pissarides, Christopher} \auth{Mortensen, Dale T.}%


As mentioned in the introduction of this chapter as well as
in the concluding remarks of chapter \use{search2},
an important difference between  matching models and search models
is whether  there are congestion externalities.  It is helpful to recall how  Lucas
and Prescott (1974) and Pissarides (1992) summarized  these distinct frameworks.  In their search-island
economy, Lucas and Prescott remarked that ``the injury a
searching worker imposes on his fellows is of exactly the same
type as the injury a seller of any good imposes on his fellow
sellers: the equilibrium expected return from job search serves
the function of any other equilibrium price of signalling to
suppliers the correct social return from an additional unit
supplied.'' Things are very different in Pissarides's
(1992) matching model with two-period-lived overlapping
generations in which workers who remain unemployed in the first period of life
lose skills. Because all unemployed workers congest the same
matching function, ``a temporary shock to employment can persist
for a long time [outlasting the maximum duration of any worker's
unemployment]. The key mechanism is a thin market externality
that reduces the supply of jobs when the duration of unemployment
increases. ... persistence and multiple equilibria are possible
even with constant returns production and matching technologies.''

Directed search in matching models disarms
congestion externalities because heterogeneous workers (and/or
heterogeneous jobs) no longer sit in the same matching
function. Directed search simultaneously simplifies  equilibrium computation and eradicates congestion
externalities.\NFootnote{Scope for beneficial government interventions remain  in matching
models with directed search  whenever the elasticity of
a matching function with respect to unemployment does not equal
 a worker's Nash bargaining power $\phi$, i.e., whenever the Hosios
efficiency condition is violated.}
%Some matching models with directed search can accommodate
%heterogeneous workers who
In some matching models with directed search, all types of
heterogeneous workers
prefer to sit in their assigned
matching functions, but there are other  models populated by some  workers
who would like  to `sneak' into another matching
function. For example, an older worker in section \use{sec:OLG2} would  prefer
to sit in a matching function with younger workers and thereby enjoy a higher
job finding probability. A firm encountering such a deviant
job applicant would be disappointed with that worker's type, but nevertheless  to recover some of  its sunk vacancy-posting cost   would  engage in Nash bargaining with the older worker and
form a match because the match surplus is positive.

